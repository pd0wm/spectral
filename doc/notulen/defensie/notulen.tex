%!TEX program=xelatex

\documentclass[12pt, a4paper]{article}

\usepackage{booktabs}
\usepackage{xcolor}
\usepackage[dutch]{babel}
\usepackage[strict,autostyle]{csquotes}
\usepackage{hyperref}
\usepackage{fancyhdr}
\usepackage{lipsum}
\usepackage{datetime}
\usepackage{fullpage}
\usepackage{eurosym}
\usepackage{xcolor,mdframed}
\usepackage{tikz}
\usepackage{framed}

\usepackage{fontspec}

% Fancy quotes
\newfontfamily\quotefont[Ligatures=TeX]{Palatino Linotype} % selects Libertine as the quote font

% \newcommand*\quotesize{60} % if quote size changes, need a way to make shifts relative
% Make commands for the quotes
% \newcommand*{\openquote}
%    {\tikz[remember picture,overlay,xshift=-4ex,yshift=-2.5ex]
%    \node (OQ) {\quotefont\fontsize{\quotesize}{\quotesize}\selectfont``};\kern0pt}

% \newcommand*{\closequote}[1]
%   {\tikz[remember picture,overlay,xshift=4ex,yshift={#1}]
%    \node (CQ) {\quotefont\fontsize{\quotesize}{\quotesize}\selectfont''};}

% select a colour for the shading
\colorlet{shadecolor}{white}

\newcommand*\shadedauthorformat{\emph} % define format for the author argument

% Now a command to allow left, right and centre alignment of the author
\newcommand*\authoralign[1]{%
  \if#1l
    \def\authorfill{}\def\quotefill{\hfill}
  \else
    \if#1r
      \def\authorfill{\hfill}\def\quotefill{}
    \else
      \if#1c
        \gdef\authorfill{\hfill}\def\quotefill{\hfill}
      \else\typeout{Invalid option}
      \fi
    \fi
  \fi}
% wrap everything in its own environment which takes one argument (author) and one optional argument
% specifying the alignment [l, r or c]
%
\newenvironment{shadequote}[2][l]%
{\authoralign{#1}
\ifblank{#2}
   {\def\shadequoteauthor{}\def\yshift{-2ex}\def\quotefill{\hfill}}
   {\def\shadequoteauthor{\par\authorfill\shadedauthorformat{#2}}\def\yshift{2ex}}
\begin{snugshade}\begin{quote}\openquote}
{\shadequoteauthor\quotefill\closequote{\yshift}\end{quote}\end{snugshade}}


% Setup actiepunten
\newenvironment{important}[1][]{%
   \begin{mdframed}[%
      backgroundcolor={red!15}, hidealllines=true,
      skipabove=0.7\baselineskip, skipbelow=0.7\baselineskip,
      splitbottomskip=2pt, splittopskip=4pt, #1]%
   \makebox[0pt]{% ignore the withd of !
      \smash{% ignor the height of !
         \fontsize{32pt}{32pt}\selectfont% make the ! bigger
         \hspace*{-19pt}% move ! to the left
         \raisebox{-2pt}{% move ! up a little
            {\color{red!70!black}\sffamily\bfseries !}% type the bold red !
         }%
      }%
   }%
}{\end{mdframed}}
\newcommand{\excl}[1]{
\begin{important}
  \textbf{#1}
\end{important}
}
\newcommand{\excll}[2]{
\begin{important}
  \textbf{#1}\\#2
\end{important}
}

% Setup fonts
\defaultfontfeatures{Ligatures=TeX} % To support LaTeX quoting style
\setmainfont[Ligatures={Common,TeX}]{Palatino Linotype}
\newfontfamily\hdrfont{Palatino Linotype}

% Setup hyperlinks
\hypersetup{
    colorlinks,
    citecolor=black,
    filecolor=black,
    linkcolor=black,
    urlcolor=black
}

% Setup header
\pagestyle{fancy}
%\rhead{\includegraphics[height=3cm]{kascomlogo.png}}
%\lhead{\includegraphics[height=3cm]{bondslogo.pdf}}
\rhead{}
\lhead{}
\chead{\hdrfont \hspace{4cm} {\Huge \bf \sc Cognitive Radio} \vspace{0.5cm} \\ \rule{15cm}{0.5pt} \vspace{0.2cm} \\ {\Large \sc Notulen sessie Defensie \displaydate{date}} \vspace{0.2cm}}
\usepackage[headsep=.5cm, top=5cm, headheight=10cm, heightrounded]{geometry}

% Setup date
\newdate{date}{08}{06}{2015}
\date{\displaydate{date}}

\begin{document}
\section{Aanwezigen}
\begin{tabular}{p{0.4\textwidth}p{0.5\textwidth}}
    Mariniers
    \begin{itemize}
        \item Koen ter Aa
        \item Hans de Boer
        \item Dennis Borst
        \item Jochem de Hollander
        \item van Houten
        \item Wouter van Maaren
        \item Eric Verheien
    \end{itemize} &

    Studenten
    \begin{itemize}
        \item Wessel Bruinsma
        \item Robin Hes
        \item Kees Kroep
        \item Dorus Leliveld
        \item Willem Melching
        \item Tom aan de Wiel
    \end{itemize} \\
\end{tabular}


\section{Punten vooraf}
\begin{itemize}
  \item Informeel overleg
  \item Aanwezigen voornamelijk `eindgebruikers' van apparatuur
  \item De gegeven achtergrondinformatie (documentje) was bruikbaar
\end{itemize}

\section{Na de presentatie}
\begin{itemize}
    \item Onduidelijkheid over de wijze van onderscheid tussen gewenste en ongewenste communicatie
    \begin{itemize}
        \item A priori informatie is sleutelwoord
        \item Interesse naar herkennen van modulatietechnieken
    \end{itemize}

    \item Interesse naar frequentiebereik van de techniek
    \begin{itemize}
        \item Communicatie is verspreid over een breed gebied (>1 GHz)
        \item Theoretisch oneindig voor oneindige tijd
        \item Afhankelijk van gebruikte hardware
    \end{itemize}

    \item Interesse naar spatiaal bereik van techniek
    \begin{itemize}
        \item Bij inzet is het belangrijk om te weten hoe ver de effecten reiken
        \item Alleen storen wat gestoord moet worden (TV, radio en mobiele diensten mogen bijv. niet verstoord worden)
    \end{itemize}

    \item De niet-technische presentatie bleek zeker niet te moeilijk te zijn
\end{itemize}

\section{Na de technische presentatie}
\begin{itemize}
    \item Meer nadruk op gebruik van meerdere samplers, niet meerdere antennes (zie illustratie)

    \item Verleg nadruk van \emph{smart jamming} naar \emph{smart sensing}
    \begin{itemize}
        \item Ook vanuit projectopdracht een goed idee
        \item Jamming is nog steeds dom, alleen gericht na smart sensing
        \item Verduidelijken dat het vooralsnog vooral om een softwareoplossing gaat
    \end{itemize}

    \item Met betrekking tot integratie
    \begin{itemize}
        \item Integratie met bestaande systemen belangrijk pluspunt
        \item Modulariteit belangrijk
        \item Formaat belangrijk
        \item Gebruiksgemak
        \item Meerdere form-factors: schip, voertuig, portable (`signaalvrije bubbel'), etc.
    \end{itemize}

    \item Ook potentie bij het zoeken van een vrij kanaal om te zenden (dynamische zendsystemen)

    \item TNO
    \begin{itemize}
        \item Bij TNO loopt mogelijk onderzoek in onze richting
        \item Mogelijk geheim, men mocht hier niet te veel over zeggen
        \item Professor Zwamborn (TU Eindhoven) mogelijk contactpersoon
        \item Introductie via e-mail door Hans de Boer
        \item Let op intellectueel eigendom
    \end{itemize}

    \item Side-effects
    \begin{itemize}
        \item Let op stralingsrisico's voor zowel personeel als omstanders
        \item Nogmaals, storen van openbare zaken als telefoon kan echt niet
        \item Let op effecten door modificaties door de eindgebruikers (voorbeeld prikkeldraad op pantservoertuig)
        \item Systeem moet hufter-proof zijn
    \end{itemize}

    \item Bij storen, probeer er voor te zorgen dat de tegenstander het niet door heeft
    \item Resolutie is een belangrijke specificatie bij aanliggende frequentiebanden
\end{itemize}

\section{Discussie}
\begin{itemize}
    \item Suggestie: spatiaal gericht storen
    \begin{itemize}
        \item Maakt jammen met minder side-effects mogelijk
        \item Mogelijk met array-antennas
        \item Bij voorkeur variabel
        \item Future work?
    \end{itemize}

    \item Scenario's
    \begin{itemize}
        \item Marine: boardingoperaties
        \begin{itemize}
            \item Anti-piraterij
            \item Niet alleen storen, maar ook kennis over communicatie van tegenstander is waardevol
            \item Jamming wordt hier al toegepast
        \end{itemize}

        \item Grootschalige operaties
        \begin{itemize}
            \item Anti-terreur, op meerdere locaties tegelijk (recent: Frankrijk)
            \item Ondermijnen van coördinatie terroristen
            \item Inzet van drones is een goede toepassing
        \end{itemize}

        \item Deconflictering
        \begin{itemize}
            \item Lastig om informatie te krijgen van bondgenoten over gebruikte frequenties
            \item Ook binnen NATO is men hier geheimzinnig over
            \item Moeilijk om zo vriendelijke en vijandige communicatie te scheiden
            \item Belangrijke hindernis
        \end{itemize}

        \item Grootschalige detectie
        \begin{itemize}
            \item Gebruik van technologie om af te luisteren frequenties te detecteren
            \item Voor grootschalige surveillance
            \item Concept bestaat al in de vorm van sweep (\url{aronia.de}, <10.000 euro)
        \end{itemize}

        \item Sales
        \begin{itemize}
            \item Defensie koopt producten, geen technieken
            \item De mogelijke prijs voor een product schaalt met de mate waarin het de veiligheid vergroot
            \item DMP-cyclus

            \begin{itemize}
                \item Behoefte $\rightarrow$ planning $\rightarrow$ aanbesteding $\rightarrow$ productie
                \item Bij geringe kosten kan afgeweken worden van de cyclus
                \item De duur van de cyclus (jaren) zorgt voor complicaties (deprecation on release)
            \end{itemize}

            \item Kleine bedrijven kunnen worden gesubsidieerd en begeleid
            \item Nogmaals, hiervoor moet een product in het vooruitzicht worden gesteld
            \item Conclusie: laat je uitkopen door bestaande leverancier

            \item Overige punten sales
            \begin{itemize}
                \item Prijs is niet te bepalen zonder idee van wat er geleverd wordt
                \item Prijzen voor softwarepakketten algemeen lopen uiteen
                \item Denk bij prijsbepaling in businessplan aan alle aspecten: levering \& installatie, hardware, software, documentatie, opleiding, service
            \end{itemize}
        \end{itemize}

        \item Concurrenten/Partners \\
        \begin{tabular}{p{0.3\textwidth}p{0.4\textwidth}}
            \begin{itemize}
                \item General Dynamic (Bowman)
                \item Thales
                \item Thomson Aerospace \& Defence
                \item Bull
                \item L-3 Communications
            \end{itemize} &

            \begin{itemize}
                \item Wolf (?)
                \item Telefunken Racoms (Airbus?)
                \item Rohde \& Schwarz
                \item Harris Corporation
                \item Raytheon
            \end{itemize} \\
        \end{tabular}

        \item Regelgeving jammers
        \begin{itemize}
            \item Binnen Nederland zeer strikte regelgeving, afhankelijk van situatie
            \item Alleen EODD heeft licentie voor gebruik
            \item Buiten Nederland hangt het af van de afspraken met lokale regering en wetgeving
            \item Maritiem lijken de regels niet-bestaand of vaag
            \item Bij natie vs. natie gelden in principe geen regels
        \end{itemize}
    \end{itemize}
\end{itemize}

\section{Overige zaken}
\begin{itemize}
    \item Kosten bepaalde radio: 70k
    \item Benchmark scannen: 1 GHz per ms
    \item Vijand $\rightarrow$ tegenstander
    \item IED $\rightarrow$ radio-controlled IED
    \item IED onschadelijk maken $\rightarrow$ IED neutraliseren
\end{itemize}

\section{Enquête}


\end{document}
