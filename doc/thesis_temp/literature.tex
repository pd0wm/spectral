%!TEX program = xelatex

\documentclass[a4paper, openany, oneside]{memoir}
\usepackage[no-math]{fontspec}
\usepackage{pgfplots}
\pgfplotsset{compat=newest}
\usepackage{commath}
\usepackage{mathtools}
\usepackage{amssymb}
\usepackage{amsthm}
\usepackage{booktabs}
\usepackage{mathtools}
\usepackage{xcolor}
\usepackage[separate-uncertainty=true, per-mode=symbol]{siunitx}
\usepackage[noabbrev, capitalize]{cleveref}
\usepackage{listings}
\usepackage[american inductor, european resistor]{circuitikz}
\usepackage{amsmath}
\usepackage{amsfonts}
\usepackage{ifxetex}
\usepackage[dutch,english]{babel}
\usepackage[backend=bibtexu,texencoding=utf8,bibencoding=utf8,style=ieee,sortlocale=en_GB,language=auto]{biblatex}
\usepackage[strict,autostyle]{csquotes}
\usepackage{parskip}
\usepackage{import}
\usepackage{standalone}
\usepackage{hyperref}
%\usepackage[toc,title,titletoc]{appendix}

\ifxetex{} % Fonts laden in het geval dat je met Xetex compiled
    \usepackage{fontspec}
    \defaultfontfeatures{Ligatures=TeX} % To support LaTeX quoting style
    \setromanfont{Palatino Linotype} % Tover ergens in Font mapje in root.
    \setmonofont{Source Code Pro}
\else % Terug val in standaard pdflatex tool chain. Geen ondersteuning voor OTT fonts
    \usepackage[T1]{fontenc}
    \usepackage[utf8]{inputenc}
\fi
\newcommand{\references}[1]{\begin{flushright}{#1}\end{flushright}}
\renewcommand{\vec}[1]{\boldsymbol{\mathbf{#1}}}
\newcommand{\uvec}[1]{\boldsymbol{\hat{\vec{#1}}}}
\newcommand{\mat}[1]{\boldsymbol{\mathbf{#1}}}
\newcommand{\fasor}[1]{\boldsymbol{\tilde{\vec{#1}}}}
\newcommand{\cmplx}[0]{\mathrm{j}}
\renewcommand{\Re}[0]{\operatorname{Re}}
\newcommand{\Cov}{\operatorname{Cov}}
\newcommand{\Var}{\operatorname{Var}}
\newcommand{\proj}{\operatorname{proj}}
\newcommand{\Perp}{\operatorname{perp}}
\newcommand{\col}{\operatorname{col}}
\newcommand{\rect}{\operatorname{rect}}
\newcommand{\sinc}{\operatorname{sinc}}
\newcommand{\IT}{\operatorname{IT}}
\newcommand{\F}{\mathcal{F}}

\newtheorem{definition}{Definition}
\newtheorem{theorem}{Theorem}


\DeclareSIUnit{\voltampere}{VA} %apparent power
\DeclareSIUnit{\pii}{\ensuremath{\pi}}

\hypersetup{%setup hyperlinks
    colorlinks,
    citecolor=black,
    filecolor=black,
    linkcolor=black,
    urlcolor=black
}

% Example boxes
\usepackage{fancybox}
\usepackage{framed}
\usepackage{adjustbox}
\newenvironment{simpages}%
{\AtBeginEnvironment{itemize}{\parskip=0pt\parsep=0pt\partopsep=0pt}
\def\FrameCommand{\fboxsep=.5\FrameSep\shadowbox}\MakeFramed{\FrameRestore}}%
{\endMakeFramed}

% Impulse train
\DeclareFontFamily{U}{wncy}{}
\DeclareFontShape{U}{wncy}{m}{n}{<->wncyr10}{}
\DeclareSymbolFont{mcy}{U}{wncy}{m}{n}
\DeclareMathSymbol{\Sha}{\mathord}{mcy}{"58}
\addbibresource{../includes/bibliography.bib}

\title{Thesis}

\author{W.P. Bruinsma \and R.P. Hes \and H.J.C. Kroep \and T.C. Leliveld \and W.M. Melching \and T.A. aan de Wiel}

\raggedbottom

\begin{document}
\frontmatter

\begin{titlingpage}
  \pagestyle{empty}
  \maketitle
\end{titlingpage}

\chapter{Reconstruction}
Let the input signal be denoted by $\vec{x} \in \mathbb{R}^{LN}$. Consider $\vec{c}_i \in \mathbb{R}^{N}$ for $i = 1,\ldots,M$.

\begin{definition}[Convolution]
    Let $\vec{x} \in \mathbb{R}^N$ and $\vec{y} \in \mathbb{R}^M$. Then $\vec{x} \ast \vec{y}$ denotes a vector $\vec{z} \in \mathbb{R}^{N+M-1}$ such that
    \begin{align*}
        (\vec{z})_i = \sum_{k=1}^{N} (\vec{x})_k (\vec{y})_{i-k+1}
    \end{align*}
    where $(\vec{x})_i=0$ for $i < 1$ and $i > N$ and $(\vec{y})_i=0$ for $i < 1$ and $i > M$.
\end{definition}

\begin{theorem}[Commutativity of Convolution]
    Let $\vec{x} \in \mathbb{R}^N$ and $\vec{y} \in \mathbb{R}^M$. Then $\vec{x} \ast \vec{y} = \vec{y} \ast \vec{x}$.
\end{theorem}
\begin{proof}
    A change of index yields that
    \begin{align*}
        (\vec{x} \ast \vec{y})_i &= \sum_{k=1}^{N} (\vec{x})_k (\vec{y})_{i-k+1} \\
        &= \sum_{k=-\infty}^{\infty} (\vec{x})_k (\vec{y})_{i-k+1} \\
        &= \sum_{k'=-\infty}^{\infty} (\vec{y})_{k'} (\vec{x})_{i-k'+1} \\
        &= \sum_{k'=1}^{M} (\vec{y})_{k'} (\vec{x})_{i-k'+1} \\
        &= (\vec{y} \ast \vec{x})_i.
    \end{align*}
\end{proof}

\begin{definition}[Correlation]
    Let $\vec{x} \in \mathbb{R}^N$ and $\vec{y} \in \mathbb{R}^M$. Then $\vec{x} \circ \vec{y}$ denotes a vector $\vec{z} \in \mathbb{R}^{N+M-1}$ such that
    \begin{align*}
        (\vec{z})_i = \sum_{k=1}^{N} (\vec{x})_k (\vec{y})_{M-i+k}
    \end{align*}
    where $(\vec{x})_i=0$ for $i < 1$ and $i > N$ and $(\vec{y})_i=0$ for $i < 1$ and $i > M$.
\end{definition}

Let $\vec{y}_i = \vec{c}_i \ast \vec{x}$ for $i = 1,\ldots,M$.

\begin{theorem}
    \begin{align*}
        \vec{y}_i \circ \vec{y}_j = (\vec{c}_i \circ \vec{c}_j) \ast (\vec{x} \circ \vec{x}).
    \end{align*}
\end{theorem}
\begin{proof}
    Note that
    \begin{align*}
        (\vec{y}_i \circ \vec{y}_j)_m
        &= [(\vec{c}_i \ast \vec{x}) \circ (\vec{c}_j \ast \vec{x})]_m \\
        &=\sum_{k''=1}^{LN+N-1}\sum_{k=1}^N (\vec{c}_i)_k (\vec{x})_{k''-k+1}\sum_{k'=1}^{N}(\vec{c}_j)_{k'}(\vec{x})_{(LN+N-1-m+k'')-k'+1} \\
        &=\sum_{k=-\infty}^\infty\sum_{k'=\infty}^{\infty}\sum_{k''=-\infty}^{\infty} (\vec{c}_i)_k (\vec{c}_j)_{k'}(\vec{x})_{k''-k+1}(\vec{x})_{(LN+N-1-m+k'')-k'+1}.
    \end{align*}
    To further evaluate this expression, we introduce a change of variables. To this end, let $k' = N -l'' +k$ and $k'' = l' + k - 1$. This transformation is invertible, so
    \begin{align*}
        (\vec{y}_i \circ \vec{y}_j)_m
        % I don't think this step is necessary
        % &=\sum_{k=-\infty}^\infty\sum_{l''=-\infty}^{\infty}\sum_{l'=\infty}^{\infty} (\vec{c}_i)_k (\vec{c}_j)_{N -l'' +k}(\vec{x})_{(l' + k - 1)-k+1}(\vec{x})_{[LN+N-1-m+(l' + k - 1)]-(N -l'' +k)+1} \\
        &=\sum_{k=-\infty}^\infty\sum_{l''=-\infty}^{\infty}\sum_{l'=\infty}^{\infty} (\vec{c}_i)_k (\vec{c}_j)_{N -l'' +k}(\vec{x})_{l'}
        (\vec{x})_{LN-(m-l'' + 1)+l'} \\
        &=\sum_{l''=1}^{2N-1}\sum_{k=1}^{N}(\vec{c}_i)_k (\vec{c}_j)_{N -l'' +k}\sum_{l'=1}^{LN}(\vec{x})_{l'}
        (\vec{x})_{LN-(m-l'' + 1)+l'} \\
        &=\sum_{k''=1}^{2N-1} \sum_{k=1}^{N} (\vec{c}_i)_k (\vec{c}_j)_{N-k''+k} \sum_{k'=1}^{LN} (\vec{x})_{k'} (\vec{x})_{LN-(m-k''+1)+k'} \\
        &=[(\vec{c}_i \circ \vec{c}_j) \ast (\vec{x} \circ \vec{x})]_m.
    \end{align*}
\end{proof}

We denote 
\begin{align*}
    \vec{r}_{y_i,y_j} = \vec{y}_i \circ \vec{y}_j = (\vec{c}_i \circ \vec{c}_j) \ast (\vec{x} \circ \vec{x}) = \vec{r}_{c_i,c_j} \ast \vec{r}_x.
\end{align*}
Using commutativity and the definition of the convolution operator, we can write this equation as
\begin{align*}
    \vec{r}_{y_i,y_j} &=  \vec{r}_{c_i,c_j} \ast \vec{r}_x \\
    &= \vec{r}_x \ast \vec{r}_{c_i,c_j} \\
    &= \begin{bmatrix}
        (\vec{r}_{c_i,c_j})_1 & 0 & 0& \cdots & & &  0 \\
        (\vec{r}_{c_i,c_j})_2 & (\vec{r}_{c_i,c_j})_1 & 0 & \cdots & & & 0 \\
        \vdots &  & & \ddots & &  & \vdots\\
        \cdots & (\vec{r}_{c_i,c_j})_{2N-1} & (\vec{r}_{c_i,c_j})_{2N-2} & \cdots & (\vec{r}_{c_i,c_j})_2 & (\vec{r}_{c_i,c_j})_1 & 0 \\
        \cdots & 0 & (\vec{r}_{c_i,c_j})_{2N-1} & \cdots & (\vec{r}_{c_i,c_j})_3 & (\vec{r}_{c_i,c_j})_2 & (\vec{r}_{c_i,c_j})_1 \\
    \end{bmatrix} \begin{bmatrix}
        (\vec{r}_x)_1 \\
        (\vec{r}_x)_2 \\
        \vdots \\
        (\vec{r}_x)_{2LN-2} \\
        (\vec{r}_x)_{2LN-1}
    \end{bmatrix} \\
    &= \mat{R}_{c_i,c_j} \vec{r}_x.
\end{align*}
Now let
\begin{align*}
    \vec{r}_y &= \begin{bmatrix}
        \vec{r}_{y_1,y_1} \\
        \vdots \\
        \vec{r}_{y_M,y_M}
    \end{bmatrix} = \begin{bmatrix}
        \mat{R}_{c_1,c_1} \vec{r}_x \\
        \vdots \\
        \mat{R}_{c_M,c_M} \vec{r}_x
    \end{bmatrix} = \begin{bmatrix}
        \mat{R}_{c_1,c_1}\\
        \vdots \\
        \mat{R}_{c_M,c_M}
    \end{bmatrix} \vec{r}_x = \mat{R} \vec{r}_x.
\end{align*}
\end{document}
