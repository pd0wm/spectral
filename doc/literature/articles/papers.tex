
\documentclass[report, oneside, a4paper, openany]{memoir}
\usepackage[no-math]{fontspec}
\usepackage{pgfplots}
\usepackage{float}
\pgfplotsset{compat=newest}
\usepackage{commath}
\usepackage{mathtools}
\usepackage{amssymb}
\usepackage{amsthm}
\usepackage{booktabs}
\usepackage{todonotes}
\usepackage{mathtools}
\usepackage{xcolor}
\usepackage[separate-uncertainty=true, per-mode=symbol]{siunitx}
\usepackage{listings}
\usepackage[american inductor, european resistor]{circuitikz}
\usepackage{amsmath}
\usepackage{amsfonts}
\usepackage{ifxetex}
\usepackage[dutch,english]{babel}
\usepackage[backend=bibtexu,texencoding=utf8,bibencoding=utf8,style=ieee,sortlocale=en_GB,language=auto]{biblatex}
\usepackage[strict,autostyle]{csquotes}
\usepackage{import}
\usepackage{standalone}
\usepackage{bookmark,hyperref}
\usepackage{xcolor,mdframed}
\usepackage{tikz}
\usepackage{framed}
\usepackage{float}
\usepackage{tabularx}
\usepackage{graphicx,adjustbox}
\usepackage{rotating}
\usepackage{pdfpages}
\usepackage{enumitem}
\usepackage{calc}
\usepackage{pgfplots}
\usepackage{filecontents}
\usepackage{caption}
\usepackage{subcaption}
\usepackage{lettrine}

\newcolumntype{Y}{>{\raggedright\arraybackslash}X} % Left-justified text in tabularx environment

\ifxetex{} % Fonts laden in het geval dat je met Xetex compiled
    \usepackage{fontspec}
    \defaultfontfeatures{Scale=MatchLowercase, Ligatures=TeX} % To support LaTeX quoting style
    %\setromanfont{Palatino Linotype} % Tover ergens in Font mapje in root.
    \setsansfont{Avenir Next LT Pro}
    \setromanfont{Adobe Caslon Pro} % Tover ergens in Font mapje in root.
    \setmonofont{Source Code Pro}
\else % Terug val in standaard pdflatex tool chain. Geen ondersteuning voor OTT fonts
    \usepackage[T1]{fontenc}
    \usepackage[utf8]{inputenc}
\fi
\usepackage[noabbrev, capitalize]{cleveref}
\usepackage{ifthen}
\usepackage{titlesec}
\usepackage{titlecaps}

\newcommand{\references}[1]{\begin{flushright}{#1}\end{flushright}}
\renewcommand{\vec}[1]{\boldsymbol{\mathbf{#1}}}
\newcommand{\uvec}[1]{\boldsymbol{\hat{\vec{#1}}}}
\newcommand{\mat}[1]{\boldsymbol{\mathbf{#1}}}
\newcommand{\fasor}[1]{\boldsymbol{\tilde{\vec{#1}}}}
\newcommand{\cmplx}[0]{\mathrm{j}}
\renewcommand{\Re}[0]{\operatorname{Re}}
\newcommand{\Cov}{\operatorname{Cov}}
\newcommand{\Var}{\operatorname{Var}}
\newcommand{\proj}{\operatorname{proj}}
\newcommand{\Perp}{\operatorname{perp}}
\newcommand{\col}{\operatorname{col}}
\newcommand{\rect}{\operatorname{rect}}
\newcommand{\sinc}{\operatorname{sinc}}
\newcommand{\lcm}{\operatorname{lcm}}
%\newcommand{\gcd}{\operatorname{gcd}}
\newcommand{\F}{\mathcal{F}}
\newcommand{\DTFT}{\mathcal{F}_*}
\newcommand{\conj}[1]{#1^*}
\renewcommand{\mod}{\operatorname{mod}}
\newcommand{\rot}{\operatorname{rot}}
\newcommand{\vecsc}[1]{\vec{\textsc{\textbf{#1}}}}
\renewcommand{\ss}[1]{_{#1}}

% Label without linebreak breaker
\newcommand{\lab}[1]{\label{#1}\nolinebreak}

\newtheorem{definition}{Definition}
\newtheorem{theorem}{Theorem}


\DeclareSIUnit{\voltampere}{VA} %apparent power
\DeclareSIUnit{\pii}{\ensuremath{\pi}}

\hypersetup{%setup hyperlinks
    colorlinks,
    citecolor=black,
    filecolor=black,
    linkcolor=black,
    urlcolor=black
}

% Example boxes
\usepackage{fancybox}
\usepackage{framed}
\usepackage{adjustbox}
\newenvironment{simpages}%
{\AtBeginEnvironment{itemize}{\parskip=0pt\parsep=0pt\partopsep=0pt}
\def\FrameCommand{\fboxsep=.5\FrameSep\shadowbox}\MakeFramed{\FrameRestore}}%
{\endMakeFramed}

% Impulse train
\DeclareFontFamily{U}{wncy}{}
\DeclareFontShape{U}{wncy}{m}{n}{<->wncyr10}{}
\DeclareSymbolFont{mcy}{U}{wncy}{m}{n}
\DeclareMathSymbol{\Sha}{\mathord}{mcy}{"58}

\setlength{\parindent}{0pt}
\nonzeroparskip

% Block environment configuration
\newcommand{\BlockLeftMargin}{20pt}
\newcommand{\BlockLeftMarginText}{25pt}
\newcommand{\BlockLeftMarginTextSpacing}{10pt}

% Own colours
\definecolor{gray75}{gray}{0.75}

% Block environment
\newenvironment{block}[3]{%
\makebox{\hspace{-\spinemargin}%
\begin{tikzpicture}[overlay]
    \draw [thick,color=gray75] (\BlockLeftMargin, 0) -- (\paperwidth - \spinemargin, 0);
    \node at (\BlockLeftMarginText, -0.9) [align=left, text width=\spinemargin - \BlockLeftMarginText - \BlockLeftMarginTextSpacing, anchor=west, text depth=1cm] {\textbf{\textsc{#1}}\newline\textit{#3}};
\end{tikzpicture}}%
\nopagebreak\\[0.25em]\ifthenelse{\equal{#2}{}}{}{(\textit{#2}.) }\nopagebreak\nolinebreak}
{\nopagebreak\\[-0.25em]%
\makebox{\hspace{-\spinemargin}%
\begin{tikzpicture}[overlay, remember picture]
    \draw [thick,color=gray75] (\spinemargin,0) -- (\paperwidth - \spinemargin,0);
\end{tikzpicture}} \vspace{0.5em}}

% Theorem
\newcounter{blockTheoremCounter}
\crefname{blockTheoremCounter}{Theorem}{Theorems}
\Crefname{blockTheoremCounter}{Theorem}{Theorems}

\newenvironment{blockTheorem}[1][]{%
\refstepcounter{blockTheoremCounter}%
\begin{block}{theorem \theblockTheoremCounter}{#1}{}}
{\end{block}}

% Definition
\newcounter{blockDefinitionCounter}
\crefname{blockDefinitionCounter}{Definition}{Definitions}
\Crefname{blockDefinitionCounter}{Definition}{Definitions}

\newenvironment{blockDefinition}[1][]{%
\refstepcounter{blockDefinitionCounter}%
\begin{block}{definition \theblockDefinitionCounter}{#1}{}}
{\end{block}}

% Proof
\newcounter{blockProofTheoremCounter}
\crefname{blockProofTheoremCounter}{Proof}{Proofs}
\Crefname{blockProofTheoremCounter}{Proof}{Proofs}

\newenvironment{blockProofTheorem}[1]{%
\refstepcounter{blockProofTheoremCounter}%
\begin{block}{proof of \\ theorem #1}{}{}}
{\qed\end{block}}

% Detail
\newcounter{blockDetailCounter}
\crefname{blockDetailCounter}{Detail}{Details}
\Crefname{blockDetailCounter}{Detail}{Details}

\newenvironment{blockDetail}[1][]{%
\refstepcounter{blockDetailCounter}%
\begin{block}{detail \theblockDetailCounter}{#1}{}}
{\end{block}}

% Redesign chapter headings
\newcommand{\chapternumber}{\thechapter}
\newcommand{\hsp}{\hspace{20pt}}
\titleformat{\chapter}[hang]{\Huge\bfseries}{\chapternumber\hsp\textcolor{gray75}{|}\hsp}{0pt}{\Huge\bfseries}

% Remove headers
% \addtopsmarks{headings}{}{
%   \createmark{chapter}{left}{nonumber}{}{}
% }
% \pagestyle{headings} % Activate changes

% Capitalise headers in a regular way
\renewcommand*{\memUChead}[1]{\titlecap{#1}}

% \hfill for math mode
\newcommand{\pushright}[1]{\intertext{\hfill$\displaystyle #1$}}
\newcommand{\pushline}{\hskip \textwidth minus \textwidth}
\newcommand{\matlab}{\textsc{Matlab}}

\definecolor{code-grey}{HTML}{DDDDDD}
\newcommand{\lib}[1]{\textsf{#1}}
\newcommand{\file}[1]{\textsf{#1}}
\newcommand{\func}[1]{\colorbox{code-grey}{\texttt{#1}}}
\newcommand{\class}[1]{\colorbox{code-grey}{\texttt{#1}}}

% Setup actiepunten
\newenvironment{important}[1][]{%
   \begin{mdframed}[%
      backgroundcolor={red!15}, hidealllines=true,
      skipabove=0.7\baselineskip, skipbelow=0.7\baselineskip,
      splitbottomskip=2pt, splittopskip=4pt, #1]%
   \makebox[0pt]{% ignore the withd of !
      \smash{% ignor the height of !
         \fontsize{32pt}{32pt}\selectfont% make the ! bigger
         \hspace*{-19pt}% move ! to the left
         \raisebox{-2pt}{% move ! up a little
            {\color{red!70!black}\sffamily\bfseries !}% type the bold red !
         }%
      }%
   }%
}{\end{mdframed}}
\newcommand{\excl}[1]{
\begin{important}
  \textbf{#1}
\end{important}
}

\makeatletter
\newcommand\footnoteref[1]{\protected@xdef\@thefnmark{\ref{#1}}\@footnotemark}
\makeatother

% Allow page breaks in display environments
%\allowdisplaybreaks
% S unit for use in Mega Samples per second
\DeclareSIUnit\sample{S}

\newcommand{\CC}{C\nolinebreak\hspace{-.05em}\raisebox{.3ex}{ \textbf{+}}\nolinebreak\hspace{-.10em}\raisebox{.3ex}{\textbf{+}}}
\def\CC{{C\nolinebreak[4]\hspace{-.05em}\raisebox{.3ex}{\textbf{++}}}}


\newcommand{\partauthor}[1]{\gdef\@partauthor{#1}}
\renewcommand{\printparttitle}[1]{
  \parttitlefont #1\\
  \vspace{1.5cm}
  \textnormal{\Large \@partauthor}
}
\addbibresource{../../includes/bibliography.bib}

\title{Literature study}
\author{H.J.C Kroep \and R.P. Hes \and W.M. Melching \and W.P. Bruinsma \and T.A. aan de Wiel \and T.C. Leliveld}


%%%%%%%%%%%%%%%%%%%%%%%%%%%%%%%%%%%%%%%%%%%%%%%%%%%%%
%%%% DIT DOCUMENT WORDT AUTOMATISCH GEGENEREERD. %%%%
%%%%       NIET MET DE HAND AANPASSEN            %%%%
%%%%%%%%%%%%%%%%%%%%%%%%%%%%%%%%%%%%%%%%%%%%%%%%%%%%%

\begin{document}
\chapter{Papers}
\section{ \cite{ariananda2011multicoset}}
%% Dibs by Dorus

A paper discussing the Power spectrum blind sampling (PBSB) in combination with multicoset sampling. They implement
\begin{itemize}
    \item a solution for the minimal sparse ruler problem
    \item adapt coprime sampling to fit PSBS
    \item discussion between the two implementations.
\end{itemize}
From this it turns out that the minimal sparse ruler offers advantages over coprime samplng in terms of
\begin{itemize}
    \item reduced sampling rates
    \item increased flexibility
    \item extended range of estimated auto-correlation lags.
\end{itemize}
\section{ \cite{romero2013wideband}}
The paper proposes a number of algorithms (SLIKES, SIIA, WLS, CWLS and SSPICE) to perform compressive sensing. All of them consider a modified version of the \emph{sample covariance matrix} (SCM), which is closely related to the standard estimates of the autocorrelation for ergodic processes. A performance review is presented for each of the algorithms using Monte Carlo simulations.
\section{ \cite{pal2011coprime}}
This paper describes the use of two uniform samples with sample
spacings $MT$ and $NT$ with  $M$ and $N$ coprime. Considering the difference set of this pair of sample spacings, one can generatate $\mathcal{O}(MN)$ consecutive samples using only $\mathcal{O}(M+N)$ samples. A novel algorithm based on spatial smoothing is proposed to efficiently use these $\mathcal{O}(MN)$ samples for super resolution spectral estimation.
\section{ \cite{vazquez2011spectrum}}
% Kees Kroep
This paper adresses the problem of Primary user detection with a correct PSD (sampled at Nyquist rate). Properties of Primary user signals, and noise are used to improve Spectrum Sensing. This theory assumes no knowledge about noise variation and channel gains (which conventional methods do assume). This theory is also tested with performance comparissons. Geert Leus refers to this in his 2013 paper as a solution for situations in which parameters are unknown. 
\section{ \cite{mishali2009blind}}
This paper describes a recovery method for sub-nyquist sampling, assuming an existing multicoset sampling. Starting point is that only the number of bands and their widths are known. Also, a theoretical lower bound on the average sampling rate required for blind sampling is developed.
\section{ \cite{axell2012spectrum}}
%% Dorus dibs

In this paper the current state of cognitive radio is discussed. A variety of technologies are discussed namely detectors and compressive sensing methods. Both multiuser and single user detectors and standard compressive sensing and multicoset sensing are discussed. For implementation details (and details in general) this paper refers to a large number of other papers.
\section{ \cite{laska2007theory}}
% Kees en Willem
By using the theory of compressive sensing we can build an analog-to-information converter using sample frequencies below the nyquist fruency.

The CS theory says that signal that is compressible in one basis (eg.if you convert the signal to the frequency domain it becomes sparse) it can be recovered from fewer samples. The signal is multiplied by a chipping sequence $p_c(t)$ consisting of pseudo random $\pm1$'s. It must alternate faster than the nyquist frequency of the input signal.

The recovery is an ill-posed problem, but because we make the assumption that the signal is compressible we can solve the problem by minimizing the $\ell_1$ norm using optimization techniques.

A theory is provided to predict the SNR performance of real-world implementations. The theory is supported with a simulation of a proof of concept real world implementation. This implementation transistor-level. Basic Pursuit with Denoising (BPDN) is used for optimalization.
\section{ \cite{gang2012secure}}
Paper discussing an algorithm to defent against Spectrum Sensing Data Falsification (SSDF). The algorithm uses compressed sensing and average consensus. This algorithm is implemented in a simulation
\section{ \cite{akyildiz2011cooperative}}
%Kees Kroep
%
A short introduction of the need for cognitive radio, and the tasks to make it happen.
The paper gives an overview of cooperative congitive radiotechniques, and tries to answer the questions:

\begin{itemize}
\item How can cognitive radios cooperate 
\item How much can be gained from cooperation 
\item What is the overhead associated with cooperation
\end{itemize}

%Note: Geert Leus does not use cooperation but tries to achgieve silimar results with one device, and therefore without cooperation delay effects.
\section{ \cite{candes2008introduction}}
This paper is an introduction to compressive sensing (CS). One of the key concepts of CS is sparsity: a signal vector may contain a lot of zero entries (be it represented using a special base matrix $\Psi$).

That is, we can represent a signal $\vec{f}$ using a sparse vector $\vec{x}$ as

\begin{align*}
  \vec{f} = \Psi \vec{x}
\end{align*}

Recording of a signal is commonly achieved by using an impulse train. Generalizing the concept of recording of a signal $f$, we can write a recording $y$ as

\begin{align*}
  \vec{y} &= \Phi \vec{f}
\end{align*}

In Compressive sensing our recording $\vec{y}$ has a lower dimensionality than
the signal $\vec{f}$. Our recording $\vec{y}$ can be represented as:

\begin{align*}\label{eq:recording_system}
  \vec{y} = \Phi \Psi \vec{x}
\end{align*}

The system of equations in \ref{eq:recording_system} is underdetermined. That is
infinitely many solutions for $\vec{x}$ exist. To approximate $\vec{x}$, one assumes that the solution which is most sparse, is the correct one.

That is, one has the following optimization problem:

\begin{align*}
  \min_{x \in \mathbb{R}^n} ||x||_{l_1}
\end{align*}
\section{ \cite{lee2008optimal}}
This paper describes a theoretical framework to optimize the sensing parameters
such that the sensing efficiency is maximized with respect to interference
avoidance contraints. Furthermore spectrum selection and scheduling methods are
proposed to exploit multiple spectrum bands.
\section{ \cite{bayarkernel}}
%% Dorus diabs

In this paper a new greedy algorithm is imposed that can provide an exact sparse solution. The algorithm is called Kernel Reconstruction.
\section{ \cite{lexa2011multi}}
Provides an introduction in multi-coset sampling and states the requirements and procedures for signal reconstruction. Multi-coset sampling is a periodic nonuniform sub-Nyquist sampling technique for acquiring continuous-time spectrally-sparse signals. Multi-coset sampling uses multiple ADCs to sample delayed versions of the input signal. Theoretically, the full spectrum may be obtained using this method.
\section{ \cite{polo2009compressive}}
% Read by: Kees Kroep
%
This paper presents a compressive wide-band spectrum sensing scheme for cognitive radios. An analog to information converter is used. This paper focusses on using the autocorrelation of the compressed signal to estimate the power spectral density.
\section{ \cite{candes2007sparsity}}
% Claim: Tom
%
Given a sampled vector $\vec{x}$, containing $n$ entries. How can we obtain this vector $\vec{x}$ using a measurement vector $\vec{y}$ which contains $m$ entries ($ m \leq n$)? 
By taking an orthogonal matrix $\mathbf{U}$ with $\mathbf{U}^{\ast}\mathbf{U} = n\mathbf{I}$ we can take $\vec{y}$ to be related to 
\section{ \cite{candes2006robust}}
This paper considers the problem of reconstructing an object from incomplete frequency samples. It concludes that a signal consisting of $|T|$ delta spikes, may be recovered by convex programming using a set of frequencies of size $\mathcal{O}(|T|\cdot \log N)$ with a probability $1-\mathcal{O}(N^{-M})$. It states that any method succeeding with a probability $1-\mathcal{O}(N^{-M})$ requires at least $|T|\cdot \log N$ samples and that therefore recovery by convex of programming requires the same order of frequency samples.
\section{ \cite{vaidyanathan2010sparse}}
% Kees Kroep
This paper is being read :P
\section{ \cite{khan2013comparative}}
%Kees Kroep
%
A Comparative Study of Spectrum Sensing Techniques in Cognitive Radio is provided. The paper is state of the art (2013) and focusses on Cooperative Spectrum Sensing. A Comnparisson is made between the performace of Fuzzy collaboration, Asynchronous Coorporation, Relay based Cooperation, Distributed Cooperation, and Non cooperative Sensing.
\section{ \cite{costa1983writing}}
%% Door Dorus

Paper introducing the Dirty Paper coding technique. It allows the channel to cancel out some interference by using knowledge about the channel. The optimal transmitter adapts its signal to the state of the channel, rather than attempting to cancel it.
\section{ \cite{mishali2010theory}}
% Claim: Willem
%
This papers describes the complete process of sub-nyquist sampling using a specific sampling method. A new sampling scheme, the modulated wideband converter (MWC) is introduced, to substitute the hard to implement multi coset sampling. The mathematical background of the MWC is presented including a method to blindly recover the original spectrum (ie. without prior knowledge of the spectrum). A digital architecture to process the signals is provided. From the simulations we can conclude that (even after optimization by collapsing channels) the number of channels  still needs to be very large ($\pm20$). 
\section{ \cite{tang2010artificial}}
%Claim: Willem
%
To improve detection we can use multiple detection methods and train an artificial neural network (ANN). This ANN is trained with an AM signal and noise with different signal to noise ratio's. Then each of the trained networks is tested with another AM signal with noise. Often the network detects signals a bit weaker (10dB) than the training signal.
\section{ \cite{zhang2008theory}}
This paper introduces an alternative to RIP (Restricted Isomery Property) analysis for CS via $\ell_1$ optimization. 

%It starts by observing that given $\vec{x}, \vec{y} \in \mathbb{R}^n$
%and $\alpha =  \supp

%\begin{align*}
	
%\end{align*}

A reasonable decoding model in CS is to look for the sparsest signal
in the solution space that produces the measured signal $b$:

\begin{align*}
	\min \{ ||x||_0 : A\vec{x} = \vec{b} \}
\end{align*} 

\noindent with the zero norm equal to the amount of non zero elements in $\vec{x}$. Solving this zero norm does not appear to be tractable. An alternative is to use the 1-norm. The paper derives what the intersection of those two solution sets are. It continues to analyze recoverability and stability without using RIP.
\section{ \cite{hossain2011wideband}}
This paper introduces a MAP (maximum a posteriori) estimator of the channel occupancy based on measurements from multiple frequency subbands.
As the MAP estimators complexity grows exponentially with the amount of subbands, an alternative structure based based om combining subband energy measurements linearly according to a MMSE criterion, is proposed to provide a sufficient statitistic for binary detection in each subband.
\section{ \cite{li2014gomp}}
This paper analyzes two algorithms that are used in sparse signal recovery:

\begin{enumerate}
	\item Orthogonal Matching Pursuit (OMP)
	\item Greedy OMP (GOMP)
\end{enumerate}

The GOMP algorithm may provide better results in signal recovery.
\section{ \cite{kirolos2006analog}}
%% Dorus paper samenvatting

Compressive sensing theory implementation in the form of an analog-to-information convertor (sub-Nyquist rates). Paper introduces algorithms, prototype implementation based on random demodulation. Simulation at transistor-level proves proof of concept under circuit nonidealities.
\section{ \cite{ariananda2012compressive}}
% Kees Kroep
%
A cognitive radio only needs to sense the Power Spectral Density (PSD). Therefore a nyquist rate sampler is much too powerhungry/expensive. This paper proposes a new type of multicoset sampling based on the minimal sparse ruler problem. The main focus is to design effective periodic sub-Nyquist sampling procedures for efficient power sectrum reconstruction. Other Papers provide ways to interpret the power spectral density with limited information about error variance and channel gain.
\end{document}