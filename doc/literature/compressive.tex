\section{Compressive Sensing}
\references{\cite{candes2008introduction}}
Compressive spectrum sensing is to exploit the fact that the original observed analog signal $y(t)$ with Nyquist rate $1/T$ can often be sampled below the Nyquist rate within an interval $t\in \left[0,N_{b}t \right)$. This results in $\vec{z}$ with dimensions $M_b \times 1$. This can be seen as 
\begin{equation}
    \label{nonuniformsampling}
    \vec{z} = \mat{\Phi}\vec{y}
\end{equation}
