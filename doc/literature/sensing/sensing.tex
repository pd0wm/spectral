%!TEX program = xelatex

\documentclass[report]{memoir}
\usepackage[no-math]{fontspec}
\usepackage{pgfplots}
\pgfplotsset{compat=newest}
\usepackage{commath}
\usepackage{mathtools}
\usepackage{amssymb}
\usepackage{amsthm}
\usepackage{booktabs}
\usepackage{mathtools}
\usepackage{xcolor}
\usepackage[separate-uncertainty=true, per-mode=symbol]{siunitx}
\usepackage[noabbrev, capitalize]{cleveref}
\usepackage{listings}
\usepackage[american inductor, european resistor]{circuitikz}
\usepackage{amsmath}
\usepackage{amsfonts}
\usepackage{ifxetex}
\usepackage[dutch,english]{babel}
\usepackage[backend=bibtexu,texencoding=utf8,bibencoding=utf8,style=ieee,sortlocale=en_GB,language=auto]{biblatex}
\usepackage[strict,autostyle]{csquotes}
\usepackage{parskip}
\usepackage{import}
\usepackage{standalone}
\usepackage{hyperref}
%\usepackage[toc,title,titletoc]{appendix}

\ifxetex{} % Fonts laden in het geval dat je met Xetex compiled
    \usepackage{fontspec}
    \defaultfontfeatures{Ligatures=TeX} % To support LaTeX quoting style
    \setromanfont{Palatino Linotype} % Tover ergens in Font mapje in root.
    \setmonofont{Source Code Pro}
\else % Terug val in standaard pdflatex tool chain. Geen ondersteuning voor OTT fonts
    \usepackage[T1]{fontenc}
    \usepackage[utf8]{inputenc}
\fi
\newcommand{\references}[1]{\begin{flushright}{#1}\end{flushright}}
\renewcommand{\vec}[1]{\boldsymbol{\mathbf{#1}}}
\newcommand{\uvec}[1]{\boldsymbol{\hat{\vec{#1}}}}
\newcommand{\mat}[1]{\boldsymbol{\mathbf{#1}}}
\newcommand{\fasor}[1]{\boldsymbol{\tilde{\vec{#1}}}}
\newcommand{\cmplx}[0]{\mathrm{j}}
\renewcommand{\Re}[0]{\operatorname{Re}}
\newcommand{\Cov}{\operatorname{Cov}}
\newcommand{\Var}{\operatorname{Var}}
\newcommand{\proj}{\operatorname{proj}}
\newcommand{\Perp}{\operatorname{perp}}
\newcommand{\col}{\operatorname{col}}
\newcommand{\rect}{\operatorname{rect}}
\newcommand{\sinc}{\operatorname{sinc}}
\newcommand{\IT}{\operatorname{IT}}
\newcommand{\F}{\mathcal{F}}

\newtheorem{definition}{Definition}
\newtheorem{theorem}{Theorem}


\DeclareSIUnit{\voltampere}{VA} %apparent power
\DeclareSIUnit{\pii}{\ensuremath{\pi}}

\hypersetup{%setup hyperlinks
    colorlinks,
    citecolor=black,
    filecolor=black,
    linkcolor=black,
    urlcolor=black
}

% Example boxes
\usepackage{fancybox}
\usepackage{framed}
\usepackage{adjustbox}
\newenvironment{simpages}%
{\AtBeginEnvironment{itemize}{\parskip=0pt\parsep=0pt\partopsep=0pt}
\def\FrameCommand{\fboxsep=.5\FrameSep\shadowbox}\MakeFramed{\FrameRestore}}%
{\endMakeFramed}

% Impulse train
\DeclareFontFamily{U}{wncy}{}
\DeclareFontShape{U}{wncy}{m}{n}{<->wncyr10}{}
\DeclareSymbolFont{mcy}{U}{wncy}{m}{n}
\DeclareMathSymbol{\Sha}{\mathord}{mcy}{"58}
\addbibresource{../../includes/bibliography.bib}

\title{Literature study}
\author{H.J.C Kroep \and R.P. Hes \and W.M. Melching \and W.P. Bruinsma \and T.A. aan de Wiel \and T.C. Leliveld}

\begin{document}

\chapter{Wideband Spectrum Sensing}
\references{\cite{axell2012spectrum,sahai2009spectrum,khan2013comparative,lee2008optimal,seshukumar2013spectrum}}{}
There are a number of techniques available to map a part of the spectrum. A number of these will be discussed in this chapter.

\section{Multiband Sensing}
\references{\cite{hossain2011wideband,quan2009optimal,ariananda2011multicoset}}
Divide the to be scanned wideband channel into subchannels. Scan each of these subchannels seperately. Problems arise due to the fact that in practice the subchannels are not independant. Primary users occupancy and noise parameters can be correlated. The detection problem becomes a composite hypothesis problem.\footnote{Multiple bands have to be evaluated to determine which hypotheses hold up. This greatly increases complexity.}

An extension to this is joint multiband sensing. Multiple cooperating sensors are used to improve the detection performance. Possibly leads to nonconvex and potentially NP-hard problems.

\section{Compressive Sensing}
\references{\cite{candes2008introduction,gang2012secure,qiao2011combination,kirolos2006analog,candes2007sparsity,mishali2010theory,laska2007theory,ariananda2012compressive,romero2013wideband,vaidyanathan2010sparse,pal2011coprime}}
Compressive spectrum sensing is to exploit the fact that the original observed analog signal $y(t)$ with Nyquist rate $1/T$ can often be sampled below the Nyquist rate within an interval $t\in \left[0,N_{b}t \right)$. This results in $\vec{z}$ with dimensions $M_b \times 1$. This can be seen as
\begin{equation}
    \label{eq:nonuniformsampling}
    \vec{z} = \mat{\Phi}\vec{y}.
\end{equation}
where $\mat{\Phi}$ is the measurement matrix.

\section{Cooperative Sensing}
\references{\cite{akyildiz2011cooperative}}
Cooperative sensing is a technique in which spectrum sensing accuracy is improved by using multiple receivers and combining their individual results. Cooperative sensing is a way to exploit spatial diversity.
Although the use of multiple receivers may yield better results, it gives rise to some issues over single receiver sensing as well.

\end{document}
