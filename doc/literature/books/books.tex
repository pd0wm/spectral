%!TEX program=xelatex

\documentclass[report, oneside, a4paper, openany]{memoir}
\usepackage[no-math]{fontspec}
\usepackage{pgfplots}
\pgfplotsset{compat=newest}
\usepackage{commath}
\usepackage{mathtools}
\usepackage{amssymb}
\usepackage{amsthm}
\usepackage{booktabs}
\usepackage{mathtools}
\usepackage{xcolor}
\usepackage[separate-uncertainty=true, per-mode=symbol]{siunitx}
\usepackage[noabbrev, capitalize]{cleveref}
\usepackage{listings}
\usepackage[american inductor, european resistor]{circuitikz}
\usepackage{amsmath}
\usepackage{amsfonts}
\usepackage{ifxetex}
\usepackage[dutch,english]{babel}
\usepackage[backend=bibtexu,texencoding=utf8,bibencoding=utf8,style=ieee,sortlocale=en_GB,language=auto]{biblatex}
\usepackage[strict,autostyle]{csquotes}
\usepackage{parskip}
\usepackage{import}
\usepackage{standalone}
\usepackage{hyperref}
%\usepackage[toc,title,titletoc]{appendix}

\ifxetex{} % Fonts laden in het geval dat je met Xetex compiled
    \usepackage{fontspec}
    \defaultfontfeatures{Ligatures=TeX} % To support LaTeX quoting style
    \setromanfont{Palatino Linotype} % Tover ergens in Font mapje in root.
    \setmonofont{Source Code Pro}
\else % Terug val in standaard pdflatex tool chain. Geen ondersteuning voor OTT fonts
    \usepackage[T1]{fontenc}
    \usepackage[utf8]{inputenc}
\fi
\newcommand{\references}[1]{\begin{flushright}{#1}\end{flushright}}
\renewcommand{\vec}[1]{\boldsymbol{\mathbf{#1}}}
\newcommand{\uvec}[1]{\boldsymbol{\hat{\vec{#1}}}}
\newcommand{\mat}[1]{\boldsymbol{\mathbf{#1}}}
\newcommand{\fasor}[1]{\boldsymbol{\tilde{\vec{#1}}}}
\newcommand{\cmplx}[0]{\mathrm{j}}
\renewcommand{\Re}[0]{\operatorname{Re}}
\newcommand{\Cov}{\operatorname{Cov}}
\newcommand{\Var}{\operatorname{Var}}
\newcommand{\proj}{\operatorname{proj}}
\newcommand{\Perp}{\operatorname{perp}}
\newcommand{\col}{\operatorname{col}}
\newcommand{\rect}{\operatorname{rect}}
\newcommand{\sinc}{\operatorname{sinc}}
\newcommand{\IT}{\operatorname{IT}}
\newcommand{\F}{\mathcal{F}}

\newtheorem{definition}{Definition}
\newtheorem{theorem}{Theorem}


\DeclareSIUnit{\voltampere}{VA} %apparent power
\DeclareSIUnit{\pii}{\ensuremath{\pi}}

\hypersetup{%setup hyperlinks
    colorlinks,
    citecolor=black,
    filecolor=black,
    linkcolor=black,
    urlcolor=black
}

% Example boxes
\usepackage{fancybox}
\usepackage{framed}
\usepackage{adjustbox}
\newenvironment{simpages}%
{\AtBeginEnvironment{itemize}{\parskip=0pt\parsep=0pt\partopsep=0pt}
\def\FrameCommand{\fboxsep=.5\FrameSep\shadowbox}\MakeFramed{\FrameRestore}}%
{\endMakeFramed}

% Impulse train
\DeclareFontFamily{U}{wncy}{}
\DeclareFontShape{U}{wncy}{m}{n}{<->wncyr10}{}
\DeclareSymbolFont{mcy}{U}{wncy}{m}{n}
\DeclareMathSymbol{\Sha}{\mathord}{mcy}{"58}
\addbibresource{../../includes/bibliography.bib}

\title{Literature study}

\author{W.P. Bruinsma \and R.P. Hes \and H.J.C. Kroep \and T.C. Leliveld \and W.M. Melching \and T.A. aan de Wiel}
%%%%%%%%%%%%%%%%%%%%%%%%%%%%%%%%%%%%%%%%%%%%%%%%%%%%%
%%%% DIT DOCUMENT WORDT AUTOMATISCH GEGENEREERD. %%%%
%%%%       NIET MET DE HAND AANPASSEN            %%%%
%%%%%%%%%%%%%%%%%%%%%%%%%%%%%%%%%%%%%%%%%%%%%%%%%%%%%

\begin{document}
\chapter{Books}

\section{A wavelet tour of signal processing \cite{mallat1999wavelet}}
The book clearly presents the standard representations with Fourier, wavelet and time-frequency transforms, and the construction of orthogonal bases with fast algorithms. The central concept of sparsity is explained and applied to signal compression, noise reduction, and inverse problems, while coverage is given to sparse representations in redundant dictionaries, super-resolution and compressive sensing applications.
\section{Introduction to Algorithms, Second Edition \cite{cormen2001introduction}}
%Willem
This book is the `bible' of algorithms, it is a 1200 page reference of most used algorithms and data structures. Each algorithm is thoroughly explained with figures and pseudo-code. Most of the time these algorithms will be already implemented in the standard libraries of C++ or Python. But when writing high performance real time software, it is critical to know their internal working.
\section{Digital \& Analog Communication Systems: International Edition \cite{couch2013digital}}
%Willem
This is the book used in the Telecommunications course in the second year of the Electrical Engineering Bachelor. It gives an introduction to signals and the spectrum of signals, analogue and digital encoding of signals, random processes and their spectral analysis and discusses performance of systems corrupted by noise. A lot of the basic concepts needed for cognitive radios is discussed in this book. 
\section{Probability and stochastic processes: a friendly introduction for electrical and computer engineers \cite{yates2005probability}}
% Willem
This book covers the basics of stochastic processes and is used in the second year of the Electrical Engineering Bachelor. This book covers the theoretical background concerning random processes needed to fully understand the papers on spectrum sensing and detection, such as properties of random signals, the processing of random signals and power spectral density.
\section{Numerical Recipes 3rd Edition: The Art of Scientific Computing \cite{press2007numerical}}
% Willem
This book also covers a lot of basic algorithms used in numerical calculation. The most interesting chapters for our research are the chapters on optimisation, linear programming, random numbers, statistics and Fourier transforms.  The book is accompanied by an example book with implementation of the algorithms in C++. This will help us implement the software needed to do compressive sensing in GNURadio.
\section{Statistical digital signal processing and modeling \cite{hayes1996statistical}}
The main thrust is to provide students with a solid understanding of a number of important and related advanced topics in digital signal processing such as Wiener filters, power spectrum estimation, signal modelling and adaptive filtering. Scores of worked examples illustrate fine points, compare techniques and algorithms and facilitate comprehension of fundamental concepts. Also features an abundance of interesting and challenging problems at the end of every chapter.
\end{document}