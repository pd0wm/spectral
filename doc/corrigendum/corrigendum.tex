%!TEX program = xelatex

\documentclass[a4paper, openany, oneside]{memoir}
\usepackage[no-math]{fontspec}
\usepackage{pgfplots}
\pgfplotsset{compat=newest}
\usepackage{commath}
\usepackage{mathtools}
\usepackage{amssymb}
\usepackage{amsthm}
\usepackage{booktabs}
\usepackage{mathtools}
\usepackage{xcolor}
\usepackage[separate-uncertainty=true, per-mode=symbol]{siunitx}
\usepackage[noabbrev, capitalize]{cleveref}
\usepackage{listings}
\usepackage[american inductor, european resistor]{circuitikz}
\usepackage{amsmath}
\usepackage{amsfonts}
\usepackage{ifxetex}
\usepackage[dutch,english]{babel}
\usepackage[backend=bibtexu,texencoding=utf8,bibencoding=utf8,style=ieee,sortlocale=en_GB,language=auto]{biblatex}
\usepackage[strict,autostyle]{csquotes}
\usepackage{parskip}
\usepackage{import}
\usepackage{standalone}
\usepackage{hyperref}
%\usepackage[toc,title,titletoc]{appendix}

\ifxetex{} % Fonts laden in het geval dat je met Xetex compiled
    \usepackage{fontspec}
    \defaultfontfeatures{Ligatures=TeX} % To support LaTeX quoting style
    \setromanfont{Palatino Linotype} % Tover ergens in Font mapje in root.
    \setmonofont{Source Code Pro}
\else % Terug val in standaard pdflatex tool chain. Geen ondersteuning voor OTT fonts
    \usepackage[T1]{fontenc}
    \usepackage[utf8]{inputenc}
\fi
\newcommand{\references}[1]{\begin{flushright}{#1}\end{flushright}}
\renewcommand{\vec}[1]{\boldsymbol{\mathbf{#1}}}
\newcommand{\uvec}[1]{\boldsymbol{\hat{\vec{#1}}}}
\newcommand{\mat}[1]{\boldsymbol{\mathbf{#1}}}
\newcommand{\fasor}[1]{\boldsymbol{\tilde{\vec{#1}}}}
\newcommand{\cmplx}[0]{\mathrm{j}}
\renewcommand{\Re}[0]{\operatorname{Re}}
\newcommand{\Cov}{\operatorname{Cov}}
\newcommand{\Var}{\operatorname{Var}}
\newcommand{\proj}{\operatorname{proj}}
\newcommand{\Perp}{\operatorname{perp}}
\newcommand{\col}{\operatorname{col}}
\newcommand{\rect}{\operatorname{rect}}
\newcommand{\sinc}{\operatorname{sinc}}
\newcommand{\IT}{\operatorname{IT}}
\newcommand{\F}{\mathcal{F}}

\newtheorem{definition}{Definition}
\newtheorem{theorem}{Theorem}


\DeclareSIUnit{\voltampere}{VA} %apparent power
\DeclareSIUnit{\pii}{\ensuremath{\pi}}

\hypersetup{%setup hyperlinks
    colorlinks,
    citecolor=black,
    filecolor=black,
    linkcolor=black,
    urlcolor=black
}

% Example boxes
\usepackage{fancybox}
\usepackage{framed}
\usepackage{adjustbox}
\newenvironment{simpages}%
{\AtBeginEnvironment{itemize}{\parskip=0pt\parsep=0pt\partopsep=0pt}
\def\FrameCommand{\fboxsep=.5\FrameSep\shadowbox}\MakeFramed{\FrameRestore}}%
{\endMakeFramed}

% Impulse train
\DeclareFontFamily{U}{wncy}{}
\DeclareFontShape{U}{wncy}{m}{n}{<->wncyr10}{}
\DeclareSymbolFont{mcy}{U}{wncy}{m}{n}
\DeclareMathSymbol{\Sha}{\mathord}{mcy}{"58}
\addbibresource{../includes/bibliography.bib}
\pagenumbering{gobble}
\usepackage{fullpage}
\usepackage{ragged2e}
\usepackage{array}
\newcolumntype{L}[1]{>{\RaggedRight\hspace{0pt}}p{#1}}

\begin{document}
% Title
\makebox[\textwidth]{
    \centering
    \fontsize{1cm}{1em}
    \textsc{corrigendum}
} \\

This document is a correction of the Bachelor graduation thesis ``An extensible toolkit for real-time high-performance wideband spectrum sensing'' by Bruinsma, Hes, Kroep, Leliveld, Melching and Wiel. \\

\centerline{
\begin{tabularx}{\linewidth}{L{3cm}L{6cm}L{6cm}}
    \textbf{Where} & \textbf{Original text} & \textbf{Corrected text} \\ \hline
    Section 6.3 & The actual sampling is not part of the system, but forms an interface between the the signal and our system. & The actual sampling is not part of the system, but forms an interface between the signal and our system. \\
    Section 8.2 & In constrast, methods of the second group are not based on the assumption the the signal is either sparse or contains few frequencies. & In contrast, methods of the second group are not based on the assumption that the signal is either sparse or contains few frequencies. \\
    Section 9.3 & Finally, the the circular sparse ruler problem, stated below, must be satisfied. & Finally, the circular sparse ruler problem, stated below, must be satisfied. \\
    Section 9.3, ``Coprime sampling'' & Finally, the bottom block shows how the the orange and greens blocks are split across many artificial signals. & Finally, the bottom block shows how the orange and greens blocks are split across many artificial signals. \\
    Section 18.3 & Upon receiving this data, Visualisation will pass it through to the the element that first triggered the request, which will then re-render the plot. & Upon receiving this data, Visualisation will pass it through to the element that first triggered the request, which will then re-render the plot. \\
    Table 11.3, third row, second column & $612\%$ & $61\%$ \\
    Section 8.4, ``Estimation of the cross-correlations'' & $\hat{r}_{y_i,y_j}[m] = \frac{1}{KL-|m|} \sum_{k=1}^{u}y_i[k]\conj{y}_j[k+m]$ & $\hat{r}_{y_i,y_j}[m] = \frac{1}{KL-|m|} \sum_{k=1}^{u}\conj{y} _i[k]y_j[k+m]$ \\
    Section 8.4, ``Efficient generation of the matrices'' & $c_{i,j}[m]=\sum_{k=-\infty}^{\infty}c_i[k]\conj{c}_j[k+m]$ & $c_{i,j}[m]=\sum_{k=-\infty}^{\infty}\conj{c}_i[k]c_j[k+m]$ \\
    Section B.2 & $r_x[n,m]=E(x[n]\conj{x}[n+m])$ & $r_x[n,m]=E(\conj{x}[n]x[n+m])$ \\
    Section F.1 & Circular Complex Gaussian & Circular symmetric complex gaussian \\
    Section F.2, &$\vec{x} = \begin{bmatrix}x[n]& x[n+1]& \cdots& x[n+L-1]\end{bmatrix}^T.$ & $
    \vec{x} = \begin{bmatrix}x[n]& x[n-1]& \cdots& x[n-L+1]\end{bmatrix}^T.$ \\
    Section F.2 & expansion of $\mat{C}_x = E(\vec{x}\vec{x}^H)$ & use corrected definition of $\vec{x}$ 
\end{tabularx}
}

\end{document}
