%!TEX program = xelatex

\documentclass[a4paper, openany, oneside]{memoir}
\usepackage[no-math]{fontspec}
\usepackage{pgfplots}
\usepackage{float}
\pgfplotsset{compat=newest}
\usepackage{commath}
\usepackage{mathtools}
\usepackage{amssymb}
\usepackage{amsthm}
\usepackage{booktabs}
\usepackage{todonotes}
\usepackage{mathtools}
\usepackage{xcolor}
\usepackage[separate-uncertainty=true, per-mode=symbol]{siunitx}
\usepackage{listings}
\usepackage[american inductor, european resistor]{circuitikz}
\usepackage{amsmath}
\usepackage{amsfonts}
\usepackage{ifxetex}
\usepackage[dutch,english]{babel}
\usepackage[backend=bibtexu,texencoding=utf8,bibencoding=utf8,style=ieee,sortlocale=en_GB,language=auto]{biblatex}
\usepackage[strict,autostyle]{csquotes}
\usepackage{import}
\usepackage{standalone}
\usepackage{bookmark,hyperref}
\usepackage{xcolor,mdframed}
\usepackage{tikz}
\usepackage{framed}
\usepackage{float}
\usepackage{tabularx}
\usepackage{graphicx,adjustbox}
\usepackage{rotating}
\usepackage{pdfpages}
\usepackage{enumitem}
\usepackage{calc}
\usepackage{pgfplots}
\usepackage{filecontents}
\usepackage{caption}
\usepackage{subcaption}
\usepackage{lettrine}

\newcolumntype{Y}{>{\raggedright\arraybackslash}X} % Left-justified text in tabularx environment

\ifxetex{} % Fonts laden in het geval dat je met Xetex compiled
    \usepackage{fontspec}
    \defaultfontfeatures{Scale=MatchLowercase, Ligatures=TeX} % To support LaTeX quoting style
    %\setromanfont{Palatino Linotype} % Tover ergens in Font mapje in root.
    \setsansfont{Avenir Next LT Pro}
    \setromanfont{Adobe Caslon Pro} % Tover ergens in Font mapje in root.
    \setmonofont{Source Code Pro}
\else % Terug val in standaard pdflatex tool chain. Geen ondersteuning voor OTT fonts
    \usepackage[T1]{fontenc}
    \usepackage[utf8]{inputenc}
\fi
\usepackage[noabbrev, capitalize]{cleveref}
\usepackage{ifthen}
\usepackage{titlesec}
\usepackage{titlecaps}

\newcommand{\references}[1]{\begin{flushright}{#1}\end{flushright}}
\renewcommand{\vec}[1]{\boldsymbol{\mathbf{#1}}}
\newcommand{\uvec}[1]{\boldsymbol{\hat{\vec{#1}}}}
\newcommand{\mat}[1]{\boldsymbol{\mathbf{#1}}}
\newcommand{\fasor}[1]{\boldsymbol{\tilde{\vec{#1}}}}
\newcommand{\cmplx}[0]{\mathrm{j}}
\renewcommand{\Re}[0]{\operatorname{Re}}
\newcommand{\Cov}{\operatorname{Cov}}
\newcommand{\Var}{\operatorname{Var}}
\newcommand{\proj}{\operatorname{proj}}
\newcommand{\Perp}{\operatorname{perp}}
\newcommand{\col}{\operatorname{col}}
\newcommand{\rect}{\operatorname{rect}}
\newcommand{\sinc}{\operatorname{sinc}}
\newcommand{\lcm}{\operatorname{lcm}}
%\newcommand{\gcd}{\operatorname{gcd}}
\newcommand{\F}{\mathcal{F}}
\newcommand{\DTFT}{\mathcal{F}_*}
\newcommand{\conj}[1]{#1^*}
\renewcommand{\mod}{\operatorname{mod}}
\newcommand{\rot}{\operatorname{rot}}
\newcommand{\vecsc}[1]{\vec{\textsc{\textbf{#1}}}}
\renewcommand{\ss}[1]{_{#1}}

% Label without linebreak breaker
\newcommand{\lab}[1]{\label{#1}\nolinebreak}

\newtheorem{definition}{Definition}
\newtheorem{theorem}{Theorem}


\DeclareSIUnit{\voltampere}{VA} %apparent power
\DeclareSIUnit{\pii}{\ensuremath{\pi}}

\hypersetup{%setup hyperlinks
    colorlinks,
    citecolor=black,
    filecolor=black,
    linkcolor=black,
    urlcolor=black
}

% Example boxes
\usepackage{fancybox}
\usepackage{framed}
\usepackage{adjustbox}
\newenvironment{simpages}%
{\AtBeginEnvironment{itemize}{\parskip=0pt\parsep=0pt\partopsep=0pt}
\def\FrameCommand{\fboxsep=.5\FrameSep\shadowbox}\MakeFramed{\FrameRestore}}%
{\endMakeFramed}

% Impulse train
\DeclareFontFamily{U}{wncy}{}
\DeclareFontShape{U}{wncy}{m}{n}{<->wncyr10}{}
\DeclareSymbolFont{mcy}{U}{wncy}{m}{n}
\DeclareMathSymbol{\Sha}{\mathord}{mcy}{"58}

\setlength{\parindent}{0pt}
\nonzeroparskip

% Block environment configuration
\newcommand{\BlockLeftMargin}{20pt}
\newcommand{\BlockLeftMarginText}{25pt}
\newcommand{\BlockLeftMarginTextSpacing}{10pt}

% Own colours
\definecolor{gray75}{gray}{0.75}

% Block environment
\newenvironment{block}[3]{%
\makebox{\hspace{-\spinemargin}%
\begin{tikzpicture}[overlay]
    \draw [thick,color=gray75] (\BlockLeftMargin, 0) -- (\paperwidth - \spinemargin, 0);
    \node at (\BlockLeftMarginText, -0.9) [align=left, text width=\spinemargin - \BlockLeftMarginText - \BlockLeftMarginTextSpacing, anchor=west, text depth=1cm] {\textbf{\textsc{#1}}\newline\textit{#3}};
\end{tikzpicture}}%
\nopagebreak\\[0.25em]\ifthenelse{\equal{#2}{}}{}{(\textit{#2}.) }\nopagebreak\nolinebreak}
{\nopagebreak\\[-0.25em]%
\makebox{\hspace{-\spinemargin}%
\begin{tikzpicture}[overlay, remember picture]
    \draw [thick,color=gray75] (\spinemargin,0) -- (\paperwidth - \spinemargin,0);
\end{tikzpicture}} \vspace{0.5em}}

% Theorem
\newcounter{blockTheoremCounter}
\crefname{blockTheoremCounter}{Theorem}{Theorems}
\Crefname{blockTheoremCounter}{Theorem}{Theorems}

\newenvironment{blockTheorem}[1][]{%
\refstepcounter{blockTheoremCounter}%
\begin{block}{theorem \theblockTheoremCounter}{#1}{}}
{\end{block}}

% Definition
\newcounter{blockDefinitionCounter}
\crefname{blockDefinitionCounter}{Definition}{Definitions}
\Crefname{blockDefinitionCounter}{Definition}{Definitions}

\newenvironment{blockDefinition}[1][]{%
\refstepcounter{blockDefinitionCounter}%
\begin{block}{definition \theblockDefinitionCounter}{#1}{}}
{\end{block}}

% Proof
\newcounter{blockProofTheoremCounter}
\crefname{blockProofTheoremCounter}{Proof}{Proofs}
\Crefname{blockProofTheoremCounter}{Proof}{Proofs}

\newenvironment{blockProofTheorem}[1]{%
\refstepcounter{blockProofTheoremCounter}%
\begin{block}{proof of \\ theorem #1}{}{}}
{\qed\end{block}}

% Detail
\newcounter{blockDetailCounter}
\crefname{blockDetailCounter}{Detail}{Details}
\Crefname{blockDetailCounter}{Detail}{Details}

\newenvironment{blockDetail}[1][]{%
\refstepcounter{blockDetailCounter}%
\begin{block}{detail \theblockDetailCounter}{#1}{}}
{\end{block}}

% Redesign chapter headings
\newcommand{\chapternumber}{\thechapter}
\newcommand{\hsp}{\hspace{20pt}}
\titleformat{\chapter}[hang]{\Huge\bfseries}{\chapternumber\hsp\textcolor{gray75}{|}\hsp}{0pt}{\Huge\bfseries}

% Remove headers
% \addtopsmarks{headings}{}{
%   \createmark{chapter}{left}{nonumber}{}{}
% }
% \pagestyle{headings} % Activate changes

% Capitalise headers in a regular way
\renewcommand*{\memUChead}[1]{\titlecap{#1}}

% \hfill for math mode
\newcommand{\pushright}[1]{\intertext{\hfill$\displaystyle #1$}}
\newcommand{\pushline}{\hskip \textwidth minus \textwidth}
\newcommand{\matlab}{\textsc{Matlab}}

\definecolor{code-grey}{HTML}{DDDDDD}
\newcommand{\lib}[1]{\textsf{#1}}
\newcommand{\file}[1]{\textsf{#1}}
\newcommand{\func}[1]{\colorbox{code-grey}{\texttt{#1}}}
\newcommand{\class}[1]{\colorbox{code-grey}{\texttt{#1}}}

% Setup actiepunten
\newenvironment{important}[1][]{%
   \begin{mdframed}[%
      backgroundcolor={red!15}, hidealllines=true,
      skipabove=0.7\baselineskip, skipbelow=0.7\baselineskip,
      splitbottomskip=2pt, splittopskip=4pt, #1]%
   \makebox[0pt]{% ignore the withd of !
      \smash{% ignor the height of !
         \fontsize{32pt}{32pt}\selectfont% make the ! bigger
         \hspace*{-19pt}% move ! to the left
         \raisebox{-2pt}{% move ! up a little
            {\color{red!70!black}\sffamily\bfseries !}% type the bold red !
         }%
      }%
   }%
}{\end{mdframed}}
\newcommand{\excl}[1]{
\begin{important}
  \textbf{#1}
\end{important}
}

\makeatletter
\newcommand\footnoteref[1]{\protected@xdef\@thefnmark{\ref{#1}}\@footnotemark}
\makeatother

% Allow page breaks in display environments
%\allowdisplaybreaks
% S unit for use in Mega Samples per second
\DeclareSIUnit\sample{S}

\newcommand{\CC}{C\nolinebreak\hspace{-.05em}\raisebox{.3ex}{ \textbf{+}}\nolinebreak\hspace{-.10em}\raisebox{.3ex}{\textbf{+}}}
\def\CC{{C\nolinebreak[4]\hspace{-.05em}\raisebox{.3ex}{\textbf{++}}}}


\newcommand{\partauthor}[1]{\gdef\@partauthor{#1}}
\renewcommand{\printparttitle}[1]{
  \parttitlefont #1\\
  \vspace{1.5cm}
  \textnormal{\Large \@partauthor}
}
\addbibresource{../../../../../includes/bibliography.bib}

\begin{document}

\section{Covariance Absolute Value detection}\label{sec:cav_derivation}

\subsection{Explanation}
Let $L$ samples of the signal $x[n]$ be collected in the vector
\begin{align*}
    \vec{x} = \begin{bmatrix}x[n]& x[n+1]& \cdots& x[n+L]\end{bmatrix}^T.
\end{align*}

Furthermore let
\begin{align*}
    \mat{C} = E[\left(\vec{x} - \mu \right)\left(\vec{x} - \mu \right)^H]
\end{align*}
denote the covariance of $\vec{x}$ with $\mu = E(\vec{x})$. In the case that $E\left(\vec{x}\right)=0$, like for noise or most communication signals, then $\mat{C}$ can be simplified to
\begin{align*}
\mat{C}_x &= E(\vec{x}\vec{x}^T) \\
&= \begin{bmatrix} 
E\left(x[n][n]\right) & E\left(x[n][n+1]\right) & \ldots & E\left(x[n][n+L-1]\right) \\
E\left(x[n+1][n]\right) & E\left(x[n+1][n+1]\right) & \ldots & E\left(x[n+1][n+L-1]\right) \\
\vdots & \vdots & \ddots & \vdots \\
E\left(x[n+L-1][n]\right) & E\left(x[n+L-1][n+1]\right) & \ldots & E\left(x[n+L-1][n+L-1]\right) \\
\end{bmatrix}.
\end{align*}
Under the assumption that $x[n]$ is a wide-sense stationary signal, we can simplify $\mat{C}$ even further. 
\begin{align*}
\mat{C}_x&= E(\vec{x}\vec{x}^T) \\
&= \begin{bmatrix} 
r_x[0] & r_x[1] & \ldots & r_x[L-1] \\
r_x[1] & r_x[0] & \ldots & r_x[L-2] \\
\vdots & \vdots & \ddots & \vdots \\
r_x[L-1] & r_x[L-2] & \ldots & r_x[0] \\
\end{bmatrix}.
\end{align*}
Note how $\mat{C}_x$ is symmetric and has a Toeplitz structure. This corresponds to the first block in \cref{tkz:cav}. As  the autocorrelation function of white noise is a delta function, it means that  if $x[n]$ is white noise with variance $\sigma_n^2$, then $\mat{C}_x = \sigma_n^2\mat{I}$.
If the signal $x[n]$ is not equal to noise, then its autocorrelation function is not equal to a delta function which results in $\mat{C}_x$ having nonzero off diagonal elements.

Covariance Absolute Value method detection uses a measure of this `diagonality of $\mat{C}_x$ as test statistic $\Lambda$.
This measure $\Lambda$ is defined as
\begin{align}\label{eq:cav_statistic}
\Lambda &= \frac{T_1}{T_2} \nonumber \\
&=\frac{\sum_{n=1}^{L} \sum_{m=1}^L \left|(\mat{C}_x)_{nm}\right|}{\sum_{k=1}^L |(\mat{C}_x)_{kk}|}
\end{align} 

where $T_1 = \sum_{n=1}^{L} \sum_{m=1}^L |(\mat{C}_x)_{nm}|/L$ and
$T_2 = \sum_{k=1}^L |(\mat{C}_x)_{kk}|/L$.
This test statistic can be computed by first taking the absolute value of $\mat{C}_x$. This is then followed by summing all the elements of the resulting matrix, $T_1$, and computing the trace of that matrix, $T_2$. Upon diving those two results one obtains the test statistic $\Lambda$. This process is depicted in \cref{tkz:cav}.

In practice one estimates the matrix $\mat{C}_x$ by using a limited amount of samples $N$ to estimate $r_x[n]$. The threshold given a desired false alarm probability
$p\ss{fa}$ is derived in \cite{zheng2009spectrum} to be
\begin{align*}
\gamma &= \frac{1+(L-1)\sqrt{2/N\pi}}{1-Q^{-1}(p\ss{fa})\sqrt{2/N}}.
\end{align*} 

This threshold, however, does not apply to the estimated autocorrelation of our reconstruction method.  \cite{zheng2009spectrum} assumes that
$N$ samples are used to estimate \emph{each} element of the autocorrelation in $\mat{C}_x$ as:

\begin{align}\label{eq:cav_rx_estimate}
r_x[m] &= \frac{1}{N}\sum_{n=0}^{N-1}x[n]\conj{x}[n-m] & m=0,1,\ldots, L-1.
\end{align}

Our reconstruction, first of all does not use \cref{eq:cav_rx_estimate} to estimate the autocorrelation. Second, as \cite{zheng2009spectrum} uses
the variance of $r_x[m]$ when estimated by \cref{eq:cav_rx_estimate}, in its threshold derivation we know that this threshold is not applicable
to our reconstructed estimate, as the variance of the estimated $r_x[m]$ is not necessarily constant (to see why consider \cref{eq:terror_cov_sx} without multiplication by $\mat{F}$) and the same as in \cref{eq:cav_rx_estimate}.  
% \subsection{Adjusting the threshold}\label{eq:threshold_cav_deriv}
% In \cite{zheng2009spectrum} it is assumed that for large $N$ the distribution of $T_1$ and $T_2$ approach Gaussian distributions. Given a fixed false alarm probability, the threshold $\gamma$ is derived under the hypothesis $\mathcal{H}_0$. Now should
% \begin{align*}
% p_{fa} &\approx P(T_1/T_2 > \gamma  | \mathcal{H}_0) \\
% &= P(T_2 < T_1/\gamma  | \mathcal{H}_0).
% \end{align*}

% Under $\mathcal{H}_0$, the distribution of $T_2 = r_x[0]$ is approximated by a Gaussian. We can therefore derive the
% threshold $\gamma$ by noting that
% \begin{align*}
% p\ss{fa} &= P\left[\frac{E\left(T_1\right)/\gamma - E\left(T_2\right) }{\sqrt{\text{Var}(T_2)}}\right]\\
% &= 1-Q\left[\frac{E\left(T_1\right)/\gamma - E\left(T_2\right) }{\sqrt{\text{Var}(T_2)}}\right].
% \end{align*}

% To solve for $\gamma$ we have to determine $E\left(T_1\right)$, $E\left(T_2\right)$ and $\Var(T_2)$. It is at this point that we cannot use the expressions from \cite{zheng2009spectrum}, since the estimation of $r_x[m]$ in \cref{sec:reconstruction-implementation} is a random variable itself, which interferes with calculating $p\ss{fa}$.

% To solve this problem, let
% \begin{align*}
%     \vec{\hat{r}_x} = \begin{bmatrix} \hat{r}_x[-LN]& \hat{r}_x[-LN+1] & \cdots & \hat{r}_x[LN]\end{bmatrix}
% \end{align*}
% denote the vector containing the by the reconstructor estimated elements of $r_x[m]$. Then 
% \begin{align*}
% E(\vec{\hat{r}}_x) &= \mat{R}^{\dagger}E(\vec{\hat{r}}_y)
% \end{align*}

% contains $r_x[m]$ for $-LN \leq m \leq LN$. Notice that this vector contains $r_x[0] = E(T_2)$ and that it can be used to construct $E(\mat{C})$. Therefore it can also be used to calculate $E(T_1)$. To obtain $\Var(r_x[0])$, we notice that the $\Var(r_x[m])$ is contained on the diagonal of the covariance matrix of $\vec{\hat{r}_x}$, which is denoted by $\mat{C}_{\hat{r}_x}$. This covariance matrix is equal to
% \begin{align*}
% \mat{C}_{\hat{r}_x} &= E(\vec{\hat{r}}_x\vec{\hat{r}}_x^T) - E(\vec{\hat{r}}_x) E(\vec{\hat{r}}_x^T)\\
% &= R^{\dagger}\mat{C}_{\hat{r}_y}(R^{\dagger})^H.
% \end{align*}
% The elements of the covariance matrix $\mat{C}_{\hat{r}_y}$ are given by \cref{eq:elem_cov_ry}. Finally, the threshold $\gamma$ can be calculated by
% \begin{align*}
% \gamma &= \frac{E(T_1)}{Q^{-1}(1-p_{fa})\sqrt{\Var(T_2)}+E(T_2)}.
% \end{align*}

\end{document}