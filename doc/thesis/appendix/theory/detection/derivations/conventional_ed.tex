%!TEX program = xelatex

\documentclass[a4paper, openany, oneside]{memoir}
\usepackage[no-math]{fontspec}
\usepackage{pgfplots}
\pgfplotsset{compat=newest}
\usepackage{commath}
\usepackage{mathtools}
\usepackage{amssymb}
\usepackage{amsthm}
\usepackage{booktabs}
\usepackage{mathtools}
\usepackage{xcolor}
\usepackage[separate-uncertainty=true, per-mode=symbol]{siunitx}
\usepackage[noabbrev, capitalize]{cleveref}
\usepackage{listings}
\usepackage[american inductor, european resistor]{circuitikz}
\usepackage{amsmath}
\usepackage{amsfonts}
\usepackage{ifxetex}
\usepackage[dutch,english]{babel}
\usepackage[backend=bibtexu,texencoding=utf8,bibencoding=utf8,style=ieee,sortlocale=en_GB,language=auto]{biblatex}
\usepackage[strict,autostyle]{csquotes}
\usepackage{parskip}
\usepackage{import}
\usepackage{standalone}
\usepackage{hyperref}
%\usepackage[toc,title,titletoc]{appendix}

\ifxetex{} % Fonts laden in het geval dat je met Xetex compiled
    \usepackage{fontspec}
    \defaultfontfeatures{Ligatures=TeX} % To support LaTeX quoting style
    \setromanfont{Palatino Linotype} % Tover ergens in Font mapje in root.
    \setmonofont{Source Code Pro}
\else % Terug val in standaard pdflatex tool chain. Geen ondersteuning voor OTT fonts
    \usepackage[T1]{fontenc}
    \usepackage[utf8]{inputenc}
\fi
\newcommand{\references}[1]{\begin{flushright}{#1}\end{flushright}}
\renewcommand{\vec}[1]{\boldsymbol{\mathbf{#1}}}
\newcommand{\uvec}[1]{\boldsymbol{\hat{\vec{#1}}}}
\newcommand{\mat}[1]{\boldsymbol{\mathbf{#1}}}
\newcommand{\fasor}[1]{\boldsymbol{\tilde{\vec{#1}}}}
\newcommand{\cmplx}[0]{\mathrm{j}}
\renewcommand{\Re}[0]{\operatorname{Re}}
\newcommand{\Cov}{\operatorname{Cov}}
\newcommand{\Var}{\operatorname{Var}}
\newcommand{\proj}{\operatorname{proj}}
\newcommand{\Perp}{\operatorname{perp}}
\newcommand{\col}{\operatorname{col}}
\newcommand{\rect}{\operatorname{rect}}
\newcommand{\sinc}{\operatorname{sinc}}
\newcommand{\IT}{\operatorname{IT}}
\newcommand{\F}{\mathcal{F}}

\newtheorem{definition}{Definition}
\newtheorem{theorem}{Theorem}


\DeclareSIUnit{\voltampere}{VA} %apparent power
\DeclareSIUnit{\pii}{\ensuremath{\pi}}

\hypersetup{%setup hyperlinks
    colorlinks,
    citecolor=black,
    filecolor=black,
    linkcolor=black,
    urlcolor=black
}

% Example boxes
\usepackage{fancybox}
\usepackage{framed}
\usepackage{adjustbox}
\newenvironment{simpages}%
{\AtBeginEnvironment{itemize}{\parskip=0pt\parsep=0pt\partopsep=0pt}
\def\FrameCommand{\fboxsep=.5\FrameSep\shadowbox}\MakeFramed{\FrameRestore}}%
{\endMakeFramed}

% Impulse train
\DeclareFontFamily{U}{wncy}{}
\DeclareFontShape{U}{wncy}{m}{n}{<->wncyr10}{}
\DeclareSymbolFont{mcy}{U}{wncy}{m}{n}
\DeclareMathSymbol{\Sha}{\mathord}{mcy}{"58}
\addbibresource{../../../../includes/bibliography.bib}

\begin{document}

% http://ieeexplore.ieee.org/stamp/stamp.jsp?tp=&arnumber=6068200
\section{Energy detector}

\subsection{Conventional Energy detector}\label{ssec:conv_ed_derivation}
This section will give a derivation of the conventional energy detector based on \cite{atapattu2014energy}.
Let $x[n]$ denote the received signal. The conventional energy detection algorithm must decide between to hypotheses:
\begin{align}\label{eq:hypotheses}
  \mathcal{H}_0&: x[n] = w[n],\\
  \mathcal{H}_1&: x[n] = s[n] + w[n],
\end{align}
in which $w[n]$ denotes additive circular complex Gaussian noise and $s[n]$ denotes a signal as transmitted by a primary user.

We assume that the noise samples are independently identically zero mean circular complex Gaussian distributed, that is $w[n] \sim \mathcal{CN}(0, \sigma_n^2)$. 
We furthermore assume that the samples of the signal $s[n]$ can be modelled independently as circular complex Gaussian $\mathcal{CN}(0, \sigma_s^2)$. % todo ref
% dat dit aannemelijk maakt That is
So

\begin{align*}
x[n] \sim 
    \begin{cases}
        \mathcal{CN}(0, \sigma_n^2) & \text{if $\mathcal{H}_0$}, \\
        \mathcal{CN}[0, (\sigma_s^2 + \sigma_n^2)] & \text{if $\mathcal{H}_1$}.
    \end{cases}
\end{align*} 

Let
\begin{align*}
    \vec{x} = \left[x[0], x[1], \ldots, x[N-1]\right]^T
\end{align*}
denote a vector containing $N$ samples of the signal $x[n]$. Then the likelihood function of $\vec{x}$ denoted as $L(\vec{x})$ is given by

\begin{align*}
    L(\vec{x}) &= \prod_{i=1}^N f_{(\vec{x})_i}\\
    &= \begin{cases}
        \sigma_n^2/\pi^N \cdot \exp(-\bar{\vec{x}}'\sigma_n^{-2}\mathbf{I}\vec{x}) & \text{if $\mathcal{H}_0$}, \\
        \sigma_n^2/\pi^N \cdot \exp[-\bar{\vec{x}}'(\sigma_n^2+\sigma_s^2)^{-1}\mathbf{I}\vec{x}] & \text{if $\mathcal{H}_1$}
    \end{cases}
\end{align*}

where $f_{(\vec{x})_i}$ denotes the probability density function of element $i$ in $\vec{x}$.  The test statistic $\Lambda(\vec{x})$ as used in the Neyman Pearson test is then given by:
\begin{align*}
\Lambda(\vec{x}) &=\frac{\sigma_n^2/\pi^N \cdot \exp(-\bar{\vec{x}}'\sigma_n^{-2}\mathbf{I}\vec{x})}{(\sigma_n^2 + \sigma_s^2)\pi^N \cdot \exp[-\bar{\vec{x}}'(\sigma_n^2+\sigma_s^2)^{-1}\mathbf{I}\vec{x}]}.
\end{align*}


By taking the logarithm of $\Lambda(x)$ we obtain a log-likelihood Ratio test statistic $\Lambda'(x)$, given by

\begin{align}
\Lambda(\vec{x})' &= \log \left\{
\frac{\sigma_n^2\exp(-\overline{\vec{x}}'\sigma_n^{-2}\mathbf{I}\vec{x})}{[\sigma_n^2 + \sigma_s^2]\exp[-\bar{\vec{x}}'(\sigma_n^2+\sigma_s^2)^{-1}\mathbf{I}\vec{x}]}\right\} \\
&= \log(\sigma_n^2) - \log(\sigma_n^2 + \sigma_s^2) +  (\frac{1}{\sigma_n^2} - \frac{1}{\sigma_n^2 + \sigma_s^2 })\sum_{i=0}^{N-1} |x[i]|^2.  \nonumber
\end{align}

Observing that the constants $\sigma_n$ and $\sigma_s$ do not depend on value of the samples, the test statistic\footnote{observe that $\frac{1}{\sigma_n^2} \geq \frac{1}{\sigma_n^2+\sigma_s^2}$}

\begin{align*}
\Lambda''(\vec{x}) &= -\sum_{n=0}^{N-1} |x[n]|^2
\end{align*} 

is proportional to $\Lambda'(x)$. If we negate $\Lambda'$ we get same test statistic as denoted by $\Lambda$ in \Cref{sec:conv_ed}. Do note that
this negation is effectively a division in the original neyman pearson test (logarithm) and therefore if
$-\Lambda'' > \gamma$, the detector decides that $\mathcal{H}_1$ is the true hypothesis. 

\subsubsection{Threshold}
In this section we will determine the threshold $\gamma$ for the energy detector test statistic $-\Lambda''(\vec{x})$.
Under $\mathcal{H}_0$ we have that $\Lambda''(\vec{x})$ is the sum of $2N$ independent zero mean Gaussian distributed variables
with variance $\sigma_n^2/2$. To see why, remember that $\vec{x}$ is circular symmetric complex gaussian and therefore for the element $x_i \in \vec{x}$ the following applies
$|x_i|^2 = x_i \cdot \conj{x_i} = a^2 + b^2$ where $a = \Re(x_i)$ and $b=\Im(x_i)$.  So

\begin{align}
    \frac{2\Lambda''(\vec{x})}{\sigma_n^2} \sim \chi^2_{2N}.
\end{align}

% Sensing Throughput Tradeoff in Cognitive Radio, Y. C. Liang
Therefore, the false alarm probability $p\ss{fa}$ is given by \cite{rugini2013small}

\begin{align*}
    p_{fa} &= P(\Lambda''(\vec{x}) > \gamma) \\
        &= 1-F_{2N} (2\gamma/\sigma_n^2)
\end{align*}

with $F_{2N}$ the cumulative distribution function of a chi square distribution with $2N$ degrees of freedom. 

If $N$ is large enough, we can approximate the test statistics' distribution by a Gaussian distribution as it it is the sum of $2N$ independent identically distributed random variables. By the central limit theorem,

\begin{align*}
F_{2N} \approx 1-Q\left(\frac{2\Lambda''(x)/\sigma^2_n-2N}{2\sqrt{N}}\right).
\end{align*}

$P\ss{fa}$ can then be approximated as 

 \begin{align}
 P\ss{fa} \approx Q\left(\frac{2\gamma/\sigma_n^2 -2N}{2\sqrt{N}}\right).\label{eq:p_fa}
 \end{align} \cite{
 %http://ieeexplore.ieee.org/stamp/stamp.jsp?tp=&arnumber=6061767
 }

Given a desired $P\ss{fa}$, the threshold $\gamma$ is given by

\begin{align}\label{eq:ed_threshold}
 \gamma &= [Q^{-1}\left(p_{fa}\right)\sqrt{N} + N]\sigma_n^2.
 \end{align} 
 %\cite{
 %http://ieeexplore.ieee.org/stamp/stamp.jsp?tp=&arnumber=6061767
 %}
 

\end{document}
