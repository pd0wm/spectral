%!TEX program = xelatex

\documentclass[a4paper, openany, oneside]{memoir}
\usepackage[no-math]{fontspec}
\usepackage{pgfplots}
\pgfplotsset{compat=newest}
\usepackage{commath}
\usepackage{mathtools}
\usepackage{amssymb}
\usepackage{amsthm}
\usepackage{booktabs}
\usepackage{mathtools}
\usepackage{xcolor}
\usepackage[separate-uncertainty=true, per-mode=symbol]{siunitx}
\usepackage[noabbrev, capitalize]{cleveref}
\usepackage{listings}
\usepackage[american inductor, european resistor]{circuitikz}
\usepackage{amsmath}
\usepackage{amsfonts}
\usepackage{ifxetex}
\usepackage[dutch,english]{babel}
\usepackage[backend=bibtexu,texencoding=utf8,bibencoding=utf8,style=ieee,sortlocale=en_GB,language=auto]{biblatex}
\usepackage[strict,autostyle]{csquotes}
\usepackage{parskip}
\usepackage{import}
\usepackage{standalone}
\usepackage{hyperref}
%\usepackage[toc,title,titletoc]{appendix}

\ifxetex{} % Fonts laden in het geval dat je met Xetex compiled
    \usepackage{fontspec}
    \defaultfontfeatures{Ligatures=TeX} % To support LaTeX quoting style
    \setromanfont{Palatino Linotype} % Tover ergens in Font mapje in root.
    \setmonofont{Source Code Pro}
\else % Terug val in standaard pdflatex tool chain. Geen ondersteuning voor OTT fonts
    \usepackage[T1]{fontenc}
    \usepackage[utf8]{inputenc}
\fi
\newcommand{\references}[1]{\begin{flushright}{#1}\end{flushright}}
\renewcommand{\vec}[1]{\boldsymbol{\mathbf{#1}}}
\newcommand{\uvec}[1]{\boldsymbol{\hat{\vec{#1}}}}
\newcommand{\mat}[1]{\boldsymbol{\mathbf{#1}}}
\newcommand{\fasor}[1]{\boldsymbol{\tilde{\vec{#1}}}}
\newcommand{\cmplx}[0]{\mathrm{j}}
\renewcommand{\Re}[0]{\operatorname{Re}}
\newcommand{\Cov}{\operatorname{Cov}}
\newcommand{\Var}{\operatorname{Var}}
\newcommand{\proj}{\operatorname{proj}}
\newcommand{\Perp}{\operatorname{perp}}
\newcommand{\col}{\operatorname{col}}
\newcommand{\rect}{\operatorname{rect}}
\newcommand{\sinc}{\operatorname{sinc}}
\newcommand{\IT}{\operatorname{IT}}
\newcommand{\F}{\mathcal{F}}

\newtheorem{definition}{Definition}
\newtheorem{theorem}{Theorem}


\DeclareSIUnit{\voltampere}{VA} %apparent power
\DeclareSIUnit{\pii}{\ensuremath{\pi}}

\hypersetup{%setup hyperlinks
    colorlinks,
    citecolor=black,
    filecolor=black,
    linkcolor=black,
    urlcolor=black
}

% Example boxes
\usepackage{fancybox}
\usepackage{framed}
\usepackage{adjustbox}
\newenvironment{simpages}%
{\AtBeginEnvironment{itemize}{\parskip=0pt\parsep=0pt\partopsep=0pt}
\def\FrameCommand{\fboxsep=.5\FrameSep\shadowbox}\MakeFramed{\FrameRestore}}%
{\endMakeFramed}

% Impulse train
\DeclareFontFamily{U}{wncy}{}
\DeclareFontShape{U}{wncy}{m}{n}{<->wncyr10}{}
\DeclareSymbolFont{mcy}{U}{wncy}{m}{n}
\DeclareMathSymbol{\Sha}{\mathord}{mcy}{"58}
\addbibresource{../includes/bibliography.bib}

\title{An Extensible Toolkit for Real-Time High-Performance Wideband Spectrum Sensing}

\author{W.P. Bruinsma \and R.P. Hes \and H.J.C. Kroep \and T.C. Leliveld \and W.M. Melching \and T.A. aan de Wiel}

\raggedbottom

\begin{document}
\chapter{Project description} \label{cha:problem_statement}
\section{Problem statement}
TODO: REVISE
To alleviate the problem of spectrum scarcity the concept of a Cognitive Radio has been introduced.
At the heart of such a cognitive radio is spectrum sensing: it is essential that Cognitive Radio's are able to sense a wide band of the spectrum to detect the presence of primary users. To minimize the interference with licensed users it is essential that this sensing process is as fast as possible. As Cognitive Radio networks are still in a conceptual stage, the need for a set of tools that can be used to sense a wide band with high efficiency arises.

This leads us to our problem statement 

\emph{"How to develop a high performance wideband sensing toolkit?"}

The development of such a toolkit requires knowledge of algorithms that can be used to perform wideband sensing. Such algorithms do not only include compressive sensing but also include detection algorithms that are to be used to actually detect users. Furthermore to implement and test those algorithms it is necessary that they can be implemented on a hardware platform. This motivates the division of this thesis in:

\begin{enumerate}
	\item theoretical research of algorithms to be used in the toolkit;
	\item efficient implementation of those algorithms in software.
\end{enumerate}

\section{Proposed solution}
TODO: beschrijf product, leg goed uit deol van part I en part I

\end{document}
