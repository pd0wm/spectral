%!TEX program = xelatex

\documentclass[a4paper, openany, oneside]{memoir}
\usepackage[no-math]{fontspec}
\usepackage{pgfplots}
\pgfplotsset{compat=newest}
\usepackage{commath}
\usepackage{mathtools}
\usepackage{amssymb}
\usepackage{amsthm}
\usepackage{booktabs}
\usepackage{mathtools}
\usepackage{xcolor}
\usepackage[separate-uncertainty=true, per-mode=symbol]{siunitx}
\usepackage[noabbrev, capitalize]{cleveref}
\usepackage{listings}
\usepackage[american inductor, european resistor]{circuitikz}
\usepackage{amsmath}
\usepackage{amsfonts}
\usepackage{ifxetex}
\usepackage[dutch,english]{babel}
\usepackage[backend=bibtexu,texencoding=utf8,bibencoding=utf8,style=ieee,sortlocale=en_GB,language=auto]{biblatex}
\usepackage[strict,autostyle]{csquotes}
\usepackage{parskip}
\usepackage{import}
\usepackage{standalone}
\usepackage{hyperref}
%\usepackage[toc,title,titletoc]{appendix}

\ifxetex{} % Fonts laden in het geval dat je met Xetex compiled
    \usepackage{fontspec}
    \defaultfontfeatures{Ligatures=TeX} % To support LaTeX quoting style
    \setromanfont{Palatino Linotype} % Tover ergens in Font mapje in root.
    \setmonofont{Source Code Pro}
\else % Terug val in standaard pdflatex tool chain. Geen ondersteuning voor OTT fonts
    \usepackage[T1]{fontenc}
    \usepackage[utf8]{inputenc}
\fi
\newcommand{\references}[1]{\begin{flushright}{#1}\end{flushright}}
\renewcommand{\vec}[1]{\boldsymbol{\mathbf{#1}}}
\newcommand{\uvec}[1]{\boldsymbol{\hat{\vec{#1}}}}
\newcommand{\mat}[1]{\boldsymbol{\mathbf{#1}}}
\newcommand{\fasor}[1]{\boldsymbol{\tilde{\vec{#1}}}}
\newcommand{\cmplx}[0]{\mathrm{j}}
\renewcommand{\Re}[0]{\operatorname{Re}}
\newcommand{\Cov}{\operatorname{Cov}}
\newcommand{\Var}{\operatorname{Var}}
\newcommand{\proj}{\operatorname{proj}}
\newcommand{\Perp}{\operatorname{perp}}
\newcommand{\col}{\operatorname{col}}
\newcommand{\rect}{\operatorname{rect}}
\newcommand{\sinc}{\operatorname{sinc}}
\newcommand{\IT}{\operatorname{IT}}
\newcommand{\F}{\mathcal{F}}

\newtheorem{definition}{Definition}
\newtheorem{theorem}{Theorem}


\DeclareSIUnit{\voltampere}{VA} %apparent power
\DeclareSIUnit{\pii}{\ensuremath{\pi}}

\hypersetup{%setup hyperlinks
    colorlinks,
    citecolor=black,
    filecolor=black,
    linkcolor=black,
    urlcolor=black
}

% Example boxes
\usepackage{fancybox}
\usepackage{framed}
\usepackage{adjustbox}
\newenvironment{simpages}%
{\AtBeginEnvironment{itemize}{\parskip=0pt\parsep=0pt\partopsep=0pt}
\def\FrameCommand{\fboxsep=.5\FrameSep\shadowbox}\MakeFramed{\FrameRestore}}%
{\endMakeFramed}

% Impulse train
\DeclareFontFamily{U}{wncy}{}
\DeclareFontShape{U}{wncy}{m}{n}{<->wncyr10}{}
\DeclareSymbolFont{mcy}{U}{wncy}{m}{n}
\DeclareMathSymbol{\Sha}{\mathord}{mcy}{"58}
\addbibresource{../../../includes/bibliography.bib}

\title{Compressive Sensing - An Overview}

\author{W.P. Bruinsma \and R.P. Hes \and H.J.C. Kroep \and T.C. Leliveld \and W.M. Melching \and T.A. aan de Wiel}

\raggedbottom

\begin{document}
\chapter{Introduction}
\cref{prt:theory} provides an in-depth description of the theory required in the design of a high-performance sensing system.
In this part, we will describe how this theory was implemented into a system that integrates sampling, reconstruction and detection and presents the results in a web-application to any user that connects.

In the following section we will state the system requirements that were set and have led to the software package in its final form. Then, starting a new chapter, the software platforms and libraries that were used will be briefly introduced, so that the reader might better understand their role in the system. We will also touch on the workflow that was followed during development. After this, we will move on the system itself and provide an overview of its architecture, so we can develop an understanding of its functioning in a structured manner. In the three chapters following this overview, we describe each of the system's three main individual components in greater detail. Next, the hardware that was used to obtain some actual spectrum data for real-time analysis will be discussed. Consequently, we will perform an analysis of the quality and performance of the final software package, in which testing, some formal guidelines and benchmarking will be deployed to determine how `good' our software is. The part will be concluded with an overview of any future work that might be interesting.

\section{Specifications}
\label{sec:implementation-specs}
This section gives a detailed description of the approach. This description consists of specifications according to which the approach will be designed. The specifications are divided into several categories. The categories are then analysed by MoSCoW prioritisation.

\subsection{General}
There are a number of general requirements that were considered at the start of the project:
\begin{itemize}
    \item The implementation needs to be fast. Existing products are all targeting high performance. The advantage of compressive sensing is that less hardware can be used at the cost of more demanding computations.
    \item The implementation needs to be modular in the sense that different kinds of approaches to the problem (both in hardware and algorithms) can be implemented and require a minimal adjustment in the software.
    \item The implementation needs to flexible enough that it could be run on different computational platforms (distributed networks, embedded devices, etc.) with minimal adjustment in the software.
\end{itemize}

To summarise, we target: \emph{performance}, \emph{modularity} and \emph{flexibility}. More specific requirements, following the MoSCoW-method, like in \cref{prt:theory}, are given in the following section.

\subsection{Software}
The software \emph{must} be developed such that
\begin{enumerate}
    \item it implements the generation and/or sampling of a theoretical spectrum;
    \item it implements the emulation of non-uniform sampling techniques on Nyquist-sampled spectra;
    \item it implements the reconstruction of the sampled spectrum, using techniques from \cref{cha:reconstruction};
    \item it implements the detection of any signal present in the reconstructed spectrum, using techniques from \cref{cha:detection} and
    \item it is able to clearly present the results of generation, reconstruction and detection to the user.
\end{enumerate}
The software \emph{should} be developed such that
\begin{enumerate}
    \item the highest possible performance may be achieved;
    \item it is designed in a structured and maintainable way and
    \item it is be usable on multiple platforms.
\end{enumerate}

\subsection{Hardware}
The software \emph{must} allow for
\begin{enumerate}
    \item the use of a Ettus Research USRP UHD N210 transceiver to obtain data;
    \item the use of such a device to implement compressive sensing and
    \item the use of two (or more) of these devices to implement collaborative sensing.
\end{enumerate}

The software \emph{should} allow for
\begin{enumerate}
    \item fully utilising the USRP's capabilities.
\end{enumerate}

\end{document}
