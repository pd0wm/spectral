%!TEX program = xelatex

\documentclass[a4paper, openany, oneside]{memoir}
\usepackage[no-math]{fontspec}
\usepackage{pgfplots}
\pgfplotsset{compat=newest}
\usepackage{commath}
\usepackage{mathtools}
\usepackage{amssymb}
\usepackage{amsthm}
\usepackage{booktabs}
\usepackage{mathtools}
\usepackage{xcolor}
\usepackage[separate-uncertainty=true, per-mode=symbol]{siunitx}
\usepackage[noabbrev, capitalize]{cleveref}
\usepackage{listings}
\usepackage[american inductor, european resistor]{circuitikz}
\usepackage{amsmath}
\usepackage{amsfonts}
\usepackage{ifxetex}
\usepackage[dutch,english]{babel}
\usepackage[backend=bibtexu,texencoding=utf8,bibencoding=utf8,style=ieee,sortlocale=en_GB,language=auto]{biblatex}
\usepackage[strict,autostyle]{csquotes}
\usepackage{parskip}
\usepackage{import}
\usepackage{standalone}
\usepackage{hyperref}
%\usepackage[toc,title,titletoc]{appendix}

\ifxetex{} % Fonts laden in het geval dat je met Xetex compiled
    \usepackage{fontspec}
    \defaultfontfeatures{Ligatures=TeX} % To support LaTeX quoting style
    \setromanfont{Palatino Linotype} % Tover ergens in Font mapje in root.
    \setmonofont{Source Code Pro}
\else % Terug val in standaard pdflatex tool chain. Geen ondersteuning voor OTT fonts
    \usepackage[T1]{fontenc}
    \usepackage[utf8]{inputenc}
\fi
\newcommand{\references}[1]{\begin{flushright}{#1}\end{flushright}}
\renewcommand{\vec}[1]{\boldsymbol{\mathbf{#1}}}
\newcommand{\uvec}[1]{\boldsymbol{\hat{\vec{#1}}}}
\newcommand{\mat}[1]{\boldsymbol{\mathbf{#1}}}
\newcommand{\fasor}[1]{\boldsymbol{\tilde{\vec{#1}}}}
\newcommand{\cmplx}[0]{\mathrm{j}}
\renewcommand{\Re}[0]{\operatorname{Re}}
\newcommand{\Cov}{\operatorname{Cov}}
\newcommand{\Var}{\operatorname{Var}}
\newcommand{\proj}{\operatorname{proj}}
\newcommand{\Perp}{\operatorname{perp}}
\newcommand{\col}{\operatorname{col}}
\newcommand{\rect}{\operatorname{rect}}
\newcommand{\sinc}{\operatorname{sinc}}
\newcommand{\IT}{\operatorname{IT}}
\newcommand{\F}{\mathcal{F}}

\newtheorem{definition}{Definition}
\newtheorem{theorem}{Theorem}


\DeclareSIUnit{\voltampere}{VA} %apparent power
\DeclareSIUnit{\pii}{\ensuremath{\pi}}

\hypersetup{%setup hyperlinks
    colorlinks,
    citecolor=black,
    filecolor=black,
    linkcolor=black,
    urlcolor=black
}

% Example boxes
\usepackage{fancybox}
\usepackage{framed}
\usepackage{adjustbox}
\newenvironment{simpages}%
{\AtBeginEnvironment{itemize}{\parskip=0pt\parsep=0pt\partopsep=0pt}
\def\FrameCommand{\fboxsep=.5\FrameSep\shadowbox}\MakeFramed{\FrameRestore}}%
{\endMakeFramed}

% Impulse train
\DeclareFontFamily{U}{wncy}{}
\DeclareFontShape{U}{wncy}{m}{n}{<->wncyr10}{}
\DeclareSymbolFont{mcy}{U}{wncy}{m}{n}
\DeclareMathSymbol{\Sha}{\mathord}{mcy}{"58}
\addbibresource{../../../includes/bibliography.bib}

\title{Compressive Sensing - An Overview}

\author{W.P. Bruinsma \and R.P. Hes \and H.J.C. Kroep \and T.C. Leliveld \and W.M. Melching \and T.A. aan de Wiel}

\raggedbottom

\begin{document}
\chapter{Software quality \& Testing}
\section{Introduction}
In this chapter we will talk about software quality. We will discuss this according to the definitions from Code Complete 2 about external and internal characteristics of quality code. External qualities are qualities the user of the software will notice. Internal qualities are only noticed by the developers of the software.

\subsection{External qualities}
\begin{description}
\item[Correctness] The degree to which a system is free from faults in its specification, design, and implementation.
\item[Usability] The easy with which users can learn and use a system.
\item[Efficiency] Minimal use of system resources, including memory and execution time.
\item[Reliability] The ability of a system to perform its required functions under state conditions whenever required --- having a long mean time between failures.
\item[Integrity] The degree to which a system prevents unauthorised or improper access to its programs and data. The idea of integrity includes restricting unauthorised user accesses as well ensuring that data is accessed properly --- that is, that tables with parallel data are modified in parallel, that date fields contain only valid dated, and so on.
\item[Adaptability] The extent to which a system can be used, without modification, in applications or environments other than those for which it was specifically designed.
\item[Accuracy] The degree to which a system, as built, is free from error, especially with respect to quantitative outputs. Accuracy differs from correctness; it is a determination of how well a system does the job it's built for rather than whether it was built correctly.
\item[Robustness] The degree to which a system continues to function in the presence of invalid inputs or stressful environmental conditions.
\end{description}

\subsection{Internal qualities}
\begin{description}
\item[Maintainability] The ease with which you modify a software system to change or add capabilities, improve performance, or correct defect.
\item[Flexibility] The extent to which you can modify a system for uses or environments other than those for which it was specifically designed.
\item[Portability] The ease with which you can modify a system to operate in an environment different from that for which it was specifically designed.
\item[Reusability] The extent to which you and the ease with which you can use parts of a system in other systems.
\item[Readability] The ease with which you can read and understand the source code of a system, especially at the detailed-statement level.
\item[Testability] The degree to which you can unit-test and system-test a system; the degree to which you can verify that the system meets its requirements.
\item[Understandability] The ease with which you can comprehend a system at both the system-organisational and detailed-statement levels. Understandability has to do with the coherence of the system at a more general level than readability does.
\end{description}

\section{Unit Tests}
\label{sec:unit-tests}

\section{Integration tests}
\label{sec:integration-tests}



\end{document}
