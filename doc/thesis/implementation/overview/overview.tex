%!TEX program = xelatex

\documentclass[a4paper, openany, oneside]{memoir}
\usepackage[no-math]{fontspec}
\usepackage{pgfplots}
\pgfplotsset{compat=newest}
\usepackage{commath}
\usepackage{mathtools}
\usepackage{amssymb}
\usepackage{amsthm}
\usepackage{booktabs}
\usepackage{mathtools}
\usepackage{xcolor}
\usepackage[separate-uncertainty=true, per-mode=symbol]{siunitx}
\usepackage[noabbrev, capitalize]{cleveref}
\usepackage{listings}
\usepackage[american inductor, european resistor]{circuitikz}
\usepackage{amsmath}
\usepackage{amsfonts}
\usepackage{ifxetex}
\usepackage[dutch,english]{babel}
\usepackage[backend=bibtexu,texencoding=utf8,bibencoding=utf8,style=ieee,sortlocale=en_GB,language=auto]{biblatex}
\usepackage[strict,autostyle]{csquotes}
\usepackage{parskip}
\usepackage{import}
\usepackage{standalone}
\usepackage{hyperref}
%\usepackage[toc,title,titletoc]{appendix}

\ifxetex{} % Fonts laden in het geval dat je met Xetex compiled
    \usepackage{fontspec}
    \defaultfontfeatures{Ligatures=TeX} % To support LaTeX quoting style
    \setromanfont{Palatino Linotype} % Tover ergens in Font mapje in root.
    \setmonofont{Source Code Pro}
\else % Terug val in standaard pdflatex tool chain. Geen ondersteuning voor OTT fonts
    \usepackage[T1]{fontenc}
    \usepackage[utf8]{inputenc}
\fi
\newcommand{\references}[1]{\begin{flushright}{#1}\end{flushright}}
\renewcommand{\vec}[1]{\boldsymbol{\mathbf{#1}}}
\newcommand{\uvec}[1]{\boldsymbol{\hat{\vec{#1}}}}
\newcommand{\mat}[1]{\boldsymbol{\mathbf{#1}}}
\newcommand{\fasor}[1]{\boldsymbol{\tilde{\vec{#1}}}}
\newcommand{\cmplx}[0]{\mathrm{j}}
\renewcommand{\Re}[0]{\operatorname{Re}}
\newcommand{\Cov}{\operatorname{Cov}}
\newcommand{\Var}{\operatorname{Var}}
\newcommand{\proj}{\operatorname{proj}}
\newcommand{\Perp}{\operatorname{perp}}
\newcommand{\col}{\operatorname{col}}
\newcommand{\rect}{\operatorname{rect}}
\newcommand{\sinc}{\operatorname{sinc}}
\newcommand{\IT}{\operatorname{IT}}
\newcommand{\F}{\mathcal{F}}

\newtheorem{definition}{Definition}
\newtheorem{theorem}{Theorem}


\DeclareSIUnit{\voltampere}{VA} %apparent power
\DeclareSIUnit{\pii}{\ensuremath{\pi}}

\hypersetup{%setup hyperlinks
    colorlinks,
    citecolor=black,
    filecolor=black,
    linkcolor=black,
    urlcolor=black
}

% Example boxes
\usepackage{fancybox}
\usepackage{framed}
\usepackage{adjustbox}
\newenvironment{simpages}%
{\AtBeginEnvironment{itemize}{\parskip=0pt\parsep=0pt\partopsep=0pt}
\def\FrameCommand{\fboxsep=.5\FrameSep\shadowbox}\MakeFramed{\FrameRestore}}%
{\endMakeFramed}

% Impulse train
\DeclareFontFamily{U}{wncy}{}
\DeclareFontShape{U}{wncy}{m}{n}{<->wncyr10}{}
\DeclareSymbolFont{mcy}{U}{wncy}{m}{n}
\DeclareMathSymbol{\Sha}{\mathord}{mcy}{"58}
\addbibresource{../../../includes/bibliography.bib}

\title{Compressive Sensing - An Overview}

\author{W.P. Bruinsma \and R.P. Hes \and H.J.C. Kroep \and T.C. Leliveld \and W.M. Melching \and T.A. aan de Wiel}

\raggedbottom

\begin{document}
\chapter{System Overview}
In this chapter we will give an overview of the chosen system architecture. In the following chapters we will describe each part of the system in more detail.

\section{Model-View-Presenter}
\label{sec:model-view-presenter}
The design of our system can best be described by a model-view-presenter architectural pattern (with passive view, referred to as MVP from now on). It is a variation on the well known model-view-controller pattern \cite{syromiatnikov2014journey}. It consists of the following components:
\begin{description}
    \item[Model] This encompasses the library \lib{spectral\_core}, that handles all the domain logic\footnote{The domain logic encompasses all the functional parts of our code. It is the ``world'' our code lives in.}.
    \item[Presenter] This encompasses part of the library \lib{spectral\_supervisor}, that handles all the application logic\footnote{The application logic is the logic that drives the application. This is amongst other things handling the input data from the user and reflecting it in the model}.
    \item[View] This encompasses part of the library \lib{spectral\_web}, that contains the GUI elements.
\end{description}
An overview can be seen in \cref{fig:MVP}. The most important advantage of MVP is that the domain logic and application logic are completely separate. The model concerns itself with the domain logic, the view with the representation of the data and the presenter manages the data and user input synchronisation. This allows us to do orthogonal design.

\begin{figure*}
    \centering
    %!TEX program = xelatex

\documentclass[a4paper, openany, oneside]{memoir}
\usepackage[no-math]{fontspec}
\usepackage{pgfplots}
\pgfplotsset{compat=newest}
\usepackage{commath}
\usepackage{mathtools}
\usepackage{amssymb}
\usepackage{amsthm}
\usepackage{booktabs}
\usepackage{mathtools}
\usepackage{xcolor}
\usepackage[separate-uncertainty=true, per-mode=symbol]{siunitx}
\usepackage[noabbrev, capitalize]{cleveref}
\usepackage{listings}
\usepackage[american inductor, european resistor]{circuitikz}
\usepackage{amsmath}
\usepackage{amsfonts}
\usepackage{ifxetex}
\usepackage[dutch,english]{babel}
\usepackage[backend=bibtexu,texencoding=utf8,bibencoding=utf8,style=ieee,sortlocale=en_GB,language=auto]{biblatex}
\usepackage[strict,autostyle]{csquotes}
\usepackage{parskip}
\usepackage{import}
\usepackage{standalone}
\usepackage{hyperref}
%\usepackage[toc,title,titletoc]{appendix}

\ifxetex{} % Fonts laden in het geval dat je met Xetex compiled
    \usepackage{fontspec}
    \defaultfontfeatures{Ligatures=TeX} % To support LaTeX quoting style
    \setromanfont{Palatino Linotype} % Tover ergens in Font mapje in root.
    \setmonofont{Source Code Pro}
\else % Terug val in standaard pdflatex tool chain. Geen ondersteuning voor OTT fonts
    \usepackage[T1]{fontenc}
    \usepackage[utf8]{inputenc}
\fi
\newcommand{\references}[1]{\begin{flushright}{#1}\end{flushright}}
\renewcommand{\vec}[1]{\boldsymbol{\mathbf{#1}}}
\newcommand{\uvec}[1]{\boldsymbol{\hat{\vec{#1}}}}
\newcommand{\mat}[1]{\boldsymbol{\mathbf{#1}}}
\newcommand{\fasor}[1]{\boldsymbol{\tilde{\vec{#1}}}}
\newcommand{\cmplx}[0]{\mathrm{j}}
\renewcommand{\Re}[0]{\operatorname{Re}}
\newcommand{\Cov}{\operatorname{Cov}}
\newcommand{\Var}{\operatorname{Var}}
\newcommand{\proj}{\operatorname{proj}}
\newcommand{\Perp}{\operatorname{perp}}
\newcommand{\col}{\operatorname{col}}
\newcommand{\rect}{\operatorname{rect}}
\newcommand{\sinc}{\operatorname{sinc}}
\newcommand{\IT}{\operatorname{IT}}
\newcommand{\F}{\mathcal{F}}

\newtheorem{definition}{Definition}
\newtheorem{theorem}{Theorem}


\DeclareSIUnit{\voltampere}{VA} %apparent power
\DeclareSIUnit{\pii}{\ensuremath{\pi}}

\hypersetup{%setup hyperlinks
    colorlinks,
    citecolor=black,
    filecolor=black,
    linkcolor=black,
    urlcolor=black
}

% Example boxes
\usepackage{fancybox}
\usepackage{framed}
\usepackage{adjustbox}
\newenvironment{simpages}%
{\AtBeginEnvironment{itemize}{\parskip=0pt\parsep=0pt\partopsep=0pt}
\def\FrameCommand{\fboxsep=.5\FrameSep\shadowbox}\MakeFramed{\FrameRestore}}%
{\endMakeFramed}

% Impulse train
\DeclareFontFamily{U}{wncy}{}
\DeclareFontShape{U}{wncy}{m}{n}{<->wncyr10}{}
\DeclareSymbolFont{mcy}{U}{wncy}{m}{n}
\DeclareMathSymbol{\Sha}{\mathord}{mcy}{"58}
\addbibresource{../../../includes/bibliography.bib}
\raggedbottom

\begin{document}
\begin{tikzpicture}
    \usetikzlibrary{arrows,automata,calc}
    \tikzset{
        state/.style={
            rectangle,
            draw=black, very thick,
            minimum height=2em,
            inner sep=2pt,
            text centered,
            anchor=north,
        },
    }
    \tikzstyle{dashedd} = [anchor=north, rectangle, inner sep=5pt, draw=black, very thick, dashed]
    \tikzstyle{fancyarrow} = [draw=black, very thick, ->, auto]
    \tikzstyle{textnodes} = [align=center]
    \node[dashedd] (model) at (0, 0) {
        \begin{tikzpicture}
            \node at (0,1) {\textbf{Model}};
            \node[state] (cogradio) {
                \begin{tabular}{l}
                    \textbf{Core}\\
                    \midrule
                    Source \\
                    Sampling \\
                    Reconstruction \\
                    Detection \\
            \end{tabular}};
        \end{tikzpicture}
    };

    \node[dashedd] (presenter) at (4.5, 0) {
        \begin{tikzpicture}
            \node at (0,1) {\textbf{Presenter}};
            \node[state] (run) at (0,0) {
                \begin{tabular}{l}
                    \textbf{Supervisor}\\
                    \midrule
                    Settings \\
                    Websocket \\
            \end{tabular}};
        \end{tikzpicture}
    };

    \node[dashedd] (view) at (9.5, 0) {
        \begin{tikzpicture}
            \node at (0,1) {\textbf{View}};
            \node[state] (vis) at (0,0){
                \begin{tabular}{l}
                    \textbf{Web}\\
                    \midrule
                    Client-side JS/CSS \\
                    Web-server
            \end{tabular}};
        \end{tikzpicture}
    };
    \draw[fancyarrow,anchor=north] (model) -- node[textnodes]{data\\(IPC)} (model-|presenter.west);
    \draw[fancyarrow,anchor=south] (presenter) -- node[textnodes]{options} (presenter-|model.east);

    \draw[fancyarrow,anchor=north] ($(presenter.east)-(0,0.5)$) -- node[textnodes]{data\\(websocket)} ($(presenter-|view.west)-(0,0.5)$);
    \draw[fancyarrow,anchor=south] (view) -- node[textnodes]{user input} (view-|presenter.east);
\end{tikzpicture}
\end{document}

    \caption{Illustration of the MVP pattern as an hierarchical way of separating our system.}
    \label{fig:MVP}
\end{figure*}

\subsection{Model}
\label{sub:model}
The goal was to implement enough primitives in these libraries that a large number of configurations can quickly be constructed together. This is to account for the variety of configurations in which this product can be used. It also allows us to test different kind of sampling techniques with reconstruction methods.

A side effect of writing code this way is that it makes our code modular. This means most ``blocks'' can be tested individually. To accommodate this, the Strategy design pattern\footnote{The strategy pattern implements a family of algorithms behind an interface. It was first introduced in \cite{designpatterns}. } is used for all the models. This is further discussed in \cref{cha:model}.

\subsection{View}
\label{sub:view}
The view should supply the user with a clear overview of what is happening in the model and allow them to change execution parameters during runtime. It fulfils the role of a \emph{graphical user interface} or GUI. An additional requirement is that the view should be accessible to multiple users at the same time. How this is achieved will be described in \cref{cha:view}.

\subsection{Presenter}
\label{sub:presenter}
Finally, the presenter assumes the role of middle-man, in the sense that it forms the `glue' between model and view. It coordinates both their actions and allows them to interact in a controlled manner. A major feature of the presenter is that it coordinates the entire system in a concurrent manner, in stead of sequentially, which should greatly increase performance, since most tasks can be handled simultaneously. The presenter will be described in-depth in \cref{cha:presenter}.

\end{document}
