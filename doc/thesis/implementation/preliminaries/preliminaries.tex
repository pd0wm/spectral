%!TEX program = xelatex

\documentclass[a4paper, openany, oneside]{memoir}
\usepackage[no-math]{fontspec}
\usepackage{pgfplots}
\usepackage{float}
\pgfplotsset{compat=newest}
\usepackage{commath}
\usepackage{mathtools}
\usepackage{amssymb}
\usepackage{amsthm}
\usepackage{booktabs}
\usepackage{todonotes}
\usepackage{mathtools}
\usepackage{xcolor}
\usepackage[separate-uncertainty=true, per-mode=symbol]{siunitx}
\usepackage{listings}
\usepackage[american inductor, european resistor]{circuitikz}
\usepackage{amsmath}
\usepackage{amsfonts}
\usepackage{ifxetex}
\usepackage[dutch,english]{babel}
\usepackage[backend=bibtexu,texencoding=utf8,bibencoding=utf8,style=ieee,sortlocale=en_GB,language=auto]{biblatex}
\usepackage[strict,autostyle]{csquotes}
\usepackage{import}
\usepackage{standalone}
\usepackage{bookmark,hyperref}
\usepackage{xcolor,mdframed}
\usepackage{tikz}
\usepackage{framed}
\usepackage{float}
\usepackage{tabularx}
\usepackage{graphicx,adjustbox}
\usepackage{rotating}
\usepackage{pdfpages}
\usepackage{enumitem}
\usepackage{calc}
\usepackage{pgfplots}
\usepackage{filecontents}
\usepackage{caption}
\usepackage{subcaption}
\usepackage{lettrine}

\newcolumntype{Y}{>{\raggedright\arraybackslash}X} % Left-justified text in tabularx environment

\ifxetex{} % Fonts laden in het geval dat je met Xetex compiled
    \usepackage{fontspec}
    \defaultfontfeatures{Scale=MatchLowercase, Ligatures=TeX} % To support LaTeX quoting style
    %\setromanfont{Palatino Linotype} % Tover ergens in Font mapje in root.
    \setsansfont{Avenir Next LT Pro}
    \setromanfont{Adobe Caslon Pro} % Tover ergens in Font mapje in root.
    \setmonofont{Source Code Pro}
\else % Terug val in standaard pdflatex tool chain. Geen ondersteuning voor OTT fonts
    \usepackage[T1]{fontenc}
    \usepackage[utf8]{inputenc}
\fi
\usepackage[noabbrev, capitalize]{cleveref}
\usepackage{ifthen}
\usepackage{titlesec}
\usepackage{titlecaps}

\newcommand{\references}[1]{\begin{flushright}{#1}\end{flushright}}
\renewcommand{\vec}[1]{\boldsymbol{\mathbf{#1}}}
\newcommand{\uvec}[1]{\boldsymbol{\hat{\vec{#1}}}}
\newcommand{\mat}[1]{\boldsymbol{\mathbf{#1}}}
\newcommand{\fasor}[1]{\boldsymbol{\tilde{\vec{#1}}}}
\newcommand{\cmplx}[0]{\mathrm{j}}
\renewcommand{\Re}[0]{\operatorname{Re}}
\newcommand{\Cov}{\operatorname{Cov}}
\newcommand{\Var}{\operatorname{Var}}
\newcommand{\proj}{\operatorname{proj}}
\newcommand{\Perp}{\operatorname{perp}}
\newcommand{\col}{\operatorname{col}}
\newcommand{\rect}{\operatorname{rect}}
\newcommand{\sinc}{\operatorname{sinc}}
\newcommand{\lcm}{\operatorname{lcm}}
%\newcommand{\gcd}{\operatorname{gcd}}
\newcommand{\F}{\mathcal{F}}
\newcommand{\DTFT}{\mathcal{F}_*}
\newcommand{\conj}[1]{#1^*}
\renewcommand{\mod}{\operatorname{mod}}
\newcommand{\rot}{\operatorname{rot}}
\newcommand{\vecsc}[1]{\vec{\textsc{\textbf{#1}}}}
\renewcommand{\ss}[1]{_{#1}}

% Label without linebreak breaker
\newcommand{\lab}[1]{\label{#1}\nolinebreak}

\newtheorem{definition}{Definition}
\newtheorem{theorem}{Theorem}


\DeclareSIUnit{\voltampere}{VA} %apparent power
\DeclareSIUnit{\pii}{\ensuremath{\pi}}

\hypersetup{%setup hyperlinks
    colorlinks,
    citecolor=black,
    filecolor=black,
    linkcolor=black,
    urlcolor=black
}

% Example boxes
\usepackage{fancybox}
\usepackage{framed}
\usepackage{adjustbox}
\newenvironment{simpages}%
{\AtBeginEnvironment{itemize}{\parskip=0pt\parsep=0pt\partopsep=0pt}
\def\FrameCommand{\fboxsep=.5\FrameSep\shadowbox}\MakeFramed{\FrameRestore}}%
{\endMakeFramed}

% Impulse train
\DeclareFontFamily{U}{wncy}{}
\DeclareFontShape{U}{wncy}{m}{n}{<->wncyr10}{}
\DeclareSymbolFont{mcy}{U}{wncy}{m}{n}
\DeclareMathSymbol{\Sha}{\mathord}{mcy}{"58}

\setlength{\parindent}{0pt}
\nonzeroparskip

% Block environment configuration
\newcommand{\BlockLeftMargin}{20pt}
\newcommand{\BlockLeftMarginText}{25pt}
\newcommand{\BlockLeftMarginTextSpacing}{10pt}

% Own colours
\definecolor{gray75}{gray}{0.75}

% Block environment
\newenvironment{block}[3]{%
\makebox{\hspace{-\spinemargin}%
\begin{tikzpicture}[overlay]
    \draw [thick,color=gray75] (\BlockLeftMargin, 0) -- (\paperwidth - \spinemargin, 0);
    \node at (\BlockLeftMarginText, -0.9) [align=left, text width=\spinemargin - \BlockLeftMarginText - \BlockLeftMarginTextSpacing, anchor=west, text depth=1cm] {\textbf{\textsc{#1}}\newline\textit{#3}};
\end{tikzpicture}}%
\nopagebreak\\[0.25em]\ifthenelse{\equal{#2}{}}{}{(\textit{#2}.) }\nopagebreak\nolinebreak}
{\nopagebreak\\[-0.25em]%
\makebox{\hspace{-\spinemargin}%
\begin{tikzpicture}[overlay, remember picture]
    \draw [thick,color=gray75] (\spinemargin,0) -- (\paperwidth - \spinemargin,0);
\end{tikzpicture}} \vspace{0.5em}}

% Theorem
\newcounter{blockTheoremCounter}
\crefname{blockTheoremCounter}{Theorem}{Theorems}
\Crefname{blockTheoremCounter}{Theorem}{Theorems}

\newenvironment{blockTheorem}[1][]{%
\refstepcounter{blockTheoremCounter}%
\begin{block}{theorem \theblockTheoremCounter}{#1}{}}
{\end{block}}

% Definition
\newcounter{blockDefinitionCounter}
\crefname{blockDefinitionCounter}{Definition}{Definitions}
\Crefname{blockDefinitionCounter}{Definition}{Definitions}

\newenvironment{blockDefinition}[1][]{%
\refstepcounter{blockDefinitionCounter}%
\begin{block}{definition \theblockDefinitionCounter}{#1}{}}
{\end{block}}

% Proof
\newcounter{blockProofTheoremCounter}
\crefname{blockProofTheoremCounter}{Proof}{Proofs}
\Crefname{blockProofTheoremCounter}{Proof}{Proofs}

\newenvironment{blockProofTheorem}[1]{%
\refstepcounter{blockProofTheoremCounter}%
\begin{block}{proof of \\ theorem #1}{}{}}
{\qed\end{block}}

% Detail
\newcounter{blockDetailCounter}
\crefname{blockDetailCounter}{Detail}{Details}
\Crefname{blockDetailCounter}{Detail}{Details}

\newenvironment{blockDetail}[1][]{%
\refstepcounter{blockDetailCounter}%
\begin{block}{detail \theblockDetailCounter}{#1}{}}
{\end{block}}

% Redesign chapter headings
\newcommand{\chapternumber}{\thechapter}
\newcommand{\hsp}{\hspace{20pt}}
\titleformat{\chapter}[hang]{\Huge\bfseries}{\chapternumber\hsp\textcolor{gray75}{|}\hsp}{0pt}{\Huge\bfseries}

% Remove headers
% \addtopsmarks{headings}{}{
%   \createmark{chapter}{left}{nonumber}{}{}
% }
% \pagestyle{headings} % Activate changes

% Capitalise headers in a regular way
\renewcommand*{\memUChead}[1]{\titlecap{#1}}

% \hfill for math mode
\newcommand{\pushright}[1]{\intertext{\hfill$\displaystyle #1$}}
\newcommand{\pushline}{\hskip \textwidth minus \textwidth}
\newcommand{\matlab}{\textsc{Matlab}}

\definecolor{code-grey}{HTML}{DDDDDD}
\newcommand{\lib}[1]{\textsf{#1}}
\newcommand{\file}[1]{\textsf{#1}}
\newcommand{\func}[1]{\colorbox{code-grey}{\texttt{#1}}}
\newcommand{\class}[1]{\colorbox{code-grey}{\texttt{#1}}}

% Setup actiepunten
\newenvironment{important}[1][]{%
   \begin{mdframed}[%
      backgroundcolor={red!15}, hidealllines=true,
      skipabove=0.7\baselineskip, skipbelow=0.7\baselineskip,
      splitbottomskip=2pt, splittopskip=4pt, #1]%
   \makebox[0pt]{% ignore the withd of !
      \smash{% ignor the height of !
         \fontsize{32pt}{32pt}\selectfont% make the ! bigger
         \hspace*{-19pt}% move ! to the left
         \raisebox{-2pt}{% move ! up a little
            {\color{red!70!black}\sffamily\bfseries !}% type the bold red !
         }%
      }%
   }%
}{\end{mdframed}}
\newcommand{\excl}[1]{
\begin{important}
  \textbf{#1}
\end{important}
}

\makeatletter
\newcommand\footnoteref[1]{\protected@xdef\@thefnmark{\ref{#1}}\@footnotemark}
\makeatother

% Allow page breaks in display environments
%\allowdisplaybreaks
% S unit for use in Mega Samples per second
\DeclareSIUnit\sample{S}

\newcommand{\CC}{C\nolinebreak\hspace{-.05em}\raisebox{.3ex}{ \textbf{+}}\nolinebreak\hspace{-.10em}\raisebox{.3ex}{\textbf{+}}}
\def\CC{{C\nolinebreak[4]\hspace{-.05em}\raisebox{.3ex}{\textbf{++}}}}


\newcommand{\partauthor}[1]{\gdef\@partauthor{#1}}
\renewcommand{\printparttitle}[1]{
  \parttitlefont #1\\
  \vspace{1.5cm}
  \textnormal{\Large \@partauthor}
}
\addbibresource{../../../includes/bibliography.bib}

\title{Compressive Sensing - An Overview}

\author{W.P. Bruinsma \and R.P. Hes \and H.J.C. Kroep \and T.C. Leliveld \and W.M. Melching \and T.A. aan de Wiel}

\raggedbottom

% TODO
% Cite all packages/libraries
% Cross-reference other chapters

\begin{document}
\chapter{Preliminaries}
\label{cha:impl:preliminaries}
There are a number of aspects that were considered at the start of the project:
\begin{itemize}
    \item The implementation needs to be fast. The products from our competitors are all high performance products. The advantage of compressive sensing is that less hardware can be used at the cost of more demanding computations.
    \item The implementation needs to be modular in the sense that different kinds of approaches to the problem (both in hardware and algorithms) can be implemented and require a minimal amount of adjustment in the software.
    \item The implementation needs to flexible enough that it could be run on different computational platforms (distributed networks, embedded devices, etc.) with minimal amount of adjustment in the software.
\end{itemize}

\section{Libraries}
\label{sec:libraries}
In this section we will discuss a number of libraries used and the advantages these have over other libraries/frameworks.

\subsection{Python}
\label{sec:python}
Initially, two platforms were considered for development: \matlab{} and Python. Taking our project preliminaries in account, Python provides the best option. Reasons for picking Python are:
\begin{itemize}
    \item The Python libraries \lib{NumPy} and \lib{SciPy} use the same linear algebra libraries\footnote{Depending on OS this can be one of the many implementations. A few popular are \lib{openBLAS}, \lib{ATLAS}, \lib{Accelerate} on Apple machines or Intel's proprietary \lib{MKL}} that \matlab{} uses for the actual calculations, so the speed difference will be marginal.
    \item Python is a non-proprietary platform, meaning it can be used commercially in the final product without additional charge.
    \item Python provides many libraries that allow for distributed networking/computing \cite{pythonmp}.
    \item Python is a full-blown programming language, meaning it has libraries for non-numerical applications such as web-frameworks, that allow for more advanced visualisation options.
\end{itemize}

\subsection{Numpy}
\label{sec:numpy}
\lib{NumPy} is Python's standard numerical calculations library, providing a faster (multi-dimensional) array and high level bindings for a variety of linear algebra routines, amongst other. \lib{NumPy} is licensed under the new-BSD\footnoteref{fn:bsd} license.

\subsection{Scipy}
\label{sec:scipy}
\lib{SciPy} is Python's replacement for a lot of default functions that are included in \matlab{}. Relying on \lib{NumPy} data structures, it gives a variety of libraries (such as \lib{signal}, their DSP-library) providing another means of quickly implementing advanced calculations. \lib{SciPy} is licensed under the new-BSD\footnote{\label{fn:bsd}The BSD 3-Clause license is free for commercial and non-commercial for more info and the template see \cite{bsdlic}.} license.

\subsection{Flask}
\label{sec:flask}
\lib{Flask} micro web framework. The choice for a web interface will further be discussed in the visualisation part of this thesis. The advantage of \lib{Flask} over other web frameworks (such as \lib{Django}) is that it is a more bare-bones minimalistic (i.e. simple) approach that suits our needs. \lib{Flask} is licensed under the new-BSD\footnoteref{fn:bsd} license.

\subsection{Twisted}
\label{sub:twisted}
\lib{Twisted} is Python's event-driven network programming framework. It was used in conjunction with \lib{Autobahn}. \lib{Twisted} is licensed under the MIT\footnoteref{fn:mit} license.

\subsection{Autobahn}
\label{sub:autobahn}
\lib{Autobahn} is a real-time framework for websockets. It was used as a means of sending data to the GUI. It is written on top of \lib{Twisted} and \lib{asyncio} and is licensed under the MIT\footnoteref{fn:mit} license.

\subsection{Pyro}
\label{sec:pyro}
A popular remote-object library. It allows us to do application programming over the network. In the future it could be an easy means of doing distributed networking but as of now it functions as a simple means of inter-process communication. \lib{Pyro4} is licensed under the MIT\footnote{\label{fn:mit}Software under the MIT license is free for both commercial and non-commercial use. For more info see \cite{mitlic}.} license.

\subsection{Bootstrap}
\label{sec:bootstrap}
\lib{Bootstrap} is a front-end CSS (Cascading Style Sheets, a language for styling web pages) framework developed by Twitter. It implements a number of primitives, much like a GUI framework does on X11 (the display server used in many Linux distributions), allowing us to quickly implement a portable GUI. It suits our demands to be able to render both on mobile and desktop views, without much effort. \lib{Bootstrap} is licensed under the MIT\footnoteref{fn:mit} license.

\subsection{Highcharts}
\label{sec:highcharts}
\lib{Highcharts} is a JavaScript plotting library capable of fulfilling all front-end plotting needs. This library was picked because of its customisability and ease-of-use. One major side note is that it is only free for non-commercial use.

\section{Workflow}
\label{sec:workflow}
Our workflow consisted of a rather standard software development environment.

\subsection{Version Control}
\label{sec:version-control}
For version control git was used. Git as a rather standard distributed version control system enabling non-linear development in the form of branching. This allows us to maintain a stable master branch, which is used to maintain a stable version of our product, a development branch in which features can be combined and minor bugs can be fixed and several feature branches for larger feature implementations. This specific workflow with git is called Git Flow\cite{driessen2010successful} and is pictured in \cref{fig:gitflow}.


\begin{figure}[p]
    \centering
    \includegraphics[width=\textwidth]{Git-branching-model.pdf}
    \caption{Git Flow}
    \label{fig:gitflow}
\end{figure}

\subsection{Testing}
\label{sec:testing}
To verify the correctness of our code we used two kinds of testing: unit tests and integration tests.

\subsubsection{Unit tests}
Unit tests are small tests that test on a function level. They test if each function returns the right values, and tests for edge cases. These unit tests should be fast and be ran often while writing code. This way you can verify the correctness of your code at all times. When the tests pass, you know the code works according to the specifications. This becomes particularity important when working on a large project with multiple people. When your codes breaks some other part of the codebase written by someone else, you immediately become aware of this. Writing unit tests also forces you to think more about your code and stimulates modular design.

It is also possible to write the tests before the implementation of a function. This is called test driven development (TDD). In our design we decided not to you use this, because the specifications of our code were not clear when we started writing code. When would do a complete rewrite of the code we would certainly use TDD.

\subsubsection{Integration tests}
Integration tests are tests that encompass a large part of the system. In our case we wrote tests where a couple of functions were tested at the same time. The output data was then compared by the output of a Matlab script that we knew was correct.


\subsection{Profiling}
\label{sec:profiling}
For profiling the standard \lib{cProfile} library was used. This is included in the standard Python distribution and is the standard method. For the call graph diagrams included later in this report the library \lib{pycallgraph} was used. This is a simple wrapper around a number of tools including the archaic \lib{GraphViz}.

\end{document}
