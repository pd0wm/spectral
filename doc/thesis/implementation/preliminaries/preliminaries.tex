%!TEX program = xelatex

\documentclass[a4paper, openany, oneside]{memoir}
\usepackage[no-math]{fontspec}
\usepackage{pgfplots}
\pgfplotsset{compat=newest}
\usepackage{commath}
\usepackage{mathtools}
\usepackage{amssymb}
\usepackage{amsthm}
\usepackage{booktabs}
\usepackage{mathtools}
\usepackage{xcolor}
\usepackage[separate-uncertainty=true, per-mode=symbol]{siunitx}
\usepackage[noabbrev, capitalize]{cleveref}
\usepackage{listings}
\usepackage[american inductor, european resistor]{circuitikz}
\usepackage{amsmath}
\usepackage{amsfonts}
\usepackage{ifxetex}
\usepackage[dutch,english]{babel}
\usepackage[backend=bibtexu,texencoding=utf8,bibencoding=utf8,style=ieee,sortlocale=en_GB,language=auto]{biblatex}
\usepackage[strict,autostyle]{csquotes}
\usepackage{parskip}
\usepackage{import}
\usepackage{standalone}
\usepackage{hyperref}
%\usepackage[toc,title,titletoc]{appendix}

\ifxetex{} % Fonts laden in het geval dat je met Xetex compiled
    \usepackage{fontspec}
    \defaultfontfeatures{Ligatures=TeX} % To support LaTeX quoting style
    \setromanfont{Palatino Linotype} % Tover ergens in Font mapje in root.
    \setmonofont{Source Code Pro}
\else % Terug val in standaard pdflatex tool chain. Geen ondersteuning voor OTT fonts
    \usepackage[T1]{fontenc}
    \usepackage[utf8]{inputenc}
\fi
\newcommand{\references}[1]{\begin{flushright}{#1}\end{flushright}}
\renewcommand{\vec}[1]{\boldsymbol{\mathbf{#1}}}
\newcommand{\uvec}[1]{\boldsymbol{\hat{\vec{#1}}}}
\newcommand{\mat}[1]{\boldsymbol{\mathbf{#1}}}
\newcommand{\fasor}[1]{\boldsymbol{\tilde{\vec{#1}}}}
\newcommand{\cmplx}[0]{\mathrm{j}}
\renewcommand{\Re}[0]{\operatorname{Re}}
\newcommand{\Cov}{\operatorname{Cov}}
\newcommand{\Var}{\operatorname{Var}}
\newcommand{\proj}{\operatorname{proj}}
\newcommand{\Perp}{\operatorname{perp}}
\newcommand{\col}{\operatorname{col}}
\newcommand{\rect}{\operatorname{rect}}
\newcommand{\sinc}{\operatorname{sinc}}
\newcommand{\IT}{\operatorname{IT}}
\newcommand{\F}{\mathcal{F}}

\newtheorem{definition}{Definition}
\newtheorem{theorem}{Theorem}


\DeclareSIUnit{\voltampere}{VA} %apparent power
\DeclareSIUnit{\pii}{\ensuremath{\pi}}

\hypersetup{%setup hyperlinks
    colorlinks,
    citecolor=black,
    filecolor=black,
    linkcolor=black,
    urlcolor=black
}

% Example boxes
\usepackage{fancybox}
\usepackage{framed}
\usepackage{adjustbox}
\newenvironment{simpages}%
{\AtBeginEnvironment{itemize}{\parskip=0pt\parsep=0pt\partopsep=0pt}
\def\FrameCommand{\fboxsep=.5\FrameSep\shadowbox}\MakeFramed{\FrameRestore}}%
{\endMakeFramed}

% Impulse train
\DeclareFontFamily{U}{wncy}{}
\DeclareFontShape{U}{wncy}{m}{n}{<->wncyr10}{}
\DeclareSymbolFont{mcy}{U}{wncy}{m}{n}
\DeclareMathSymbol{\Sha}{\mathord}{mcy}{"58}
\addbibresource{../../../includes/bibliography.bib}

\title{Compressive Sensing - An Overview}

\author{W.P. Bruinsma \and R.P. Hes \and H.J.C. Kroep \and T.C. Leliveld \and W.M. Melching \and T.A. aan de Wiel}

\raggedbottom

\begin{document}
\chapter{Preliminaries}
\section{Libraries}
\label{sec:libraries}
In this section we will discuss a number of used libraries and the advantages these have over other libraries/frameworks.

\subsection{Python}
\label{sec:python}
Initially, two platforms were considered for development: \matlab{} and Python. Taking our project preliminaries in account, Python provides the best option. Reasons for picking Python are:
\begin{itemize}
    \item The Python libraries \lib{NumPy} and \lib{SciPy} use the same linear algebra libraries\footnote{Depending on OS this can be one of the many implementations. A few popular are \lib{openBLAS}, \lib{ATLAS}, \lib{Accelerate} on Apple machines or Intel's proprietary \lib{MKL}.} that \matlab{} uses for the actual calculations, so the speed difference will be marginal.
    \item Python is a non-proprietary platform, meaning it can be used commercially in the final product without additional charge.
    \item Python provides many libraries that allow for distributed networking/computing \cite{pythonmp}.
    \item Python is a full-blown programming language, meaning it has libraries for non-numerical applications such as web-frameworks, that allow for more advanced visualisation options.
\end{itemize}
The Python version used is 2.7. We are forced to use Python 2 (instead of the newer Python 3) because \lib{GNU Radio} does not support Python 3. The original reference and used reference are respectively \cite{Rossum:1995:PRM:869369}\cite{python2}.

\subsection{GNU Radio}
\label{sec:gnu-radio}
\lib{GNU Radio} is a comprehensive library of signal processing tools. It offers a framework and a vast collection of digital signal processing functions. The initial version of our system was written in this framework. By building custom signal processing blocks we could extend the functionality of \lib{GNU Radio} with our own compressive sensing functions.

However, soon this became too slow and complex for our needs. In the final version of our system we only used the USRP block from the \lib{GNU Radio} library in one of our sources to receive samples from the USRP.


\subsection{Numpy}
\label{sec:numpy}
\lib{NumPy} is Python's standard numerical calculations library, providing a faster (multi-dimensional) array and high level bindings for a variety of linear algebra routines, amongst other. \lib{NumPy} is licensed under the new-BSD\footnoteref{fn:bsd} license. The reference manual is available at \cite{numpyscipy}.

\subsection{Scipy}
\label{sec:scipy}
\lib{SciPy} is Python's replacement for a lot of default functions that are included in \matlab{}. Relying on \lib{NumPy} data structures, it gives a variety of subpackages (such as \lib{signal}, their DSP-library) providing another means of quickly implementing advanced calculations. \lib{SciPy} is licensed under the new-BSD\footnote{\label{fn:bsd}The BSD 3-Clause license is free for commercial and non-commercial use for more info and the template see \cite{bsdlic}.} license. The reference manual is available at \cite{numpyscipy}.

\subsection{Flask}
\label{sec:flask}
\lib{Flask} is a micro web framework. The choice for a web interface will further be discussed in \cref{cha:view}. The advantage of \lib{Flask} over other web frameworks (such as \lib{Django}) is that it is a more bare-bones minimalistic (i.e. simple) approach that suits our needs. \lib{Flask} is licensed under the new-BSD\footnoteref{fn:bsd} license. The reference manual is available at \cite{flask}.

\subsection{Twisted}
\label{sub:twisted}
\lib{Twisted} is Python's event-driven network programming framework. It was used in conjunction with \lib{Autobahn}. \lib{Twisted} is licensed under the MIT\footnoteref{fn:mit} license. The reference is available at \cite{twisted}.

\subsection{Autobahn}
\label{sub:autobahn}
\lib{Autobahn} is a real-time framework for websockets. It was used as a means of sending data to the GUI. It is written on top of \lib{Twisted} and \lib{asyncio} and is licensed under the MIT\footnoteref{fn:mit} license. The reference is available at \cite{autobahn}.

\subsection{Pyro}
\label{sec:pyro}
\lib{Pyro} is a popular remote-object library. It allows us to do application programming over the network. In the future it could be an easy means of doing distributed networking but as of now it functions as a simple means of inter-process communication. \lib{Pyro} is licensed under the MIT\footnote{\label{fn:mit}Software under the MIT license is free for both commercial and non-commercial use. For more info see \cite{mitlic}.} license. The reference manual is available at \cite{pyro4}.

\subsection{Bootstrap}
\label{sec:bootstrap}
\lib{Bootstrap} is a front-end CSS\footnote{Cascading Style Sheets, a language for styling web pages.} and JavaScript framework. The framework was developed by Twitter. It implements a number of primitives, much like a GUI framework does on X11 (the display server used in many Linux distributions), allowing us to quickly implement a portable GUI. It suits our demands to be able to render both on mobile and desktop views, without much effort. \lib{Bootstrap} is licensed under the MIT\footnoteref{fn:mit} license. The reference manual is available at \cite{bootstrap}.

\subsection{Highcharts}
\label{sec:highcharts}
\lib{Highcharts} is a JavaScript plotting library capable of fulfilling all front-end plotting needs. This library was picked because of its customisability and ease-of-use. One major side note is that it is only free for non-commercial use. The Highcharts reference is available at \cite{highcharts}.

\section{Workflow}
\label{sec:workflow}

\subsection{Version Control}
\label{sec:version-control}
For version control git was used. Git is a distributed version control system enabling non-linear development in the form of branching. This allows us to maintain a stable master branch, which is used to maintain a stable version of our product, a development branch in which features can be combined and minor bugs can be fixed and several feature branches for larger feature implementations. This specific workflow with git is called Git Flow \cite{driessen2010successful}. We used a GitHub private repository to host our project on.

\subsection{Testing}
\label{sec:testing}
To verify the correctness of our code we used two kinds of testing: unit tests and integration tests.

\subsubsection{Unit tests}
Unit tests are small tests that test on a function level. They test if each function returns the right values for given input signals and tests for edge cases. These unit tests should be fast and be run often while writing code. This way, the correctness of one's code may be verified at all times. When the tests pass, the code is guaranteed to work according to the specifications. This becomes particulary important when working on a large project with multiple people. Any change in code can be verified quickly. Writing unit tests also forces the developer to think more about their code and stimulates modular design.

It is also possible to write the tests before the implementation of a function. This is called test driven development (TDD). In our design we decided not to use this, because the specifications were not entirely clear at the start of the project. In case of an abundance of additional time, TDD would definitely be a preferred approach.

\subsubsection{Integration tests}
Integration tests are tests that encompass a large part of the system. In our case we wrote tests where a couple of functions were tested at the same time. The output data was then compared by the output of a \matlab{} script that we knew was correct.

\subsubsection{Nose}
\label{ssub:Nose}
In addition to the standard Python testing library \lib{Unittest}, the library \lib{nose} was used. This library provides some additional test commands and asserts which allow us to set up a more extensive test framework.

\subsection{Profiling}
\label{sec:profiling}
For profiling the standard \lib{cProfile} library was used. This is included in the standard Python distribution and is the de facto standard for profiling in Python.

\end{document}
