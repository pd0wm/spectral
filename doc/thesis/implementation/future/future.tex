%!TEX program = xelatex

\documentclass[a4paper, openany, oneside]{memoir}
\usepackage[no-math]{fontspec}
\usepackage{pgfplots}
\pgfplotsset{compat=newest}
\usepackage{commath}
\usepackage{mathtools}
\usepackage{amssymb}
\usepackage{amsthm}
\usepackage{booktabs}
\usepackage{mathtools}
\usepackage{xcolor}
\usepackage[separate-uncertainty=true, per-mode=symbol]{siunitx}
\usepackage[noabbrev, capitalize]{cleveref}
\usepackage{listings}
\usepackage[american inductor, european resistor]{circuitikz}
\usepackage{amsmath}
\usepackage{amsfonts}
\usepackage{ifxetex}
\usepackage[dutch,english]{babel}
\usepackage[backend=bibtexu,texencoding=utf8,bibencoding=utf8,style=ieee,sortlocale=en_GB,language=auto]{biblatex}
\usepackage[strict,autostyle]{csquotes}
\usepackage{parskip}
\usepackage{import}
\usepackage{standalone}
\usepackage{hyperref}
%\usepackage[toc,title,titletoc]{appendix}

\ifxetex{} % Fonts laden in het geval dat je met Xetex compiled
    \usepackage{fontspec}
    \defaultfontfeatures{Ligatures=TeX} % To support LaTeX quoting style
    \setromanfont{Palatino Linotype} % Tover ergens in Font mapje in root.
    \setmonofont{Source Code Pro}
\else % Terug val in standaard pdflatex tool chain. Geen ondersteuning voor OTT fonts
    \usepackage[T1]{fontenc}
    \usepackage[utf8]{inputenc}
\fi
\newcommand{\references}[1]{\begin{flushright}{#1}\end{flushright}}
\renewcommand{\vec}[1]{\boldsymbol{\mathbf{#1}}}
\newcommand{\uvec}[1]{\boldsymbol{\hat{\vec{#1}}}}
\newcommand{\mat}[1]{\boldsymbol{\mathbf{#1}}}
\newcommand{\fasor}[1]{\boldsymbol{\tilde{\vec{#1}}}}
\newcommand{\cmplx}[0]{\mathrm{j}}
\renewcommand{\Re}[0]{\operatorname{Re}}
\newcommand{\Cov}{\operatorname{Cov}}
\newcommand{\Var}{\operatorname{Var}}
\newcommand{\proj}{\operatorname{proj}}
\newcommand{\Perp}{\operatorname{perp}}
\newcommand{\col}{\operatorname{col}}
\newcommand{\rect}{\operatorname{rect}}
\newcommand{\sinc}{\operatorname{sinc}}
\newcommand{\IT}{\operatorname{IT}}
\newcommand{\F}{\mathcal{F}}

\newtheorem{definition}{Definition}
\newtheorem{theorem}{Theorem}


\DeclareSIUnit{\voltampere}{VA} %apparent power
\DeclareSIUnit{\pii}{\ensuremath{\pi}}

\hypersetup{%setup hyperlinks
    colorlinks,
    citecolor=black,
    filecolor=black,
    linkcolor=black,
    urlcolor=black
}

% Example boxes
\usepackage{fancybox}
\usepackage{framed}
\usepackage{adjustbox}
\newenvironment{simpages}%
{\AtBeginEnvironment{itemize}{\parskip=0pt\parsep=0pt\partopsep=0pt}
\def\FrameCommand{\fboxsep=.5\FrameSep\shadowbox}\MakeFramed{\FrameRestore}}%
{\endMakeFramed}

% Impulse train
\DeclareFontFamily{U}{wncy}{}
\DeclareFontShape{U}{wncy}{m}{n}{<->wncyr10}{}
\DeclareSymbolFont{mcy}{U}{wncy}{m}{n}
\DeclareMathSymbol{\Sha}{\mathord}{mcy}{"58}
\addbibresource{../../../includes/bibliography.bib}

\title{Compressive Sensing - An Overview}

\author{W.P. Bruinsma \and R.P. Hes \and H.J.C. Kroep \and T.C. Leliveld \and W.M. Melching \and T.A. aan de Wiel}

\raggedbottom

\begin{document}
\chapter{Future Work}
\label{cha:futurework}
As of now we have build a flexible software framwork. We split the implementation in part hardware and part product implementations.

\section{Hardware Implementations}
\label{sec:hardware_implementations}
There are a number of configurations discussed in \cref{cha:sampling_methods} that can be applied with different advantages and disadvantages.

\subsection{Circular-sparse-ruler hardware implementation}
\label{sub:minimal_sparse_ruler_hardware_implementation}
A hardware implementation of samplers combined in a setup discussed in \cref{sub:ci-circ}. It allows for a high performance setup with optimal reconstruction resolution. This implementation is focussed on high-performance compressive spectrum sensing.

\subsection{Coprime hardware implementation}
\label{sub:coprime_hardwa}
A hardware implementation of two samplers in coprime setup discussed in \cref{sec:coprime}. It would involve two samplers syncing on a combined period whilst the individual sample frequencies of the samplers would have a coprime ratio. The use case would involve optimal performance for least cost. Coprime uses non optimal ``ruler'' but is implementable with two samplers only.

The software coprime truncater uses two input sample vectors, with a different number of samples from two samplers with different sample rates. The sampling matrix is constructed as with the normal coprime sampler. The two input vectors are then transformed into one output vector that matches the sampling matrix.

\subsection{Synchronisation of USRPs to implement coprime sampling}
For our implementation of co-prime sampling we needed to synchronise the sampling of two USRPs. This can be accomplished by connection them with a special MIMO cable. This allows them to internally synchronise their sampling clock. It even allows them to schedule commands at exactly the same time, so they can start sampling at exactly the same moment.

To verify that our synchronisation is correct, we dumped two frames from the same moment in time to a file. If we plot them on the same time scale, as in \cref{fig:plot_samplers}, we can see that they align almost perfectly. If we zoom in on the second burst in the signal, see \cref{fig:plot_samplers_zoom}, we can see the offset due to the local oscillator offset.

The only offset we have is from the phase offset between the two local oscillators of the USRPs. After each tune command the two USRPs have a random phase offset of \SIrange[retain-explicit-plus]{-180}{+180}{\degree}.  We don't know yet if that influences our results. If that is the case we can calculate this lag by looking at the cross-correlation between the two USRPs or by transmitting a reference pulse after each tune command and calculating the delay.

At this moment the synchronised sampling works, but something is going wrong in the processing or visualisation of the data. We did not have the time to find these problems and solve them. This is something for future work.

\begin{figure}[H]
\centering
\begin{tikzpicture}
\begin{axis}[
	xlabel={Samples},
	ylabel={Amplitude},
  ymin=-1e-2,
  ymax=1e-2,
  xmin=0,
  xmax=1e5,
	width=\textwidth, height=0.8\textwidth,
	xmajorgrids, xminorgrids, ymajorgrids,
  name=sampler_a
	]

%% Set the plot options, and load a csv formatted file %%
\addplot [
	color=black,
	solid,
	mark=.,
	]
	table [col sep=comma]{plots/coprime_orig_a.csv};
\end{axis}

\begin{axis}[
	xlabel={Samples},
	ylabel={Amplitude},
  ymin=-1e-2,
  ymax=1e-2,
  xmin=0,
  xmax=4e4,
	width=\textwidth, height=0.8\textwidth,
	xmajorgrids, xminorgrids, ymajorgrids,
  at=(sampler_a.below south), anchor=above north
	]

%% Set the plot options, and load a csv formatted file %%
\addplot [
	color=black,
	solid,
	mark=.,
	]
	table [col sep=comma]{plots/coprime_orig_b.csv};
\end{axis}
\end{tikzpicture}
\caption{Output of the two samplers}
\label{fig:plot_samplers}
\end{figure}

\begin{figure}[H]
\centering
\begin{tikzpicture}
\begin{axis}[
	xlabel={Samples},
	ylabel={Amplitude},
  ymin=-1e-2,
  ymax=1e-2,
  xmin=0,
  xmax=100,
	width=\textwidth, height=0.8\textwidth,
	xmajorgrids, xminorgrids, ymajorgrids,
	]

%% Set the plot options, and load a csv formatted file %%
\addplot [
	color=blue,
	solid,
	mark=.,
	]
	table [col sep=comma]{plots/coprime_resamp_a.csv};

%% Set the plot options, and load a csv formatted file %%
\addplot [
	color=red,
	solid,
	mark=.,
	]
	table [col sep=comma]{plots/coprime_resamp_b.csv};
\end{axis}
\end{tikzpicture}
\caption{Zoomed in version of the start of the second burst}
\label{fig:plot_samplers_zoom}
\end{figure}


\section{Product Implementations}
\label{sec:product_implementations}
% After our visit to the Ministry of Defence to present our technology, we learned about a number of possible implementations. Besides this the discussions we had with Koen Bertels during the business plan part of the bachelor graduation project were very constructive in finding possible applications.

\subsection{Distributed network}
\label{sec:distributed}
In the current implementation a lot of multiprocessing is done as discussed in \cref{cha:presenter}. A variation on this is to run a distributed network of nodes all reconstructing part of or the entire spectrum and recombining this on the views. This distributes the computing power amongst several nodes while the views  have to do very little processing. Because there are several libraries available (such as \lib{Pyro4}) which allow to do this kind of computing on a standard LAN network existing networks can be used for this purpose.

% A possible product for this could be aboard of naval vessels. In our conversation with the Ministry of Defence a need for high speed spectrum sensers was voiced. It could be used for boarding operations where they want to map the communication of the opponent.

\subsection{Mobile platform}
\label{sub:mobile_platform}
Our system could also be used in a distributed system that uses mobile devices (e.g. mobile phones) for collaborative spectrum sensing. This enables mobile network providers to keep track of a database of the current spectrum usage. This could then be used for dynamic spectrum allocation where devices can communicate on the unused part of the licensed spectrum if the primary user (i.e. the user who paid for the license) is not using the specific frequency.



\end{document}
