%!TEX program = xelatex

\documentclass[a4paper, openany, oneside]{memoir}
\usepackage[no-math]{fontspec}
\usepackage{pgfplots}
\pgfplotsset{compat=newest}
\usepackage{commath}
\usepackage{mathtools}
\usepackage{amssymb}
\usepackage{amsthm}
\usepackage{booktabs}
\usepackage{mathtools}
\usepackage{xcolor}
\usepackage[separate-uncertainty=true, per-mode=symbol]{siunitx}
\usepackage[noabbrev, capitalize]{cleveref}
\usepackage{listings}
\usepackage[american inductor, european resistor]{circuitikz}
\usepackage{amsmath}
\usepackage{amsfonts}
\usepackage{ifxetex}
\usepackage[dutch,english]{babel}
\usepackage[backend=bibtexu,texencoding=utf8,bibencoding=utf8,style=ieee,sortlocale=en_GB,language=auto]{biblatex}
\usepackage[strict,autostyle]{csquotes}
\usepackage{parskip}
\usepackage{import}
\usepackage{standalone}
\usepackage{hyperref}
%\usepackage[toc,title,titletoc]{appendix}

\ifxetex{} % Fonts laden in het geval dat je met Xetex compiled
    \usepackage{fontspec}
    \defaultfontfeatures{Ligatures=TeX} % To support LaTeX quoting style
    \setromanfont{Palatino Linotype} % Tover ergens in Font mapje in root.
    \setmonofont{Source Code Pro}
\else % Terug val in standaard pdflatex tool chain. Geen ondersteuning voor OTT fonts
    \usepackage[T1]{fontenc}
    \usepackage[utf8]{inputenc}
\fi
\newcommand{\references}[1]{\begin{flushright}{#1}\end{flushright}}
\renewcommand{\vec}[1]{\boldsymbol{\mathbf{#1}}}
\newcommand{\uvec}[1]{\boldsymbol{\hat{\vec{#1}}}}
\newcommand{\mat}[1]{\boldsymbol{\mathbf{#1}}}
\newcommand{\fasor}[1]{\boldsymbol{\tilde{\vec{#1}}}}
\newcommand{\cmplx}[0]{\mathrm{j}}
\renewcommand{\Re}[0]{\operatorname{Re}}
\newcommand{\Cov}{\operatorname{Cov}}
\newcommand{\Var}{\operatorname{Var}}
\newcommand{\proj}{\operatorname{proj}}
\newcommand{\Perp}{\operatorname{perp}}
\newcommand{\col}{\operatorname{col}}
\newcommand{\rect}{\operatorname{rect}}
\newcommand{\sinc}{\operatorname{sinc}}
\newcommand{\IT}{\operatorname{IT}}
\newcommand{\F}{\mathcal{F}}

\newtheorem{definition}{Definition}
\newtheorem{theorem}{Theorem}


\DeclareSIUnit{\voltampere}{VA} %apparent power
\DeclareSIUnit{\pii}{\ensuremath{\pi}}

\hypersetup{%setup hyperlinks
    colorlinks,
    citecolor=black,
    filecolor=black,
    linkcolor=black,
    urlcolor=black
}

% Example boxes
\usepackage{fancybox}
\usepackage{framed}
\usepackage{adjustbox}
\newenvironment{simpages}%
{\AtBeginEnvironment{itemize}{\parskip=0pt\parsep=0pt\partopsep=0pt}
\def\FrameCommand{\fboxsep=.5\FrameSep\shadowbox}\MakeFramed{\FrameRestore}}%
{\endMakeFramed}

% Impulse train
\DeclareFontFamily{U}{wncy}{}
\DeclareFontShape{U}{wncy}{m}{n}{<->wncyr10}{}
\DeclareSymbolFont{mcy}{U}{wncy}{m}{n}
\DeclareMathSymbol{\Sha}{\mathord}{mcy}{"58}
\addbibresource{../../../includes/bibliography.bib}

\title{Compressive Sensing - An Overview}

\author{W.P. Bruinsma \and R.P. Hes \and H.J.C. Kroep \and T.C. Leliveld \and W.M. Melching \and T.A. aan de Wiel}

\raggedbottom

\begin{document}
\chapter{Future Work}
As of now we have build a flexible software framwork. We split the implementation in part hardware and part product implementations.

\section{Hardware Implementations}
\label{sec:hardware_implementations}
There are a number of configurations discussed in \cref{cha:sampling_methods} that can be applied with different advantages and disadvantages.

\subsection{Circular-sparse-ruler hardware implementation}
\label{sub:minimal_sparse_ruler_hardware_implementation}
A hardware implementation of samplers combined in a setup discussed in \cref{sub:ci-circ}. It allows for a high performance setup with optimal reconstruction resolution. This implementation is focussed on high-performance compressive spectrum sensing.

\subsection{Coprime hardware implementation}
\label{sub:coprime_hardwa}
A hardware implementation of two samplers in coprime setup discussed in \cref{sec:coprime}. It would involve two samplers syncing on a combined period whilst the individual sample frequencies of the samplers would have a coprime ratio. The use case would involve optimal performance for least cost. Coprime uses non optimal ``ruler'' but is implementable with two samplers.

\section{Product Implementations}
\label{sec:product_implementations}
After our visit to the Ministry of Defence to present our technology, we learned about a number of possible implementations. Besides this the discussions we had with Koen Bertels during the business plan part of the bachelor graduation project were very constructive in finding possible applications.

\subsection{Distributed network}
\label{sec:distributed}
In the current implementation a lot of multiprocessing is done as discussed in \cref{cha:presenter}. A variation on this is to run a distributed network of nodes all reconstructing part of or the entire spectrum and recombining this on the views. This distributes the computing power amongst several nodes while the views  have to do very little processing. Because there are several libraries available (such as \lib{Pyro4}) which allow to do this kind of computing on a standard LAN network existing networks can be used for this purpose.

A possible product for this could be aboard of naval vessels. In our conversation with the Ministry of Defence a need for high speed spectrum sensers was voiced. It could be used for boarding operations where they want to map the communication of the opponent.

\subsection{Mobile platform}
\label{sub:mobile_platform}
Our system could also be used in a distributed system that uses mobile devices (e.g. mobile phones) for collaborative spectrum sensing. This enables mobile network providers to keep track of a database of the current spectrum usage. This could then be used for dynamic spectrum allocation where devices can communicate on the unused part of the licensed spectrum if the primary user (i.e. the user who paid for the license) is not using the specific frequency.



\end{document}
