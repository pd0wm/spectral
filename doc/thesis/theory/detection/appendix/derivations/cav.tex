%!TEX program = xelatex

\documentclass[a4paper, openany, oneside]{memoir}
\usepackage[no-math]{fontspec}
\usepackage{pgfplots}
\pgfplotsset{compat=newest}
\usepackage{commath}
\usepackage{mathtools}
\usepackage{amssymb}
\usepackage{amsthm}
\usepackage{booktabs}
\usepackage{mathtools}
\usepackage{xcolor}
\usepackage[separate-uncertainty=true, per-mode=symbol]{siunitx}
\usepackage[noabbrev, capitalize]{cleveref}
\usepackage{listings}
\usepackage[american inductor, european resistor]{circuitikz}
\usepackage{amsmath}
\usepackage{amsfonts}
\usepackage{ifxetex}
\usepackage[dutch,english]{babel}
\usepackage[backend=bibtexu,texencoding=utf8,bibencoding=utf8,style=ieee,sortlocale=en_GB,language=auto]{biblatex}
\usepackage[strict,autostyle]{csquotes}
\usepackage{parskip}
\usepackage{import}
\usepackage{standalone}
\usepackage{hyperref}
%\usepackage[toc,title,titletoc]{appendix}

\ifxetex{} % Fonts laden in het geval dat je met Xetex compiled
    \usepackage{fontspec}
    \defaultfontfeatures{Ligatures=TeX} % To support LaTeX quoting style
    \setromanfont{Palatino Linotype} % Tover ergens in Font mapje in root.
    \setmonofont{Source Code Pro}
\else % Terug val in standaard pdflatex tool chain. Geen ondersteuning voor OTT fonts
    \usepackage[T1]{fontenc}
    \usepackage[utf8]{inputenc}
\fi
\newcommand{\references}[1]{\begin{flushright}{#1}\end{flushright}}
\renewcommand{\vec}[1]{\boldsymbol{\mathbf{#1}}}
\newcommand{\uvec}[1]{\boldsymbol{\hat{\vec{#1}}}}
\newcommand{\mat}[1]{\boldsymbol{\mathbf{#1}}}
\newcommand{\fasor}[1]{\boldsymbol{\tilde{\vec{#1}}}}
\newcommand{\cmplx}[0]{\mathrm{j}}
\renewcommand{\Re}[0]{\operatorname{Re}}
\newcommand{\Cov}{\operatorname{Cov}}
\newcommand{\Var}{\operatorname{Var}}
\newcommand{\proj}{\operatorname{proj}}
\newcommand{\Perp}{\operatorname{perp}}
\newcommand{\col}{\operatorname{col}}
\newcommand{\rect}{\operatorname{rect}}
\newcommand{\sinc}{\operatorname{sinc}}
\newcommand{\IT}{\operatorname{IT}}
\newcommand{\F}{\mathcal{F}}

\newtheorem{definition}{Definition}
\newtheorem{theorem}{Theorem}


\DeclareSIUnit{\voltampere}{VA} %apparent power
\DeclareSIUnit{\pii}{\ensuremath{\pi}}

\hypersetup{%setup hyperlinks
    colorlinks,
    citecolor=black,
    filecolor=black,
    linkcolor=black,
    urlcolor=black
}

% Example boxes
\usepackage{fancybox}
\usepackage{framed}
\usepackage{adjustbox}
\newenvironment{simpages}%
{\AtBeginEnvironment{itemize}{\parskip=0pt\parsep=0pt\partopsep=0pt}
\def\FrameCommand{\fboxsep=.5\FrameSep\shadowbox}\MakeFramed{\FrameRestore}}%
{\endMakeFramed}

% Impulse train
\DeclareFontFamily{U}{wncy}{}
\DeclareFontShape{U}{wncy}{m}{n}{<->wncyr10}{}
\DeclareSymbolFont{mcy}{U}{wncy}{m}{n}
\DeclareMathSymbol{\Sha}{\mathord}{mcy}{"58}
\addbibresource{../../../../../includes/bibliography.bib}

\begin{document}

\section{Covariance Absolute Value detection}\label{sec:cav_derivation}

\subsection{Explanation}
Let $L$ samples of the signal $x[n]$ be collected in the vector $\vec{x} = \left[x[n], x[n+1], \ldots, x[n+L]\right]^T$. 
Furthermore let $\mat{C} = E\left(\left(\vec{x} - \mu \right)\left(\vec{x} - \mu \right)^H\right)$ denote the covariance of $\vec{x}$ with $\mu = E(\vec{x})$.

In the case that $E\left(\vec{x}\right)=0$, like for noise or most communication signals, then $\mat{C}$ can be simplified to

\begin{align*}
\mat{C}_x &= E\left(\vec{x}\vec{x}^T\right) \\
&= \begin{bmatrix} 
E\left(x[n][n]\right) & E\left(x[n][n+1]\right) & \ldots & E\left(x[n][n+L-1]\right) \\
E\left(x[n+1][n]\right) & E\left(x[n+1][n+1]\right) & \ldots & E\left(x[n+1][n+L-1]\right) \\
\vdots & \vdots & \ddots & \vdots \\
E\left(x[n+L-1][n]\right) & E\left(x[n+L-1][n+1]\right) & \ldots & E\left(x[n+L-1][n+L-1]\right) \\
\end{bmatrix}.
\end{align*}
Under the assumption that $x[n]$ is a wide-sense stationary signal, we can simplify $\mat{C}$ even further:
\begin{align*}
\mat{C}&= E\left[\vec{x}\vec{x}^T\right] \\
&= \begin{bmatrix} 
r_x[0] & r_x[1] & \ldots & r_x[L-1] \\
r_x[1] & r_x[0] & \ldots & r_x[L-2] \\
\vdots & \vdots & \ddots & \vdots \\
r_x[L-1] & r_x[L-2] & \ldots & r_x[0] \\
\end{bmatrix}.
\end{align*}
Note how $\mat{C}$ is symmetric and Toeplitz: this is the first block in \cref{tkz:cav}. As  the autocorrelation function of white noise is a delta function, it means that  if $x[n]$ is white noise with variance $\sigma_n^2$ then $\mat{C}_x = \sigma_n^2\mat{I}$.
If the signal $x[n]$ is not equal to noise, then its autocorrelation function is not equal to a delta function which results in $\mat{C}$ having non-zero off diagonal elements.

Covariance absolute value method detection uses a measure of this ``diagonality'' of $\mat{C}$ as test statistic $\Lambda$.
This measure $\Lambda$ is defined as
\begin{align}\label{eq:cav_statistic}
\Lambda &= \frac{T_1}{T_2} \nonumber \\
&=\frac{\sum_{n=1}^{L} \sum_{m=1}^L \left|\mat{C}_{nm}\right|}{\sum_{k=1}^L |\mat{C}_{kk}}
\end{align} 

with $T_1 = \frac{\sum_{n=1}^{L} \sum_{m=1}^L \left|\mat{C}_{nm}\right|}{L}$ and
$T_2 = \frac{\sum_{k=1}^L |\mat{C}_{kk}}{L}$.
This test statistic can be computed by first taking the absolute value of $\mat{C}$. This is then followed by summing all the elements of the resulting matrix (forming the numerator in \cref{eq:cav_statistic}) and computing the trace (the denominator in \cref{tkz:conv_ed}) of that matrix. Upon diving those two results one obtains the test statistic $\Lambda$. This process is depicted in \cref{tkz:cav}.

In practice one estimates the matrix $\mat{C}$ by using a limited amount of samples $N$ to estimate $r_x[n]$. The threshold given a desired false alarm probability
$p_{fa}$ is derived in \cite{zheng2009spectrum} to be

\begin{align*}
\gamma &= \frac{\left(1+(L-1)\sqrt{\frac{2}{N\pi}}\right)}{1-Q^{-1}(p_{fa})\sqrt{\frac{2}{N}}}.
\end{align*} 

\subsection{Adjusting the threshold}\label{eq:threshold_cav_deriv}
In \cite{zheng2009spectrum} it is assumed that for large $N$ the distribution of $T_1$ and $T_2$ approach Gaussian distributions. Given a fixed false alarm probability, the threshold $\gamma$ is derived under the hypothesis $\mathcal{H}_0$. The following should hold

\begin{align*}
p_{fa} &\approx P\left(\frac{T_1}{T_2} > \gamma \big | \mathcal{H}_0\right) \\
&= P\left(T_2 < \frac{T_1}{\gamma} \big | \mathcal{H}_0\right)
\end{align*}

Under $\mathcal{H}_0$, the distribution of $T_2 = r_x[0]$ is approximated by a Gaussian. We can therefore derive the
threshold $\gamma$ by noting that.

\begin{align*}
p_{fa} &= P\left(\frac{\frac{E\left(T_1\right)}{\gamma} - E\left(T_2\right) }{\sqrt{\text{Var}(T_2)}}\right)\\
&= 1-Q\left(\frac{\frac{E\left(T_1\right)}{\gamma} - E\left(T_2\right) }{\sqrt{\text{Var}(T_2)}}\right) 
\end{align*}

To solve for $\gamma$ we have to determine $E\left(T_1\right)$, $E\left(T_2\right)$ and $\Var(T_2)$. It is at this point that we cannot use the expressions from \cite{zheng2009spectrum} as they assume that $r_x[m]$ is estimated by the sample autocorrelation function. 

Let $\vec{\hat{r}_x} = \left[\hat{r}_x[-LN], \hat{r}_x[-LN+1] , \ldots, \hat{r}_x[LN]\right]$ denote the vector containing the, by the reconstructor, estimated elements of $r_x[m]$. Then 

\begin{align*}
E(\vec{\hat{r}}_x) &= \mat{R}^{\dagger}E(\vec{\hat{r}}_y)
\end{align*}

contains $r_x[m]$ for $-LN \leq m \leq LN$. Notice that this vector contains $r_x[0] = E(T_2)$ and that it can be used to construct $E(\mat{C})$. Therefore it can also be used to calculate $E(T_1)$.

To obtain $\Var(r_x[0])$ we notice that the $\Var(r_x[m])$ is contained (for $-LN \leq m \leq LN$) on the diagonal of the covariance matrix of $\vec{\hat{r}_x}$, which is denoted by $\mat{C}_{\hat{r}_x}$.

This covariance matrix is equal to
\begin{align*}
\mat{C}_{\hat{r}_x} &= E(\vec{\hat{r}}_x\vec{\hat{r}}_x^T) - E(\vec{\hat{r}}_x) E(\vec{\hat{r}}_x^T)\\
&= R^{\dagger}C_{\hat{r}_y}(R^{\dagger})^H.
\end{align*}

The elements of the covariance matrix $C_{\hat{r}_y}$ are given by \cref{eq:elem_cov_ry}.

The threshold $\gamma$ can then be calculated as

\begin{align*}
\gamma &= \frac{E(T_1)}{Q^{-1}(1-p_{fa})\sqrt{\Var(T_2)}+E(T_2)}
\end{align*}

\end{document}