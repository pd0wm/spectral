%!TEX program = xelatex

\documentclass[a4paper, openany, oneside]{memoir}
\usepackage[no-math]{fontspec}
\usepackage{pgfplots}
\usepackage{float}
\pgfplotsset{compat=newest}
\usepackage{commath}
\usepackage{mathtools}
\usepackage{amssymb}
\usepackage{amsthm}
\usepackage{booktabs}
\usepackage{todonotes}
\usepackage{mathtools}
\usepackage{xcolor}
\usepackage[separate-uncertainty=true, per-mode=symbol]{siunitx}
\usepackage{listings}
\usepackage[american inductor, european resistor]{circuitikz}
\usepackage{amsmath}
\usepackage{amsfonts}
\usepackage{ifxetex}
\usepackage[dutch,english]{babel}
\usepackage[backend=bibtexu,texencoding=utf8,bibencoding=utf8,style=ieee,sortlocale=en_GB,language=auto]{biblatex}
\usepackage[strict,autostyle]{csquotes}
\usepackage{import}
\usepackage{standalone}
\usepackage{bookmark,hyperref}
\usepackage{xcolor,mdframed}
\usepackage{tikz}
\usepackage{framed}
\usepackage{float}
\usepackage{tabularx}
\usepackage{graphicx,adjustbox}
\usepackage{rotating}
\usepackage{pdfpages}
\usepackage{enumitem}
\usepackage{calc}
\usepackage{pgfplots}
\usepackage{filecontents}
\usepackage{caption}
\usepackage{subcaption}
\usepackage{lettrine}

\newcolumntype{Y}{>{\raggedright\arraybackslash}X} % Left-justified text in tabularx environment

\ifxetex{} % Fonts laden in het geval dat je met Xetex compiled
    \usepackage{fontspec}
    \defaultfontfeatures{Scale=MatchLowercase, Ligatures=TeX} % To support LaTeX quoting style
    %\setromanfont{Palatino Linotype} % Tover ergens in Font mapje in root.
    \setsansfont{Avenir Next LT Pro}
    \setromanfont{Adobe Caslon Pro} % Tover ergens in Font mapje in root.
    \setmonofont{Source Code Pro}
\else % Terug val in standaard pdflatex tool chain. Geen ondersteuning voor OTT fonts
    \usepackage[T1]{fontenc}
    \usepackage[utf8]{inputenc}
\fi
\usepackage[noabbrev, capitalize]{cleveref}
\usepackage{ifthen}
\usepackage{titlesec}
\usepackage{titlecaps}

\newcommand{\references}[1]{\begin{flushright}{#1}\end{flushright}}
\renewcommand{\vec}[1]{\boldsymbol{\mathbf{#1}}}
\newcommand{\uvec}[1]{\boldsymbol{\hat{\vec{#1}}}}
\newcommand{\mat}[1]{\boldsymbol{\mathbf{#1}}}
\newcommand{\fasor}[1]{\boldsymbol{\tilde{\vec{#1}}}}
\newcommand{\cmplx}[0]{\mathrm{j}}
\renewcommand{\Re}[0]{\operatorname{Re}}
\newcommand{\Cov}{\operatorname{Cov}}
\newcommand{\Var}{\operatorname{Var}}
\newcommand{\proj}{\operatorname{proj}}
\newcommand{\Perp}{\operatorname{perp}}
\newcommand{\col}{\operatorname{col}}
\newcommand{\rect}{\operatorname{rect}}
\newcommand{\sinc}{\operatorname{sinc}}
\newcommand{\lcm}{\operatorname{lcm}}
%\newcommand{\gcd}{\operatorname{gcd}}
\newcommand{\F}{\mathcal{F}}
\newcommand{\DTFT}{\mathcal{F}_*}
\newcommand{\conj}[1]{#1^*}
\renewcommand{\mod}{\operatorname{mod}}
\newcommand{\rot}{\operatorname{rot}}
\newcommand{\vecsc}[1]{\vec{\textsc{\textbf{#1}}}}
\renewcommand{\ss}[1]{_{#1}}

% Label without linebreak breaker
\newcommand{\lab}[1]{\label{#1}\nolinebreak}

\newtheorem{definition}{Definition}
\newtheorem{theorem}{Theorem}


\DeclareSIUnit{\voltampere}{VA} %apparent power
\DeclareSIUnit{\pii}{\ensuremath{\pi}}

\hypersetup{%setup hyperlinks
    colorlinks,
    citecolor=black,
    filecolor=black,
    linkcolor=black,
    urlcolor=black
}

% Example boxes
\usepackage{fancybox}
\usepackage{framed}
\usepackage{adjustbox}
\newenvironment{simpages}%
{\AtBeginEnvironment{itemize}{\parskip=0pt\parsep=0pt\partopsep=0pt}
\def\FrameCommand{\fboxsep=.5\FrameSep\shadowbox}\MakeFramed{\FrameRestore}}%
{\endMakeFramed}

% Impulse train
\DeclareFontFamily{U}{wncy}{}
\DeclareFontShape{U}{wncy}{m}{n}{<->wncyr10}{}
\DeclareSymbolFont{mcy}{U}{wncy}{m}{n}
\DeclareMathSymbol{\Sha}{\mathord}{mcy}{"58}

\setlength{\parindent}{0pt}
\nonzeroparskip

% Block environment configuration
\newcommand{\BlockLeftMargin}{20pt}
\newcommand{\BlockLeftMarginText}{25pt}
\newcommand{\BlockLeftMarginTextSpacing}{10pt}

% Own colours
\definecolor{gray75}{gray}{0.75}

% Block environment
\newenvironment{block}[3]{%
\makebox{\hspace{-\spinemargin}%
\begin{tikzpicture}[overlay]
    \draw [thick,color=gray75] (\BlockLeftMargin, 0) -- (\paperwidth - \spinemargin, 0);
    \node at (\BlockLeftMarginText, -0.9) [align=left, text width=\spinemargin - \BlockLeftMarginText - \BlockLeftMarginTextSpacing, anchor=west, text depth=1cm] {\textbf{\textsc{#1}}\newline\textit{#3}};
\end{tikzpicture}}%
\nopagebreak\\[0.25em]\ifthenelse{\equal{#2}{}}{}{(\textit{#2}.) }\nopagebreak\nolinebreak}
{\nopagebreak\\[-0.25em]%
\makebox{\hspace{-\spinemargin}%
\begin{tikzpicture}[overlay, remember picture]
    \draw [thick,color=gray75] (\spinemargin,0) -- (\paperwidth - \spinemargin,0);
\end{tikzpicture}} \vspace{0.5em}}

% Theorem
\newcounter{blockTheoremCounter}
\crefname{blockTheoremCounter}{Theorem}{Theorems}
\Crefname{blockTheoremCounter}{Theorem}{Theorems}

\newenvironment{blockTheorem}[1][]{%
\refstepcounter{blockTheoremCounter}%
\begin{block}{theorem \theblockTheoremCounter}{#1}{}}
{\end{block}}

% Definition
\newcounter{blockDefinitionCounter}
\crefname{blockDefinitionCounter}{Definition}{Definitions}
\Crefname{blockDefinitionCounter}{Definition}{Definitions}

\newenvironment{blockDefinition}[1][]{%
\refstepcounter{blockDefinitionCounter}%
\begin{block}{definition \theblockDefinitionCounter}{#1}{}}
{\end{block}}

% Proof
\newcounter{blockProofTheoremCounter}
\crefname{blockProofTheoremCounter}{Proof}{Proofs}
\Crefname{blockProofTheoremCounter}{Proof}{Proofs}

\newenvironment{blockProofTheorem}[1]{%
\refstepcounter{blockProofTheoremCounter}%
\begin{block}{proof of \\ theorem #1}{}{}}
{\qed\end{block}}

% Detail
\newcounter{blockDetailCounter}
\crefname{blockDetailCounter}{Detail}{Details}
\Crefname{blockDetailCounter}{Detail}{Details}

\newenvironment{blockDetail}[1][]{%
\refstepcounter{blockDetailCounter}%
\begin{block}{detail \theblockDetailCounter}{#1}{}}
{\end{block}}

% Redesign chapter headings
\newcommand{\chapternumber}{\thechapter}
\newcommand{\hsp}{\hspace{20pt}}
\titleformat{\chapter}[hang]{\Huge\bfseries}{\chapternumber\hsp\textcolor{gray75}{|}\hsp}{0pt}{\Huge\bfseries}

% Remove headers
% \addtopsmarks{headings}{}{
%   \createmark{chapter}{left}{nonumber}{}{}
% }
% \pagestyle{headings} % Activate changes

% Capitalise headers in a regular way
\renewcommand*{\memUChead}[1]{\titlecap{#1}}

% \hfill for math mode
\newcommand{\pushright}[1]{\intertext{\hfill$\displaystyle #1$}}
\newcommand{\pushline}{\hskip \textwidth minus \textwidth}
\newcommand{\matlab}{\textsc{Matlab}}

\definecolor{code-grey}{HTML}{DDDDDD}
\newcommand{\lib}[1]{\textsf{#1}}
\newcommand{\file}[1]{\textsf{#1}}
\newcommand{\func}[1]{\colorbox{code-grey}{\texttt{#1}}}
\newcommand{\class}[1]{\colorbox{code-grey}{\texttt{#1}}}

% Setup actiepunten
\newenvironment{important}[1][]{%
   \begin{mdframed}[%
      backgroundcolor={red!15}, hidealllines=true,
      skipabove=0.7\baselineskip, skipbelow=0.7\baselineskip,
      splitbottomskip=2pt, splittopskip=4pt, #1]%
   \makebox[0pt]{% ignore the withd of !
      \smash{% ignor the height of !
         \fontsize{32pt}{32pt}\selectfont% make the ! bigger
         \hspace*{-19pt}% move ! to the left
         \raisebox{-2pt}{% move ! up a little
            {\color{red!70!black}\sffamily\bfseries !}% type the bold red !
         }%
      }%
   }%
}{\end{mdframed}}
\newcommand{\excl}[1]{
\begin{important}
  \textbf{#1}
\end{important}
}

\makeatletter
\newcommand\footnoteref[1]{\protected@xdef\@thefnmark{\ref{#1}}\@footnotemark}
\makeatother

% Allow page breaks in display environments
%\allowdisplaybreaks
% S unit for use in Mega Samples per second
\DeclareSIUnit\sample{S}

\newcommand{\CC}{C\nolinebreak\hspace{-.05em}\raisebox{.3ex}{ \textbf{+}}\nolinebreak\hspace{-.10em}\raisebox{.3ex}{\textbf{+}}}
\def\CC{{C\nolinebreak[4]\hspace{-.05em}\raisebox{.3ex}{\textbf{++}}}}


\newcommand{\partauthor}[1]{\gdef\@partauthor{#1}}
\renewcommand{\printparttitle}[1]{
  \parttitlefont #1\\
  \vspace{1.5cm}
  \textnormal{\Large \@partauthor}
}
\addbibresource{../../../../includes/bibliography.bib}

\begin{document}

% http://ieeexplore.ieee.org/stamp/stamp.jsp?tp=&arnumber=6068200
\section{Energy detector}

\subsection{Conventional Energy detector}\label{ssec:conv_ed_derivation}
This section will give a derivation of the conventional energy detector based on ???.
Let $x[n]$ denote the received signal. The conventional energy detection algorithm must decide between to hypotheses:
\begin{align}\label{eq:hypotheses}
  \mathcal{H}_0&: x[n] = w[n]\\
  \mathcal{H}_1&: x[n] = s[n] + w[n]
\end{align}
in which $w[n]$ denotes additive circular complex Gaussian noise and $s[n]$ denotes a signal as transmitted by a primary user.

We assume that the noise samples are i.i.d. zero mean circular complex Gaussian distributed; $w[n] \sim \mathcal{CN}(0, \sigma_n^2)$. 
We furthermore assume that the samples of the signal $s[n]$ can be modelled independently as circular complex Gaussian $\mathcal{CN}(0, \sigma_s^2)$. % todo ref
% dat dit aannemelijk maakt That is

\begin{align*}
x[n] \sim 
    \begin{cases}
        \mathcal{CN}(0, \sigma_n^2) & \text{under $\mathcal{H}_0$} \\
        \mathcal{CN}(0, (\sigma_s^2 + \sigma_n^2)) & \text{under $\mathcal{H}_1$}
    \end{cases}
\end{align*} 

Let $\vec{x} = \left[x[0], x[1], \ldots, x[N-1]\right]^T$ denote a vector containing $N$ samples of the signal $x$. Then the likelihood function of $\vec{x}$ denotes as $L(\vec{x})$ is given by:

\begin{align*}
    L(\vec{x}) &= \prod_{i=1}^N f_{(\vec{x})_i}\\
    &= \begin{cases}
        \frac{\sigma_n^2}{\pi^N } \exp(-\bar{\vec{x}}'\sigma_n^2\mathbf{I}\vec{x}) & \text{under $\mathcal{H}_0$} \\
        \frac{\sigma_n^2}{\pi^N } \exp(-\bar{\vec{x}}'(\sigma_n^2+\sigma_s^2)\mathbf{I}\vec{x}) & \text{under $\mathcal{H}_1$}
    \end{cases}
\end{align*}

Where $f_{(\vec{x})_i}$ denotes the probability density function of element $i$ in $\vec{x}$.  The test statistic $\Lambda(\vec{x})$ as used in the Neyman Pearson test is then given by:
\begin{align*}
\Lambda(\vec{x}) &=\frac{\frac{\sigma_n^2}{\pi^N } \exp(-\bar{\vec{x}}'\sigma_n^2\mathbf{I}\vec{x})}{\frac{(\sigma_n^2 + \sigma_s^2)}{\pi^N } \exp(-\bar{\vec{x}}'(\sigma_n^2+\sigma_s^2)\mathbf{I}\vec{x})}
\end{align*}


By taking the logarithm of $\Lambda(x)$ we obtain a Log Likelihood Ratio test statistic $\Lambda'(x)$, given by

\begin{align*}
\Lambda(\vec{x})' &= \log \left(
\frac{\sigma_n^2\exp(-\overline{\vec{x}}'\sigma_n^2\mathbf{I}\vec{x})}{(\sigma_n^2 + \sigma_s^2)\exp(-\bar{\vec{x}}'(\sigma_n^2+\sigma_s^2)\mathbf{I}\vec{x})}\right) \\
&= \log\left(\sigma_n^2\right) - \log\left(\sigma_n^2 + \sigma_s^2\right) +  (\sigma_s^2) \sum_{i=0}^{N-1} |x[i]|^2. 
\end{align*}

Observing that the constants $\sigma_n$ and $\sigma_s$ do not depend on value of the samples, the test statistic 

\begin{align*}
\Lambda''(\vec{x}) &= \sum_{n=0}^{N-1} |x[n]|^2
\end{align*} 

is proportional to $\Lambda'(x)$. Note that $\Lambda''$ is the same test statistic as $\Lambda$ in \Cref{sec:conv_ed}.

\subsubsection{Threshold}
In this section we will determine the threshold $\gamma$, for the energy detector test statistic $\Lambda''(\vec{x})$.
Under $\mathcal{H}_0$ we have that $\Lambda''(\vec{x})$ is the sum of $2N$ independent zero mean Gaussian distributed variables
with variance $\frac{\sigma_n^2}{2}$. Therefore the following applies:

\begin{align}
    \frac{2\Lambda''(\vec{x})}{\sigma_n^2} \sim \chi^2_{2N}.
\end{align}

% Sensing Throughput Tradeoff in Cognitive Radio, Y. C. Liang
Therefore the false alarm probability $p_{fa}$ is given by \cite{rugini2013small}:

\begin{align*}
    P_{fa} &= P\left[(\Lambda''(\vec{x}) > \gamma)\right] \\
        &= 1-F_{2N} (\frac{2\gamma}{\sigma_n^2})
\end{align*}

with $F_{2N}$ the CDF of a chi square distribution with $2N$ degrees of freedom. 

If $N$ is large enough, we can approximate the test statistics' distribution by a Gaussian distribution as it it is the sum of $2N$ i.i.d. random variables, using the Central Limit theorem:

\begin{align*}
F_{2N} \approx 1-Q(\frac{\frac{2\Lambda''(x)}{\sigma^2_n}-2N}{2\sqrt{N}}).
\end{align*}

$P_{fa}$ can then be approximated as 

 \begin{align}
 P_{fa} \approx Q\left(\frac{\frac{2\gamma}{\sigma_n^2} -2N}{2\sqrt{N}}\right).\label{eq:p_fa}
 \end{align} \cite{
 %http://ieeexplore.ieee.org/stamp/stamp.jsp?tp=&arnumber=6061767
 }

 and given a desired $P_{fa}$ the threshold $\gamma$ is given by

\begin{align}\label{eq:ed_threshold}
 \gamma &= \left(Q^{-1}\left(P_{fa}\right)\sqrt{N} + N\right)\sigma_n^2
 \end{align} 
 %\cite{
 %http://ieeexplore.ieee.org/stamp/stamp.jsp?tp=&arnumber=6061767
 %}
 

\end{document}
