%!TEX program = xelatex

\documentclass[a4paper, openany, oneside]{memoir}
\usepackage[no-math]{fontspec}
\usepackage{pgfplots}
\pgfplotsset{compat=newest}
\usepackage{commath}
\usepackage{mathtools}
\usepackage{amssymb}
\usepackage{amsthm}
\usepackage{booktabs}
\usepackage{mathtools}
\usepackage{xcolor}
\usepackage[separate-uncertainty=true, per-mode=symbol]{siunitx}
\usepackage[noabbrev, capitalize]{cleveref}
\usepackage{listings}
\usepackage[american inductor, european resistor]{circuitikz}
\usepackage{amsmath}
\usepackage{amsfonts}
\usepackage{ifxetex}
\usepackage[dutch,english]{babel}
\usepackage[backend=bibtexu,texencoding=utf8,bibencoding=utf8,style=ieee,sortlocale=en_GB,language=auto]{biblatex}
\usepackage[strict,autostyle]{csquotes}
\usepackage{parskip}
\usepackage{import}
\usepackage{standalone}
\usepackage{hyperref}
%\usepackage[toc,title,titletoc]{appendix}

\ifxetex{} % Fonts laden in het geval dat je met Xetex compiled
    \usepackage{fontspec}
    \defaultfontfeatures{Ligatures=TeX} % To support LaTeX quoting style
    \setromanfont{Palatino Linotype} % Tover ergens in Font mapje in root.
    \setmonofont{Source Code Pro}
\else % Terug val in standaard pdflatex tool chain. Geen ondersteuning voor OTT fonts
    \usepackage[T1]{fontenc}
    \usepackage[utf8]{inputenc}
\fi
\newcommand{\references}[1]{\begin{flushright}{#1}\end{flushright}}
\renewcommand{\vec}[1]{\boldsymbol{\mathbf{#1}}}
\newcommand{\uvec}[1]{\boldsymbol{\hat{\vec{#1}}}}
\newcommand{\mat}[1]{\boldsymbol{\mathbf{#1}}}
\newcommand{\fasor}[1]{\boldsymbol{\tilde{\vec{#1}}}}
\newcommand{\cmplx}[0]{\mathrm{j}}
\renewcommand{\Re}[0]{\operatorname{Re}}
\newcommand{\Cov}{\operatorname{Cov}}
\newcommand{\Var}{\operatorname{Var}}
\newcommand{\proj}{\operatorname{proj}}
\newcommand{\Perp}{\operatorname{perp}}
\newcommand{\col}{\operatorname{col}}
\newcommand{\rect}{\operatorname{rect}}
\newcommand{\sinc}{\operatorname{sinc}}
\newcommand{\IT}{\operatorname{IT}}
\newcommand{\F}{\mathcal{F}}

\newtheorem{definition}{Definition}
\newtheorem{theorem}{Theorem}


\DeclareSIUnit{\voltampere}{VA} %apparent power
\DeclareSIUnit{\pii}{\ensuremath{\pi}}

\hypersetup{%setup hyperlinks
    colorlinks,
    citecolor=black,
    filecolor=black,
    linkcolor=black,
    urlcolor=black
}

% Example boxes
\usepackage{fancybox}
\usepackage{framed}
\usepackage{adjustbox}
\newenvironment{simpages}%
{\AtBeginEnvironment{itemize}{\parskip=0pt\parsep=0pt\partopsep=0pt}
\def\FrameCommand{\fboxsep=.5\FrameSep\shadowbox}\MakeFramed{\FrameRestore}}%
{\endMakeFramed}

% Impulse train
\DeclareFontFamily{U}{wncy}{}
\DeclareFontShape{U}{wncy}{m}{n}{<->wncyr10}{}
\DeclareSymbolFont{mcy}{U}{wncy}{m}{n}
\DeclareMathSymbol{\Sha}{\mathord}{mcy}{"58}
\addbibresource{../../../../includes/bibliography.bib}

\begin{document}


\section{Proofs}
\label{sec:proofs_detection}
In this section the stated theorems will be proven.

% \begin{blockProofTheorem}{\ref{th:abs-psd}}
%     \begin{align*}
%         \exp \left [ \frac{2\pi j (N-1)}{2N-1} \right] \sum_{n=1}^{2N-1} r[n] \exp \left[ \frac{2\pi j (n-1)(k-1)}{2N-1}\right] \\
%         \sum_{n=1}^{2N-1} r[n] \exp \left[ \frac{2\pi j (n-N)(k-1)}{2N-1}\right] \\
%         \sum_{n=1}^{2N-1} \sum_{\tau=1}^{N} x[\tau - n] \overline{x}[n] \exp \left[ \frac{2\pi j (n-1)(k-1)}{2N-1}\right]  \\        
%     \end{align*}
% \end{blockProofTheorem}

\begin{blockProofTheorem}{\ref{th:psd_distr}}
First notice that if the elements of $\vec{r_x}$ have a gaussian distribution, then so will have the elements of $\vec{s}_x$ as they are a linear combination of the elements in $\vec{r}_x$.

To determine the distribution of the elements of $\vec{\hat{r}}_x$, we notice that 
\begin{align*}
\vec{\hat{r}}_x &= \mat{R}^{\dagger}\vec{\hat{r}'_y}.
\end{align*}

As $\mat{R}$ is constant, we will focus on $\vec{\hat{r}'_y}$. Again, if the elements of $\vec{\hat{r}}_x$ are to be gaussian distributed, then so are the elements of $\vec{\hat{r}'_y}$.

That is, we'd like to show that the elements of $r_{y_i,y_j}$ are approximately gaussian distributed.
\begin{align*}
(\vec{\hat{r}}'_{y_i,y_j})_{u+KL-L} &= \left(\left(\vec{c}_i\ast\vec{x}\right)' \circ \left(\vec{c}_j\ast\vec{x}\right)' \right)_{u+KL-L} \\
&= \sum_{k=1}^{KL} \left(\vec{c}_i\ast\vec{x}\right)'_k \left(\vec{\overline{c}}_j\ast\vec{\overline{x}}\right)'_{KL-(u+KL-L)+k} \\
&= \sum_{k=1}^{KL} \left(\vec{c}_i\ast\vec{x}\right)_{kN} \left(\vec{\overline{c}}_j\ast\vec{\overline{x}}\right)_{kN+KLN-N(u+KL-L)}\\
&=  \sum_{k=1}^{KL} \left(\sum_{l=1}^N \left(\vec{c}_i\right)_l\left(\vec{x}\right)_{kN-l+1}\right) \cdot \left(\sum_{m=1}^{N} \left(\vec{\overline{c}_j}\right)_m\left(\vec{\overline{x}}\right)_{kN - Nu + NL - m+1}\right)\\
&= \sum_{l=1}^N\sum_{m-1}^N \left(\vec{c}_i\right)_l \left(\vec{\overline{c}}_j\right)_m \sum_{k=1}^{KL} (\vec{x})_{kN-l+1} \cdot (\vec{\overline{x}})_{(k-u+L)N - m+1}
\end{align*}

As $\vec{c}_i$ and $\vec{c}_j$ are constant vectors, we will only have to focus on the last term. Let us introduce the helper variable $a =  \sum_{k=1}^{KL} (\vec{x})_{kN-l+1} \cdot (\vec{\overline{x}})_{(k-u+L)N - m+1}$. Notice that if

\begin{enumerate}
	\item $u=L, l=m$, then $a = \sum_{k=1}^{KL}\vec{x}_{kN-l+1}\vec{\overline{x}}_{kN-l+1}$. That is, $a$ follows a $\chi^2$ distribution (by definition). As with the convential energy detector, we can approximate the distribution of the sum as a whole as gaussian if $KL$ is large enough.
	\item $l \neq m$. Notice that as 
	% \begin{align*}\{(\vec{x})_{kN-l+1} : 1 \leq k \leq KL\} \cap 
	% \{(\vec{x})_{(k-u+L)N - m+1} : 1 \leq k \leq KL\}  &= \emptyset
	% \end{align*}

	\begin{align*}
	\bmod(kN-l+1,N) &= \bmod(-l+1, N)
	\end{align*}
	and
	\begin{align*}
	\bmod((k-u+L)N - m+1,N) &= \bmod(-m+1, N)
	\end{align*}
	the product in the sum
	That is $(\vec{x})_{kN-l+1}$  can never equal  $(\vec{x})_{(k-u+L)N - m+1}$ in $a$. Because each element of $\vec{x}$ has the same distribution, the product of two distinct elements in $\vec{x}$ will always yield the same distribution (whatever it may be). By the Central Limit Theorem we can approximate the distribution of the sum as a whole as gaussian if $KL$ is large enough.

	\item $u\neq L, l=m$ In this case, the product in the sum contains elements of $\vec{x}$ with a minimum displacement of $N$ in their indices. Unlike case 2, the specified sets \emph{can} have elements in common. To see why, consider \cref{fig:ex_dep}. Notice, that the dependence of a product in the sum on other products is limited: for each element of $\vec{x}$ there will be a maximum of two terms in $a$ dependent on that element. Without loss of generality, we will assume $L-u \geq 0$. This does not pose a problem as similar analysis for $L-u < 0$ is possible.  Notice that as $u\neq L$ the first $L-u$ multiplicands in $a$ can \emph{never} appear as multiplier in $a$. The first $L-u$ multipliers, however \emph{can} appear as multiplicand in $a$.  Using this observation it is possible to rewrite $a$ using two sums, which alternatingly sum $L-u$ consecutive terms (which guarantees that the multiplicands in the sum can never equal the multiplier) of the original sum in $a$:
	
	\begin{align*}
	a &= \sum_{q=1}^{\left\lfloor{\frac{KL}{2(L-u)}}\right\rfloor} \sum_{r=1}^{L-u} \vec{x}_{((2q-1)(L-u)+r)N-l+1}\vec{\overline{x}}_{((2q-1)(L-u)+r)N-l+1} \\
	   & \;+ \sum_{s=1}^{\left\lfloor{\frac{KL}{2(L-u)}}\right\rfloor} \sum_{t=1}^{L-u} \vec{x}_{(2s(L-u)+t)N-l+1}\vec{\overline{x}}_{(2s(L-u)+t)N-l+1}
	\end{align*}
	
	Therefore, if $\left\lfloor{\frac{KL}{2(L-u)}}\right\rfloor$ is large enough, $a$ is the sum of two approximately gaussian distributed variables, and therefore itself is approximately gaussian distributed.			
\end{enumerate}

\begin{figure}
\begin{tikzpicture}
\node (species1) {
\begin{tabular}{|c|c|c|c|c|c|c|c|}\hline
    $(\vec{x})_{2N-l+1}$ & $\ldots$  & $(\vec{x})_{3N-l+1}$ & $\ldots$ & $(\vec{x})_{4N-l+1}$ & $\ldots$ & $(\vec{x})_{5N-l+1}$  & $\ldots$ \\ \hline 
    \end{tabular}
  
};
\node (species2) [below right= 0.2cm and -14.5cm of species1] {
      \begin{tabular}{|c|c|c|c|c|c|c|c|}\hline
    $(\vec{x})_{N-l+1}$ & $\ldots$  & $(\vec{x})_{2N-l+1}$ & $\ldots$ & $(\vec{x})_{3N-l+1}$ & $\ldots$ & $(\vec{x})_{4N-l+1}$  & $\ldots$ \\ \hline 
    \end{tabular}
};
\end{tikzpicture}
\caption{Example illustrating dependence, $u=L-1$}
\label{fig:ex_dep}
\end{figure}


\end{blockProofTheorem}

\end{document}