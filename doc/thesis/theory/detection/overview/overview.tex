%!TEX program = xelatex

\documentclass[a4paper, openany, oneside]{memoir}
\usepackage[no-math]{fontspec}
\usepackage{pgfplots}
\usepackage{float}
\pgfplotsset{compat=newest}
\usepackage{commath}
\usepackage{mathtools}
\usepackage{amssymb}
\usepackage{amsthm}
\usepackage{booktabs}
\usepackage{todonotes}
\usepackage{mathtools}
\usepackage{xcolor}
\usepackage[separate-uncertainty=true, per-mode=symbol]{siunitx}
\usepackage{listings}
\usepackage[american inductor, european resistor]{circuitikz}
\usepackage{amsmath}
\usepackage{amsfonts}
\usepackage{ifxetex}
\usepackage[dutch,english]{babel}
\usepackage[backend=bibtexu,texencoding=utf8,bibencoding=utf8,style=ieee,sortlocale=en_GB,language=auto]{biblatex}
\usepackage[strict,autostyle]{csquotes}
\usepackage{import}
\usepackage{standalone}
\usepackage{bookmark,hyperref}
\usepackage{xcolor,mdframed}
\usepackage{tikz}
\usepackage{framed}
\usepackage{float}
\usepackage{tabularx}
\usepackage{graphicx,adjustbox}
\usepackage{rotating}
\usepackage{pdfpages}
\usepackage{enumitem}
\usepackage{calc}
\usepackage{pgfplots}
\usepackage{filecontents}
\usepackage{caption}
\usepackage{subcaption}
\usepackage{lettrine}

\newcolumntype{Y}{>{\raggedright\arraybackslash}X} % Left-justified text in tabularx environment

\ifxetex{} % Fonts laden in het geval dat je met Xetex compiled
    \usepackage{fontspec}
    \defaultfontfeatures{Scale=MatchLowercase, Ligatures=TeX} % To support LaTeX quoting style
    %\setromanfont{Palatino Linotype} % Tover ergens in Font mapje in root.
    \setsansfont{Avenir Next LT Pro}
    \setromanfont{Adobe Caslon Pro} % Tover ergens in Font mapje in root.
    \setmonofont{Source Code Pro}
\else % Terug val in standaard pdflatex tool chain. Geen ondersteuning voor OTT fonts
    \usepackage[T1]{fontenc}
    \usepackage[utf8]{inputenc}
\fi
\usepackage[noabbrev, capitalize]{cleveref}
\usepackage{ifthen}
\usepackage{titlesec}
\usepackage{titlecaps}

\newcommand{\references}[1]{\begin{flushright}{#1}\end{flushright}}
\renewcommand{\vec}[1]{\boldsymbol{\mathbf{#1}}}
\newcommand{\uvec}[1]{\boldsymbol{\hat{\vec{#1}}}}
\newcommand{\mat}[1]{\boldsymbol{\mathbf{#1}}}
\newcommand{\fasor}[1]{\boldsymbol{\tilde{\vec{#1}}}}
\newcommand{\cmplx}[0]{\mathrm{j}}
\renewcommand{\Re}[0]{\operatorname{Re}}
\newcommand{\Cov}{\operatorname{Cov}}
\newcommand{\Var}{\operatorname{Var}}
\newcommand{\proj}{\operatorname{proj}}
\newcommand{\Perp}{\operatorname{perp}}
\newcommand{\col}{\operatorname{col}}
\newcommand{\rect}{\operatorname{rect}}
\newcommand{\sinc}{\operatorname{sinc}}
\newcommand{\lcm}{\operatorname{lcm}}
%\newcommand{\gcd}{\operatorname{gcd}}
\newcommand{\F}{\mathcal{F}}
\newcommand{\DTFT}{\mathcal{F}_*}
\newcommand{\conj}[1]{#1^*}
\renewcommand{\mod}{\operatorname{mod}}
\newcommand{\rot}{\operatorname{rot}}
\newcommand{\vecsc}[1]{\vec{\textsc{\textbf{#1}}}}
\renewcommand{\ss}[1]{_{#1}}

% Label without linebreak breaker
\newcommand{\lab}[1]{\label{#1}\nolinebreak}

\newtheorem{definition}{Definition}
\newtheorem{theorem}{Theorem}


\DeclareSIUnit{\voltampere}{VA} %apparent power
\DeclareSIUnit{\pii}{\ensuremath{\pi}}

\hypersetup{%setup hyperlinks
    colorlinks,
    citecolor=black,
    filecolor=black,
    linkcolor=black,
    urlcolor=black
}

% Example boxes
\usepackage{fancybox}
\usepackage{framed}
\usepackage{adjustbox}
\newenvironment{simpages}%
{\AtBeginEnvironment{itemize}{\parskip=0pt\parsep=0pt\partopsep=0pt}
\def\FrameCommand{\fboxsep=.5\FrameSep\shadowbox}\MakeFramed{\FrameRestore}}%
{\endMakeFramed}

% Impulse train
\DeclareFontFamily{U}{wncy}{}
\DeclareFontShape{U}{wncy}{m}{n}{<->wncyr10}{}
\DeclareSymbolFont{mcy}{U}{wncy}{m}{n}
\DeclareMathSymbol{\Sha}{\mathord}{mcy}{"58}

\setlength{\parindent}{0pt}
\nonzeroparskip

% Block environment configuration
\newcommand{\BlockLeftMargin}{20pt}
\newcommand{\BlockLeftMarginText}{25pt}
\newcommand{\BlockLeftMarginTextSpacing}{10pt}

% Own colours
\definecolor{gray75}{gray}{0.75}

% Block environment
\newenvironment{block}[3]{%
\makebox{\hspace{-\spinemargin}%
\begin{tikzpicture}[overlay]
    \draw [thick,color=gray75] (\BlockLeftMargin, 0) -- (\paperwidth - \spinemargin, 0);
    \node at (\BlockLeftMarginText, -0.9) [align=left, text width=\spinemargin - \BlockLeftMarginText - \BlockLeftMarginTextSpacing, anchor=west, text depth=1cm] {\textbf{\textsc{#1}}\newline\textit{#3}};
\end{tikzpicture}}%
\nopagebreak\\[0.25em]\ifthenelse{\equal{#2}{}}{}{(\textit{#2}.) }\nopagebreak\nolinebreak}
{\nopagebreak\\[-0.25em]%
\makebox{\hspace{-\spinemargin}%
\begin{tikzpicture}[overlay, remember picture]
    \draw [thick,color=gray75] (\spinemargin,0) -- (\paperwidth - \spinemargin,0);
\end{tikzpicture}} \vspace{0.5em}}

% Theorem
\newcounter{blockTheoremCounter}
\crefname{blockTheoremCounter}{Theorem}{Theorems}
\Crefname{blockTheoremCounter}{Theorem}{Theorems}

\newenvironment{blockTheorem}[1][]{%
\refstepcounter{blockTheoremCounter}%
\begin{block}{theorem \theblockTheoremCounter}{#1}{}}
{\end{block}}

% Definition
\newcounter{blockDefinitionCounter}
\crefname{blockDefinitionCounter}{Definition}{Definitions}
\Crefname{blockDefinitionCounter}{Definition}{Definitions}

\newenvironment{blockDefinition}[1][]{%
\refstepcounter{blockDefinitionCounter}%
\begin{block}{definition \theblockDefinitionCounter}{#1}{}}
{\end{block}}

% Proof
\newcounter{blockProofTheoremCounter}
\crefname{blockProofTheoremCounter}{Proof}{Proofs}
\Crefname{blockProofTheoremCounter}{Proof}{Proofs}

\newenvironment{blockProofTheorem}[1]{%
\refstepcounter{blockProofTheoremCounter}%
\begin{block}{proof of \\ theorem #1}{}{}}
{\qed\end{block}}

% Detail
\newcounter{blockDetailCounter}
\crefname{blockDetailCounter}{Detail}{Details}
\Crefname{blockDetailCounter}{Detail}{Details}

\newenvironment{blockDetail}[1][]{%
\refstepcounter{blockDetailCounter}%
\begin{block}{detail \theblockDetailCounter}{#1}{}}
{\end{block}}

% Redesign chapter headings
\newcommand{\chapternumber}{\thechapter}
\newcommand{\hsp}{\hspace{20pt}}
\titleformat{\chapter}[hang]{\Huge\bfseries}{\chapternumber\hsp\textcolor{gray75}{|}\hsp}{0pt}{\Huge\bfseries}

% Remove headers
% \addtopsmarks{headings}{}{
%   \createmark{chapter}{left}{nonumber}{}{}
% }
% \pagestyle{headings} % Activate changes

% Capitalise headers in a regular way
\renewcommand*{\memUChead}[1]{\titlecap{#1}}

% \hfill for math mode
\newcommand{\pushright}[1]{\intertext{\hfill$\displaystyle #1$}}
\newcommand{\pushline}{\hskip \textwidth minus \textwidth}
\newcommand{\matlab}{\textsc{Matlab}}

\definecolor{code-grey}{HTML}{DDDDDD}
\newcommand{\lib}[1]{\textsf{#1}}
\newcommand{\file}[1]{\textsf{#1}}
\newcommand{\func}[1]{\colorbox{code-grey}{\texttt{#1}}}
\newcommand{\class}[1]{\colorbox{code-grey}{\texttt{#1}}}

% Setup actiepunten
\newenvironment{important}[1][]{%
   \begin{mdframed}[%
      backgroundcolor={red!15}, hidealllines=true,
      skipabove=0.7\baselineskip, skipbelow=0.7\baselineskip,
      splitbottomskip=2pt, splittopskip=4pt, #1]%
   \makebox[0pt]{% ignore the withd of !
      \smash{% ignor the height of !
         \fontsize{32pt}{32pt}\selectfont% make the ! bigger
         \hspace*{-19pt}% move ! to the left
         \raisebox{-2pt}{% move ! up a little
            {\color{red!70!black}\sffamily\bfseries !}% type the bold red !
         }%
      }%
   }%
}{\end{mdframed}}
\newcommand{\excl}[1]{
\begin{important}
  \textbf{#1}
\end{important}
}

\makeatletter
\newcommand\footnoteref[1]{\protected@xdef\@thefnmark{\ref{#1}}\@footnotemark}
\makeatother

% Allow page breaks in display environments
%\allowdisplaybreaks
% S unit for use in Mega Samples per second
\DeclareSIUnit\sample{S}

\newcommand{\CC}{C\nolinebreak\hspace{-.05em}\raisebox{.3ex}{ \textbf{+}}\nolinebreak\hspace{-.10em}\raisebox{.3ex}{\textbf{+}}}
\def\CC{{C\nolinebreak[4]\hspace{-.05em}\raisebox{.3ex}{\textbf{++}}}}


\newcommand{\partauthor}[1]{\gdef\@partauthor{#1}}
\renewcommand{\printparttitle}[1]{
  \parttitlefont #1\\
  \vspace{1.5cm}
  \textnormal{\Large \@partauthor}
}
\addbibresource{../../../../includes/bibliography.bib}

\begin{document}

\section{Overview}
The fundamental problem of detection is to discern noise from a signal. This problem can be formulated as a binary hypothesis test. Given a signal $x[n]$ a detector has to decide between the hypotheses

\begin{align*}
	\begin{array}{ll}
		\mathcal{H}_0: & x[n] = w[n], \\
		\mathcal{H}_1: & x[n] = w[n] + s[n]
	\end{array}
\end{align*}

where $w[n]$ denotes a noise signal and $s[n]$ a signal different from noise. Under $\mathcal{H}_0$ the signal $x[n]$ contains only noise. Under $\mathcal{H}_1$ another signal besides noise is present in $x[n]$. Besides this hypothesis test, the detector also has to deal with frequency bands: which frequency bands are occupied by some signal and which are unoccupied?
\clearpage
We can distinguish between two types of detection methods that provide an answer to this problem:
\begin{enumerate}
\item The first type of methods are detection methods based on information of the type of signals to be detected, such called \emph{prior information}.
A popular example of this type is cyclostationary detection which uses the knowledge of the modulation techniques of the signals to perform the detection \cite{axell2012spectrum,quan2009optimal}. Another example is a matched filter detector. This detector correlates the received signal with a known signal that is to be detected and therefore needs exact knowledge of the signal \cite{Kapoor2011Communication,couch2013digital}. 
\item The second type of methods are detection methods methods working without any prior knowledge of the signals to be detected, such called \emph{blind} detection methods.
A basic detection method of this type is energy detection \cite{axell2012spectrum} which compares the energy contained in a signal to a certain threshold based on the noise power. Other examples include Covariance Absolute Value detection \cite{zheng2009spectrum} and wavelet-based detection \cite{han2013novel}. 
\end{enumerate}

A well known problem in the context of cognitive radio is the \emph{SNR wall}. This problem states that below a certain signal-to-noise ratio\footnote{The signal-to-noise ratio is the ratio of the signal power and the noise power.} a detector fails to detect a signal no matter how long the sensing time \cite{sahai2009spectrum}. This problem arises from uncertainties and imperfections in the model assumptions of that detector. Detectors that use a threshold that is independent from noise or signal parameters are not affected by this SNR wall \cite{axell2012spectrum} and possess the property of constant false-alarm rate (\emph{CFAR}). This property states the probability that the detector decides that its input signal is a signal different from noise, where in fact the signal contains only noise, is constant. 

In this chapter we will focus on detection methods that do not assume to have any prior knowledge available. Although methods that use \emph{prior information} can operate at lower SNRs than blind detectors, they generally result in more (computationally) complex algorithms. Furthermore, using a detector that depends on \emph{prior information} cannot be used to detect arbitrary signals as demanded in \cref{sec:theory-specs}. \footnote{We will not focus on spread-spectrum modulation in this thesis. Spread spectrum signals occupy a wide band of the spectrum and are designed to be resilient against interception by other detectors than those of the intended receivers \cite{pickholtz1982theory,gevargiz1989adaptive}.}

This chapter will be concerned with the theoretical analysis of the following two detection methods:

\begin{enumerate}
	\item Energy detection. This detection method uses a test statistic which is easy to compute. Its threshold, as stated before, is dependent on 
	the noise power. This noise power has to be estimated in real detectors for example by using an empty frequency band as reference. This detection method is therefore subject to the SNR-wall phenomenon as there will be uncertainty in the noise estimate.
	\item Covariance Absolute Value detection (CAV). This detection method uses a test statistic based on the autocorrelation of 
	the signal. It uses a threshold that does not depend on any signal or noise parameters. Therefore, this detector does posses the \emph{CFAR} property \cite{sharma2011sensors}.
\end{enumerate}

This analysis represents the search for a suitable detector to be used in our system.  The analsysis of energy detection is motivated by the relative simplicity of the algorithm and low comlexity of its implementation. Analysing CAV is motivated by the fact that it is a \emph{CFAR} detector which is convenient in practical applications \cite{axell2012spectrum}.  Based on this analysis this chapter will conclude why the final system is using an energy detector. 

\end{document}
