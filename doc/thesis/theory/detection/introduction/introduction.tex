%!TEX program = xelatex

\documentclass[a4paper, openany, oneside]{memoir}
\usepackage[no-math]{fontspec}
\usepackage{pgfplots}
\pgfplotsset{compat=newest}
\usepackage{commath}
\usepackage{mathtools}
\usepackage{amssymb}
\usepackage{amsthm}
\usepackage{booktabs}
\usepackage{mathtools}
\usepackage{xcolor}
\usepackage[separate-uncertainty=true, per-mode=symbol]{siunitx}
\usepackage[noabbrev, capitalize]{cleveref}
\usepackage{listings}
\usepackage[american inductor, european resistor]{circuitikz}
\usepackage{amsmath}
\usepackage{amsfonts}
\usepackage{ifxetex}
\usepackage[dutch,english]{babel}
\usepackage[backend=bibtexu,texencoding=utf8,bibencoding=utf8,style=ieee,sortlocale=en_GB,language=auto]{biblatex}
\usepackage[strict,autostyle]{csquotes}
\usepackage{parskip}
\usepackage{import}
\usepackage{standalone}
\usepackage{hyperref}
%\usepackage[toc,title,titletoc]{appendix}

\ifxetex{} % Fonts laden in het geval dat je met Xetex compiled
    \usepackage{fontspec}
    \defaultfontfeatures{Ligatures=TeX} % To support LaTeX quoting style
    \setromanfont{Palatino Linotype} % Tover ergens in Font mapje in root.
    \setmonofont{Source Code Pro}
\else % Terug val in standaard pdflatex tool chain. Geen ondersteuning voor OTT fonts
    \usepackage[T1]{fontenc}
    \usepackage[utf8]{inputenc}
\fi
\newcommand{\references}[1]{\begin{flushright}{#1}\end{flushright}}
\renewcommand{\vec}[1]{\boldsymbol{\mathbf{#1}}}
\newcommand{\uvec}[1]{\boldsymbol{\hat{\vec{#1}}}}
\newcommand{\mat}[1]{\boldsymbol{\mathbf{#1}}}
\newcommand{\fasor}[1]{\boldsymbol{\tilde{\vec{#1}}}}
\newcommand{\cmplx}[0]{\mathrm{j}}
\renewcommand{\Re}[0]{\operatorname{Re}}
\newcommand{\Cov}{\operatorname{Cov}}
\newcommand{\Var}{\operatorname{Var}}
\newcommand{\proj}{\operatorname{proj}}
\newcommand{\Perp}{\operatorname{perp}}
\newcommand{\col}{\operatorname{col}}
\newcommand{\rect}{\operatorname{rect}}
\newcommand{\sinc}{\operatorname{sinc}}
\newcommand{\IT}{\operatorname{IT}}
\newcommand{\F}{\mathcal{F}}

\newtheorem{definition}{Definition}
\newtheorem{theorem}{Theorem}


\DeclareSIUnit{\voltampere}{VA} %apparent power
\DeclareSIUnit{\pii}{\ensuremath{\pi}}

\hypersetup{%setup hyperlinks
    colorlinks,
    citecolor=black,
    filecolor=black,
    linkcolor=black,
    urlcolor=black
}

% Example boxes
\usepackage{fancybox}
\usepackage{framed}
\usepackage{adjustbox}
\newenvironment{simpages}%
{\AtBeginEnvironment{itemize}{\parskip=0pt\parsep=0pt\partopsep=0pt}
\def\FrameCommand{\fboxsep=.5\FrameSep\shadowbox}\MakeFramed{\FrameRestore}}%
{\endMakeFramed}

% Impulse train
\DeclareFontFamily{U}{wncy}{}
\DeclareFontShape{U}{wncy}{m}{n}{<->wncyr10}{}
\DeclareSymbolFont{mcy}{U}{wncy}{m}{n}
\DeclareMathSymbol{\Sha}{\mathord}{mcy}{"58}
\addbibresource{../../../../includes/bibliography.bib}

\begin{document}

\section{Introduction}
The fundamental problem of detection is to discern noise from a signal. This problem can be formulated as a binary hypothesis test. Given a signal $x[n]$ a detector has to decide between the following two hypotheses:

\begin{align*}
	\begin{cases}
		\mathcal{H}_0 & x[n] = w[n] \\
		\mathcal{H}_1 & x[n] = w[n] + s[n]
	\end{cases}
\end{align*}

where $w[n]$ denotes a noise signal and $s[n]$ a signal different from noise. Under $\mathcal{H}_0$ the signal $x[n]$ contains only noise. Under $\mathcal{H}_1$ another signal besides noise is present in $x[n]$.

Besides this hypothesis test, the detector also has to deal with frequency bands: which frequency bands are occupied by some signal and which are unoccupied?

This chapter examines three detectors that can be used in combination with our reconstructor:

\begin{enumerate}
	\item Conventional energy detection;
	\item Energy detection adapted to the reconstruction method;  
	\item Covariance absolute value detection (CAV).
\end{enumerate}

% Many other detectors exist, some of them using \emph{prior information} about the signals to be detected. Although the use of prior information
% makes detection of signals at a very low signal to noise ratio possible, we eventually have chosen not to depend on this \emph{prior information}. The reason for this is twofold:
% \begin{enumerate}
% 	\item By not depending on \emph{prior information} the detector can be used in a variety of situations,
% 	effectively providing us with a ``general '' purpose detector.
% 	\item The methods listed above have a relatively low complexity and therefore need less care to be implemented
% 	in our testing platform.
% \end{enumerate}

% This chapter will continue with a description and comparison of the three detection methods.

% \section{Energy detection}
% Conventional energy detection is a detection method that measures the energy of a signal $x[n]$, $\Lambda$, during a certain period. By comparing this measured energy to a threshold $\gamma$ the detector decides whether the input signal $x[n]$ contains only noise or that another signal besides noise is present in $x[n]$:

% \begin{align*}
% 	\begin{cases}
% 		\mathcal{H}_0 & \text{if } \Lambda < \gamma \\
% 		\mathcal{H}_1 & \text{if } \Lambda > \gamma
% 	\end{cases}
% \end{align*}
 
% Given $N$ samples of $x[n]$, the measure of energy, $\Lambda$, which is used as test statistic is given by 

% \begin{align}\label{eq:test_ed}
% 	\Lambda &= \sum_{n=0}^{N-1} |x[n]|^2.
% \end{align}

% The threshold $\gamma$ depends on the expected noise power $\sigma_n^2$ and the desired false alarm probability $p_{fa}$: the probability that the detector decides that hypothesis 2 is true, while hypothesis 1 is the true hypothesis. The derivation of $\gamma$ (see Appendix ...) gives the following result

% \begin{align*}
% \gamma = \left[Q^{-1}(p_{fa})\sqrt{N} + N\right]2\sigma_n^2
% \end{align*}

% where $Q^{-1}$ denotes the the inverse tail probability function of the standard normal distribution and $p_{fa}$ is the desired false alarm probability.
% Using a bandpass filter $H(f)$ one can filter the signal $x[n]$ such that it can be used to detect a signal in a particular frequency band. This whole process is depicted
% in \cref{tkz:conv_ed}
% \begin{figure}[H]
% \centering
% \begin{tikzpicture}

% \node at (-6,2) {$x[n]$};
% \node at (1.8,2) {$\mathcal{H}_0 \lor \mathcal{H}_1$};

% \draw [>=latex, ->] (-5.5,2) -- (-5,2);
% \draw (-5,2.5) rectangle (-4,1.5) node[pos=.5]{$H(f)$};

% \draw [>=latex, ->] (-4,2) -- (-3.5,2);
% \draw (-3.5,2.5) rectangle (-2.5,1.5) node[pos=.5]{$|\cdot|^2$};

% \draw [>=latex, ->] (-2.5,2) -- (-2,2);
% \draw (-2,2.5) rectangle (-1,1.5) node[pos=.5]{$\sum$};

% \draw [>=latex, ->] (-1,2) -- (-0.5,2); 
% \draw (-0.5,2.5) rectangle (0.5,1.5) node[pos=.5]{$_<^ > \gamma$};

% \draw [>=latex, ->] (0.5,2) -- (1,2);

% \end{tikzpicture}
% \caption{Conventional Energy Detector}\label{tkz:conv_ed}
% \end{figure}

% % As can be seen from  \cref{eq:test_ed} the test statistic used by the conventional energy detector does not take into account that the signal to be detected resides in a certain frequency band. Therefore it is necessary that the input signal $x[n]$ is filtered before $\Lambda$ is computed.

% As can be noted from \cref{eq:test_ed} the test statistic depends on the signal itself, not on the autocorrelation which serves as input of our detector.
% In the rest of this section, we will show how the test statistic can be modified
% such that the autocorrelation function can be used to detect the presence of  a signal in a certain frequency band.

% Let 
% \begin{align*}
% \Lambda' &= \frac{\Lambda}{N}\\
% 	&= \frac{\sum_{n=0}^{N-1} |x[n]|^2}{N},
% \end{align*}
% then $\Lambda'$ is an estimate of the average power $E\left[|x[n]|^2\right]$ of the signal $x$. We can use $\Lambda'$ as test statistic by using a modified threshold $\gamma' = \frac{\gamma}{N}$. By definition of the power spectral density we know that integral over the power spectral density equals the expected average power:

% \begin{align}\label{eq:average_power_psd}
% E\left[\left|x[n]\right|^2\right] = \frac{1}{2\pi} \int_{-\pi}^{\pi}\mathcal{P}_x(\omega) \text{d}\omega.
% \end{align}
% Furthermore, by the Wiener-Khinchin theorem we have that, if $x[n]$ is wide-sense stationary (as is assumed in the reconstructor),
% \begin{align}\label{eq:wiener_psd}
% 	\mathcal{P}_x(\omega) = \sum_{n=-\infty}^{\infty} r_x[n] \exp [2\pi j\omega].
% \end{align}
% By combining \cref{eq:average_power_psd} and \cref{eq:wiener_psd} we can relate the autocorrelation function to the test statistic $\Lambda'$.
% The test statistic $\Lambda'$, however, still does not allow for detection in a certain frequency band.

%  Notice that if $x[n]=w[n]$, its autocorrelation function equals a delta function: $r_x[n] = \sigma_n^2\delta[n]$, with $\sigma_n^2$ the average noise power. 
%  Therefore the power spectral density of $x[n]$, denoted by $\mathcal{P}(\omega)$ equals  
%  \begin{align*}
%  \mathcal{P}_x(\omega) = \sum_{n=-\infty}^{\infty}r_x[n]e^{-jn\omega} = \sigma_n^2.
%  \end{align*} The power spectral density is constant. If we want to detect the presence of a signal in a certain frequency band $W$, we can make use of this characteristic. Let

%  \begin{align*}
%  \Lambda'' &= \int_W \mathcal{P}(\omega) \text{d}\omega
%  \end{align*}
% denote the test statistic to detect a signal in the frequency band W.
%  If we would integrate the power spectral density of noise over $W$ we obtain an average power of  $\frac{W}{2\pi} \sigma_n^2$. Therefore if we are to use $\Lambda''$ as test statistic, we have to use the modified threshold $\gamma'' = \frac{W}{2\pi} \gamma'$.

%  By applying the fourier transform to the autocorrelation one can use this test statistic to apply the detection process to several frequency bands, as depicted in \cref{tkz:my_ed}.

%  \begin{figure}[H]
% \centering
% \begin{tikzpicture}

% \node at (-6.5,2.5) {$r_x[n]$};

% \draw [>=latex, ->] (-6,2.5) -- (-5.5,2.5);
% \draw (-5.5,3) rectangle (-4.5,2) node[pos=.5]{$\mathcal{F}$};

% \draw [>=latex, ->] (-4.5,2.5) -- (-3.5,2.5);
% \draw (-3.5,3) rectangle (-2.5,2) node[pos=.5]{$\int P_x$};

% \draw [>=latex, ->] (-2.5,2.5) -- (-2,2.5);
% \draw (-2,3) rectangle (-1,2) node[pos=.5]{$_<^ > \gamma$};

% \draw [>=latex, ->] (-1,2.5) -- (-0.5,2.5);
% \node at (0.3,2.5) {$\mathcal{H}_0 \lor \mathcal{H}_1$};


% \draw [>=latex, ->] (-4,1) -- (-3.5,1);
% \draw (-3.5,1.5) rectangle (-2.5,0.5) node[pos=.5]{$\int P_x$};

% \draw [>=latex, ->] (-2.5,1) -- (-2,1);
% \draw (-2,1.5) rectangle (-1,0.5) node[pos=.5]{$_<^ > \gamma$};

% \draw [>=latex, ->] (-1,1) -- (-0.5,1);
% \node at (0.3,1) {$\mathcal{H}_0 \lor \mathcal{H}_1$};


% \draw [>=latex, ->] (-4,-1) -- (-3.5,-1);
% \draw (-3.5,-0.5) rectangle (-2.5,-1.5) node[pos=.5]{$\int P_x$};

% \draw [>=latex, ->] (-2.5,-1) -- (-2,-1);
% \draw (-2,-0.5) rectangle (-1,-1.5) node[pos=.5]{$_<^ > \gamma$};

% \draw [>=latex, ->] (-1,-1)-- (-0.5,-1);
% \node at (0.3,-1) {$\mathcal{H}_0 \lor \mathcal{H}_1$};

% \draw (-4,2.5) -- (-4,-1);


% \node at (-3,0.11) {\vdots};
% \node at (-1.5,0.11) {\vdots};
% \node at (0.3,0.11) {\vdots};

% \end{tikzpicture}
% \caption{Energy detection, multiple frequency bands}\label{tkz:my_ed}
% \end{figure}


% \subsection{Noise variance}
% As indicated, knowledge of the noise variance is necessary for the energy detector to work. In practical situations this noise variance may be estimated from a reference measurement in an empty frequency band. 

% \subsection{Energy detection with knowledge of the reconstructor}
% The energy detector as described in \cite{ariananda2012compressive} uses the power spectral density $\mathcal{P}_x(\omega)$ as test statistic. With a frequency specific threshold, $\gamma_{\omega}$, the detector decides per frequency whether $\mathcal{H}_0$ or $\mathcal{H}_1$ is true, see \cref{tkz:ed_ari_overview}.

% \begin{figure}[H]
% \centering
% \begin{tikzpicture}

% \node at (-3,4) {$r_x[n]$};

% \draw [>=latex, ->] (-2.5,4) -- (-2,4);

% \draw (-2,4.5) rectangle (-1,3.5) node[pos=.5]{FFT};

% \draw [>=latex, ->] (-1,4) -- (-0.5,4);
% \node at (0.3,4) {PSD[k]};

% \node at (-3.2,2.5) {$PSD[i]$};

% \draw [>=latex, ->] (-2.5,2.5) -- (-2,2.5);

% \draw (-2,3) rectangle (-1,2) node[pos=.5]{$_<^ > \gamma_i$};

% \draw [>=latex, ->] (-1,2.5) -- (-0.5,2.5);
% \node at (0.3,2.5) {$\mathcal{H}_0 \lor \mathcal{H}_1$};

% \end{tikzpicture}
% \caption{Energy detection with knowledge of the reconstructor}\label{tkz:ed_ari_overview}
% \end{figure}

%  To determine the threshold $\gamma_{\omega}$ for each element, the signal $x[n]$ is assumed to contain purely additive circular complex gaussian noise. The distribution of the elements of the reconstructed power spectral density is then approximated as a normal distribution, see Appendix \textbf{TODO} for a derivation.

% \begin{figure}[H]
% \centering
% \begin{tikzpicture}

% \draw [fill=lightgray] (-2,0) rectangle (1,-1);
% \draw (1,3) rectangle (3,-1);
% \draw [fill=cyan, opacity=0.2] (1,3) rectangle (3,-1);
% \draw [fill=lightgray](3,0) rectangle (9,-1);


% \draw [>=latex,<->] (-2.1,-1) -- (-2.1,0);
% \node at (-2.4,-.5) {$\sigma^2_n$};

% \draw [red] (-2,.1) -- (9.2,.1);

% \draw [>=latex,<->] (9.2,-1) -- (9.2,.1);
% \node at (9.4,-.4 ) {$\gamma$};

% \node at (-2,-1.5) {0 rad/s};
% \node at (9,-1.5) {$2\pi$ rad/s};
% \end{tikzpicture}
% \caption{Energy detection with power spectral density}\label{tkz:ed_ari}
% \end{figure}

% Given $\mu (\omega) = E[\mathcal{P}(\omega)]$ and $\sigma^2 (\omega) = \text{Var}(\mathcal{P}(\omega))$, a threshold for each frequency $\omega$ given a false alarm probability can be calculated as:

% \begin{align*}
% \gamma(\omega) &= Q^{-1}(p_{fa})\sigma (\omega) + \mu (\omega) 
% \end{align*}

%  \subsection{Determining $\mu(\omega)$ and $\sigma^2(\omega)$}
% In this section we will describe how $\mu(\omega$ and $\sigma^2(\omega)$ can be derived given a reconstructed autocorrelation.

% Let $\mat{F}$ denote the $(2NL+1) \times (2NL+1) $ DFT matrix, then
% \begin{align*}
% \vec{s}_x = \mat{F}\mat{R}^{\dagger}\vec{r}_y
% \end{align*}
% denotes the vector containing the elements of the reconstructed power spectral density of $x[n]$. That is the element $\left(\vec{s}_{x}\right)_{i}$ denotes the component of the power spectral density at $\omega = \frac{2\pi (i-1)}{2LN-1}$.

% To obtain $\text{Var}(\vec{s}_x)_i$ for each element in $\vec{s}_x$, we compute the covariance matrix $\mat{C}_{s_x}$ of $\vec{s}_x$ as 

% \begin{align*}
% 	\mat{C}_{s_x} &= E\left[\vec{s}\vec{s}^H\right] - E\left[\vec{s}\right]E\left[\vec{s}^H\right]
% \end{align*} 

% The variance $\sigma^2$ of the elements of $\vec{s}_x$ will then given by the diagonal of $\mat{C}_{s_x}$.
% We can express $\mat{C}_{s_x}$ as

% \begin{align*}
% \mat{C}_{s_x} &= E\left[\vec{s}_x\vec{s}_x^H\right] - E\left[\vec{s}\right]E\left[\vec{s}^H\right] \\
% &= E\left[\left(\mat{F} \mat{R}^{\dagger}\vec{r}_y\right) \left( \mat{F}\mat{R}^{\dagger}\vec{r}_y\right)^H\right] - E\left[\mat{F}\mat{R}^{\dagger}\vec{r}_y\right]E\left[\left(\mat{F}\mat{R}^{\dagger}\vec{r}_y\right)^H\right] \\
% &= \mat{F}\mat{R}^{\dagger} E\left[ \vec{r}_y\vec{r}_y^H\right]  \mat{R}^{\dagger H}\mat{F}^H -  \mat{F}\mat{R}^{\dagger} E\left[ \vec{r}_y\right] E\left[\vec{r}_y^H\right]   \mat{R}^{\dagger H}\mat{F}^H \\
% &= \mat{F}\mat{R}^{\dagger} \mat{C}_{r_y} \mat{R}^{\dagger H}\mat{F}^H.
% \end{align*}

% $\mat{C}_{r_y}$ denotes the covariance matrix of $\vec{r}_y$ and is defined as

% \begin{align*}
%  E\left[\vec{r}_y\vec{r}_y^H\right] - E\left[\vec{r}_y\right]E\left[\vec{r}_y^H\right]
% \end{align*}

% In case that $x[n]$ contains white noise with variance $\sigma_n$, the elements of $\mat{C}_{r_y}$ are given by

% \begin{align*}
% \text{Cov}\left(r_{y_r,y_s}[t],r_{y_u,y_v}[w]\right) &= \sigma_n^4 r_{c_r,c_u}[0]r_{c_s,c_v}[0]\delta[w-t]
% \end{align*}

% That is, by computing the theoretical covariance $C_{r_y}$ in the case that $x[n]$ contains only noise, we can compute $\mat{C}_{s_x}$.
% By taking the diagonal of $\mat{C}_{s_x}$ we obtain the vector $[\sigma^2(\frac{2\pi k}{d} $
% \section{Covariance absolute value method}
% The covariance absolute value detection method as introduced does \emph{not} depend on the noise power. Its test statistic is based on the structure of the autocorrelation function $r_x[n]$.

% Let $L$ samples of the signal $x[n]$ be collected in the vector $\vec{x} = \left[x[n], x[n+1], \ldots, x[n+L]\right]$. 
% Furthermore let $\mat{C}_x = E\left[\left(\vec{x} - \mu \right)\left(\vec{x} - \mu \right)^H\right]$ denote the covariance of $\vec{x}$ with $\mu = E[\vec{x}]$.

% In the case that $E\left[\vec{x}\right]=0$, like for noise or most communication signals, then $\mat{C}_x$ can be simplified to

% \begin{align*}
% \mat{C}_x &= E\left[\vec{x}\vec{x}^T\right] \\
% &= \begin{bmatrix} 
% E\left(x[n][n]\right) & E\left(x[n][n+1]\right) & \ldots & E\left(x[n][n+L-1]\right) \\
% E\left(x[n+1][n]\right) & E\left(x[n+1][n+1]\right) & \ldots & E\left(x[n+1][n+L-1]\right) \\
% \vdots & \vdots & \ddots & \vdots \\
% E\left(x[n+L-1][n]\right) & E\left(x[n+L-1][n+1]\right) & \ldots & E\left(x[n+L-1][n+L-1]\right) \\
% \end{bmatrix}.
% \end{align*}
% Under the assumption that $x[n]$ is a wide-sense stationary signal, we can simplify $\mat{C}_x$ even further:
% \begin{align*}
% \mat{C}_x &= E\left[\vec{x}\vec{x}^T\right] \\
% &= \begin{bmatrix} 
% r_x[0] & r_x[1] & \ldots & r_x[L-1] \\
% r_x[1] & r_x[0] & \ldots & r_x[L-2] \\
% \vdots & \vdots & \ddots & \vdots \\
% r_x[L-1] & r_x[L-2] & \ldots & r_x[0] \\
% \end{bmatrix}.
% \end{align*}

% As stated in section \textbf{verwijs}, the autocorrelation function of white noise is a delta function. This implies that if $x[n]$ is white noise with variance $\sigma_n^2$ then $\mat{C}_x = \sigma_n^2\mat{I}$.
% If the signal $x[n]$ is not equal to noise, then its autocorrelation function is not equal to a delta function which results in $\mat{C}_x$ having non-zero off diagonal elements.

% Covariance absolute value method detection uses a measure of this ``diagonality'' of $\mat{C}_x$ as test statistic $\Lambda$.
% This measure $\Lambda$ is defined as
% \begin{align*}
% \Lambda &= \frac{\sum_{n=1}^{L} \sum_{m=1}^L \left|\mat{C}_{nm}\right|}{\sum_{k=1}^L |\mat{C}_{kk}}
% \end{align*} 

% In practice one estimates the matrix $\mat{C}_x$ by using a limited amount of samples $N$ to estimate $r_x[n]$. The threshold given a desired false alarm probability
% $p_{fa}$ is derived in \textbf{cite} to be

% \begin{align*}
% \gamma &= \frac{(1+(L-1)\sqrt{\frac{2}{N\pi}}}{1-Q^{-1}(p_{fa})\sqrt{\frac{2}{N}}}
% \end{align*} 

% The whole process of the CAV detector is depicted in \cref{tkz:cav}.

% \begin{figure}[H]
% \centering
% \begin{tikzpicture}

% \node at (-4,4.5) {$r_x[n]$};

% \draw [>=latex, ->] (-3.5,4.5) -- (-3,4.5);

% \draw (-3,5) rectangle (-2,4) node[pos=.5]{};
% \draw [>=latex, ->] (-2,4.5) -- (-1.5,4.5);

% \draw (-1.5,5) rectangle (-0.5,4) node[pos=.5]{$|\cdot|$};
% \draw [>=latex, ->] (-0.5,4.5) -- (0.5,4.5);

% \draw (2.5,5) rectangle (3.5,4) node[pos=.5]{$_<^ > \gamma$};
% \draw [>=latex, ->] (3.5,4.5) -- (4,4.5);

% \draw (0.5,5) rectangle (1.5,4) node[pos=.5]{$\sum$};
% \draw [>=latex, ->] (1.5,4.5) -- (2.5,4.5);

% \draw (0.5,3.5) rectangle (1.5,2.5) node[pos=.5]{Tr[$\cdot$]};

% \node at (4.8,4.5) {$\mathcal{H}_0 \lor \mathcal{H}_1$};

% \draw (0,4.5) -- (0,3);
% \draw (2,4.5) -- (2,3);
% \draw (2,3) -- (1.5,3);
% \draw [>=latex,->] (0,3) -- (0.5,3);

% \draw (-2.6,4.3) -- (-2.7,4.4);

% \draw (-2.45,4.3) -- (-2.7,4.55);
% \draw (-2.3,4.3) -- (-2.7,4.7);
% \draw (-2.3,4.45) -- (-2.55,4.7);

% \draw (-2.3,4.6) -- (-2.4,4.7);

% \draw (-2.8,4.8) -- (-2.8,4.2);
% \draw (-2.8,4.8) -- (-2.7,4.8);
% \draw (-2.8,4.2) -- (-2.7,4.2);

% \draw (-2.2,4.8) -- (-2.2,4.2);
% \draw (-2.2,4.2) -- (-2.3,4.2);
% \draw (-2.2,4.8) -- (-2.3,4.8);


% \end{tikzpicture}
% \caption{CAV detector}\label{tkz:cav}
% \end{figure}



% % \subimport{detection/preliminaries/}{preliminaries}
% % \subimport{detection/main_analysis/}{main_analysis}
% % \subimport{detection/algorithm/}{algorithm}
% % \subimport{detection/proofs/}{proofs}
% \end{document}
