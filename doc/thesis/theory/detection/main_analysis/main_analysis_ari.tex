%!TEX program = xelatex

\documentclass[a4paper, openany, oneside]{memoir}
\usepackage[no-math]{fontspec}
\usepackage{pgfplots}
\usepackage{float}
\pgfplotsset{compat=newest}
\usepackage{commath}
\usepackage{mathtools}
\usepackage{amssymb}
\usepackage{amsthm}
\usepackage{booktabs}
\usepackage{todonotes}
\usepackage{mathtools}
\usepackage{xcolor}
\usepackage[separate-uncertainty=true, per-mode=symbol]{siunitx}
\usepackage{listings}
\usepackage[american inductor, european resistor]{circuitikz}
\usepackage{amsmath}
\usepackage{amsfonts}
\usepackage{ifxetex}
\usepackage[dutch,english]{babel}
\usepackage[backend=bibtexu,texencoding=utf8,bibencoding=utf8,style=ieee,sortlocale=en_GB,language=auto]{biblatex}
\usepackage[strict,autostyle]{csquotes}
\usepackage{import}
\usepackage{standalone}
\usepackage{bookmark,hyperref}
\usepackage{xcolor,mdframed}
\usepackage{tikz}
\usepackage{framed}
\usepackage{float}
\usepackage{tabularx}
\usepackage{graphicx,adjustbox}
\usepackage{rotating}
\usepackage{pdfpages}
\usepackage{enumitem}
\usepackage{calc}
\usepackage{pgfplots}
\usepackage{filecontents}
\usepackage{caption}
\usepackage{subcaption}
\usepackage{lettrine}

\newcolumntype{Y}{>{\raggedright\arraybackslash}X} % Left-justified text in tabularx environment

\ifxetex{} % Fonts laden in het geval dat je met Xetex compiled
    \usepackage{fontspec}
    \defaultfontfeatures{Scale=MatchLowercase, Ligatures=TeX} % To support LaTeX quoting style
    %\setromanfont{Palatino Linotype} % Tover ergens in Font mapje in root.
    \setsansfont{Avenir Next LT Pro}
    \setromanfont{Adobe Caslon Pro} % Tover ergens in Font mapje in root.
    \setmonofont{Source Code Pro}
\else % Terug val in standaard pdflatex tool chain. Geen ondersteuning voor OTT fonts
    \usepackage[T1]{fontenc}
    \usepackage[utf8]{inputenc}
\fi
\usepackage[noabbrev, capitalize]{cleveref}
\usepackage{ifthen}
\usepackage{titlesec}
\usepackage{titlecaps}

\newcommand{\references}[1]{\begin{flushright}{#1}\end{flushright}}
\renewcommand{\vec}[1]{\boldsymbol{\mathbf{#1}}}
\newcommand{\uvec}[1]{\boldsymbol{\hat{\vec{#1}}}}
\newcommand{\mat}[1]{\boldsymbol{\mathbf{#1}}}
\newcommand{\fasor}[1]{\boldsymbol{\tilde{\vec{#1}}}}
\newcommand{\cmplx}[0]{\mathrm{j}}
\renewcommand{\Re}[0]{\operatorname{Re}}
\newcommand{\Cov}{\operatorname{Cov}}
\newcommand{\Var}{\operatorname{Var}}
\newcommand{\proj}{\operatorname{proj}}
\newcommand{\Perp}{\operatorname{perp}}
\newcommand{\col}{\operatorname{col}}
\newcommand{\rect}{\operatorname{rect}}
\newcommand{\sinc}{\operatorname{sinc}}
\newcommand{\lcm}{\operatorname{lcm}}
%\newcommand{\gcd}{\operatorname{gcd}}
\newcommand{\F}{\mathcal{F}}
\newcommand{\DTFT}{\mathcal{F}_*}
\newcommand{\conj}[1]{#1^*}
\renewcommand{\mod}{\operatorname{mod}}
\newcommand{\rot}{\operatorname{rot}}
\newcommand{\vecsc}[1]{\vec{\textsc{\textbf{#1}}}}
\renewcommand{\ss}[1]{_{#1}}

% Label without linebreak breaker
\newcommand{\lab}[1]{\label{#1}\nolinebreak}

\newtheorem{definition}{Definition}
\newtheorem{theorem}{Theorem}


\DeclareSIUnit{\voltampere}{VA} %apparent power
\DeclareSIUnit{\pii}{\ensuremath{\pi}}

\hypersetup{%setup hyperlinks
    colorlinks,
    citecolor=black,
    filecolor=black,
    linkcolor=black,
    urlcolor=black
}

% Example boxes
\usepackage{fancybox}
\usepackage{framed}
\usepackage{adjustbox}
\newenvironment{simpages}%
{\AtBeginEnvironment{itemize}{\parskip=0pt\parsep=0pt\partopsep=0pt}
\def\FrameCommand{\fboxsep=.5\FrameSep\shadowbox}\MakeFramed{\FrameRestore}}%
{\endMakeFramed}

% Impulse train
\DeclareFontFamily{U}{wncy}{}
\DeclareFontShape{U}{wncy}{m}{n}{<->wncyr10}{}
\DeclareSymbolFont{mcy}{U}{wncy}{m}{n}
\DeclareMathSymbol{\Sha}{\mathord}{mcy}{"58}

\setlength{\parindent}{0pt}
\nonzeroparskip

% Block environment configuration
\newcommand{\BlockLeftMargin}{20pt}
\newcommand{\BlockLeftMarginText}{25pt}
\newcommand{\BlockLeftMarginTextSpacing}{10pt}

% Own colours
\definecolor{gray75}{gray}{0.75}

% Block environment
\newenvironment{block}[3]{%
\makebox{\hspace{-\spinemargin}%
\begin{tikzpicture}[overlay]
    \draw [thick,color=gray75] (\BlockLeftMargin, 0) -- (\paperwidth - \spinemargin, 0);
    \node at (\BlockLeftMarginText, -0.9) [align=left, text width=\spinemargin - \BlockLeftMarginText - \BlockLeftMarginTextSpacing, anchor=west, text depth=1cm] {\textbf{\textsc{#1}}\newline\textit{#3}};
\end{tikzpicture}}%
\nopagebreak\\[0.25em]\ifthenelse{\equal{#2}{}}{}{(\textit{#2}.) }\nopagebreak\nolinebreak}
{\nopagebreak\\[-0.25em]%
\makebox{\hspace{-\spinemargin}%
\begin{tikzpicture}[overlay, remember picture]
    \draw [thick,color=gray75] (\spinemargin,0) -- (\paperwidth - \spinemargin,0);
\end{tikzpicture}} \vspace{0.5em}}

% Theorem
\newcounter{blockTheoremCounter}
\crefname{blockTheoremCounter}{Theorem}{Theorems}
\Crefname{blockTheoremCounter}{Theorem}{Theorems}

\newenvironment{blockTheorem}[1][]{%
\refstepcounter{blockTheoremCounter}%
\begin{block}{theorem \theblockTheoremCounter}{#1}{}}
{\end{block}}

% Definition
\newcounter{blockDefinitionCounter}
\crefname{blockDefinitionCounter}{Definition}{Definitions}
\Crefname{blockDefinitionCounter}{Definition}{Definitions}

\newenvironment{blockDefinition}[1][]{%
\refstepcounter{blockDefinitionCounter}%
\begin{block}{definition \theblockDefinitionCounter}{#1}{}}
{\end{block}}

% Proof
\newcounter{blockProofTheoremCounter}
\crefname{blockProofTheoremCounter}{Proof}{Proofs}
\Crefname{blockProofTheoremCounter}{Proof}{Proofs}

\newenvironment{blockProofTheorem}[1]{%
\refstepcounter{blockProofTheoremCounter}%
\begin{block}{proof of \\ theorem #1}{}{}}
{\qed\end{block}}

% Detail
\newcounter{blockDetailCounter}
\crefname{blockDetailCounter}{Detail}{Details}
\Crefname{blockDetailCounter}{Detail}{Details}

\newenvironment{blockDetail}[1][]{%
\refstepcounter{blockDetailCounter}%
\begin{block}{detail \theblockDetailCounter}{#1}{}}
{\end{block}}

% Redesign chapter headings
\newcommand{\chapternumber}{\thechapter}
\newcommand{\hsp}{\hspace{20pt}}
\titleformat{\chapter}[hang]{\Huge\bfseries}{\chapternumber\hsp\textcolor{gray75}{|}\hsp}{0pt}{\Huge\bfseries}

% Remove headers
% \addtopsmarks{headings}{}{
%   \createmark{chapter}{left}{nonumber}{}{}
% }
% \pagestyle{headings} % Activate changes

% Capitalise headers in a regular way
\renewcommand*{\memUChead}[1]{\titlecap{#1}}

% \hfill for math mode
\newcommand{\pushright}[1]{\intertext{\hfill$\displaystyle #1$}}
\newcommand{\pushline}{\hskip \textwidth minus \textwidth}
\newcommand{\matlab}{\textsc{Matlab}}

\definecolor{code-grey}{HTML}{DDDDDD}
\newcommand{\lib}[1]{\textsf{#1}}
\newcommand{\file}[1]{\textsf{#1}}
\newcommand{\func}[1]{\colorbox{code-grey}{\texttt{#1}}}
\newcommand{\class}[1]{\colorbox{code-grey}{\texttt{#1}}}

% Setup actiepunten
\newenvironment{important}[1][]{%
   \begin{mdframed}[%
      backgroundcolor={red!15}, hidealllines=true,
      skipabove=0.7\baselineskip, skipbelow=0.7\baselineskip,
      splitbottomskip=2pt, splittopskip=4pt, #1]%
   \makebox[0pt]{% ignore the withd of !
      \smash{% ignor the height of !
         \fontsize{32pt}{32pt}\selectfont% make the ! bigger
         \hspace*{-19pt}% move ! to the left
         \raisebox{-2pt}{% move ! up a little
            {\color{red!70!black}\sffamily\bfseries !}% type the bold red !
         }%
      }%
   }%
}{\end{mdframed}}
\newcommand{\excl}[1]{
\begin{important}
  \textbf{#1}
\end{important}
}

\makeatletter
\newcommand\footnoteref[1]{\protected@xdef\@thefnmark{\ref{#1}}\@footnotemark}
\makeatother

% Allow page breaks in display environments
%\allowdisplaybreaks
% S unit for use in Mega Samples per second
\DeclareSIUnit\sample{S}

\newcommand{\CC}{C\nolinebreak\hspace{-.05em}\raisebox{.3ex}{ \textbf{+}}\nolinebreak\hspace{-.10em}\raisebox{.3ex}{\textbf{+}}}
\def\CC{{C\nolinebreak[4]\hspace{-.05em}\raisebox{.3ex}{\textbf{++}}}}


\newcommand{\partauthor}[1]{\gdef\@partauthor{#1}}
\renewcommand{\printparttitle}[1]{
  \parttitlefont #1\\
  \vspace{1.5cm}
  \textnormal{\Large \@partauthor}
}
\addbibresource{../../../../includes/bibliography.bib}

\begin{document}

\section{Main analysis Ariananda}

We assume that if our signal $x[n]$ purely contains noise, the elements of $\vec{x}$ are i.i.d. circular complex gaussian distributed; $(\vec{x})_i \sim \mathcal{CN}(0,\sigma_n)$. 

In this section we will analyze how the elements of the PSD can serve as the test statistic as used in the Neyman-Pearson test. 

\subsection{Distribution of PSD elements}
To employ the Neyman-Pearson theorem to determine the threshold at frequency $\omega$, it is necessary that one knows the distribution of $\mathcal{P}\left(\omega\right)$. Only if this distribution is known, the threshold $\eta_{\omega}$ can correctly be set to yield the desired false alarm probability.

Note that 

\begin{align*}
\mathcal{P}\left(\omega\right) &= \sum_{n=-\infty}^{\infty} r_{x}[n] \exp\left[-jn\omega\right]
\end{align*}

and therefore the distribution of $\mathcal{P}(\omega)$ depends on the distribution of \emph{all} elements of the autocorrelation. Furthermore, the elements of the autocorrelation, can in general not be regarded as independent random variables. 

In the rest of this section we will show that under the assumption that $\vec{x}$ is circular complex gaussian noise and $L-1 \ll KL$, the elements of $\vec{s}_x$ are approximately gaussian distributed.

First notice that if the elements of $\vec{\hat{r}}_x$ have a gaussian distribution, then so will have the elements of $\vec{s}_x$ as they are a linear combination of the elements in $\vec{\hat{r}}_x$.

To determine the distribution of the elements of $\vec{\hat{r}}_x$, we notice that 
\begin{align*}
\vec{\hat{r}}_x &= \mat{R}^{\dagger}\vec{\hat{r}'_y}.
\end{align*}

As $\mat{R}$ is constant, we will focus on $\vec{\hat{r}'_y}$. Again, if the elements of $\vec{\hat{r}}_x$ are to be gaussian distributed, then so are the elements of $\vec{\hat{r}'_y}$.

That is, we'd like to show that the elements of $\vec{\hat{r}}'_{y_i,y_j}$ are approximately gaussian distributed.
\begin{align*}
(\vec{\hat{r}}'_{y_i,y_j})_{u+KL-L} &= \left(\left(\vec{c}_i\ast\vec{x}\right)' \circ \left(\vec{c}_j\ast\vec{x}\right)' \right)_{u+KL-L} \\
&= \sum_{k=1}^{KL} \left(\vec{c}_i\ast\vec{x}\right)'_k \left(\vec{\overline{c}}_j\ast\vec{\overline{x}}\right)'_{KL-(u+KL-L)+k} \\
&= \sum_{k=1}^{KL} \left(\vec{c}_i\ast\vec{x}\right)_{kN} \left(\vec{\overline{c}}_j\ast\vec{\overline{x}}\right)_{kN+KLN-N(u+KL-L)}\\
&=  \sum_{k=1}^{KL} \left(\sum_{l=1}^N \left(\vec{c}_i\right)_l\left(\vec{x}\right)_{kN-l+1}\right) \cdot \left(\sum_{m=1}^{N} \left(\vec{\overline{c}_j}\right)_m\left(\vec{\overline{x}}\right)_{kN - Nu + NL - m+1}\right)\\
&= \sum_{l=1}^N\sum_{m-1}^N \left(\vec{c}_i\right)_l \left(\vec{\overline{c}}_j\right)_m \sum_{k=1}^{KL} (\vec{x})_{kN-l+1} \cdot (\vec{\overline{x}})_{(k-u+L)N - m+1}
\end{align*}

As $\vec{c}_i$ and $\vec{c}_j$ are constant vectors, we will only have to focus on the last term. Notice that if

\begin{enumerate}
	\item $u=L, l=m$, then $\sum_{k=1}^{KL}\vec{x}_{kN-l+1}\vec{\overline{x}}_{kN-l+1}$. This term follows a $\chi^2$ distribution (by definition). As with the convential energy detector, we can approximate the distribution of the sum as a whole as gaussian if $KL$ is large enough.
	\item $l \neq m$, in this case 
	\begin{align*}\{(\vec{x})_{kN-l+1} : 1 \leq k \leq KL\} \cap 
	\{(\vec{x})_{(k-u+L)N - m+1} : 1 \leq k \leq KL\}  &= \emptyset
	\end{align*}
	That is, all products in the sum have absolutely no terms in common. However, as each element of $\vec{x}$ has the same distribution, the product of two distinct elements in $\vec{x}$ will always yield the same distribution (whatever it may be). By the Central Limit Theorem we can approximate the distribution of the sum as a whole as gaussian if $KL$ is large enough.
	\item $u\neq L, l=m$ In this case, the product in the sum contains elements of $\vec{x}$ with a minimum displacement of $N$ in their indices. Unlike case 2, the specified sets \emph{do} have elements in common (but only if $KL \geq N$). To see why, consider Figure TODO TIKZ PRO. Notice, that the dependence of a product in the sum on other products is limited is limited. Therefore, if $KL$ is large enough we can again regard the distribution of the sum as a whole as gaussian.			
\end{enumerate}

To compute the variance of each element of $(\vec{s})_x$, we first observe that the Covariance matrix $C_{sx}$ of $\vec{s}_x$ contains the variance of each element  on the diagonal.

Furthermore as:

\begin{align*}
\vec{s_x} = F
\end{align*}

Using this approximation, it is possible to determine the distribution of the elements  of $\vec{s}_x$: by determining the expected value ($\mu_{\omega}$) and the variance $\sigma_{\omega}^2$ for each element of $\vec{s}_x$, the distribution is completely specified.

Using the Neyman Pearson Test statistic, we obtain that our threshold $\gamma_{\omega}$ given a false alarm probability $p_{fa}$ should be set to $Q^{-1}(p_{fa})\sigma_{\omega} + \mu_{\omega}$.

\end{document}