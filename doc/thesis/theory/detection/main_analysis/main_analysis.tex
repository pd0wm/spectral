%!TEX program = xelatex

\documentclass[a4paper, openany, oneside]{memoir}
\usepackage[no-math]{fontspec}
\usepackage{pgfplots}
\pgfplotsset{compat=newest}
\usepackage{commath}
\usepackage{mathtools}
\usepackage{amssymb}
\usepackage{amsthm}
\usepackage{booktabs}
\usepackage{mathtools}
\usepackage{xcolor}
\usepackage[separate-uncertainty=true, per-mode=symbol]{siunitx}
\usepackage[noabbrev, capitalize]{cleveref}
\usepackage{listings}
\usepackage[american inductor, european resistor]{circuitikz}
\usepackage{amsmath}
\usepackage{amsfonts}
\usepackage{ifxetex}
\usepackage[dutch,english]{babel}
\usepackage[backend=bibtexu,texencoding=utf8,bibencoding=utf8,style=ieee,sortlocale=en_GB,language=auto]{biblatex}
\usepackage[strict,autostyle]{csquotes}
\usepackage{parskip}
\usepackage{import}
\usepackage{standalone}
\usepackage{hyperref}
%\usepackage[toc,title,titletoc]{appendix}

\ifxetex{} % Fonts laden in het geval dat je met Xetex compiled
    \usepackage{fontspec}
    \defaultfontfeatures{Ligatures=TeX} % To support LaTeX quoting style
    \setromanfont{Palatino Linotype} % Tover ergens in Font mapje in root.
    \setmonofont{Source Code Pro}
\else % Terug val in standaard pdflatex tool chain. Geen ondersteuning voor OTT fonts
    \usepackage[T1]{fontenc}
    \usepackage[utf8]{inputenc}
\fi
\newcommand{\references}[1]{\begin{flushright}{#1}\end{flushright}}
\renewcommand{\vec}[1]{\boldsymbol{\mathbf{#1}}}
\newcommand{\uvec}[1]{\boldsymbol{\hat{\vec{#1}}}}
\newcommand{\mat}[1]{\boldsymbol{\mathbf{#1}}}
\newcommand{\fasor}[1]{\boldsymbol{\tilde{\vec{#1}}}}
\newcommand{\cmplx}[0]{\mathrm{j}}
\renewcommand{\Re}[0]{\operatorname{Re}}
\newcommand{\Cov}{\operatorname{Cov}}
\newcommand{\Var}{\operatorname{Var}}
\newcommand{\proj}{\operatorname{proj}}
\newcommand{\Perp}{\operatorname{perp}}
\newcommand{\col}{\operatorname{col}}
\newcommand{\rect}{\operatorname{rect}}
\newcommand{\sinc}{\operatorname{sinc}}
\newcommand{\IT}{\operatorname{IT}}
\newcommand{\F}{\mathcal{F}}

\newtheorem{definition}{Definition}
\newtheorem{theorem}{Theorem}


\DeclareSIUnit{\voltampere}{VA} %apparent power
\DeclareSIUnit{\pii}{\ensuremath{\pi}}

\hypersetup{%setup hyperlinks
    colorlinks,
    citecolor=black,
    filecolor=black,
    linkcolor=black,
    urlcolor=black
}

% Example boxes
\usepackage{fancybox}
\usepackage{framed}
\usepackage{adjustbox}
\newenvironment{simpages}%
{\AtBeginEnvironment{itemize}{\parskip=0pt\parsep=0pt\partopsep=0pt}
\def\FrameCommand{\fboxsep=.5\FrameSep\shadowbox}\MakeFramed{\FrameRestore}}%
{\endMakeFramed}

% Impulse train
\DeclareFontFamily{U}{wncy}{}
\DeclareFontShape{U}{wncy}{m}{n}{<->wncyr10}{}
\DeclareSymbolFont{mcy}{U}{wncy}{m}{n}
\DeclareMathSymbol{\Sha}{\mathord}{mcy}{"58}
\addbibresource{../../../../includes/bibliography.bib}

\begin{document}

\section{Main analysis}

Let $\vec{x} \in \mathbb{C}^{N}$ denote the received signal. 

The algorithm must decide between to hypotheses:

\begin{align}\label{eq:hypotheses}
  \mathcal{H}_0&: \vec{x} = \vec{n}\\
  \mathcal{H}_1&: \vec{x} = \vec{s} + \vec{n}
\end{align}

in which $\vec{n} \in \mathbb{C}^{N} $ denotes a noise signal and $\vec{s} \in \mathbb{C}^{N}$ denotes a signal as transmitted by a primary user.

\subsection{Energy Detection}
We assume that the noise samples are i.i.d. zero mean circular complex gaussian distributed; $(\vec{n})_i \sim \mathcal{CN}(0, \sigma_n^2 \mathbf{I})$. 
We furthermore assume that the samples of the signal $\vec{s}$ can be modelled independtly as circular complex gaussian $\mathcal{CN}(0, \sigma_s^2 \mathbf{I})$. % todo ref
% dat dit aannemelijk maakt That is

\begin{align*}
\vec{x} \sim 
    \begin{cases}
        \mathcal{CN}(0, (\sigma_n^2)\mathbf{I} & \text{under $\mathcal{H}_0$} \\
        \mathcal{CN}(0, (\sigma_s^2 + \sigma_n^2)\mathbf{I} & \text{under $\mathcal{H}_1$}
    \end{cases}
\end{align*} 

The likelihood function $L(\vec{x})$ of the signal $\vec{x}$ is given by:

\begin{align*}
    L(\vec{x}) &= \prod_{i=1}^N f_{(\vec{x})_i}\\
    &= \begin{cases}
        \frac{1}{\pi^N \sigma_n^2} \exp(-\bar{\vec{x}}'\sigma_n^2\mathbf{I}\vec{x}) & \text{under $\mathcal{H}_0$} \\
        \frac{1}{\pi^N \sigma_n^2} \exp(-\bar{\vec{x}}'(\sigma_n^2+\sigma_s^2)\mathbf{I}\vec{x}) & \text{under $\mathcal{H}_1$}
    \end{cases}
\end{align*}

Where $f_{(\vec{x})_i}$ denotes the probability density function of element $i$ in $\vec{x}$.  The test statistic $\Lambda(\vec{x})$ as used in the Neyman Pearson test is then given by:
\begin{align*}
\Lambda(\vec{x}) &=\frac{\frac{1}{\pi^N \sigma_n^2} \exp(-\bar{\vec{x}}'\sigma_n^2\mathbf{I}\vec{x})}{\frac{1}{\pi^N (\sigma_n^2 + \sigma_s^2)} \exp(-\bar{\vec{x}}'(\sigma_n^2+\sigma_s^2)\mathbf{I}\vec{x})}
\end{align*}


By taking the logarithm of $\Lambda(x)$ we obtain a Log Likelihood Ratio test statistic $\Lambda'(x)$, given by

\begin{align*}
\Lambda(\vec{x}) &= \log(\frac{\frac{1}{\pi^N \sigma_n^2} \exp(-\bar{\vec{x}}'\sigma_n^2\mathbf{I}\vec{x})}{\frac{1}{\pi^N (\sigma_n^2 + \sigma_s^2)} \exp(-\bar{\vec{x}}'(\sigma_n^2+\sigma_s^2)\mathbf{I}\vec{x})}) \\
&= \log(\frac{1}{\pi^N \sigma_n^2}) - \log(\frac{1}{\pi^N (\sigma_n^2 + \sigma_s^2)}) +  (\sigma_s^2) \sum_{i=1}^N |x|^2. \\
\end{align*}

Observing that the constants $\sigma_n$ and $\sigma_s$ do not depend on value of the samples, the test statistic 

\begin{align*}
\Lambda''(\vec{x}) &= \sum_{i=1}^N |x|^2
\end{align*} 

is proportional to $\Lambda'(x)$. A detector using the test statistic $\Lambda''(x)$ is  referred to as the \emph{Energy detector}. 

\subsubsection{Threshold}

Without loss of generality, let $\sigma_n = N^2$ \footnote{To determine the threshold $\eta$ for $\sigma_n \neq N^2$, observe that we can normalize $(\vec{x})_i$, by dividing it by its standard deviation $\sigma_n$.}. Under $\mathcal{H}_0$ the following applies\cite{teguigspectrum2014}  

\begin{align}
    \frac{\Lambda''(x)}{N} &= \sum_{i=1}^N \left|\frac{x}{N}\right|^2 \sim \chi^2_{2N}.
\end{align}

% Sensing Throughput Tradeoff in Cognitive Radio, Y. C. Liang

Therefore the false alarm probability $P_{fa}$ is given by \cite{rugini2013small}:

\begin{align*}
    P_{fa} &= P\left[\frac{\Lambda''(\vec{x} > \eta)}{N}\right] \\
        &= 1-F_{2N} (\eta)
\end{align*}

with $F_{2N}$ the CDF of a chi square distribution with $2N$ degrees of freedom. That is, $E[\frac{\Lambda''(x)}{N}] = 2N$ and $\text{Var}[\frac{\Lambda''(x)}{N}] = 4N$.

If $N$ is large enough, we can approximate the test statistics' distribtion by a gaussian distribution, using the Central Limit theorem:

\begin{align*}
F_{2N} \approx 1-Q(\frac{\frac{\Lambda''(x)}{N}-2N}{2\sqrt{N}}).
\end{align*}

$P_{fa}$ can then be approximated as 

 \begin{align}
 P_{fa} = P(\Lambda'(\vec{x}) > \eta) = Q(\frac{\eta -\sigma_n^2}{\sqrt{\frac{2}{N}\sigma^2_n}}).\label{eq:p_fa}
 \end{align} \cite{
 %http://ieeexplore.ieee.org/stamp/stamp.jsp?tp=&arnumber=6061767
 }

 and given a desired $P_{fa}$ the threshold $\eta$ is given by

\begin{align}
 \eta = Q^{-1}(P_{fa}) P(\Lambda'(\vec{x}) > \eta) = Q(\frac{\eta -\sigma_n^2}{\sqrt{\frac{2}{N}\sigma^2_n}}).
 \end{align} \cite{
 %http://ieeexplore.ieee.org/stamp/stamp.jsp?tp=&arnumber=6061767
 }
 

 \subsection{The energy detector in the fourier domain}
In this section we discuss how the test statistic of the energy detector relates to the power spectral density of the signal that serves as input of the detector.

\subsubsection{Power spectral density \& autocorrelation}
Consider a discrete time signal $x[n]$. We define its \emph{expected} power as $E\left[\left|x[n]\right|^2\right]$, which equals the autocorrelation function of $x$ at zero lag. 

By the Wiener-Khinchin Theorem we have that 

 \begin{align*}
S(j\omega) = \sum_{k=-\infty}^{\infty} r_{xx}[k] \exp \left( -j2\pi \omega k) \right)
 \end{align*}

where $r_{xx}[k]$ denotes the autocorrelation function of $x$ at lag $k$ and $S(\omega$ denotes the spectral power density (PSD).
By definition of the PSD 

\begin{align*}
E\left[\left|x[n]\right|^2\right] = \int_{\omega = -\infty}^{\infty} S(j\omega)\text{d}\omega.
\end{align*}

\subsubsection{Detection of Bandlimited signals}
By our assumption that the noise is circular complex gaussian
\begin{align*}
    r_{nn}[k] = \sigma^2_n \delta[k]
\end{align*}
and 
\begin{align*}
    S(j\omega) = \sigma^2_n.
\end{align*}

If just \emph{one} bandlimited signal $s$ is present in the received signal $x$, then under the assumption that it is independent from the noise, we can add the PSD of $s$ to that of the $n$ to obtain the PSD of $x$. If the lowest frequency present in $s$ is $\omega_1$ and the highest frequency $\omega_2$, then, given a bandpass filter with a frequency response 

\begin{align*}
H(j\omega) &= \begin{cases}1 &\text{$\omega_1 \leq |\omega| \leq \omega_2$} \\
0 & \text{elsewhere} \end{cases} 
\end{align*}
the PSD of $x$ is given by
\begin{align}
S_x(j\omega) &= H(j\omega) \cdot S(j\omega) + \sigma_n^2 \cdot \left(-H(j\omega+1) \right).
\end{align}

% Let us now consider the vector $\vec{x} \in \mathbb{C}^{2M+1}$ for which $(\vec{x})_{i+M+1} = x[i]$.
\subsection{The algorithm}

To detect the presence of a signal in a certain frequency band, given the autocorrelation of the received signal, $r_{xx}$,  we propose the following algorithm

\begin{enumerate}   
    \item Estimate the PSD of $x$, $\hat{S}_x(j\Omega)$  by taking the absolute value of the DFT of $r_{xx}$.
    \item Estimate the noise variance $\sigma_n^2$ from the estimated PSD
    \item Summate $S(j\Omega)$ in the desired frequency band.
    \item Add $\sigma_n^2 \cdot W$ to the result of the previous step, where $W$ denotes the bandwidth of
    $x$ minus that of the certain frequency band. 
    \item Compare the result of the previous step with a threshold $\gamma$.
\end{enumerate}

% \begin{align*}
%     \mathcal{P}[k] = \sum_{n=1}^{2N-1} (\vec{x} \circ \vec{x})_n e^{-\omega(n-1)(k-1)}
% \end{align*}

% with $\omega = \frac{2\pi\cdot j}{2N-1}$

% That is, using the inverse fourier transform, we can retrieve Nth element of $\vec{x}\circ \vec{x}$:

% \begin{align*}
% (\vec{x} \circ \vec{x})_N &= \frac{1}{2N-1} \sum_{k=1}^{2N-1}  \mathcal{P}[k] e^{\omega (N-1)(k-1)} \\
% &=  \frac{1}{2N-1} \sum_{k=1}^{2N-1} \sum_{n=1}^{2N-1} \left((\vec{x} \circ \vec{x})_n e^{-\omega(n-1)(k-1)}\right)e^{\omega(N-1)(k-1)} \\
% % &=  \frac{1}{2N-1} \sum_{k=1}^{2N-1} ||x||^2 + \sum_{n=1}^{2N-1} (\vec{x} \circ \vec{x})_n e^{-\omega(n-N)(k-1)} + \overline{(\vec{x} \circ \vec{x})_n e^{-\omega(n-N)(k-1)}} \\
% % &=\frac{1}{2N-1} \sum_{k=1}^{2N-1} ||x||^2 + \sum_{n=1}^{N-1} \Re{(\vec{x} \circ \vec{x})_n e^{-\omega(n-N)(k-1)}}
% \end{align*}

% Note that 
% \begin{align*}
% (\vec{x} \circ \vec{x})_i &= \overline{(\vec{x} \circ \vec{x})}_{i+N} & \text{for $i=1,2 \ldots, N-1$}
% \end{align*}

% \begin{align*}
% \sum_{n=1}^{2N-1} \left((\vec{x} \circ \vec{x})_n e^{-\omega(n-1)(k-1)}\right)e^{\omega(N-1)(k-1)} &= d
% % &=  \frac{1}{2N-1} \sum_{k=1}^{2N-1} ||x||^2 + \sum_{n=1}^{2N-1} (\vec{x} \circ \vec{x})_n e^{-\omega(n-N)(k-1)} + \overline{(\vec{x} \circ \vec{x})_n e^{-\omega(n-N)(k-1)}} \\
% % &=\frac{1}{2N-1} \sum_{k=1}^{2N-1} ||x||^2 + \sum_{n=1}^{N-1} \Re{(\vec{x} \circ \vec{x})_n e^{-\omega(n-N)(k-1)}}
% \end{align*}




\subsection{Estimation of the noise variance}
To determine the threshold for the test statistic, it is necessary that we can estimate the noise variance 

% http://ieeexplore.ieee.org/stamp/stamp.jsp?tp=&arnumber=6809311
% http://ieeexplore.ieee.org/stamp/stamp.jsp?tp=&arnumber=6061767

\end{document}
