%!TEX program = xelatex

\documentclass[a4paper, openany, oneside]{memoir}
\usepackage[no-math]{fontspec}
\usepackage{pgfplots}
\pgfplotsset{compat=newest}
\usepackage{commath}
\usepackage{mathtools}
\usepackage{amssymb}
\usepackage{amsthm}
\usepackage{booktabs}
\usepackage{mathtools}
\usepackage{xcolor}
\usepackage[separate-uncertainty=true, per-mode=symbol]{siunitx}
\usepackage[noabbrev, capitalize]{cleveref}
\usepackage{listings}
\usepackage[american inductor, european resistor]{circuitikz}
\usepackage{amsmath}
\usepackage{amsfonts}
\usepackage{ifxetex}
\usepackage[dutch,english]{babel}
\usepackage[backend=bibtexu,texencoding=utf8,bibencoding=utf8,style=ieee,sortlocale=en_GB,language=auto]{biblatex}
\usepackage[strict,autostyle]{csquotes}
\usepackage{parskip}
\usepackage{import}
\usepackage{standalone}
\usepackage{hyperref}
%\usepackage[toc,title,titletoc]{appendix}

\ifxetex{} % Fonts laden in het geval dat je met Xetex compiled
    \usepackage{fontspec}
    \defaultfontfeatures{Ligatures=TeX} % To support LaTeX quoting style
    \setromanfont{Palatino Linotype} % Tover ergens in Font mapje in root.
    \setmonofont{Source Code Pro}
\else % Terug val in standaard pdflatex tool chain. Geen ondersteuning voor OTT fonts
    \usepackage[T1]{fontenc}
    \usepackage[utf8]{inputenc}
\fi
\newcommand{\references}[1]{\begin{flushright}{#1}\end{flushright}}
\renewcommand{\vec}[1]{\boldsymbol{\mathbf{#1}}}
\newcommand{\uvec}[1]{\boldsymbol{\hat{\vec{#1}}}}
\newcommand{\mat}[1]{\boldsymbol{\mathbf{#1}}}
\newcommand{\fasor}[1]{\boldsymbol{\tilde{\vec{#1}}}}
\newcommand{\cmplx}[0]{\mathrm{j}}
\renewcommand{\Re}[0]{\operatorname{Re}}
\newcommand{\Cov}{\operatorname{Cov}}
\newcommand{\Var}{\operatorname{Var}}
\newcommand{\proj}{\operatorname{proj}}
\newcommand{\Perp}{\operatorname{perp}}
\newcommand{\col}{\operatorname{col}}
\newcommand{\rect}{\operatorname{rect}}
\newcommand{\sinc}{\operatorname{sinc}}
\newcommand{\IT}{\operatorname{IT}}
\newcommand{\F}{\mathcal{F}}

\newtheorem{definition}{Definition}
\newtheorem{theorem}{Theorem}


\DeclareSIUnit{\voltampere}{VA} %apparent power
\DeclareSIUnit{\pii}{\ensuremath{\pi}}

\hypersetup{%setup hyperlinks
    colorlinks,
    citecolor=black,
    filecolor=black,
    linkcolor=black,
    urlcolor=black
}

% Example boxes
\usepackage{fancybox}
\usepackage{framed}
\usepackage{adjustbox}
\newenvironment{simpages}%
{\AtBeginEnvironment{itemize}{\parskip=0pt\parsep=0pt\partopsep=0pt}
\def\FrameCommand{\fboxsep=.5\FrameSep\shadowbox}\MakeFramed{\FrameRestore}}%
{\endMakeFramed}

% Impulse train
\DeclareFontFamily{U}{wncy}{}
\DeclareFontShape{U}{wncy}{m}{n}{<->wncyr10}{}
\DeclareSymbolFont{mcy}{U}{wncy}{m}{n}
\DeclareMathSymbol{\Sha}{\mathord}{mcy}{"58}
\addbibresource{../../includes/bibliography.bib}

\begin{document}

\section{Main analysis}

Let $\vec{x} \in \mathbb{C}^{N}$ denote the received signal. %The input of the algorithm consists of an estimate of $E(\vec{r}_x)$), which we will denote by $\vec{r}'_x$. 

The algorithm must decide between to hypotheses:

\begin{align}\label{eq:hypotheses}
  \mathcal{H}_0&: \vec{x} = \vec{n}\\
  \mathcal{H}_1&: \vec{x} = \vec{s} + \vec{n}
\end{align}

in which $\vec{n}$ denotes a noise signal and $\vec{s}$ denotes the signal
at the receiver. 

% Under the assumption that $\vec{n}$ represents AWGN, we have that $(\vec{n})_i \sim \mathcal{N}(0, \sigma_n^2)$. Furthermore, its autocorrelation, which will be denoted by $\vec{r}_n$, is equal to:

% \begin{align*}
%   \vec{r}_n = \vec{\delta}
% \end{align*}

\subsection{Energy Detection}
Energy detection is based on the \emph{Neyman-Pearson Test} to decide wether there is or isn't a signal present in the received signal $\vec{x}$. Instead of using the likelihood function in the test statistic, energy detection, resorts to the \emph{log-likelihood}. Therefore,  given
the signal $\vec{x}$

\begin{align*}
\Lambda(\vec{x}) &= \frac{\log (\prod_{i=1}^N \frac{1}{\sqrt{2 \pi \sigma_n^2}} \exp ( \frac{(\vec{x})_i^2}{2\sigma_n^2}))}{\log (\prod_{i=1}^N \frac{1}{\sqrt{2 \pi (\sigma_n^2+\sigma_s^2)}} \exp ( \frac{(\vec{x})_i^2}{2(\sigma_n^2 + \sigma_s^2)}))} \\
&= \frac{n}{2}(2\pi\sigma_s^2 ) +( -\frac{1}{2\sigma_n^2} + \frac{1}{2(\sigma_n^2 + \sigma^2_s)})(\vec{x} \cdot \vec{x}) \\
&= a + b  (\vec{x} \cdot \vec{x})
\end{align*}

Observing that the constants $a$ and $b$ do not depend on the signal itself, we can simplify the test statistic to:

\begin{align*}
\Lambda'(\vec{x}) &= (\vec{x} \cdot \vec{x})
\end{align*} 

As stated in \cite{axell2012spectrum}, a constant false alarm rate (CFAR)of the detector is a desirable property. Under $\mathcal{H}_0$, $ \frac{2\Lambda'(\vec{x})}{\sigma_n^2}$ follows a chi-square distribution with 2N degrees of freedom \cite{%http://ieeexplore.ieee.org/stamp/stamp.jsp?tp=&arnumber=6584537.
 }. Approximating this chi-square distribution by a Gaussian, using the Central Limit theorem, it can be found that
 the probability of false alarm is equal to

 \begin{align}
 P_{fa} = P(\Lambda'(\vec{x}) > \eta) = Q(\frac{\eta -\sigma_n^2}{\sqrt{\frac{2}{N}\sigma^2_n}}).\label{eq:p_fa}
 \end{align} \cite{
 %http://ieeexplore.ieee.org/stamp/stamp.jsp?tp=&arnumber=6061767
 }
 Observe the dependency on the noise variance $\sigma_n^2$ in  \cref{eq:p_fa}. That is, to determine $\eta$ for a certain false alarm rate,
 it is necessary that one knows (an estimate of) $\sigma_n^2$.

By dividing $\Lambda'$ by $N$, we create another test statistic $\Lambda'' = E[(x)_i^2] = (r_{(\vec{x})_0})$, where $r_{\vec{x}}$ denotes the autocorrelation of the received signal. We can create an analogue test statistic for samples in the spectral domain by nothing that

\begin{align}
    E[(x)_i^2]  = \int_{-\infty}^{\infty} S_x (f) df
\end{align}

where $S_x(f)$ denotes the power spectral density of the signal. 


\subsection{Estimation of the noise variance}
To determine the threshold for the test statistic, it is necessary that we can estimate the noise variance 

% http://ieeexplore.ieee.org/stamp/stamp.jsp?tp=&arnumber=6809311
% http://ieeexplore.ieee.org/stamp/stamp.jsp?tp=&arnumber=6061767
 

\end{document}