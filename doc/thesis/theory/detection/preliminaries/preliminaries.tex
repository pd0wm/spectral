%!TEX program = xelatex

\documentclass[a4paper, openany, oneside]{memoir}
\usepackage[no-math]{fontspec}
\usepackage{pgfplots}
\pgfplotsset{compat=newest}
\usepackage{commath}
\usepackage{mathtools}
\usepackage{amssymb}
\usepackage{amsthm}
\usepackage{booktabs}
\usepackage{mathtools}
\usepackage{xcolor}
\usepackage[separate-uncertainty=true, per-mode=symbol]{siunitx}
\usepackage[noabbrev, capitalize]{cleveref}
\usepackage{listings}
\usepackage[american inductor, european resistor]{circuitikz}
\usepackage{amsmath}
\usepackage{amsfonts}
\usepackage{ifxetex}
\usepackage[dutch,english]{babel}
\usepackage[backend=bibtexu,texencoding=utf8,bibencoding=utf8,style=ieee,sortlocale=en_GB,language=auto]{biblatex}
\usepackage[strict,autostyle]{csquotes}
\usepackage{parskip}
\usepackage{import}
\usepackage{standalone}
\usepackage{hyperref}
%\usepackage[toc,title,titletoc]{appendix}

\ifxetex{} % Fonts laden in het geval dat je met Xetex compiled
    \usepackage{fontspec}
    \defaultfontfeatures{Ligatures=TeX} % To support LaTeX quoting style
    \setromanfont{Palatino Linotype} % Tover ergens in Font mapje in root.
    \setmonofont{Source Code Pro}
\else % Terug val in standaard pdflatex tool chain. Geen ondersteuning voor OTT fonts
    \usepackage[T1]{fontenc}
    \usepackage[utf8]{inputenc}
\fi
\newcommand{\references}[1]{\begin{flushright}{#1}\end{flushright}}
\renewcommand{\vec}[1]{\boldsymbol{\mathbf{#1}}}
\newcommand{\uvec}[1]{\boldsymbol{\hat{\vec{#1}}}}
\newcommand{\mat}[1]{\boldsymbol{\mathbf{#1}}}
\newcommand{\fasor}[1]{\boldsymbol{\tilde{\vec{#1}}}}
\newcommand{\cmplx}[0]{\mathrm{j}}
\renewcommand{\Re}[0]{\operatorname{Re}}
\newcommand{\Cov}{\operatorname{Cov}}
\newcommand{\Var}{\operatorname{Var}}
\newcommand{\proj}{\operatorname{proj}}
\newcommand{\Perp}{\operatorname{perp}}
\newcommand{\col}{\operatorname{col}}
\newcommand{\rect}{\operatorname{rect}}
\newcommand{\sinc}{\operatorname{sinc}}
\newcommand{\IT}{\operatorname{IT}}
\newcommand{\F}{\mathcal{F}}

\newtheorem{definition}{Definition}
\newtheorem{theorem}{Theorem}


\DeclareSIUnit{\voltampere}{VA} %apparent power
\DeclareSIUnit{\pii}{\ensuremath{\pi}}

\hypersetup{%setup hyperlinks
    colorlinks,
    citecolor=black,
    filecolor=black,
    linkcolor=black,
    urlcolor=black
}

% Example boxes
\usepackage{fancybox}
\usepackage{framed}
\usepackage{adjustbox}
\newenvironment{simpages}%
{\AtBeginEnvironment{itemize}{\parskip=0pt\parsep=0pt\partopsep=0pt}
\def\FrameCommand{\fboxsep=.5\FrameSep\shadowbox}\MakeFramed{\FrameRestore}}%
{\endMakeFramed}

% Impulse train
\DeclareFontFamily{U}{wncy}{}
\DeclareFontShape{U}{wncy}{m}{n}{<->wncyr10}{}
\DeclareSymbolFont{mcy}{U}{wncy}{m}{n}
\DeclareMathSymbol{\Sha}{\mathord}{mcy}{"58}
\addbibresource{../../../../includes/bibliography.bib}

\begin{document}


\section{Preliminaries}

% \begin{blockDefinition}[Delta vector]
% The $N$ dimensional delta vector is represented by a vector $\vec{\delta} \in \mathbb{C}^N$such that:

% \begin{align*}
%   (\vec{\delta})_i &= 1 &&\text{for $i = \lfloor \frac{N}{2} \rfloor$} \\ 
%   (\vec{\delta})_i &= 0 &&\text{elsewhere.}
% \end{align*}
% \end{blockDefinition}
% \begin{blockDefinition}[Element wise multiplication]
% The result of the element wise multiplication of two $N$ dimensional vectors $\vec{a}$ and $\vec{b}$, denoted by

% \begin{align*}
%   \vec{a} \diamond \vec{b}
% \end{align*}
%  is given by a $N$ dimension vector $\vec{c}$ with element $(\vec{c})_i = (\vec{a})_i (\vec{b})_i$. 
% \end{blockDefinition}

\begin{blockDefinition}[Likelihood function under Hypothesis]
Given a hypothesis $\mathcal{H}$ and a realisation $\mathbf{x}$ of a random variable $\mathbf{X}$. Then $f_{\mathbf{X} | \mathcal{H}}(\mathbf{x})$ denotes the likelihood function of $\mathbf{x}$ given $\mathcal{H}$.
\end{blockDefinition}

\begin{blockDefinition}[Neyman-Pearson Test]
Given a continous random vector $\vec{\mathfrak{x}}$, and two hypotheses $\mathcal{H}_0$ and $\mathcal{H}_1$, the Neyman-Pearson test rejects that $\mathbf{x}$ has been produced under $\mathcal{H}_0$ in favor of $\mathcal{H}_1$ when

\begin{align*}
    \Lambda (\mathbf{x}) &= \frac{f_{\vec{\mathfrak{x}} | \mathcal{H}_0} (\mathbf{x})}{f_{(\vec{\mathfrak{x}} | \mathcal{H}_1}(\mathbf{x})} < \eta 
\end{align*}

The decision threshold $\eta$ is chosen such that $P(\Lambda(\vec{\mathfrak{x}}) < \eta) = \alpha$.
\end{blockDefinition}


\end{document}