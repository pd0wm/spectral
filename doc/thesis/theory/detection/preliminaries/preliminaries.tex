%!TEX program = xelatex

\documentclass[a4paper, openany, oneside]{memoir}
\usepackage[no-math]{fontspec}
\usepackage{pgfplots}
\pgfplotsset{compat=newest}
\usepackage{commath}
\usepackage{mathtools}
\usepackage{amssymb}
\usepackage{amsthm}
\usepackage{booktabs}
\usepackage{mathtools}
\usepackage{xcolor}
\usepackage[separate-uncertainty=true, per-mode=symbol]{siunitx}
\usepackage[noabbrev, capitalize]{cleveref}
\usepackage{listings}
\usepackage[american inductor, european resistor]{circuitikz}
\usepackage{amsmath}
\usepackage{amsfonts}
\usepackage{ifxetex}
\usepackage[dutch,english]{babel}
\usepackage[backend=bibtexu,texencoding=utf8,bibencoding=utf8,style=ieee,sortlocale=en_GB,language=auto]{biblatex}
\usepackage[strict,autostyle]{csquotes}
\usepackage{parskip}
\usepackage{import}
\usepackage{standalone}
\usepackage{hyperref}
%\usepackage[toc,title,titletoc]{appendix}

\ifxetex{} % Fonts laden in het geval dat je met Xetex compiled
    \usepackage{fontspec}
    \defaultfontfeatures{Ligatures=TeX} % To support LaTeX quoting style
    \setromanfont{Palatino Linotype} % Tover ergens in Font mapje in root.
    \setmonofont{Source Code Pro}
\else % Terug val in standaard pdflatex tool chain. Geen ondersteuning voor OTT fonts
    \usepackage[T1]{fontenc}
    \usepackage[utf8]{inputenc}
\fi
\newcommand{\references}[1]{\begin{flushright}{#1}\end{flushright}}
\renewcommand{\vec}[1]{\boldsymbol{\mathbf{#1}}}
\newcommand{\uvec}[1]{\boldsymbol{\hat{\vec{#1}}}}
\newcommand{\mat}[1]{\boldsymbol{\mathbf{#1}}}
\newcommand{\fasor}[1]{\boldsymbol{\tilde{\vec{#1}}}}
\newcommand{\cmplx}[0]{\mathrm{j}}
\renewcommand{\Re}[0]{\operatorname{Re}}
\newcommand{\Cov}{\operatorname{Cov}}
\newcommand{\Var}{\operatorname{Var}}
\newcommand{\proj}{\operatorname{proj}}
\newcommand{\Perp}{\operatorname{perp}}
\newcommand{\col}{\operatorname{col}}
\newcommand{\rect}{\operatorname{rect}}
\newcommand{\sinc}{\operatorname{sinc}}
\newcommand{\IT}{\operatorname{IT}}
\newcommand{\F}{\mathcal{F}}

\newtheorem{definition}{Definition}
\newtheorem{theorem}{Theorem}


\DeclareSIUnit{\voltampere}{VA} %apparent power
\DeclareSIUnit{\pii}{\ensuremath{\pi}}

\hypersetup{%setup hyperlinks
    colorlinks,
    citecolor=black,
    filecolor=black,
    linkcolor=black,
    urlcolor=black
}

% Example boxes
\usepackage{fancybox}
\usepackage{framed}
\usepackage{adjustbox}
\newenvironment{simpages}%
{\AtBeginEnvironment{itemize}{\parskip=0pt\parsep=0pt\partopsep=0pt}
\def\FrameCommand{\fboxsep=.5\FrameSep\shadowbox}\MakeFramed{\FrameRestore}}%
{\endMakeFramed}

% Impulse train
\DeclareFontFamily{U}{wncy}{}
\DeclareFontShape{U}{wncy}{m}{n}{<->wncyr10}{}
\DeclareSymbolFont{mcy}{U}{wncy}{m}{n}
\DeclareMathSymbol{\Sha}{\mathord}{mcy}{"58}
\addbibresource{../../../../includes/bibliography.bib}

\begin{document}

\section{Preliminaries}

\begin{blockDefinition}[Likelihood function under Hypothesis]
Given a hypothesis $\mathcal{H}$ and a realisation $\mathbf{x}$ of a random variable $\mathbf{X}$. Then $L_{\mathbf{X} | \mathcal{H}}(\mathbf{x})$ denotes the likelihood function of $\mathbf{x}$ given $\mathcal{H}$.
\end{blockDefinition}

% \begin{blockDefinition}{Deterministic autocorrelation of a discrete signal}
% Given a discrete time signal $y[k]$, its autocorrelation function is defined as

% \begin{align}
%     r_{yy}[k] = \sum_{i=-\infty}^{\infty}y[i]\overline{y}[i-j]
% \end{align}

% \end{blockDefinition}

\begin{blockDefinition}[Generalized Chi-square distribution]
A random variable $\mathbf{X} \in \mathbb{C}$ that follows a chi-square distribution with $N$ degrees of freedom, denoted by $\mathbf{X} \sim \chi^2_N$, is the sum of $N$ independent circular complex normal distributed random variables with zero mean and standard deviation equal to 1. 
\end{blockDefinition}

\begin{blockDefinition}[Neyman-Pearson Test]
Given a continous random vector $\vec{X}$, and two hypotheses $\mathcal{H}_0$ and $\mathcal{H}_1$, the Neyman-Pearson test rejects that the realization of $\vec{X}$, $\vec{x}$, has been produced under $\mathcal{H}_0$ in favor of $\mathcal{H}_1$
when
\begin{align*}
    \Lambda (\mathbf{x}) &= \frac{L_{\vec{X} | \mathcal{H}_0} (\mathbf{x})}{L_{(\vec{X} | \mathcal{H}_1}(\mathbf{x})} > \eta. 
\end{align*}

Where $\eta$, the decision threshold, is chosen such that $P(\Lambda(\vec{x} < \eta) | \mathcal{H}_1)$ is minimized subject to $P(\Lambda(\vec{x}) > \eta) | \mathcal{H}_0) = P_{fa}$, where $P_{fa}$ denotes the false alarm probability. % insert ref naar boek
\end{blockDefinition}

\end{document}
