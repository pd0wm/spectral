%!TEX program = xelatex

\documentclass[a4paper, openany, oneside]{memoir}
\usepackage[no-math]{fontspec}
\usepackage{pgfplots}
\usepackage{float}
\pgfplotsset{compat=newest}
\usepackage{commath}
\usepackage{mathtools}
\usepackage{amssymb}
\usepackage{amsthm}
\usepackage{booktabs}
\usepackage{todonotes}
\usepackage{mathtools}
\usepackage{xcolor}
\usepackage[separate-uncertainty=true, per-mode=symbol]{siunitx}
\usepackage{listings}
\usepackage[american inductor, european resistor]{circuitikz}
\usepackage{amsmath}
\usepackage{amsfonts}
\usepackage{ifxetex}
\usepackage[dutch,english]{babel}
\usepackage[backend=bibtexu,texencoding=utf8,bibencoding=utf8,style=ieee,sortlocale=en_GB,language=auto]{biblatex}
\usepackage[strict,autostyle]{csquotes}
\usepackage{import}
\usepackage{standalone}
\usepackage{bookmark,hyperref}
\usepackage{xcolor,mdframed}
\usepackage{tikz}
\usepackage{framed}
\usepackage{float}
\usepackage{tabularx}
\usepackage{graphicx,adjustbox}
\usepackage{rotating}
\usepackage{pdfpages}
\usepackage{enumitem}
\usepackage{calc}
\usepackage{pgfplots}
\usepackage{filecontents}
\usepackage{caption}
\usepackage{subcaption}
\usepackage{lettrine}

\newcolumntype{Y}{>{\raggedright\arraybackslash}X} % Left-justified text in tabularx environment

\ifxetex{} % Fonts laden in het geval dat je met Xetex compiled
    \usepackage{fontspec}
    \defaultfontfeatures{Scale=MatchLowercase, Ligatures=TeX} % To support LaTeX quoting style
    %\setromanfont{Palatino Linotype} % Tover ergens in Font mapje in root.
    \setsansfont{Avenir Next LT Pro}
    \setromanfont{Adobe Caslon Pro} % Tover ergens in Font mapje in root.
    \setmonofont{Source Code Pro}
\else % Terug val in standaard pdflatex tool chain. Geen ondersteuning voor OTT fonts
    \usepackage[T1]{fontenc}
    \usepackage[utf8]{inputenc}
\fi
\usepackage[noabbrev, capitalize]{cleveref}
\usepackage{ifthen}
\usepackage{titlesec}
\usepackage{titlecaps}

\newcommand{\references}[1]{\begin{flushright}{#1}\end{flushright}}
\renewcommand{\vec}[1]{\boldsymbol{\mathbf{#1}}}
\newcommand{\uvec}[1]{\boldsymbol{\hat{\vec{#1}}}}
\newcommand{\mat}[1]{\boldsymbol{\mathbf{#1}}}
\newcommand{\fasor}[1]{\boldsymbol{\tilde{\vec{#1}}}}
\newcommand{\cmplx}[0]{\mathrm{j}}
\renewcommand{\Re}[0]{\operatorname{Re}}
\newcommand{\Cov}{\operatorname{Cov}}
\newcommand{\Var}{\operatorname{Var}}
\newcommand{\proj}{\operatorname{proj}}
\newcommand{\Perp}{\operatorname{perp}}
\newcommand{\col}{\operatorname{col}}
\newcommand{\rect}{\operatorname{rect}}
\newcommand{\sinc}{\operatorname{sinc}}
\newcommand{\lcm}{\operatorname{lcm}}
%\newcommand{\gcd}{\operatorname{gcd}}
\newcommand{\F}{\mathcal{F}}
\newcommand{\DTFT}{\mathcal{F}_*}
\newcommand{\conj}[1]{#1^*}
\renewcommand{\mod}{\operatorname{mod}}
\newcommand{\rot}{\operatorname{rot}}
\newcommand{\vecsc}[1]{\vec{\textsc{\textbf{#1}}}}
\renewcommand{\ss}[1]{_{#1}}

% Label without linebreak breaker
\newcommand{\lab}[1]{\label{#1}\nolinebreak}

\newtheorem{definition}{Definition}
\newtheorem{theorem}{Theorem}


\DeclareSIUnit{\voltampere}{VA} %apparent power
\DeclareSIUnit{\pii}{\ensuremath{\pi}}

\hypersetup{%setup hyperlinks
    colorlinks,
    citecolor=black,
    filecolor=black,
    linkcolor=black,
    urlcolor=black
}

% Example boxes
\usepackage{fancybox}
\usepackage{framed}
\usepackage{adjustbox}
\newenvironment{simpages}%
{\AtBeginEnvironment{itemize}{\parskip=0pt\parsep=0pt\partopsep=0pt}
\def\FrameCommand{\fboxsep=.5\FrameSep\shadowbox}\MakeFramed{\FrameRestore}}%
{\endMakeFramed}

% Impulse train
\DeclareFontFamily{U}{wncy}{}
\DeclareFontShape{U}{wncy}{m}{n}{<->wncyr10}{}
\DeclareSymbolFont{mcy}{U}{wncy}{m}{n}
\DeclareMathSymbol{\Sha}{\mathord}{mcy}{"58}

\setlength{\parindent}{0pt}
\nonzeroparskip

% Block environment configuration
\newcommand{\BlockLeftMargin}{20pt}
\newcommand{\BlockLeftMarginText}{25pt}
\newcommand{\BlockLeftMarginTextSpacing}{10pt}

% Own colours
\definecolor{gray75}{gray}{0.75}

% Block environment
\newenvironment{block}[3]{%
\makebox{\hspace{-\spinemargin}%
\begin{tikzpicture}[overlay]
    \draw [thick,color=gray75] (\BlockLeftMargin, 0) -- (\paperwidth - \spinemargin, 0);
    \node at (\BlockLeftMarginText, -0.9) [align=left, text width=\spinemargin - \BlockLeftMarginText - \BlockLeftMarginTextSpacing, anchor=west, text depth=1cm] {\textbf{\textsc{#1}}\newline\textit{#3}};
\end{tikzpicture}}%
\nopagebreak\\[0.25em]\ifthenelse{\equal{#2}{}}{}{(\textit{#2}.) }\nopagebreak\nolinebreak}
{\nopagebreak\\[-0.25em]%
\makebox{\hspace{-\spinemargin}%
\begin{tikzpicture}[overlay, remember picture]
    \draw [thick,color=gray75] (\spinemargin,0) -- (\paperwidth - \spinemargin,0);
\end{tikzpicture}} \vspace{0.5em}}

% Theorem
\newcounter{blockTheoremCounter}
\crefname{blockTheoremCounter}{Theorem}{Theorems}
\Crefname{blockTheoremCounter}{Theorem}{Theorems}

\newenvironment{blockTheorem}[1][]{%
\refstepcounter{blockTheoremCounter}%
\begin{block}{theorem \theblockTheoremCounter}{#1}{}}
{\end{block}}

% Definition
\newcounter{blockDefinitionCounter}
\crefname{blockDefinitionCounter}{Definition}{Definitions}
\Crefname{blockDefinitionCounter}{Definition}{Definitions}

\newenvironment{blockDefinition}[1][]{%
\refstepcounter{blockDefinitionCounter}%
\begin{block}{definition \theblockDefinitionCounter}{#1}{}}
{\end{block}}

% Proof
\newcounter{blockProofTheoremCounter}
\crefname{blockProofTheoremCounter}{Proof}{Proofs}
\Crefname{blockProofTheoremCounter}{Proof}{Proofs}

\newenvironment{blockProofTheorem}[1]{%
\refstepcounter{blockProofTheoremCounter}%
\begin{block}{proof of \\ theorem #1}{}{}}
{\qed\end{block}}

% Detail
\newcounter{blockDetailCounter}
\crefname{blockDetailCounter}{Detail}{Details}
\Crefname{blockDetailCounter}{Detail}{Details}

\newenvironment{blockDetail}[1][]{%
\refstepcounter{blockDetailCounter}%
\begin{block}{detail \theblockDetailCounter}{#1}{}}
{\end{block}}

% Redesign chapter headings
\newcommand{\chapternumber}{\thechapter}
\newcommand{\hsp}{\hspace{20pt}}
\titleformat{\chapter}[hang]{\Huge\bfseries}{\chapternumber\hsp\textcolor{gray75}{|}\hsp}{0pt}{\Huge\bfseries}

% Remove headers
% \addtopsmarks{headings}{}{
%   \createmark{chapter}{left}{nonumber}{}{}
% }
% \pagestyle{headings} % Activate changes

% Capitalise headers in a regular way
\renewcommand*{\memUChead}[1]{\titlecap{#1}}

% \hfill for math mode
\newcommand{\pushright}[1]{\intertext{\hfill$\displaystyle #1$}}
\newcommand{\pushline}{\hskip \textwidth minus \textwidth}
\newcommand{\matlab}{\textsc{Matlab}}

\definecolor{code-grey}{HTML}{DDDDDD}
\newcommand{\lib}[1]{\textsf{#1}}
\newcommand{\file}[1]{\textsf{#1}}
\newcommand{\func}[1]{\colorbox{code-grey}{\texttt{#1}}}
\newcommand{\class}[1]{\colorbox{code-grey}{\texttt{#1}}}

% Setup actiepunten
\newenvironment{important}[1][]{%
   \begin{mdframed}[%
      backgroundcolor={red!15}, hidealllines=true,
      skipabove=0.7\baselineskip, skipbelow=0.7\baselineskip,
      splitbottomskip=2pt, splittopskip=4pt, #1]%
   \makebox[0pt]{% ignore the withd of !
      \smash{% ignor the height of !
         \fontsize{32pt}{32pt}\selectfont% make the ! bigger
         \hspace*{-19pt}% move ! to the left
         \raisebox{-2pt}{% move ! up a little
            {\color{red!70!black}\sffamily\bfseries !}% type the bold red !
         }%
      }%
   }%
}{\end{mdframed}}
\newcommand{\excl}[1]{
\begin{important}
  \textbf{#1}
\end{important}
}

\makeatletter
\newcommand\footnoteref[1]{\protected@xdef\@thefnmark{\ref{#1}}\@footnotemark}
\makeatother

% Allow page breaks in display environments
%\allowdisplaybreaks
% S unit for use in Mega Samples per second
\DeclareSIUnit\sample{S}

\newcommand{\CC}{C\nolinebreak\hspace{-.05em}\raisebox{.3ex}{ \textbf{+}}\nolinebreak\hspace{-.10em}\raisebox{.3ex}{\textbf{+}}}
\def\CC{{C\nolinebreak[4]\hspace{-.05em}\raisebox{.3ex}{\textbf{++}}}}


\newcommand{\partauthor}[1]{\gdef\@partauthor{#1}}
\renewcommand{\printparttitle}[1]{
  \parttitlefont #1\\
  \vspace{1.5cm}
  \textnormal{\Large \@partauthor}
}
\addbibresource{../../../../includes/bibliography.bib}

\begin{document}

\section{Preliminaries}

\begin{blockDefinition}[Likelihood function under Hypothesis]
Given a hypothesis $\mathcal{H}$ and a realisation $\mathbf{x}$ of a random variable $\mathbf{X}$, then $L_{\mathbf{X} | \mathcal{H}}(\mathbf{x})$ denotes the likelihood function of $\mathbf{x}$ given $\mathcal{H}$.
\end{blockDefinition}

\begin{blockDefinition}[Complex Gaussian Random Variable]
Given the complex gaussian random variable $Z = X + jY$, then $Z$ is said to be complex gaussian distributed if $X$ and $Y$ are jointly-gaussian distributed.
\end{blockDefinition}

\begin{blockDefinition}[Circular Complex Gaussian Random Variable]
Given a complex gaussian random variable $Z = X + jY$, then $Z$ is circular complex gaussian distributed if $Z$ has the same distribution as $Ze^{j\theta}$ with $\theta \in \mathbb{R}$. Since
\begin{align*}
	E\left[Z\right] = E\left[e^{j\theta}Z\right]  = e^{j\theta}E\left[Z\right] 
\end{align*}
and
\begin{align*}
	E\left[ZZ\right] = E\left[e^{j\theta}Z e^j\theta{Z}\right]  = e^{2j\theta}E\left[Z^2\right] 
\end{align*}
it follows that $Z$  has an expectation value of zero ($E\left[Z\right] = 0$) and that %$E\left[Z^2\right]= E\left[X^2 - Y^2 + 2jXY] = 0$.  
This implies that $E\left[X\right] = E\left[Y\right] = 0$ and that $\text{Var}\left(X\right) = \text{Var}\left(Y\right)$. That is, $X$ and $Y$ are gaussian distributed
with zero mean and equal variance. 
A circular complex gaussian random variable $Z$ is referred to as $Z \sim \mathcal{CN}(0,\sigma^2)$ with $\sigma^2= E\left[Z\overline{Z}\right] = \text{Var}\left(X\right) + \text{Var}\left(Y\right)$. 
\end{blockDefinition}

\begin{blockDefinition}[Circular Complex Gaussian Random Vector]
Given the random vector $\vec{Z} \in \mathbb{C}^N$, then $\vec{Z} = \vec{X} + j\vec{Y}$ is a circular complex random vector if $\vec{X}$ and $\vec{Y}$ are jointly gaussian distributed and $\vec{Z}$ has the same distribution as $e^{j\theta}\vec{Z}$ with $\theta \in \mathbb{R}$. Similar to a circular complex gaussian variable, $E\left[\vec{Z}\right] = 0$
and $E\left[\vec{Z}\vec{Z}^T\right] = 0$. A circular complex gaussian random vector $\vec{Z}$ is referred to as $\vec{Z} \sim \mathcal{CN}(0,\mathbf{\Gamma})$ with $\mathbf{\Gamma} = E\left[\vec{Z}\overline{\vec{Z}} \right]$. For the standard circular complex gaussian random vector, $\mathbf{\Gamma} = \mathbf{I}$. 

The probability density function of a circular complex gaussian vector $\vec{Z}\in \mathbb{C}^N$ is given by:

\begin{align*}
	\frac{1}{\pi^N \text{det}(\mathbf{\Gamma})} \exp \left(-\overline{\vec{z}}^T \mathbf{\Gamma}^{-1}\vec{z}\right)
\end{align*}
\end{blockDefinition}

\begin{blockDefinition}[Chi-square distribution for complex random variables]
Given a random vector $\vec{Z} \in \mathbb{C}^N$, with $\vec{Z}i \sim \mathcal{CN}(0, 2\mathbf{I})$ then the random variable $\mathbf{X}$ defined as

\begin{align*}
	\mathbf{X} &= \sum_{n=1}^N \left|(\vec{Z})_n\right|^2 % http://dsp-book.narod.ru/DSPMW/60.PDF
\end{align*}

% % https://books.google.nl/books?id=KwkgAwAAQBAJ&pg=PA158&lpg=PA158&dq=circular+complex+gaussian+chi+square&source=bl&ots=9e7czQCFaN&sig=yuMXCjiFC21c_0EgmSM_yzefVFk&hl=nl&sa=X&ei=MbB2VfTXBsizswGi34DgCw&ved=0CFIQ6AEwBg#v=onepage&q=circular%20complex%20gaussian%20chi%20square&f=false

% % https://books.google.nl/books?id=ERLrAQAAQBAJ&pg=PA145&dq=circular+complex+gaussian&hl=nl&sa=X&ei=YLd2Vc7HLMmmsgHhx4TgCQ&ved=0CCEQ6AEwAA#v=onepage&q=circular%20complex%20gaussian&f=false

% % http://lib.tkk.fi/Diss/2010/isbn9789526030319/article6.pdf

% % https://www.ee.iitb.ac.in/~sarva/courses/EE703/2013/Slides/CircularlySymmetricGaussian.pdf

% % http://www.ifp.illinois.edu/~pramodv/Chapters_PDF/Fundamentals_Wireless_Communication_AppendixA.pdf

follows a chi-square distribution with $2N$ degrees of freedom, denoted by $\mathbf{X} \sim \chi^2_{2N}$.
$E[X] = 2N$ and $\text{Var}[X] = 4N$.
\end{blockDefinition}

\begin{blockDefinition}[Neyman-Pearson Test]
Given a continous random vector $\vec{X}$, and two hypotheses $\mathcal{H}_0$ and $\mathcal{H}_1$, the Neyman-Pearson test rejects that the realization of $\vec{X}$, $\vec{x}$, has been produced under $\mathcal{H}_0$ in favor of $\mathcal{H}_1$
when
\begin{align*}
    \Lambda (\mathbf{x}) &= \frac{L_{\vec{X} | \mathcal{H}_0} (\mathbf{x})}{L_{(\vec{X} | \mathcal{H}_1}(\mathbf{x})} > \eta. 
\end{align*}
Where $\eta$, the decision threshold, is chosen such that $P(\Lambda(\vec{x} < \eta) | \mathcal{H}_1)$ is minimized subject to $P(\Lambda(\vec{x}) > \eta) | \mathcal{H}_0) = P_{fa}$, where $P_{fa}$ denotes the false alarm probability. % insert ref naar boek
\end{blockDefinition}

\begin{blockDefinition}[Covariance Matrix]
Given a vector $\vec{X}$, its covariance matrix is defined as $C_{X} = E\left[XX^H\right]-E\left[X\right]E\left[X^H\right]$
\end{blockDefinition}
\end{document}
