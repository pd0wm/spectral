%!TEX program = xelatex

\documentclass[a4paper, openany, oneside]{memoir}
\usepackage[no-math]{fontspec}
\usepackage{pgfplots}
\pgfplotsset{compat=newest}
\usepackage{commath}
\usepackage{mathtools}
\usepackage{amssymb}
\usepackage{amsthm}
\usepackage{booktabs}
\usepackage{mathtools}
\usepackage{xcolor}
\usepackage[separate-uncertainty=true, per-mode=symbol]{siunitx}
\usepackage[noabbrev, capitalize]{cleveref}
\usepackage{listings}
\usepackage[american inductor, european resistor]{circuitikz}
\usepackage{amsmath}
\usepackage{amsfonts}
\usepackage{ifxetex}
\usepackage[dutch,english]{babel}
\usepackage[backend=bibtexu,texencoding=utf8,bibencoding=utf8,style=ieee,sortlocale=en_GB,language=auto]{biblatex}
\usepackage[strict,autostyle]{csquotes}
\usepackage{parskip}
\usepackage{import}
\usepackage{standalone}
\usepackage{hyperref}
%\usepackage[toc,title,titletoc]{appendix}

\ifxetex{} % Fonts laden in het geval dat je met Xetex compiled
    \usepackage{fontspec}
    \defaultfontfeatures{Ligatures=TeX} % To support LaTeX quoting style
    \setromanfont{Palatino Linotype} % Tover ergens in Font mapje in root.
    \setmonofont{Source Code Pro}
\else % Terug val in standaard pdflatex tool chain. Geen ondersteuning voor OTT fonts
    \usepackage[T1]{fontenc}
    \usepackage[utf8]{inputenc}
\fi
\newcommand{\references}[1]{\begin{flushright}{#1}\end{flushright}}
\renewcommand{\vec}[1]{\boldsymbol{\mathbf{#1}}}
\newcommand{\uvec}[1]{\boldsymbol{\hat{\vec{#1}}}}
\newcommand{\mat}[1]{\boldsymbol{\mathbf{#1}}}
\newcommand{\fasor}[1]{\boldsymbol{\tilde{\vec{#1}}}}
\newcommand{\cmplx}[0]{\mathrm{j}}
\renewcommand{\Re}[0]{\operatorname{Re}}
\newcommand{\Cov}{\operatorname{Cov}}
\newcommand{\Var}{\operatorname{Var}}
\newcommand{\proj}{\operatorname{proj}}
\newcommand{\Perp}{\operatorname{perp}}
\newcommand{\col}{\operatorname{col}}
\newcommand{\rect}{\operatorname{rect}}
\newcommand{\sinc}{\operatorname{sinc}}
\newcommand{\IT}{\operatorname{IT}}
\newcommand{\F}{\mathcal{F}}

\newtheorem{definition}{Definition}
\newtheorem{theorem}{Theorem}


\DeclareSIUnit{\voltampere}{VA} %apparent power
\DeclareSIUnit{\pii}{\ensuremath{\pi}}

\hypersetup{%setup hyperlinks
    colorlinks,
    citecolor=black,
    filecolor=black,
    linkcolor=black,
    urlcolor=black
}

% Example boxes
\usepackage{fancybox}
\usepackage{framed}
\usepackage{adjustbox}
\newenvironment{simpages}%
{\AtBeginEnvironment{itemize}{\parskip=0pt\parsep=0pt\partopsep=0pt}
\def\FrameCommand{\fboxsep=.5\FrameSep\shadowbox}\MakeFramed{\FrameRestore}}%
{\endMakeFramed}

% Impulse train
\DeclareFontFamily{U}{wncy}{}
\DeclareFontShape{U}{wncy}{m}{n}{<->wncyr10}{}
\DeclareSymbolFont{mcy}{U}{wncy}{m}{n}
\DeclareMathSymbol{\Sha}{\mathord}{mcy}{"58}
\addbibresource{../../includes/bibliography.bib}

\begin{document}

\chapter{System evaluation}
\section{Introduction}

\cref{cha:sampling}, \cref{cha:reconstruction}, \cref{cha:sampling_methods}, \cref{cha:detection} have analysed and described several
sampling, reconstruction and detection techniques. It is in this chapter that we will evaluate their performance such that a discussion and a conclusion can be draw based on the specifications as listed in ???.

\section{Overview}
A block diagram of the complete system is depicted in ???

\section{Testing}

To evaluate the performance of our system we will 
\begin{enumerate}
	\item evaluate the performance of the reconstruction and feeding it an artificial constructed signal \textbf{wat zullen we hiermee doen? allemaal bandjes maken en dezelfde ook de detector in mikken?} while

	\begin{enumerate}
		\item varying the sampling technique used to sample the signal (as described in \cref{cha:sampling_methods}).
		\item varying the compression rate and the oversampling factor (as described in \cref{cha:reconstruction}).
	\end{enumerate}

	By comparing the reconstructed the power spectral density to the power spectral density recovered directly from the constructed test signal, we can evaluate the correctness of reconstruction when used with a specific sampling technique.
	Furthermore the dependence on the oversampling factor and
	the compression factor can be evaluated from those test results. 
	
	\item Adress the performance of detection on the reconstructed signal. The energy detector as described in \cref{ssec:ari_ed} will be evaluated by testing on an artificial input signal. 

	This is signal is constructed by generating complex white noise which is filtered using (\textbf{several???}) bandpass filter. To this filtered signal we add additive white noise. This signal is then artificially sampled and fed to the reconstructor. 
	\textbf{plaatje welkom denk ik?}
	
	\begin{enumerate}
		\item varying the signal-to-noise ratio of the input signal; 
		\item varying the false alarm probability the detector 
		should allow;
		\item varying the compression rate of the reconstructor.
	\end{enumerate}
	Using these results we can construct the receiver operating characteristic curves (ROC-curves) which plot the detection probability versus the false alarm probability of the detector. 
	From these curves the minimum signal strength and the detectors performance on a random signal can be evaluated. Furthermore the detection performance of the system as a whole can be evaluated by comparing these curves for different compression rates.
	\end{enumerate}

\section{Discussion}
From the results we see that insert hier uw discussie.

\section{Conclusion}
het werkt. Future work -> collaborative detection/prior information/neural netwerk ding?

\end{document}
