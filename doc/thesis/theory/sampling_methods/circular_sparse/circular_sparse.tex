%!TEX program = xelatex

\documentclass[a4paper, openany, oneside]{memoir}
\usepackage[no-math]{fontspec}
\usepackage{pgfplots}
\pgfplotsset{compat=newest}
\usepackage{commath}
\usepackage{mathtools}
\usepackage{amssymb}
\usepackage{amsthm}
\usepackage{booktabs}
\usepackage{mathtools}
\usepackage{xcolor}
\usepackage[separate-uncertainty=true, per-mode=symbol]{siunitx}
\usepackage[noabbrev, capitalize]{cleveref}
\usepackage{listings}
\usepackage[american inductor, european resistor]{circuitikz}
\usepackage{amsmath}
\usepackage{amsfonts}
\usepackage{ifxetex}
\usepackage[dutch,english]{babel}
\usepackage[backend=bibtexu,texencoding=utf8,bibencoding=utf8,style=ieee,sortlocale=en_GB,language=auto]{biblatex}
\usepackage[strict,autostyle]{csquotes}
\usepackage{parskip}
\usepackage{import}
\usepackage{standalone}
\usepackage{hyperref}
%\usepackage[toc,title,titletoc]{appendix}

\ifxetex{} % Fonts laden in het geval dat je met Xetex compiled
    \usepackage{fontspec}
    \defaultfontfeatures{Ligatures=TeX} % To support LaTeX quoting style
    \setromanfont{Palatino Linotype} % Tover ergens in Font mapje in root.
    \setmonofont{Source Code Pro}
\else % Terug val in standaard pdflatex tool chain. Geen ondersteuning voor OTT fonts
    \usepackage[T1]{fontenc}
    \usepackage[utf8]{inputenc}
\fi
\newcommand{\references}[1]{\begin{flushright}{#1}\end{flushright}}
\renewcommand{\vec}[1]{\boldsymbol{\mathbf{#1}}}
\newcommand{\uvec}[1]{\boldsymbol{\hat{\vec{#1}}}}
\newcommand{\mat}[1]{\boldsymbol{\mathbf{#1}}}
\newcommand{\fasor}[1]{\boldsymbol{\tilde{\vec{#1}}}}
\newcommand{\cmplx}[0]{\mathrm{j}}
\renewcommand{\Re}[0]{\operatorname{Re}}
\newcommand{\Cov}{\operatorname{Cov}}
\newcommand{\Var}{\operatorname{Var}}
\newcommand{\proj}{\operatorname{proj}}
\newcommand{\Perp}{\operatorname{perp}}
\newcommand{\col}{\operatorname{col}}
\newcommand{\rect}{\operatorname{rect}}
\newcommand{\sinc}{\operatorname{sinc}}
\newcommand{\IT}{\operatorname{IT}}
\newcommand{\F}{\mathcal{F}}

\newtheorem{definition}{Definition}
\newtheorem{theorem}{Theorem}


\DeclareSIUnit{\voltampere}{VA} %apparent power
\DeclareSIUnit{\pii}{\ensuremath{\pi}}

\hypersetup{%setup hyperlinks
    colorlinks,
    citecolor=black,
    filecolor=black,
    linkcolor=black,
    urlcolor=black
}

% Example boxes
\usepackage{fancybox}
\usepackage{framed}
\usepackage{adjustbox}
\newenvironment{simpages}%
{\AtBeginEnvironment{itemize}{\parskip=0pt\parsep=0pt\partopsep=0pt}
\def\FrameCommand{\fboxsep=.5\FrameSep\shadowbox}\MakeFramed{\FrameRestore}}%
{\endMakeFramed}

% Impulse train
\DeclareFontFamily{U}{wncy}{}
\DeclareFontShape{U}{wncy}{m}{n}{<->wncyr10}{}
\DeclareSymbolFont{mcy}{U}{wncy}{m}{n}
\DeclareMathSymbol{\Sha}{\mathord}{mcy}{"58}
\addbibresource{../../../../includes/bibliography.bib}

\begin{document}

\section{Sampling and reconstruction}
In this previous section we discussed three multi-coset sampling methods. In \cref{sec:reconstruction-algorithm}, we saw that the sampling method determines the configuration, which is the way the samplers sample the input signal. The configuration then yields $c_i[n]$. Remember that the sampling signal $c_i[n]$ represents the way sampler $i$ samples the input signal. If these $c_i[n]$ satisfy the requirements stated in \cref{sec:reconstruction-implementation}, then the algorithm described in \cref{sec:reconstruction-algorithm} can be used to reconstruct the power spectral density of the sampled signal. We will reconsider these requirements one by one.

First, every $c_i[n]=0$ for $n < 0$ and $n > N-1$. Since $c_i[n]=1$ if coset $i$ samples the $N-n$'th sample of every group of $N$ samples of the input signal, which was discussed in \cref{sec:reconstruction-algorithm}, this requirement is satisfied.

Second, all $c_i[n]=1$ for a single $n=n_i$ and otherwise zero. Every coset $i$ samples a single sample of every group of $N$ samples of the input signal. Therefore, this requirement is also satisfied. It is very important to notice this requirement in conjuction with the first requirement tells us that $n_i$ determine the offsets of the decimations of the different cosets.

Finally, the the circular sparse ruler problem, stated below, must be satisfied.

\begin{description}
    \item[Circular sparse ruler problem] Consider $n_i$ for every coset $i$ as described in the second requirement. The set consisting of $|n_i - n_j|$ and $N-|n_i-n_j|$ for all combinations of cosets $i$ and $j$ must make up $0,\ldots,N-1$.
\end{description}

Let $d=|n_i - n_j|$ or $d=N-|n_i-n_j|$ for a combination of cosets $i$ and $j$. Equivalently, $d$ is the integer difference between the offsets of the decimations of cosets $i$ and $j$. We now say that \textit{lag $d$ is estimated}. We will use this terminology throughout the chapter. The circular sparse ruler problem can be reformulated as follows.

\begin{description}
    \item[Circular sparse ruler problem (alternative formulation)] Lags $0,\ldots,N-1$ must be estimated.
\end{description}

We have abstracted the satisfaction of the requirements necessary for reconstruction to the alternative formulation of the circular sparse ruler problem. The circular sparse ruler problem is treated in a more abstract way in \cref{ap:circ-ruler}. From now on, we assume that we have solutions to the circular sparse ruler problem readily available.

\subsection{Circular sparse ruler sampling}\label{sub:ci-circ}
Since the solutions to the circular sparse ruler problem are readily available, we can choose all $c_i[n]$ such that the problem is satisfied. Then all requirements stated in \cref{sec:reconstruction-implementation} are satisfied and the sampled signal can be reconstructed. It is important to note that $c_i[n]$ is defined for $i = 1,\ldots,M$, which was explained in \cref{sec:reconstruction-algorithm}. Therefore, the solution to the circular sparse ruler problem yields a value for $M$. The parameter $M$ determines the amount of samplers which are required. Since we want to minimise the amount of samplers, we should choose a solution to the circular sparse ruler problem such that $M$ is minimal. This is discussed in \cref{ap:circ-ruler}.



\end{document}
