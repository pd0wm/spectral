%!TEX program = xelatex

\documentclass[a4paper, openany, oneside]{memoir}
\usepackage[no-math]{fontspec}
\usepackage{pgfplots}
\pgfplotsset{compat=newest}
\usepackage{commath}
\usepackage{mathtools}
\usepackage{amssymb}
\usepackage{amsthm}
\usepackage{booktabs}
\usepackage{mathtools}
\usepackage{xcolor}
\usepackage[separate-uncertainty=true, per-mode=symbol]{siunitx}
\usepackage[noabbrev, capitalize]{cleveref}
\usepackage{listings}
\usepackage[american inductor, european resistor]{circuitikz}
\usepackage{amsmath}
\usepackage{amsfonts}
\usepackage{ifxetex}
\usepackage[dutch,english]{babel}
\usepackage[backend=bibtexu,texencoding=utf8,bibencoding=utf8,style=ieee,sortlocale=en_GB,language=auto]{biblatex}
\usepackage[strict,autostyle]{csquotes}
\usepackage{parskip}
\usepackage{import}
\usepackage{standalone}
\usepackage{hyperref}
%\usepackage[toc,title,titletoc]{appendix}

\ifxetex{} % Fonts laden in het geval dat je met Xetex compiled
    \usepackage{fontspec}
    \defaultfontfeatures{Ligatures=TeX} % To support LaTeX quoting style
    \setromanfont{Palatino Linotype} % Tover ergens in Font mapje in root.
    \setmonofont{Source Code Pro}
\else % Terug val in standaard pdflatex tool chain. Geen ondersteuning voor OTT fonts
    \usepackage[T1]{fontenc}
    \usepackage[utf8]{inputenc}
\fi
\newcommand{\references}[1]{\begin{flushright}{#1}\end{flushright}}
\renewcommand{\vec}[1]{\boldsymbol{\mathbf{#1}}}
\newcommand{\uvec}[1]{\boldsymbol{\hat{\vec{#1}}}}
\newcommand{\mat}[1]{\boldsymbol{\mathbf{#1}}}
\newcommand{\fasor}[1]{\boldsymbol{\tilde{\vec{#1}}}}
\newcommand{\cmplx}[0]{\mathrm{j}}
\renewcommand{\Re}[0]{\operatorname{Re}}
\newcommand{\Cov}{\operatorname{Cov}}
\newcommand{\Var}{\operatorname{Var}}
\newcommand{\proj}{\operatorname{proj}}
\newcommand{\Perp}{\operatorname{perp}}
\newcommand{\col}{\operatorname{col}}
\newcommand{\rect}{\operatorname{rect}}
\newcommand{\sinc}{\operatorname{sinc}}
\newcommand{\IT}{\operatorname{IT}}
\newcommand{\F}{\mathcal{F}}

\newtheorem{definition}{Definition}
\newtheorem{theorem}{Theorem}


\DeclareSIUnit{\voltampere}{VA} %apparent power
\DeclareSIUnit{\pii}{\ensuremath{\pi}}

\hypersetup{%setup hyperlinks
    colorlinks,
    citecolor=black,
    filecolor=black,
    linkcolor=black,
    urlcolor=black
}

% Example boxes
\usepackage{fancybox}
\usepackage{framed}
\usepackage{adjustbox}
\newenvironment{simpages}%
{\AtBeginEnvironment{itemize}{\parskip=0pt\parsep=0pt\partopsep=0pt}
\def\FrameCommand{\fboxsep=.5\FrameSep\shadowbox}\MakeFramed{\FrameRestore}}%
{\endMakeFramed}

% Impulse train
\DeclareFontFamily{U}{wncy}{}
\DeclareFontShape{U}{wncy}{m}{n}{<->wncyr10}{}
\DeclareSymbolFont{mcy}{U}{wncy}{m}{n}
\DeclareMathSymbol{\Sha}{\mathord}{mcy}{"58}

\title{Coprime sampling\\Constant C-matrix derivation}

\author{W.P. Bruinsma \and R.P. Hes \and H.J.C. Kroep \and T.C. Leliveld \and W.M. Melching \and T.A. aan de Wiel}
\raggedbottom{}

\begin{document}


\begin{titlingpage}
  \pagestyle{empty}
  \maketitle
\end{titlingpage}
\chapter{Overview of variables}
\begin{center}
    \begin{tabular}{llr}
        \toprule
         & Variabele & vorm \\
        \midrule
        M & downsample factor of one of the coprime sampler & scalar \\
        N & downsample factor of the other coprime sampler & scalar \\
        \bottomrule
    \end{tabular}
\end{center}

\chapter{Analysis}
The goal of this chapter will be to deduce a constant $C$-matrix that describes the sampling process of a coprime sampler.

\begin{blockDefinition}[Coprime integers]
\label{def:coprime}
    Two numbers are called coprime if the only positive integer that evenly divides both is $1$. This can be formulated as two numbers are coprime if and only if $\gcd(a,b) = 1$.
\end{blockDefinition}

\begin{blockTheorem}[Least common multiple of coprime numbers]\label{th:cplcm}
    For coprime numbers $a$ and $b$ the least common multiple is $ab$.
\end{blockTheorem}

\begin{blockProofTheorem}{\ref{th:cplcm}}
    It is trivial to show that the least common multiple problem can be rewritten as \begin{equation*}
        \lcm(a,b) = \frac{\abs{a \cdot b}}{\gcd(a,b)}.
    \end{equation*}
    Coprime numbers greatest common divider is $1$ by definition as seen in \cref{def:coprime}. From this we can conclude that for $a$, $b$ coprime, their least common multiple is $ab$.
\end{blockProofTheorem}

\begin{blockDefinition}[Coprime sampling]
\label{def:coprimesampling}
    Two samplers are coprime sampling if their sampling rates can be expressed as
    \begin{equation*}
        \frac{F_{s,1}}{F_{s,2}} = \frac{a}{b},
        \end{equation*}
        with $a/b$ reduced and $a$, $b$ coprime.
    \end{blockDefinition}

    \begin{blockTheorem}[Downsample factor of coprime sampler]\label{th:downsample}
    For a pair coprime samplers, a common period can be found. Let $F_{s,1}$ and $F_{s,2}$ be the sample factors of the samplers respectively. Let $a$ and $b$ be the factors found in \cref{def:coprimesampling}. The total frequency is equal to $b \cdot F_{s,1} = a \cdot F_{s,2}$, because $a$ and $b$ are coprime (\cref{th:cplcm}). In this context $a$ and $b$ are called the downsampling factor with respect to the total period of the coprime samplers.
\end{blockTheorem}

Let $S_1$ and $S_2$ denote two coprime samplers with downsample factors $a$ and $b$ and input signals $\vec{y_1}$ and $\vec{y_2}$ respectively. As can be seen from \cref{th:downsample} the samplers can be seen as two downsamplers from a sampler at the total frequency of the coprime samplers with downsample factors $a, b$. Let the input of the total sampler be $\vec{x}$. It is known that with regard to the total sampling frequency the coprime samplers are repeated every $n\cdot ab$ samples with $n$ in $\mathbb{N}_0$

Let $A$ be the set that contains union of $na$ with $n a < ab, n \in \mathbb{N}_0$ and $n b$ with $n b < ab, n \in \mathbb{N}_0$. Let $a_i$ denote the elements of $A$ with $a_i \leq a_{i+1}$. We can define the matrix $C$, the coprime sample matrix, as the matrix with elements $(C)_{i,a_i} = 1$, otherwise $0$.


\end{document}

