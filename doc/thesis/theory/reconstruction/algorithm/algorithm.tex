%!TEX program = xelatex

\documentclass[a4paper, openany, oneside]{memoir}
\usepackage[no-math]{fontspec}
\usepackage{pgfplots}
\usepackage{float}
\pgfplotsset{compat=newest}
\usepackage{commath}
\usepackage{mathtools}
\usepackage{amssymb}
\usepackage{amsthm}
\usepackage{booktabs}
\usepackage{todonotes}
\usepackage{mathtools}
\usepackage{xcolor}
\usepackage[separate-uncertainty=true, per-mode=symbol]{siunitx}
\usepackage{listings}
\usepackage[american inductor, european resistor]{circuitikz}
\usepackage{amsmath}
\usepackage{amsfonts}
\usepackage{ifxetex}
\usepackage[dutch,english]{babel}
\usepackage[backend=bibtexu,texencoding=utf8,bibencoding=utf8,style=ieee,sortlocale=en_GB,language=auto]{biblatex}
\usepackage[strict,autostyle]{csquotes}
\usepackage{import}
\usepackage{standalone}
\usepackage{bookmark,hyperref}
\usepackage{xcolor,mdframed}
\usepackage{tikz}
\usepackage{framed}
\usepackage{float}
\usepackage{tabularx}
\usepackage{graphicx,adjustbox}
\usepackage{rotating}
\usepackage{pdfpages}
\usepackage{enumitem}
\usepackage{calc}
\usepackage{pgfplots}
\usepackage{filecontents}
\usepackage{caption}
\usepackage{subcaption}
\usepackage{lettrine}

\newcolumntype{Y}{>{\raggedright\arraybackslash}X} % Left-justified text in tabularx environment

\ifxetex{} % Fonts laden in het geval dat je met Xetex compiled
    \usepackage{fontspec}
    \defaultfontfeatures{Scale=MatchLowercase, Ligatures=TeX} % To support LaTeX quoting style
    %\setromanfont{Palatino Linotype} % Tover ergens in Font mapje in root.
    \setsansfont{Avenir Next LT Pro}
    \setromanfont{Adobe Caslon Pro} % Tover ergens in Font mapje in root.
    \setmonofont{Source Code Pro}
\else % Terug val in standaard pdflatex tool chain. Geen ondersteuning voor OTT fonts
    \usepackage[T1]{fontenc}
    \usepackage[utf8]{inputenc}
\fi
\usepackage[noabbrev, capitalize]{cleveref}
\usepackage{ifthen}
\usepackage{titlesec}
\usepackage{titlecaps}

\newcommand{\references}[1]{\begin{flushright}{#1}\end{flushright}}
\renewcommand{\vec}[1]{\boldsymbol{\mathbf{#1}}}
\newcommand{\uvec}[1]{\boldsymbol{\hat{\vec{#1}}}}
\newcommand{\mat}[1]{\boldsymbol{\mathbf{#1}}}
\newcommand{\fasor}[1]{\boldsymbol{\tilde{\vec{#1}}}}
\newcommand{\cmplx}[0]{\mathrm{j}}
\renewcommand{\Re}[0]{\operatorname{Re}}
\newcommand{\Cov}{\operatorname{Cov}}
\newcommand{\Var}{\operatorname{Var}}
\newcommand{\proj}{\operatorname{proj}}
\newcommand{\Perp}{\operatorname{perp}}
\newcommand{\col}{\operatorname{col}}
\newcommand{\rect}{\operatorname{rect}}
\newcommand{\sinc}{\operatorname{sinc}}
\newcommand{\lcm}{\operatorname{lcm}}
%\newcommand{\gcd}{\operatorname{gcd}}
\newcommand{\F}{\mathcal{F}}
\newcommand{\DTFT}{\mathcal{F}_*}
\newcommand{\conj}[1]{#1^*}
\renewcommand{\mod}{\operatorname{mod}}
\newcommand{\rot}{\operatorname{rot}}
\newcommand{\vecsc}[1]{\vec{\textsc{\textbf{#1}}}}
\renewcommand{\ss}[1]{_{#1}}

% Label without linebreak breaker
\newcommand{\lab}[1]{\label{#1}\nolinebreak}

\newtheorem{definition}{Definition}
\newtheorem{theorem}{Theorem}


\DeclareSIUnit{\voltampere}{VA} %apparent power
\DeclareSIUnit{\pii}{\ensuremath{\pi}}

\hypersetup{%setup hyperlinks
    colorlinks,
    citecolor=black,
    filecolor=black,
    linkcolor=black,
    urlcolor=black
}

% Example boxes
\usepackage{fancybox}
\usepackage{framed}
\usepackage{adjustbox}
\newenvironment{simpages}%
{\AtBeginEnvironment{itemize}{\parskip=0pt\parsep=0pt\partopsep=0pt}
\def\FrameCommand{\fboxsep=.5\FrameSep\shadowbox}\MakeFramed{\FrameRestore}}%
{\endMakeFramed}

% Impulse train
\DeclareFontFamily{U}{wncy}{}
\DeclareFontShape{U}{wncy}{m}{n}{<->wncyr10}{}
\DeclareSymbolFont{mcy}{U}{wncy}{m}{n}
\DeclareMathSymbol{\Sha}{\mathord}{mcy}{"58}

\setlength{\parindent}{0pt}
\nonzeroparskip

% Block environment configuration
\newcommand{\BlockLeftMargin}{20pt}
\newcommand{\BlockLeftMarginText}{25pt}
\newcommand{\BlockLeftMarginTextSpacing}{10pt}

% Own colours
\definecolor{gray75}{gray}{0.75}

% Block environment
\newenvironment{block}[3]{%
\makebox{\hspace{-\spinemargin}%
\begin{tikzpicture}[overlay]
    \draw [thick,color=gray75] (\BlockLeftMargin, 0) -- (\paperwidth - \spinemargin, 0);
    \node at (\BlockLeftMarginText, -0.9) [align=left, text width=\spinemargin - \BlockLeftMarginText - \BlockLeftMarginTextSpacing, anchor=west, text depth=1cm] {\textbf{\textsc{#1}}\newline\textit{#3}};
\end{tikzpicture}}%
\nopagebreak\\[0.25em]\ifthenelse{\equal{#2}{}}{}{(\textit{#2}.) }\nopagebreak\nolinebreak}
{\nopagebreak\\[-0.25em]%
\makebox{\hspace{-\spinemargin}%
\begin{tikzpicture}[overlay, remember picture]
    \draw [thick,color=gray75] (\spinemargin,0) -- (\paperwidth - \spinemargin,0);
\end{tikzpicture}} \vspace{0.5em}}

% Theorem
\newcounter{blockTheoremCounter}
\crefname{blockTheoremCounter}{Theorem}{Theorems}
\Crefname{blockTheoremCounter}{Theorem}{Theorems}

\newenvironment{blockTheorem}[1][]{%
\refstepcounter{blockTheoremCounter}%
\begin{block}{theorem \theblockTheoremCounter}{#1}{}}
{\end{block}}

% Definition
\newcounter{blockDefinitionCounter}
\crefname{blockDefinitionCounter}{Definition}{Definitions}
\Crefname{blockDefinitionCounter}{Definition}{Definitions}

\newenvironment{blockDefinition}[1][]{%
\refstepcounter{blockDefinitionCounter}%
\begin{block}{definition \theblockDefinitionCounter}{#1}{}}
{\end{block}}

% Proof
\newcounter{blockProofTheoremCounter}
\crefname{blockProofTheoremCounter}{Proof}{Proofs}
\Crefname{blockProofTheoremCounter}{Proof}{Proofs}

\newenvironment{blockProofTheorem}[1]{%
\refstepcounter{blockProofTheoremCounter}%
\begin{block}{proof of \\ theorem #1}{}{}}
{\qed\end{block}}

% Detail
\newcounter{blockDetailCounter}
\crefname{blockDetailCounter}{Detail}{Details}
\Crefname{blockDetailCounter}{Detail}{Details}

\newenvironment{blockDetail}[1][]{%
\refstepcounter{blockDetailCounter}%
\begin{block}{detail \theblockDetailCounter}{#1}{}}
{\end{block}}

% Redesign chapter headings
\newcommand{\chapternumber}{\thechapter}
\newcommand{\hsp}{\hspace{20pt}}
\titleformat{\chapter}[hang]{\Huge\bfseries}{\chapternumber\hsp\textcolor{gray75}{|}\hsp}{0pt}{\Huge\bfseries}

% Remove headers
% \addtopsmarks{headings}{}{
%   \createmark{chapter}{left}{nonumber}{}{}
% }
% \pagestyle{headings} % Activate changes

% Capitalise headers in a regular way
\renewcommand*{\memUChead}[1]{\titlecap{#1}}

% \hfill for math mode
\newcommand{\pushright}[1]{\intertext{\hfill$\displaystyle #1$}}
\newcommand{\pushline}{\hskip \textwidth minus \textwidth}
\newcommand{\matlab}{\textsc{Matlab}}

\definecolor{code-grey}{HTML}{DDDDDD}
\newcommand{\lib}[1]{\textsf{#1}}
\newcommand{\file}[1]{\textsf{#1}}
\newcommand{\func}[1]{\colorbox{code-grey}{\texttt{#1}}}
\newcommand{\class}[1]{\colorbox{code-grey}{\texttt{#1}}}

% Setup actiepunten
\newenvironment{important}[1][]{%
   \begin{mdframed}[%
      backgroundcolor={red!15}, hidealllines=true,
      skipabove=0.7\baselineskip, skipbelow=0.7\baselineskip,
      splitbottomskip=2pt, splittopskip=4pt, #1]%
   \makebox[0pt]{% ignore the withd of !
      \smash{% ignor the height of !
         \fontsize{32pt}{32pt}\selectfont% make the ! bigger
         \hspace*{-19pt}% move ! to the left
         \raisebox{-2pt}{% move ! up a little
            {\color{red!70!black}\sffamily\bfseries !}% type the bold red !
         }%
      }%
   }%
}{\end{mdframed}}
\newcommand{\excl}[1]{
\begin{important}
  \textbf{#1}
\end{important}
}

\makeatletter
\newcommand\footnoteref[1]{\protected@xdef\@thefnmark{\ref{#1}}\@footnotemark}
\makeatother

% Allow page breaks in display environments
%\allowdisplaybreaks
% S unit for use in Mega Samples per second
\DeclareSIUnit\sample{S}

\newcommand{\CC}{C\nolinebreak\hspace{-.05em}\raisebox{.3ex}{ \textbf{+}}\nolinebreak\hspace{-.10em}\raisebox{.3ex}{\textbf{+}}}
\def\CC{{C\nolinebreak[4]\hspace{-.05em}\raisebox{.3ex}{\textbf{++}}}}


\newcommand{\partauthor}[1]{\gdef\@partauthor{#1}}
\renewcommand{\printparttitle}[1]{
  \parttitlefont #1\\
  \vspace{1.5cm}
  \textnormal{\Large \@partauthor}
}
\addbibresource{../../../../includes/bibliography.bib}

\begin{document}


\section{Algorithm}
\label{sec:reconstruction-algorithm}
This section will discuss an algorithm to estimate the power spectral density of a signal which is sampled at sub-Nyquist frequencies. We consider multi-coset sampling such as described in \cref{cha:sampling}. The algorithm will be derived step-by-step in \cref{sec:reconstruction-derivation}.

Let the input signal sampled at the Nyquist-frequency be denoted by $x[n]$. Let the number of cosets be given by $M$. Every coset $i$ samples the input signal every $N$ samples, which means that the output of a coset is an $N$-decimation of the input signal. Let the output of coset $i$ be denoted by $y_i[n]$. Our reconstruction method estimates the power spectral density of $x[n]$ by making use of the outputs of all cosets. The Wiener-Khinchin theorem shows that estimating the power spectral density of $x[n]$ is equivalent to estimating the autocorrelation of $x[n]$. Therefore, our reconstruction method aims to estimate the autocorrelation of $x[n]$. Let the autocorrelation of the input signal be denoted by $r_x[m]$. Let the cross-correlation of the output of cosets $i$ and $j$ be denoted by $r_{y_i,y_j}[m]$. 

It is clear that $y_i[n]$ is part of the input of the algorithm. The way all cosets sample the input signal is determined by the sampling method. The configuration refers to the way all cosets sample the input signal. Possible configurations are discussed in \ref{cha:sampling} and will be further discussed in \ref{cha:sampling_methods}. Furthermore, the input of the algorithm also consists of the parameters $M$, $N$, $L$ and $K$. The parameter $L$ determines the support of $r_{y_i,y_j}[m]$\footnote{If $r_{y_i,y_j}[m]$ is limited in support from $m=-L$ to $m=L$, then this means that $r_{y_i,y_j}[m]=0$ for $|m|>L$.} and influences the length of the estimated autocorrelation, which determines the resolution of the estimated power spectral density of $x[n]$. The parameter $K$ influences the measurement time, which determines the accuracy of the estimated power spectral density. The measurement time consists of the time required to obtain $KL$ samples of the output of every coset. The inputs and output of the algorithm are summarised in Table \ref{tab:reconstruction-algorithm-inputs-outputs}. The algorithm consists of several steps.

\begin{table}
    \centering
    \begin{tabularx}{\textwidth}{llY}
        \textbf{Type} & \textbf{Parameter} & \textbf{Description} \\
        Input & $M$ & Number of cosets \\
        Input & $N$ & Downsampling factor. Every coset samples the input signal once per $N$ samples. \\
        Input & $y_i[n]$ & Output of coset $i$ \\
        Input & $L$ & Support of $r_{y_i,y_j}[m]$. This parameter influences the resolution of the estimated power spectral density of $x[n]$. \\
        Input & $K$ & Oversampling factor. This parameter influences the accuracy of the estimated power spectral density of $x[n]$, but increases the measurement time. The measurement time consists of the time required to obtain $KL$ samples of the output of every coset. \\
        Output & $r_x[m]$ & Autocorrelation of $x[n]$. The autocorrelation $r_x[m]$ is estimated from $m=-LN$ to $m=LN$.
    \end{tabularx}
    \caption{Input and outputs of the reconstruction algorithm}
    \label{tab:reconstruction-algorithm-inputs-outputs}
\end{table}

The algoritm to estimate the autocorrelation is as follows.

\begin{tabularx}{\textwidth}{rY}
    Step 1: & Determine $c_i[n]$ for every coset $i$.
    \newline \newline
    The signal $c_i[n]$ is defined in \cref{eq:yi}. This equation will be discussed in the derivation. The configuration determines $c_i[n]$ for every coset $i$. This is explained in \cref{sub:ci-circ,sub:ci-collab,sub:ci-coprime}.
    \newline \\
    Step 2: & Construct $\mat{R}$.
    \newline \newline
    The matrix $\mat{R}$ is defined in \cref{eq:ry-R-rx}. This equation will be discussed in the derivation.
    \newline \\
    Step 3: & Measure $y_i[n]$ for $KL$ samples for every coset $i$. \newline \\
    Step 4: & Estimate $r_{y_i,y_j}[m]$ for every combination of cosets $i$ and $j$.
    \newline \newline
    \cref{sec:reconstruction-estimation} discusses how $KL$ samples of $y_i[n]$ can be used to estimate $r_{y_i,y_j}[m]$.
    \newline \\
    Step 5: & Construct $\vec{r}_y$ and solve \cref{eq:normal-rx} to estimate $\vec{r}_x$, which yields an estimation of $r_x[m]$.
    \newline \newline
    The vectors $\vec{r}_y$ and $\vec{r}_x$ are defined in respectively \cref{eq:ry-R-rx,eq:def-rx}. Once more, these equations will be discussed in the derivation.  \\
\end{tabularx}

\end{document}