%!TEX program = xelatex
%!TEX root = ../../theory.tex

\documentclass[a4paper, openany, oneside]{memoir}
\usepackage[no-math]{fontspec}
\usepackage{pgfplots}
\pgfplotsset{compat=newest}
\usepackage{commath}
\usepackage{mathtools}
\usepackage{amssymb}
\usepackage{amsthm}
\usepackage{booktabs}
\usepackage{mathtools}
\usepackage{xcolor}
\usepackage[separate-uncertainty=true, per-mode=symbol]{siunitx}
\usepackage[noabbrev, capitalize]{cleveref}
\usepackage{listings}
\usepackage[american inductor, european resistor]{circuitikz}
\usepackage{amsmath}
\usepackage{amsfonts}
\usepackage{ifxetex}
\usepackage[dutch,english]{babel}
\usepackage[backend=bibtexu,texencoding=utf8,bibencoding=utf8,style=ieee,sortlocale=en_GB,language=auto]{biblatex}
\usepackage[strict,autostyle]{csquotes}
\usepackage{parskip}
\usepackage{import}
\usepackage{standalone}
\usepackage{hyperref}
%\usepackage[toc,title,titletoc]{appendix}

\ifxetex{} % Fonts laden in het geval dat je met Xetex compiled
    \usepackage{fontspec}
    \defaultfontfeatures{Ligatures=TeX} % To support LaTeX quoting style
    \setromanfont{Palatino Linotype} % Tover ergens in Font mapje in root.
    \setmonofont{Source Code Pro}
\else % Terug val in standaard pdflatex tool chain. Geen ondersteuning voor OTT fonts
    \usepackage[T1]{fontenc}
    \usepackage[utf8]{inputenc}
\fi
\newcommand{\references}[1]{\begin{flushright}{#1}\end{flushright}}
\renewcommand{\vec}[1]{\boldsymbol{\mathbf{#1}}}
\newcommand{\uvec}[1]{\boldsymbol{\hat{\vec{#1}}}}
\newcommand{\mat}[1]{\boldsymbol{\mathbf{#1}}}
\newcommand{\fasor}[1]{\boldsymbol{\tilde{\vec{#1}}}}
\newcommand{\cmplx}[0]{\mathrm{j}}
\renewcommand{\Re}[0]{\operatorname{Re}}
\newcommand{\Cov}{\operatorname{Cov}}
\newcommand{\Var}{\operatorname{Var}}
\newcommand{\proj}{\operatorname{proj}}
\newcommand{\Perp}{\operatorname{perp}}
\newcommand{\col}{\operatorname{col}}
\newcommand{\rect}{\operatorname{rect}}
\newcommand{\sinc}{\operatorname{sinc}}
\newcommand{\IT}{\operatorname{IT}}
\newcommand{\F}{\mathcal{F}}

\newtheorem{definition}{Definition}
\newtheorem{theorem}{Theorem}


\DeclareSIUnit{\voltampere}{VA} %apparent power
\DeclareSIUnit{\pii}{\ensuremath{\pi}}

\hypersetup{%setup hyperlinks
    colorlinks,
    citecolor=black,
    filecolor=black,
    linkcolor=black,
    urlcolor=black
}

% Example boxes
\usepackage{fancybox}
\usepackage{framed}
\usepackage{adjustbox}
\newenvironment{simpages}%
{\AtBeginEnvironment{itemize}{\parskip=0pt\parsep=0pt\partopsep=0pt}
\def\FrameCommand{\fboxsep=.5\FrameSep\shadowbox}\MakeFramed{\FrameRestore}}%
{\endMakeFramed}

% Impulse train
\DeclareFontFamily{U}{wncy}{}
\DeclareFontShape{U}{wncy}{m}{n}{<->wncyr10}{}
\DeclareSymbolFont{mcy}{U}{wncy}{m}{n}
\DeclareMathSymbol{\Sha}{\mathord}{mcy}{"58}
\addbibresource{../../../../includes/bibliography.bib}

\begin{document}


\section{Main Analysis}
Let $L$, $N$ and $M$ be integer parameters such that $M$ cosets provide $KL$ samples downsampled by a factor $N$. Thus $KLN$ samples of the input signal are required. Let the input signal be denoted by $\vec{x} \in \mathbb{C}^{KLN}$. The main analysis will be considered a limiting process of $K \in \mathbb{N}$ where $K \to \infty$. To this end, let $\varepsilon > 0$.

Let the sampling vector for coset $i$ be given by $\vec{c}_i \in \mathbb{C}^{N}$. The sampling vector determines the output of a coset. The exact relationship has yet to be derived. Let the pseudo output for coset $i$ be given by $\vec{y}_i = \vec{c}_i \ast \vec{x}$. The pseudo output will be used to derive the output of a coset.

\begin{blockTheorem} \label{th:conv-corr}
    \makebox[\textwidth]{\centering $(\vec{c}_i \ast \vec{x}) \circ (\vec{c}_j \ast \vec{x}) = (\vec{c}_i \circ \vec{c}_j) \ast (\vec{x} \circ \vec{x})$.} \nolinebreak
\end{blockTheorem}

Let the correlation of the pseudo outputs of cosets $i$ and $j$ be given by $\vec{r}_{y_i,y_j} = \vec{y}_i \circ \vec{y}_j$ and let the correlation of the sampling vectors of cosets $i$ and $j$ be given by $\vec{r}_{c_i,c_j} = \vec{c}_i \circ \vec{c}_j$. Then
\begin{align*}
    \vec{r}_{y_i,y_j} =(\vec{c}_i \ast \vec{x}) \circ (\vec{c}_j \ast \vec{x}) = (\vec{c}_i \circ \vec{c}_j) \ast (\vec{x} \circ \vec{x}) = \vec{r}_{c_i,c_j} \ast \vec{r}_x.
\end{align*}
Denote a truncated version of $\vec{r}_{y_i,y_j}$ by $\hat{\vec{r}}_{y_i,y_j} = \vec{r}_{y_i,y_j}[KLN-LN+2N-1,KLN+LN-1]$. Denote a truncated version of $\vec{r}_x$ by $\hat{\vec{r}}_x = \vec{r}_x [KLN-LN+1,KLN+LN-1]$. Let $\mat{R}_{c_i,c_j}$ be such that
\begin{align*}
    \hat{\vec{r}}_{y_i,y_j}
    &= (\vec{r}_{c_i,c_j} \ast \vec{r}_x)[KLN-LN+2N-1,KLN+LN-1]\\
    &= (\vec{r}_x \ast \vec{r}_{c_i,c_j})[KLN-LN+2N-1,KLN+LN-1] \\
    &= \pushline \\
    \pushright{
    \begin{bmatrix}
        % (\vec{r}_{c_i,c_j})_1 & 0 & 0& \cdots & 0 \\
        % (\vec{r}_{c_i,c_j})_2 & (\vec{r}_{c_i,c_j})_1 & 0 & \cdots & 0 \\
        % &  & \ddots &  & \\
        % 0 &  \cdots & 0 & (\vec{r}_{c_i,c_j})_{2N-1} & (\vec{r}_{c_i,c_j})_{2N-2} \\
        % 0 &  \cdots & 0& 0 & (\vec{r}_{c_i,c_j})_{2N - 1} \\
        (\vec{r}_{c_i,c_j})_{2N-1} & (\vec{r}_{c_i,c_j})_{2N-2} & \cdots &(\vec{r}_{c_i,c_j})_{1} & 0 & \cdots  \\
        0 & (\vec{r}_{c_i,c_j})_{2N-1} & \cdots & (\vec{r}_{c_i,c_j})_{2} & (\vec{r}_{c_i,c_j})_{1} & \cdots \\
        && \multicolumn{2}{c}{\ddots} \\
        \cdots & (\vec{r}_{c_i,c_j})_{2N-1} & (\vec{r}_{c_i,c_j})_{2N-2} & \cdots & (\vec{r}_{c_i,c_j})_{1} & 0 \\
        \cdots & 0 & (\vec{r}_{c_i,c_j})_{2N-1} & \cdots &(\vec{r}_{c_i,c_j})_{2} & (\vec{r}_{c_i,c_j})_{1} 
    \end{bmatrix} \pushline}
    \pushright{\begin{bmatrix}
        (\vec{r}_x)_{KLN-LN+1} \\
        (\vec{r}_x)_{KLN-LN+2} \\
        \vdots \\
        (\vec{r}_x)_{KLN+LN-2} \\
        (\vec{r}_x)_{KLN+LN-1}
    \end{bmatrix} }
    &= \mat{R}_{c_i,c_j} \hat{\vec{r}}_x.
\end{align*}
Let the $2L-1\times 2LN-2N+1$ decimation matrix be defined by $(\mat{D})_{i,(i-1)N+1} = 1$ for $i=1,\ldots,2L-1$ and otherwise zero. Let the decimated correlation of the pseudo outputs of cosets $i$ and $j$ by given by $\hat{\vec{r}}'_{y_i,y_j} = \mat{D} \hat{\vec{r}}_{y_i,y_j}$.
% Then let $\mat{R}$ be such that
% \begin{align*}
%     \begin{bmatrix}
%         \hat{\vec{r}}'_{y_1,y_1} \\
%         \vdots \\
%         \hat{\vec{r}}'_{y_M,y_M}
%     \end{bmatrix}
%     = \begin{bmatrix}
%         \mat{D}\mat{R}_{c_1,c_1} \hat{\vec{r}}_x \\
%         \vdots \\
%         \mat{D}\mat{R}_{c_M,c_M} \hat{\vec{r}}_x
%     \end{bmatrix}
%     = \begin{bmatrix}
%         \mat{D}\mat{R}_{c_1,c_1}\\
%         \vdots \\
%         \mat{D}\mat{R}_{c_M,c_M}
%     \end{bmatrix} \hat{\vec{r}}_x
%     = \mat{R} \hat{\vec{r}}_x.
% \end{align*}
% Note that we used \cref{def:dots-extended} here.
% We now investigate $\vec{r}'_{y_i,y_j}$'s structure. Notice that
% \begin{align*}
%     (\hat{\vec{r}}'_{y_i,y_j})_m = (\mat{D} \hat{\vec{r}}_{y_i,y_j})_{m} = (\hat{\vec{r}}_{y_i,y_j})_{(m-1)N+1}.
% \end{align*}
% Thus $\hat{\vec{r}}'_{y_i,y_j}$ is the $N$-decimation of $\hat{\vec{r}}_{y_i,y_j}$. 
Let $\vec{y}'_i$ denote the $N$-decimation of $\vec{y}_i$. Then $\vec{y}'_i$ corresponds to the output of coset $i$. Now let $\vec{r}_{y'_i,y'_j} = \vec{y}'_i \circ \vec{y}'_j$. Denote a truncated version of $\vec{r}_{y'_i,y'_j}$ by $\hat{\vec{r}}_{y'_i,y'_j}=\vec{r}_{y'_i,y'_j}[KL-L+1,KL+L-1]$.

\begin{blockTheorem} \lab{th:deci-corr}
    Let $Y_i[n]$ and $Y_j[n]$ be wide sense stationary stochastic processes such that $(\vec{y}_i)_m = Y_i[m]$ and $(\vec{y}_j)_m = Y_j[m]$ for $m=1,\ldots,KLN+N-1$. Furthermore, let $Y'_i[n]$ and $Y'_j[n]$ be wide sense stationary stochastic processes such that $(\vec{y}'_i)_m = Y'_i[m]$ and $(\vec{y}'_j)_m = Y'_j[m]$ for $m=1,\ldots,KL$. Then $N\hat{\vec{r}}_{y'_i,y'_j}$ is an unbiased estimator of $E( \vec{w} \odot \hat{\vec{r}}'_{y_i,y_j})$ where

    \makebox[\textwidth]{\centering
        $\vec{w} = \begin{bmatrix}
            KLN+(-L+2)N-1 \\
            KLN+(-L+3)N-1 \\
            \vdots \\
            KLN+0\cdot N-1 \\
            KLN+1\cdot N-1 \\
            KLN+0\cdot N-1 \\
            \vdots \\
            KLN+(-L+3)N-1 \\
            KLN+(-L+2)N-1
         \end{bmatrix} \odot \begin{bmatrix}
            [KLN+(-L+1)N]^{-1} \\
            [KLN+(-L+2)N]^{-1} \\
            \vdots \\
            (KLN -1 \cdot N)^{-1} \\
            (KLN+ 0 \cdot N)^{-1} \\
            (KLN -1 \cdot N)^{-1} \\
            \vdots \\
            [KLN+(-L+2)N]^{-1} \\
            [KLN+(-L+1)N]^{-1}
         \end{bmatrix}.$
    }
\end{blockTheorem}


% \Cref{th:deci-corr} shows that the correlation of the decimated pseudo outputs of cosets $i$ and $j$ can be used to estimate the decimated correlation of the pseudo outputs of cosets $i$ and $j$. To this end, let
% \begin{align*}
%     \hat{\vec{r}}'_y = N \begin{bmatrix}
%         \hat{\vec{r}}_{y'_1,y'_1} \\
%         \vdots \\
%         \hat{\vec{r}}_{y'_M,y'_M}
%     \end{bmatrix}.
% \end{align*}
We now identify the expected value of $\hat{\vec{r}}_x$ and $\hat{\vec{r}}_{y'_i,y'_j}$. Let $X[n]$ be a wide sense stationary stochastic process such that $(\vec{x})_i = X[i]$ for $i = 1,\ldots,KLN$. Also, let $\vecsc{r}_x \in \mathbb{C}^{2LN-1}$ be such that $(\vecsc{r}_x)_{i+LN} = R_X[i]$ for $i = -LN + 1, LN-1$. Furthermore, let $\vecsc{r}_{y'_i,y'_j} \in \mathbb{C}^{2L-1}$ be such that $(\vecsc{r}_{y'_i,y'_j})_{i+L}=R_{Y'_i,Y'_j}[i]$ for $i = -L+1,L-1$.
Note that $\vecsc{r}_x$ represents the unbiased autocorrelation of the input signal and that $\vecsc{r}_{y'_i,y'_j}$ represents the unbiased crosscorrelation of the decimated pseudo outputs of cosets $i$ and $j$. Then by \cref{th:correlation-bias}, $E(\hat{\vec{r}}_x) = \vec{b}_x \odot \vecsc{r}_x$ and $E(\hat{\vec{r}}_{y'_i,y'_j})=\vec{b}_y \odot \vecsc{r}_{y'_i,y'_j}$ where
\begin{align*}
    \vec{b}_{x} =  \begin{bmatrix}
        (K-1)LN+1 \\
        (K-1)LN+2 \\
        \vdots \\
        KLN \\
        \vdots \\
        (K-1)LN+2 \\
        (K-1)LN+1
    \end{bmatrix},\vec{b}_y = \begin{bmatrix}
        (K-1)L + 1 \\
        (K-1)L + 2 \\
        \vdots \\
        KL \\
        \vdots \\
        (K-1)L + 2 \\
        (K-1)L + 1
    \end{bmatrix}.
\end{align*}

% Then
% \begin{align*}
%     E(\hat{\vec{r}}'_y) &= \begin{bmatrix}
%         E(N\hat{\vec{r}}_{y'_1,y'_1}) \\
%         \vdots \\
%         E(N\hat{\vec{r}}_{y'_M,y'_M})
%     \end{bmatrix} \\
%     &= \begin{bmatrix}
%         N\vec{b}_y \odot \vec{r}_{y'_1,y'_1,u} \\
%         \vdots \\
%         N\vec{b}_y \odot \vec{r}_{y'_M,y'_j,M}
%     \end{bmatrix} \\
%     &= \begin{bmatrix} 
%         \vec{w} \odot E(\hat{\vec{r}}'_{y_1,y_1}) \\
%         \vdots \\
%         \vec{w} \odot E(\hat{\vec{r}}'_{y_M,y_M}) \\
%     \end{bmatrix} \\
%     &= \begin{bmatrix}
%         \vec{w} \odot \mat{D}\mat{R}_{c_1,c_1} E(\hat{\vec{r}}_x) \\
%         \vdots \\
%         \vec{w} \odot \mat{D}\mat{R}_{c_M,c_M} E(\hat{\vec{r}}_x) \\
%     \end{bmatrix} \\
%     &= \begin{bmatrix}
%         \vec{w} \odot \mat{D}\mat{R}_{c_1,c_1} (\vec{b}_x \odot \vec{r}_{x,u}) \\
%         \vdots \\
%         \vec{w} \odot \mat{D}\mat{R}_{c_M,c_M} (\vec{b}_x \odot \vec{r}_{x,u}) \\
%     \end{bmatrix}.
%     % = E\left(\begin{bmatrix}
%     %     \vec{w} \odot \vec{r}'_{y_1,y_1} \\
%     %     \vdots \\
%     %     \vec{w} \odot \vec{r}'_{y_M,y_M} \\
%     % \end{bmatrix}\right) = E(\mat{R} \hat{\vec{r}}_x) = \mat{R} E(\hat{\vec{r}}_x).
% \end{align*}
Now $E(N\hat{\vec{r}}_{y'_i,y'_i}) = N\vec{b}_y \odot \vecsc{r}_{y'_i,y'_i}$, whilst also
\begin{align*}
    E(N\hat{\vec{r}}_{y'_i,y'_i}) = \vec{w} \odot E(\hat{\vec{r}}'_{y_i,y_i}) = \vec{w} \odot \mat{D}\mat{R}_{c_i,c_i} E(\hat{\vec{r}}_x) = \vec{w} \odot \mat{D}\mat{R}_{c_i,c_i} (\vec{b}_x \odot \vecsc{r}_{x}).
\end{align*}
Therefore equating and dividing by $KLN$ yields that
\begin{align*}
    \left(\frac{\vec{b}_y}{KL}\right) \odot \vecsc{r}_{y'_i,y'_j} = \vec{w} \odot \mat{D}\mat{R}_{c_i,c_i} \left[\left(\frac{\vec{b}_x}{KLN}\right) \odot \vecsc{r}_{x}\right].
\end{align*}
We have now related $\vecsc{r}_{y'_i,y'_j}$ to $\vecsc{r}_{x}$. However, this relationship involves element-wise multiplication, which can be considered as effects due to windowing.


todo: CONSEQUENTIE VAN E en haakjes, elimineer > in defs
\subsection{Limiting Case}
Assume that $R_{Y'_i,Y'_j}[m] = 0$ for $-KL+1 \le m \le -L$ and $L \le m \le KL-1$. The proof of \cref{th:conv-corr} then shows that $R_{Y_i,Y_j}[m]=0$ for $m \le -LN$ and $m \ge LN$, which implies that $[E(\vec{r}_{y_i,y_j})]_{m}=0$ for $m \le KLN - LN + N - 1$ and $m \ge KLN +LN + N - 1$. Since $\vec{r}_{y_i,y_j} = \vec{r}_{c} \ast \vec{r}_x$, $[E(\vec{r}_x)]_m=0$ for $m \le KLN - LN$ and $m \ge KLN + LN$. This is achieved by letting $(\vec{x})_m=0$ for $m \ge LN + 1$.

Imply limiting process

To eliminate these effects, we assume that $R_X[i]=0$ for $i \ge LN$ and $i \le -LN$. To achieve this, let $(\vec{x})_i=0$ for $i \ge LN$ and let $K \to \infty$. Because $K \to \infty$, we notice that $\vec{b}_x/KLN \to \vec{1}_{2LN-1}$, $\vec{b}_y/KL \to \vec{1}_{2L-1}$ and $\vec{w} \to \vec{1}_{2L-1}$. Then $\vecsc{r}_{y'_i,y'_j} = \mat{D}\mat{R}_{c_i,c_i} \vecsc{r}_{x}$. Now let $\vecsc{r}_y$ and $\vec{R}$ be such that
\begin{align*}
    \vecsc{r}_y = \begin{bmatrix}
        \vecsc{r}_{y'_1,y'_1} \\ \vdots \\ \vecsc{r}_{y'_M,y'_M}
    \end{bmatrix} = \begin{bmatrix}
        \mat{D}\mat{R}_{c_1,c_1} \\ \vdots \\ \mat{R}_{c_M,c_M}
    \end{bmatrix} \vecsc{r}_x = \mat{R} \vecsc{r}_x.
\end{align*}
This concludes the main analysis.

Consider \cref{th:deci-corr}. Here $[E(N \vec{r}_{y'_i,y'_j})]_{m + L}$ is an estimator of $R_{Y'_i,Y'_j}[m]$ for $m = -L+1,L-1$. Since one is only able to estimate $R_{Y'_i,Y'_j}[m]$ for $m=-L+1,L-1$, it is often assumed that $R_{Y'_i,Y'_j}[m]=0$ for $m \le -L$ and $m \ge L$. But then $R_{Y'_i,Y'_j}[m]=0$ for $m \le -LN$ and $m \ge LN$, which implies that $E[(\vec{r}_{y_i,y_j})]_{m+LN + N-1}=0$ for $m \le -LN$ and $m \ge LN$. Therefore the first and last $N-1$ elements of $E[(\vec{r}_{y_i,y_j})]$ are zero. By the structure of $\mat{R}_{c_i,c_j}$, the first $N-1$ elements of $E(\vec{r}_x)$ consist of the first $N-1$ elements of $\vec{r}_{y_i,y_j}$, which are zero. Therefore, the first $N-1$ elements of $E(\vec{r}_x)$ are zero. Similarly, the last $N-1$ elements of $E(\vec{r}_x)$ are zero.



% So $\vec{r}_y'$ is an unbiased estimator of $\mat{R} E(\hat{\vec{r}}_x)$, which we can use to determine $E(\hat{\vec{r}}_x)$. Denote $\vec{x}_m = \vec{x}[(m-1)N+1,mN]$ for $m = 1,\ldots,L$. Thus $\vec{x}_m$ corresponds to the $m$'th interval of $N$ samples of $\vec{x}$. Finally, note that
% \begin{align*}
%     (\vec{y}'_i)_m = (\vec{y}_i)_{mN} = (\vec{c}_i \ast \vec{x})_{mN} = \sum_{k=1}^N (\vec{c}_i)_k (\vec{x})_{mN - k + 1} = \vec{d}_i \cdot \vec{x}_m
% \end{align*}
% where $\vec{d}_i$ denotes $\vec{c}_i$ reversed. Therefore, the reverse of the sampling vector for a coset determines the output of the coset for every interval of $N$ samples of $\vec{x}$. Accordingly, let
% \begin{align*}
%     \vec{w}_m = \begin{bmatrix}
%         (\vec{y}'_1)_m \\
%         \vdots \\
%         (\vec{y}'_M)_m
%     \end{bmatrix} = \begin{bmatrix}
%         \vec{d}_1 \cdot \vec{x}_m \\
%         \vdots \\
%         \vec{d}_M \cdot \vec{x}_m
%     \end{bmatrix} = \begin{bmatrix}
%         \vec{d}_1^T\\
%         \vdots \\
%         \vec{d}_M^T
%     \end{bmatrix} \vec{x}_m.
% \end{align*}
% Thus $\vec{w}_m$ aggregates the output of all cosets in the $m$'th interval of $N$ samples of $\vec{x}$. This concludes the main analysis.


\end{document}