%!TEX program = xelatex

\documentclass[a4paper, openany, oneside]{memoir}
\usepackage[no-math]{fontspec}
\usepackage{pgfplots}
\pgfplotsset{compat=newest}
\usepackage{commath}
\usepackage{mathtools}
\usepackage{amssymb}
\usepackage{amsthm}
\usepackage{booktabs}
\usepackage{mathtools}
\usepackage{xcolor}
\usepackage[separate-uncertainty=true, per-mode=symbol]{siunitx}
\usepackage[noabbrev, capitalize]{cleveref}
\usepackage{listings}
\usepackage[american inductor, european resistor]{circuitikz}
\usepackage{amsmath}
\usepackage{amsfonts}
\usepackage{ifxetex}
\usepackage[dutch,english]{babel}
\usepackage[backend=bibtexu,texencoding=utf8,bibencoding=utf8,style=ieee,sortlocale=en_GB,language=auto]{biblatex}
\usepackage[strict,autostyle]{csquotes}
\usepackage{parskip}
\usepackage{import}
\usepackage{standalone}
\usepackage{hyperref}
%\usepackage[toc,title,titletoc]{appendix}

\ifxetex{} % Fonts laden in het geval dat je met Xetex compiled
    \usepackage{fontspec}
    \defaultfontfeatures{Ligatures=TeX} % To support LaTeX quoting style
    \setromanfont{Palatino Linotype} % Tover ergens in Font mapje in root.
    \setmonofont{Source Code Pro}
\else % Terug val in standaard pdflatex tool chain. Geen ondersteuning voor OTT fonts
    \usepackage[T1]{fontenc}
    \usepackage[utf8]{inputenc}
\fi
\newcommand{\references}[1]{\begin{flushright}{#1}\end{flushright}}
\renewcommand{\vec}[1]{\boldsymbol{\mathbf{#1}}}
\newcommand{\uvec}[1]{\boldsymbol{\hat{\vec{#1}}}}
\newcommand{\mat}[1]{\boldsymbol{\mathbf{#1}}}
\newcommand{\fasor}[1]{\boldsymbol{\tilde{\vec{#1}}}}
\newcommand{\cmplx}[0]{\mathrm{j}}
\renewcommand{\Re}[0]{\operatorname{Re}}
\newcommand{\Cov}{\operatorname{Cov}}
\newcommand{\Var}{\operatorname{Var}}
\newcommand{\proj}{\operatorname{proj}}
\newcommand{\Perp}{\operatorname{perp}}
\newcommand{\col}{\operatorname{col}}
\newcommand{\rect}{\operatorname{rect}}
\newcommand{\sinc}{\operatorname{sinc}}
\newcommand{\IT}{\operatorname{IT}}
\newcommand{\F}{\mathcal{F}}

\newtheorem{definition}{Definition}
\newtheorem{theorem}{Theorem}


\DeclareSIUnit{\voltampere}{VA} %apparent power
\DeclareSIUnit{\pii}{\ensuremath{\pi}}

\hypersetup{%setup hyperlinks
    colorlinks,
    citecolor=black,
    filecolor=black,
    linkcolor=black,
    urlcolor=black
}

% Example boxes
\usepackage{fancybox}
\usepackage{framed}
\usepackage{adjustbox}
\newenvironment{simpages}%
{\AtBeginEnvironment{itemize}{\parskip=0pt\parsep=0pt\partopsep=0pt}
\def\FrameCommand{\fboxsep=.5\FrameSep\shadowbox}\MakeFramed{\FrameRestore}}%
{\endMakeFramed}

% Impulse train
\DeclareFontFamily{U}{wncy}{}
\DeclareFontShape{U}{wncy}{m}{n}{<->wncyr10}{}
\DeclareSymbolFont{mcy}{U}{wncy}{m}{n}
\DeclareMathSymbol{\Sha}{\mathord}{mcy}{"58}
\addbibresource{../../../../includes/bibliography.bib}

\begin{document}


\section{Main Analysis}

Let $L$, $K$, $N$ and $M$ be parameters such that $M$ cosets will provide $KL$ samples downsampled by a factor $N$. Thus $KLN$ samples of the input signal are required. Let the input signal be denoted by $\vec{x} \in \mathbb{C}^{KLN}$. Let for coset $i$ the sampling vector be given by $\vec{c}_i \in \mathbb{C}^{N}$. The sampling vector determines the output of a coset. The exact relationship has yet to be derived. Let for coset $i$ the pseudo output be given by $\vec{y}_i = \vec{c}_i \ast \vec{x}$. The pseudo output will be used to derive the output of a coset.

\begin{blockTheorem} \label{th:conv-corr}
    \makebox[\textwidth]{\centering $(\vec{c}_i \ast \vec{x}) \circ (\vec{c}_j \ast \vec{x}) = (\vec{c}_i \circ \vec{c}_j) \ast (\vec{x} \circ \vec{x})$.} \nolinebreak
\end{blockTheorem}

Let the correlation of the pseudo outputs of cosets $i$ and $j$ be given by $\vec{r}_{y_i,y_j} = \vec{y}_i \circ \vec{y}_j$ and let the correlation of the sampling vectors of cosets $i$ and $j$ be given by $\vec{r}_{c_i,c_j} = \vec{c}_i \circ \vec{c}_j$. Then
\begin{align*}
    \vec{r}_{y_i,y_j} =(\vec{c}_i \ast \vec{x}) \circ (\vec{c}_j \ast \vec{x}) = (\vec{c}_i \circ \vec{c}_j) \ast (\vec{x} \circ \vec{x}) = \vec{r}_{c_i,c_j} \ast \vec{r}_x.
\end{align*}
Denote a truncated version of $\vec{r}_{y_i,y_j}$ by $\hat{\vec{r}}_{y_i,y_j} = \vec{r}_{y_i,y_j}[KLN-LN+2N-1,KLN+LN-1]$. Denote a truncated version of $\vec{r}_x$ by $\hat{\vec{r}}_x = \vec{r}_x [KLN-LN+1,KLN+LN-1]$. Commutativity and the definition of the convolution operator yield that
\begin{align*}
    \hat{\vec{r}}_{y_i,y_j}
    &= (\vec{r}_{c_i,c_j} \ast \vec{r}_x)[KLN-LN+2N-1,KLN+LN-1]\\
    &= (\vec{r}_x \ast \vec{r}_{c_i,c_j})[KLN-LN+2N-1,KLN+LN-1] \\
    &= \pushline \\
    \pushright{
    \begin{bmatrix}
        % (\vec{r}_{c_i,c_j})_1 & 0 & 0& \cdots & 0 \\
        % (\vec{r}_{c_i,c_j})_2 & (\vec{r}_{c_i,c_j})_1 & 0 & \cdots & 0 \\
        % &  & \ddots &  & \\
        % 0 &  \cdots & 0 & (\vec{r}_{c_i,c_j})_{2N-1} & (\vec{r}_{c_i,c_j})_{2N-2} \\
        % 0 &  \cdots & 0& 0 & (\vec{r}_{c_i,c_j})_{2N - 1} \\
        (\vec{r}_{c_i,c_j})_{2N-1} & (\vec{r}_{c_i,c_j})_{2N-2} & \cdots &(\vec{r}_{c_i,c_j})_{1} & 0 & \cdots  \\
        0 & (\vec{r}_{c_i,c_j})_{2N-1} & \cdots & (\vec{r}_{c_i,c_j})_{2} & (\vec{r}_{c_i,c_j})_{1} & \cdots \\
        && \multicolumn{2}{c}{\ddots} \\
        \cdots & (\vec{r}_{c_i,c_j})_{2N-1} & (\vec{r}_{c_i,c_j})_{2N-2} & \cdots & (\vec{r}_{c_i,c_j})_{1} & 0 \\
        \cdots & 0 & (\vec{r}_{c_i,c_j})_{2N-1} & \cdots &(\vec{r}_{c_i,c_j})_{2} & (\vec{r}_{c_i,c_j})_{1} 
    \end{bmatrix} \pushline}
    \pushright{\begin{bmatrix}
        (\vec{r}_x)_{KLN-LN+1} \\
        (\vec{r}_x)_{KLN-LN+2} \\
        \vdots \\
        (\vec{r}_x)_{KLN+LN-2} \\
        (\vec{r}_x)_{KLN+LN-1}
    \end{bmatrix} }
    &= \mat{R}_{c_i,c_j} \hat{\vec{r}}_x.
\end{align*}
Let the $2L-1\times 2LN-2N+1$ decimation matrix be defined by $(\mat{D})_{i,iN} = 1$ for $i=1,\ldots,2L-1$ and otherwise zero. The decimation matrix is used to derive the output of a coset from its pseudo output. Let the decimated correlation of the pseudo outputs of cosets $i$ and $j$ by given by $\vec{r}'_{y_i,y_j} = \mat{D} \hat{\vec{r}}_{y_i,y_j}$. Then let $\mat{R}$ be such that
\begin{align*}
    \begin{bmatrix}
        \vec{r}'_{y_1,y_1} \\
        \vdots \\
        \vec{r}'_{y_M,y_M}
    \end{bmatrix}
    = \begin{bmatrix}
        \mat{D}\mat{R}_{c_1,c_1} \hat{\vec{r}}_x \\
        \vdots \\
        \mat{D}\mat{R}_{c_M,c_M} \hat{\vec{r}}_x
    \end{bmatrix}
    = \begin{bmatrix}
        \mat{D}\mat{R}_{c_1,c_1}\\
        \vdots \\
        \mat{D}\mat{R}_{c_M,c_M}
    \end{bmatrix} \hat{\vec{r}}_x
    = \mat{R} \hat{\vec{r}}_x.
\end{align*}
Note that we used \cref{def:dots-extended} here. We now investigate $\vec{r}'_{y_i,y_j}$'s structure. Notice that
\begin{align*}
    (\vec{r}'_{y_i,y_j})_m = (\mat{D} \hat{\vec{r}}_{y_i,y_j})_{m} = (\hat{\vec{r}}_{y_i,y_j})_{mN}.
\end{align*}
Thus $\vec{r}'_{y_i,y_j}$ is the $N$-decimation of $\hat{\vec{r}}_{y_i,y_j}$. Let $\vec{y}'_i$ denote the $N$-decimation of $\vec{y}_i$. Then $\vec{y}'_i$ corresponds to the output of coset $i$. Now let $\vec{r}_{y'_i,y'_j} = \vec{y}'_i \circ \vec{y}'_j$.

\begin{blockTheorem} \lab{th:deci-corr}
    Let $Y_i[n]$ and $Y_j[n]$ be wide sense stationary stochastic processes such that $(\vec{y}_i)_m = Y_i[m]$ and $(\vec{y}_j)_m = Y_j[m]$ for $m=1,\ldots,KLN+N-1$. Furthermore, let $Y'_i[n]$ and $Y'_j[n]$ be wide sense stationary stochastic processes such that $(\vec{y}'_i)_m = Y'_i[m]$ and $(\vec{y}'_j)_m = Y'_j[m]$ for $m=1,\ldots,L$. Then $N\vec{r}_{y'_i,y'_j}$ is an unbiased estimator of $E( \vec{r}'_{y_i,y_j})$.
\end{blockTheorem}

\Cref{th:deci-corr} shows that the correlation of the outputs of cosets $i$ and $j$ can be used to estimate the $N$-decimation of the correlation of the pseudo outputs of cosets $i$ and $j$. To this end, let
\begin{align*}
    \vec{r}'_y = N \begin{bmatrix}
        \vec{y}'_1 \circ \vec{y}'_1 \\
        \vdots \\
        \vec{y}'_M \circ \vec{y}'_M
    \end{bmatrix}.
\end{align*}
Then
\begin{align*}
    E(\vec{r}'_y) &= \begin{bmatrix}
        E(N\vec{r}_{y'_1,y'_1}) \\
        \vdots \\
        E(N\vec{r}_{y'_M,y'_M})
    \end{bmatrix}
    = E\left(\begin{bmatrix}
        \vec{r}'_{y_1,y_1} \\
        \vdots \\
        \vec{r}'_{y_M,y_M} \\
    \end{bmatrix}\right) = E(\mat{R} \hat{\vec{r}}_x) = \mat{R} E(\hat{\vec{r}}_x).
\end{align*}
So $\vec{r}_y'$ is an unbiased estimator of $\mat{R} E(\hat{\vec{r}}_x)$, which we can use to determine $E(\hat{\vec{r}}_x)$. Denote $\vec{x}_m = \vec{x}[(m-1)N+1,mN]$ for $m = 1,\ldots,L$. Thus $\vec{x}_m$ corresponds to the $m$'th interval of $N$ samples of $\vec{x}$. Finally, note that
\begin{align*}
    (\vec{y}'_i)_m = (\vec{y}_i)_{mN} = (\vec{c}_i \ast \vec{x})_{mN} = \sum_{k=1}^N (\vec{c}_i)_k (\vec{x})_{mN - k + 1} = \vec{d}_i \cdot \vec{x}_m
\end{align*}
where $\vec{d}_i$ denotes $\vec{c}_i$ reversed. Therefore, the reverse of the sampling vector for a coset determines the output of the coset for every interval of $N$ samples of $\vec{x}$. Accordingly, let
\begin{align*}
    \vec{w}_m = \begin{bmatrix}
        (\vec{y}'_1)_m \\
        \vdots \\
        (\vec{y}'_M)_m
    \end{bmatrix} = \begin{bmatrix}
        \vec{d}_1 \cdot \vec{x}_m \\
        \vdots \\
        \vec{d}_M \cdot \vec{x}_m
    \end{bmatrix} = \begin{bmatrix}
        \vec{d}_1^T\\
        \vdots \\
        \vec{d}_M^T
    \end{bmatrix} \vec{x}_m.
\end{align*}
Thus $\vec{w}_m$ aggregates the output of all cosets in the $m$'th interval of $N$ samples of $\vec{x}$. This concludes the main analysis.


\end{document}