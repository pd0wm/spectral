%!TEX program = xelatex

\documentclass[a4paper, openany, oneside]{memoir}
\usepackage[no-math]{fontspec}
\usepackage{pgfplots}
\pgfplotsset{compat=newest}
\usepackage{commath}
\usepackage{mathtools}
\usepackage{amssymb}
\usepackage{amsthm}
\usepackage{booktabs}
\usepackage{mathtools}
\usepackage{xcolor}
\usepackage[separate-uncertainty=true, per-mode=symbol]{siunitx}
\usepackage[noabbrev, capitalize]{cleveref}
\usepackage{listings}
\usepackage[american inductor, european resistor]{circuitikz}
\usepackage{amsmath}
\usepackage{amsfonts}
\usepackage{ifxetex}
\usepackage[dutch,english]{babel}
\usepackage[backend=bibtexu,texencoding=utf8,bibencoding=utf8,style=ieee,sortlocale=en_GB,language=auto]{biblatex}
\usepackage[strict,autostyle]{csquotes}
\usepackage{parskip}
\usepackage{import}
\usepackage{standalone}
\usepackage{hyperref}
%\usepackage[toc,title,titletoc]{appendix}

\ifxetex{} % Fonts laden in het geval dat je met Xetex compiled
    \usepackage{fontspec}
    \defaultfontfeatures{Ligatures=TeX} % To support LaTeX quoting style
    \setromanfont{Palatino Linotype} % Tover ergens in Font mapje in root.
    \setmonofont{Source Code Pro}
\else % Terug val in standaard pdflatex tool chain. Geen ondersteuning voor OTT fonts
    \usepackage[T1]{fontenc}
    \usepackage[utf8]{inputenc}
\fi
\newcommand{\references}[1]{\begin{flushright}{#1}\end{flushright}}
\renewcommand{\vec}[1]{\boldsymbol{\mathbf{#1}}}
\newcommand{\uvec}[1]{\boldsymbol{\hat{\vec{#1}}}}
\newcommand{\mat}[1]{\boldsymbol{\mathbf{#1}}}
\newcommand{\fasor}[1]{\boldsymbol{\tilde{\vec{#1}}}}
\newcommand{\cmplx}[0]{\mathrm{j}}
\renewcommand{\Re}[0]{\operatorname{Re}}
\newcommand{\Cov}{\operatorname{Cov}}
\newcommand{\Var}{\operatorname{Var}}
\newcommand{\proj}{\operatorname{proj}}
\newcommand{\Perp}{\operatorname{perp}}
\newcommand{\col}{\operatorname{col}}
\newcommand{\rect}{\operatorname{rect}}
\newcommand{\sinc}{\operatorname{sinc}}
\newcommand{\IT}{\operatorname{IT}}
\newcommand{\F}{\mathcal{F}}

\newtheorem{definition}{Definition}
\newtheorem{theorem}{Theorem}


\DeclareSIUnit{\voltampere}{VA} %apparent power
\DeclareSIUnit{\pii}{\ensuremath{\pi}}

\hypersetup{%setup hyperlinks
    colorlinks,
    citecolor=black,
    filecolor=black,
    linkcolor=black,
    urlcolor=black
}

% Example boxes
\usepackage{fancybox}
\usepackage{framed}
\usepackage{adjustbox}
\newenvironment{simpages}%
{\AtBeginEnvironment{itemize}{\parskip=0pt\parsep=0pt\partopsep=0pt}
\def\FrameCommand{\fboxsep=.5\FrameSep\shadowbox}\MakeFramed{\FrameRestore}}%
{\endMakeFramed}

% Impulse train
\DeclareFontFamily{U}{wncy}{}
\DeclareFontShape{U}{wncy}{m}{n}{<->wncyr10}{}
\DeclareSymbolFont{mcy}{U}{wncy}{m}{n}
\DeclareMathSymbol{\Sha}{\mathord}{mcy}{"58}
\addbibresource{../../../../includes/bibliography.bib}

\begin{document}


\section{Introduction}
This chapter introduces a method to reconstruct the power spectral density of a signal which is sampled at sub-Nyquist frequencies. The method is largely based on \cite{ariananda2012compressive}. However, \cite{ariananda2012compressive} considers the reconstruction from a statistical perspective, where we choose a deterministic point of view. Besides novelty and innovation, there are a few practical reason for taking this point of view.

First of all, \cite{ariananda2012compressive} starts the analysis of its time-domain reconstruction approach assuming that the correlation of the output signal is known. They then proceed to discuss an unbiased estimator of this correlation. In statistics and signal processing, one commonly uses a minimum mean squared error estimator to estimate a quantity. Such an estimator minimises the mean squared error by optimally balancing the bias and the variance of the estimator. Ideally, we want to use a minimal mean squared error estimator to estimate the correlation of the outputsignal. However, from the time-domain reconstruction method described in \cite{ariananda2012compressive}, it is not clear whether the reconstruction method should change if a biased estimator is used. Also, it is not clear what the effects of using a biased estimator are. In this chapter we study a reconstruction method based on a biased estimator. This biased estimator can yield a better estimation of the correlation by having less variance. Also, it turns out that our reconstruction method can also be used when using an unbiased estimator. This makes the method more general and convenient to use.

Second, in the study of our reconstruction method, we obtain an expression which summarises the effects caused by the bias of the estimator. By having knowledge of this relationship, the effects can be either compensated for, or they can be used to one's advantage.

Finally, we make use of operators that are often available in software used for digital signal processing, such as MATLAB. This makes the analysis easy to follow and the method easy to implement.
\end{document}