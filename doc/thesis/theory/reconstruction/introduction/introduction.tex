%!TEX program = xelatex

\documentclass[a4paper, openany, oneside]{memoir}
\usepackage[no-math]{fontspec}
\usepackage{pgfplots}
\pgfplotsset{compat=newest}
\usepackage{commath}
\usepackage{mathtools}
\usepackage{amssymb}
\usepackage{amsthm}
\usepackage{booktabs}
\usepackage{mathtools}
\usepackage{xcolor}
\usepackage[separate-uncertainty=true, per-mode=symbol]{siunitx}
\usepackage[noabbrev, capitalize]{cleveref}
\usepackage{listings}
\usepackage[american inductor, european resistor]{circuitikz}
\usepackage{amsmath}
\usepackage{amsfonts}
\usepackage{ifxetex}
\usepackage[dutch,english]{babel}
\usepackage[backend=bibtexu,texencoding=utf8,bibencoding=utf8,style=ieee,sortlocale=en_GB,language=auto]{biblatex}
\usepackage[strict,autostyle]{csquotes}
\usepackage{parskip}
\usepackage{import}
\usepackage{standalone}
\usepackage{hyperref}
%\usepackage[toc,title,titletoc]{appendix}

\ifxetex{} % Fonts laden in het geval dat je met Xetex compiled
    \usepackage{fontspec}
    \defaultfontfeatures{Ligatures=TeX} % To support LaTeX quoting style
    \setromanfont{Palatino Linotype} % Tover ergens in Font mapje in root.
    \setmonofont{Source Code Pro}
\else % Terug val in standaard pdflatex tool chain. Geen ondersteuning voor OTT fonts
    \usepackage[T1]{fontenc}
    \usepackage[utf8]{inputenc}
\fi
\newcommand{\references}[1]{\begin{flushright}{#1}\end{flushright}}
\renewcommand{\vec}[1]{\boldsymbol{\mathbf{#1}}}
\newcommand{\uvec}[1]{\boldsymbol{\hat{\vec{#1}}}}
\newcommand{\mat}[1]{\boldsymbol{\mathbf{#1}}}
\newcommand{\fasor}[1]{\boldsymbol{\tilde{\vec{#1}}}}
\newcommand{\cmplx}[0]{\mathrm{j}}
\renewcommand{\Re}[0]{\operatorname{Re}}
\newcommand{\Cov}{\operatorname{Cov}}
\newcommand{\Var}{\operatorname{Var}}
\newcommand{\proj}{\operatorname{proj}}
\newcommand{\Perp}{\operatorname{perp}}
\newcommand{\col}{\operatorname{col}}
\newcommand{\rect}{\operatorname{rect}}
\newcommand{\sinc}{\operatorname{sinc}}
\newcommand{\IT}{\operatorname{IT}}
\newcommand{\F}{\mathcal{F}}

\newtheorem{definition}{Definition}
\newtheorem{theorem}{Theorem}


\DeclareSIUnit{\voltampere}{VA} %apparent power
\DeclareSIUnit{\pii}{\ensuremath{\pi}}

\hypersetup{%setup hyperlinks
    colorlinks,
    citecolor=black,
    filecolor=black,
    linkcolor=black,
    urlcolor=black
}

% Example boxes
\usepackage{fancybox}
\usepackage{framed}
\usepackage{adjustbox}
\newenvironment{simpages}%
{\AtBeginEnvironment{itemize}{\parskip=0pt\parsep=0pt\partopsep=0pt}
\def\FrameCommand{\fboxsep=.5\FrameSep\shadowbox}\MakeFramed{\FrameRestore}}%
{\endMakeFramed}

% Impulse train
\DeclareFontFamily{U}{wncy}{}
\DeclareFontShape{U}{wncy}{m}{n}{<->wncyr10}{}
\DeclareSymbolFont{mcy}{U}{wncy}{m}{n}
\DeclareMathSymbol{\Sha}{\mathord}{mcy}{"58}
\addbibresource{../../../../includes/bibliography.bib}

\begin{document}

% - intro (done)
% - afweging methoden (done)
% - algoritme (done)
% - afleiding (done)
% - garantie full rank -> lags (in afleiding) (done)
% - implementatie (generate + estimation) (todo)
% - conclusion (todo)

\section{Introduction}
In the previous chapter, we introduced the concept of multi-coset sampling. In the introduction, we argued that we want to avoid sampling at the Nyquist frequency. Therefore, the cosets in multi-coset sampling will sample at sub-Nyquist frequencies. The samples obtained by these cosets will be used to reconstruct the power spectral density of the sampled signal, which, once more, we argued to be important during detection.

This chapter introduces a method to reconstruct the power spectral density of a signal which is sampled at sub-Nyquist frequencies. Usually, sampling at sub-Nyquist frequencies causes aliasing, which means that the obtained signal is distorted. Our reconstruction method should be able to reconstruct the power spectral density of the signal in real-time, despite the fact that the sampled signal is aliased. We focus on an accessible, yet complete explanation of such a reconstruction method which discusses all details necessary for implementation.

First of all, we will consider various methods which allow for reconstruction of the power spectral density of signals sampled at sub-Nyquist frequencies. We will then choose one of these methods, based on careful considerations. Next, we will discuss an algorithm based on this method. Afterwards, we will discuss some details which turn out to be essential when implementing the algorithm.

\section{Overview}
There are various methods to reconstruct the power spectral density of a signal. These methods can be roughly divided in two groups: methodes based on $l_1$-optimisation\footnote{$l_1$-optimisation consists of optimising the $l_1$-norm of a vector.}, such as \cite{bayarkernel, candes2006robust, candes2007sparsity, candes2008introduction, kirolos2006analog, li2014gomp, polo2009compressive, pal2011coprime}, and methods which are not based on such optimisation, such as \cite{ariananda2011multicoset,ariananda2012compressive}.

Methods of the first group are based on the assumption that the signal contains few frequencies.%
\footnote{More specifically, methods of the first group are based on the assumption that the signal is sparse in a certain basis. Let $\vec{x}$ denote the signal sampled at the Nyquist-frequency. Also, let the matrix $\mat{\Phi}$ consist of rows of the identity matrix such that $\vec{y}=\mat{\Phi}\vec{x}$ represents the signal sampled at a sub-Nyquist frequency. If $\vec{x}$ is sparse in the basis $\mat{\Psi}$, then there exists a sparse vector $\vec{s}$ such that $\vec{x}=\mat{\Psi}\vec{s}$. Given $\vec{y}$, the original signal $\vec{x}$ can be obtained by finding the most sparse vector $\vec{\hat{s}}$ such that $\vec{y}=\mat{\Phi}\mat{\Psi}\vec{\hat{s}}$. Then $\vec{x} = \mat{\Psi}\vec{\hat{s}}$. This problem can be formulated as an $l_1$-optimisation problem. Since it is assumed that the signal contains few frequencies, $\mat{\Psi}$ is often chosen to be the discrete Fourier transform matrix.}
They then proceed to find a signal which contains few frequencies and matches the signal sampled at sub-Nyquist frequencies. This can be done by optimisation or by algorithms such as the MUSIC algorithm \cite{pal2011coprime}. There are a few problems with this approach. First, the assumption that the signal contains few frequencies may be not true in our case. Second, solving an optimisation problem or making use of an algorithm such as the MUSIC algorithm real-time yields a computationally expensive and complex reconstruction method. Based on these two reasons, we decide to discuss a reconstruction method based on the method discussed in \cite{ariananda2012compressive}.


\end{document}