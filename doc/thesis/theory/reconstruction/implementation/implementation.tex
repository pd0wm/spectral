%!TEX program = xelatex

\documentclass[a4paper, openany, oneside]{memoir}
\usepackage[no-math]{fontspec}
\usepackage{pgfplots}
\usepackage{float}
\pgfplotsset{compat=newest}
\usepackage{commath}
\usepackage{mathtools}
\usepackage{amssymb}
\usepackage{amsthm}
\usepackage{booktabs}
\usepackage{todonotes}
\usepackage{mathtools}
\usepackage{xcolor}
\usepackage[separate-uncertainty=true, per-mode=symbol]{siunitx}
\usepackage{listings}
\usepackage[american inductor, european resistor]{circuitikz}
\usepackage{amsmath}
\usepackage{amsfonts}
\usepackage{ifxetex}
\usepackage[dutch,english]{babel}
\usepackage[backend=bibtexu,texencoding=utf8,bibencoding=utf8,style=ieee,sortlocale=en_GB,language=auto]{biblatex}
\usepackage[strict,autostyle]{csquotes}
\usepackage{import}
\usepackage{standalone}
\usepackage{bookmark,hyperref}
\usepackage{xcolor,mdframed}
\usepackage{tikz}
\usepackage{framed}
\usepackage{float}
\usepackage{tabularx}
\usepackage{graphicx,adjustbox}
\usepackage{rotating}
\usepackage{pdfpages}
\usepackage{enumitem}
\usepackage{calc}
\usepackage{pgfplots}
\usepackage{filecontents}
\usepackage{caption}
\usepackage{subcaption}
\usepackage{lettrine}

\newcolumntype{Y}{>{\raggedright\arraybackslash}X} % Left-justified text in tabularx environment

\ifxetex{} % Fonts laden in het geval dat je met Xetex compiled
    \usepackage{fontspec}
    \defaultfontfeatures{Scale=MatchLowercase, Ligatures=TeX} % To support LaTeX quoting style
    %\setromanfont{Palatino Linotype} % Tover ergens in Font mapje in root.
    \setsansfont{Avenir Next LT Pro}
    \setromanfont{Adobe Caslon Pro} % Tover ergens in Font mapje in root.
    \setmonofont{Source Code Pro}
\else % Terug val in standaard pdflatex tool chain. Geen ondersteuning voor OTT fonts
    \usepackage[T1]{fontenc}
    \usepackage[utf8]{inputenc}
\fi
\usepackage[noabbrev, capitalize]{cleveref}
\usepackage{ifthen}
\usepackage{titlesec}
\usepackage{titlecaps}

\newcommand{\references}[1]{\begin{flushright}{#1}\end{flushright}}
\renewcommand{\vec}[1]{\boldsymbol{\mathbf{#1}}}
\newcommand{\uvec}[1]{\boldsymbol{\hat{\vec{#1}}}}
\newcommand{\mat}[1]{\boldsymbol{\mathbf{#1}}}
\newcommand{\fasor}[1]{\boldsymbol{\tilde{\vec{#1}}}}
\newcommand{\cmplx}[0]{\mathrm{j}}
\renewcommand{\Re}[0]{\operatorname{Re}}
\newcommand{\Cov}{\operatorname{Cov}}
\newcommand{\Var}{\operatorname{Var}}
\newcommand{\proj}{\operatorname{proj}}
\newcommand{\Perp}{\operatorname{perp}}
\newcommand{\col}{\operatorname{col}}
\newcommand{\rect}{\operatorname{rect}}
\newcommand{\sinc}{\operatorname{sinc}}
\newcommand{\lcm}{\operatorname{lcm}}
%\newcommand{\gcd}{\operatorname{gcd}}
\newcommand{\F}{\mathcal{F}}
\newcommand{\DTFT}{\mathcal{F}_*}
\newcommand{\conj}[1]{#1^*}
\renewcommand{\mod}{\operatorname{mod}}
\newcommand{\rot}{\operatorname{rot}}
\newcommand{\vecsc}[1]{\vec{\textsc{\textbf{#1}}}}
\renewcommand{\ss}[1]{_{#1}}

% Label without linebreak breaker
\newcommand{\lab}[1]{\label{#1}\nolinebreak}

\newtheorem{definition}{Definition}
\newtheorem{theorem}{Theorem}


\DeclareSIUnit{\voltampere}{VA} %apparent power
\DeclareSIUnit{\pii}{\ensuremath{\pi}}

\hypersetup{%setup hyperlinks
    colorlinks,
    citecolor=black,
    filecolor=black,
    linkcolor=black,
    urlcolor=black
}

% Example boxes
\usepackage{fancybox}
\usepackage{framed}
\usepackage{adjustbox}
\newenvironment{simpages}%
{\AtBeginEnvironment{itemize}{\parskip=0pt\parsep=0pt\partopsep=0pt}
\def\FrameCommand{\fboxsep=.5\FrameSep\shadowbox}\MakeFramed{\FrameRestore}}%
{\endMakeFramed}

% Impulse train
\DeclareFontFamily{U}{wncy}{}
\DeclareFontShape{U}{wncy}{m}{n}{<->wncyr10}{}
\DeclareSymbolFont{mcy}{U}{wncy}{m}{n}
\DeclareMathSymbol{\Sha}{\mathord}{mcy}{"58}

\setlength{\parindent}{0pt}
\nonzeroparskip

% Block environment configuration
\newcommand{\BlockLeftMargin}{20pt}
\newcommand{\BlockLeftMarginText}{25pt}
\newcommand{\BlockLeftMarginTextSpacing}{10pt}

% Own colours
\definecolor{gray75}{gray}{0.75}

% Block environment
\newenvironment{block}[3]{%
\makebox{\hspace{-\spinemargin}%
\begin{tikzpicture}[overlay]
    \draw [thick,color=gray75] (\BlockLeftMargin, 0) -- (\paperwidth - \spinemargin, 0);
    \node at (\BlockLeftMarginText, -0.9) [align=left, text width=\spinemargin - \BlockLeftMarginText - \BlockLeftMarginTextSpacing, anchor=west, text depth=1cm] {\textbf{\textsc{#1}}\newline\textit{#3}};
\end{tikzpicture}}%
\nopagebreak\\[0.25em]\ifthenelse{\equal{#2}{}}{}{(\textit{#2}.) }\nopagebreak\nolinebreak}
{\nopagebreak\\[-0.25em]%
\makebox{\hspace{-\spinemargin}%
\begin{tikzpicture}[overlay, remember picture]
    \draw [thick,color=gray75] (\spinemargin,0) -- (\paperwidth - \spinemargin,0);
\end{tikzpicture}} \vspace{0.5em}}

% Theorem
\newcounter{blockTheoremCounter}
\crefname{blockTheoremCounter}{Theorem}{Theorems}
\Crefname{blockTheoremCounter}{Theorem}{Theorems}

\newenvironment{blockTheorem}[1][]{%
\refstepcounter{blockTheoremCounter}%
\begin{block}{theorem \theblockTheoremCounter}{#1}{}}
{\end{block}}

% Definition
\newcounter{blockDefinitionCounter}
\crefname{blockDefinitionCounter}{Definition}{Definitions}
\Crefname{blockDefinitionCounter}{Definition}{Definitions}

\newenvironment{blockDefinition}[1][]{%
\refstepcounter{blockDefinitionCounter}%
\begin{block}{definition \theblockDefinitionCounter}{#1}{}}
{\end{block}}

% Proof
\newcounter{blockProofTheoremCounter}
\crefname{blockProofTheoremCounter}{Proof}{Proofs}
\Crefname{blockProofTheoremCounter}{Proof}{Proofs}

\newenvironment{blockProofTheorem}[1]{%
\refstepcounter{blockProofTheoremCounter}%
\begin{block}{proof of \\ theorem #1}{}{}}
{\qed\end{block}}

% Detail
\newcounter{blockDetailCounter}
\crefname{blockDetailCounter}{Detail}{Details}
\Crefname{blockDetailCounter}{Detail}{Details}

\newenvironment{blockDetail}[1][]{%
\refstepcounter{blockDetailCounter}%
\begin{block}{detail \theblockDetailCounter}{#1}{}}
{\end{block}}

% Redesign chapter headings
\newcommand{\chapternumber}{\thechapter}
\newcommand{\hsp}{\hspace{20pt}}
\titleformat{\chapter}[hang]{\Huge\bfseries}{\chapternumber\hsp\textcolor{gray75}{|}\hsp}{0pt}{\Huge\bfseries}

% Remove headers
% \addtopsmarks{headings}{}{
%   \createmark{chapter}{left}{nonumber}{}{}
% }
% \pagestyle{headings} % Activate changes

% Capitalise headers in a regular way
\renewcommand*{\memUChead}[1]{\titlecap{#1}}

% \hfill for math mode
\newcommand{\pushright}[1]{\intertext{\hfill$\displaystyle #1$}}
\newcommand{\pushline}{\hskip \textwidth minus \textwidth}
\newcommand{\matlab}{\textsc{Matlab}}

\definecolor{code-grey}{HTML}{DDDDDD}
\newcommand{\lib}[1]{\textsf{#1}}
\newcommand{\file}[1]{\textsf{#1}}
\newcommand{\func}[1]{\colorbox{code-grey}{\texttt{#1}}}
\newcommand{\class}[1]{\colorbox{code-grey}{\texttt{#1}}}

% Setup actiepunten
\newenvironment{important}[1][]{%
   \begin{mdframed}[%
      backgroundcolor={red!15}, hidealllines=true,
      skipabove=0.7\baselineskip, skipbelow=0.7\baselineskip,
      splitbottomskip=2pt, splittopskip=4pt, #1]%
   \makebox[0pt]{% ignore the withd of !
      \smash{% ignor the height of !
         \fontsize{32pt}{32pt}\selectfont% make the ! bigger
         \hspace*{-19pt}% move ! to the left
         \raisebox{-2pt}{% move ! up a little
            {\color{red!70!black}\sffamily\bfseries !}% type the bold red !
         }%
      }%
   }%
}{\end{mdframed}}
\newcommand{\excl}[1]{
\begin{important}
  \textbf{#1}
\end{important}
}

\makeatletter
\newcommand\footnoteref[1]{\protected@xdef\@thefnmark{\ref{#1}}\@footnotemark}
\makeatother

% Allow page breaks in display environments
%\allowdisplaybreaks
% S unit for use in Mega Samples per second
\DeclareSIUnit\sample{S}

\newcommand{\CC}{C\nolinebreak\hspace{-.05em}\raisebox{.3ex}{ \textbf{+}}\nolinebreak\hspace{-.10em}\raisebox{.3ex}{\textbf{+}}}
\def\CC{{C\nolinebreak[4]\hspace{-.05em}\raisebox{.3ex}{\textbf{++}}}}


\newcommand{\partauthor}[1]{\gdef\@partauthor{#1}}
\renewcommand{\printparttitle}[1]{
  \parttitlefont #1\\
  \vspace{1.5cm}
  \textnormal{\Large \@partauthor}
}
\addbibresource{../../../../includes/bibliography.bib}

\begin{document}

\section{Implementational details}
\label{sec:reconstruction-implementation}
This section will discuss some details of the algorithm which should be considered when implementing the algorithm.

\subsection{Restrictions on the sampling signals}
\label{sub:reconstruction-ci}
This subsection discusses sufficient restrictions on $c_i[n]$ which yield a correct estimation. The derivation of these restrictions can be found in \cref{sec:reconstruction-derivation}. Every $c_i[n]=0$ for $n < 0$ and $n > N-1$. Furthermore, all $c_i[n]=1$ for $n=n_i$ and otherwise zero. Finally, the following problem must be satisfied.

\begin{description}
    \item[Circular Sparse Ruler Problem] The set consisting of $|n_i - n_j|$ and $N-|n_i-n_j|$ for all combinations of $i$ and $j$ make up $0,\ldots,N-1$.
\end{description}

This problem will be further discussed in \cref{cha:sampling_methods}.

\subsection{Estimation of the cross-correlations and autocorrelation}
\label{sub:reconstruction-estimation}
In Step 4 of the algorithm we have to estimate $r_{y_i,y_j}[m]$. Suppose that $y_i[n]$ is known for $n = 0,\ldots,KL-1$. Since we assumed that $r_{y_i,y_j}[m]=0$ for $|m|>L$, it remains to estimate $r_{y_i,y_j}[m]$ for $|m| \le L$. An estimator of $r_{y_i,y_j}[m]$ is given by
\begin{align} \label{eq:ryij-est}
    \hat{r}_{y_i,y_j}[m] = \frac{1}{KL-|m|}\sum_{k=l}^{u}y_i[k]\conj{y}_j[k+m]
\end{align}
where $l=-\max\{0,m\}$ and $u=KL-1-\min\{0,m\}$ \cite{hayes1996statistical}. It is worthwhile to notice that $E(\hat{r}_{y_i,y_j}[m])=r_{y_i,y_j}[m]$, which means that $\hat{r}_{y_i,y_j}[m]$ is an unbiased estimator of $r_{y_i,y_j}[m]$.
% Similarly, we define
% \begin{align*}
%     \hat{\vec{r}}_{y_i,y_j} = \begin{bmatrix}
%         \hat{r}_{y_i,y_j}[L] & \cdots & \hat{r}_{y_i,y_j}[-L]
%     \end{bmatrix}^T.
% \end{align*}
% Then $E(\hat{\vec{r}}_{y_i,y_j})=\vec{r}_{y_i,y_j}$, which means that $\hat{\vec{r}}_{y_i,y_j}$ is an unbiased estimator of $\vec{r}_{y_i,y_j}$. Now consider \cref{eq:ry-R-rx}. Notice that
% \begin{align*}
%     E(\hat{\vec{r}}_{y_i,y_j}) = \vec{r}_{y_i,y_j} = \mat{R} \vec{r}_x.
% \end{align*}
% Thus $\hat{\vec{r}}_{y_i,y_j}$ can be used to estimate $\vec{r}_x$. Denote this estimation of $\vec{r}_x$ by $\hat{\vec{r}}_x$.

\subsection{Sparsity}
\label{sub:reconstruction-sparsity}
Operations on large matrices can be computationally expensive. If a matrix is sparse, then functions designed for sparse matrices can be used. These functions can provide significant speedups. In the discussion on the unicity of the estimation during the derivation of the algorithm, we argued that every row of $\mat{R}$ has exactly one nonzero element. Since $\mat{R}$ has $2LN+1$ columns, the fration of nonzero elements of $\mat{R}$ is given by
\begin{align*}
    \rho_{\text{nz}}=\frac{1}{2LN+1}.
\end{align*}
Since $LN$ determines the resolution of the estimated power spectral density of the input signal, the product $LN$ is usually chosen large. This means that $\mat{R}$ is usually a sparse matrix.

\subsection{Efficient generation of the matrices}
\label{sub:reconstruction-generation}
It turns out that $\mat{R}$ can be generated efficiently using commonly available functions. To generate $\mat{R}$, we make use of the functions $\operatorname{toeplitz}$ and $\operatorname{tril}$. Given a vector, $\operatorname{toeplitz}$ generates a Toeplitz matrix from this vector. Given a matrix, $\operatorname{tril}$ keeps the lower triangular matrix of this matrix and sets the other elements to zero. The functions $\operatorname{toeplitz}$ and $\operatorname{tril}$ are often found in software used for digital signal processing. The operation of these functions is illustrated in \cref{sec:reconstruction-generation-algorithm}. The algorithm to construct $\mat{R}$ is as follows. An example of the algorithm which illustrates its correctness is given in \cref{sec:reconstruction-generation-algorithm}.

\begin{tabularx}{\textwidth}{rY}
    Step 1: & Let \begin{align*}
                \vec{c} = \begin{bmatrix} c_{i,j}[N-1] \; \cdots \; c_{i,j}[-N+1] \end{bmatrix}
\end{align*} where \begin{align*}
                c_{i,j}[m] = \sum_{k=-\infty}^{\infty}c_i[k] \conj{c}_j[k+m].
            \end{align*} \\
    Step 2: & Append $\vec{c}$ with $2(L-1)N+1$ zeros. \\ \\
    Step 3: & Calculate $\operatorname{tril}[\operatorname{toeplitz}(\vec{c})]$. \\ \\
    Step 4: & Omit the first $N-1$ rows and the last $N-1$ columns. \\ \\
    Step 5: & Keep rows $1,N,\ldots,2LN+1$. This yields $\mat{R}_{i,j}$ \\ \\
    Step 6: & Construct \begin{align*}
        \mat{R} = \begin{bmatrix}
            \mat{R}_{1,1} \\ \vdots \\ \mat{R}_{M,M}
        \end{bmatrix}.
    \end{align*} \\
\end{tabularx}

The advantage is that this algorithm takes advantage of highly optimised implementations of $\operatorname{toeplitz}$ and $\operatorname{tril}$. Furthermore, Step 4 and 5 can be vectorised.
\end{document}