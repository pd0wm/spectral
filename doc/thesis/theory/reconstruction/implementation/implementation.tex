%!TEX program = xelatex

\documentclass[a4paper, openany, oneside]{memoir}
\usepackage[no-math]{fontspec}
\usepackage{pgfplots}
\pgfplotsset{compat=newest}
\usepackage{commath}
\usepackage{mathtools}
\usepackage{amssymb}
\usepackage{amsthm}
\usepackage{booktabs}
\usepackage{mathtools}
\usepackage{xcolor}
\usepackage[separate-uncertainty=true, per-mode=symbol]{siunitx}
\usepackage[noabbrev, capitalize]{cleveref}
\usepackage{listings}
\usepackage[american inductor, european resistor]{circuitikz}
\usepackage{amsmath}
\usepackage{amsfonts}
\usepackage{ifxetex}
\usepackage[dutch,english]{babel}
\usepackage[backend=bibtexu,texencoding=utf8,bibencoding=utf8,style=ieee,sortlocale=en_GB,language=auto]{biblatex}
\usepackage[strict,autostyle]{csquotes}
\usepackage{parskip}
\usepackage{import}
\usepackage{standalone}
\usepackage{hyperref}
%\usepackage[toc,title,titletoc]{appendix}

\ifxetex{} % Fonts laden in het geval dat je met Xetex compiled
    \usepackage{fontspec}
    \defaultfontfeatures{Ligatures=TeX} % To support LaTeX quoting style
    \setromanfont{Palatino Linotype} % Tover ergens in Font mapje in root.
    \setmonofont{Source Code Pro}
\else % Terug val in standaard pdflatex tool chain. Geen ondersteuning voor OTT fonts
    \usepackage[T1]{fontenc}
    \usepackage[utf8]{inputenc}
\fi
\newcommand{\references}[1]{\begin{flushright}{#1}\end{flushright}}
\renewcommand{\vec}[1]{\boldsymbol{\mathbf{#1}}}
\newcommand{\uvec}[1]{\boldsymbol{\hat{\vec{#1}}}}
\newcommand{\mat}[1]{\boldsymbol{\mathbf{#1}}}
\newcommand{\fasor}[1]{\boldsymbol{\tilde{\vec{#1}}}}
\newcommand{\cmplx}[0]{\mathrm{j}}
\renewcommand{\Re}[0]{\operatorname{Re}}
\newcommand{\Cov}{\operatorname{Cov}}
\newcommand{\Var}{\operatorname{Var}}
\newcommand{\proj}{\operatorname{proj}}
\newcommand{\Perp}{\operatorname{perp}}
\newcommand{\col}{\operatorname{col}}
\newcommand{\rect}{\operatorname{rect}}
\newcommand{\sinc}{\operatorname{sinc}}
\newcommand{\IT}{\operatorname{IT}}
\newcommand{\F}{\mathcal{F}}

\newtheorem{definition}{Definition}
\newtheorem{theorem}{Theorem}


\DeclareSIUnit{\voltampere}{VA} %apparent power
\DeclareSIUnit{\pii}{\ensuremath{\pi}}

\hypersetup{%setup hyperlinks
    colorlinks,
    citecolor=black,
    filecolor=black,
    linkcolor=black,
    urlcolor=black
}

% Example boxes
\usepackage{fancybox}
\usepackage{framed}
\usepackage{adjustbox}
\newenvironment{simpages}%
{\AtBeginEnvironment{itemize}{\parskip=0pt\parsep=0pt\partopsep=0pt}
\def\FrameCommand{\fboxsep=.5\FrameSep\shadowbox}\MakeFramed{\FrameRestore}}%
{\endMakeFramed}

% Impulse train
\DeclareFontFamily{U}{wncy}{}
\DeclareFontShape{U}{wncy}{m}{n}{<->wncyr10}{}
\DeclareSymbolFont{mcy}{U}{wncy}{m}{n}
\DeclareMathSymbol{\Sha}{\mathord}{mcy}{"58}
\addbibresource{../../../../includes/bibliography.bib}

\begin{document}

\section{Implementation}
\label{sec:reconstruction-implementation}
This section will discuss some details of the algorithm which are relevant when implementing the algorithm.

\subsection{Estimation of the cross-correlations and autocorrelation}
\label{sub:reconstruction-estimation}
In Step 4 of the algorithm we have to estimate $r_{y_i,y_j}[m]$. Suppose that $y_i[m]$ is known for $m = 0,\ldots,KL-1$. Since we assumed that $r_{y_i,y_j}[m]=0$ for $|m|>L$, it remains to estimate $r_{y_i,y_j}[m]$ for $|m| \le L$. An estimator of $r_{y_i,y_j}[m]$ is given by
\begin{align*}
    \hat{r}_{y_i,y_j}[m] = \frac{1}{KL-|m|}\sum_{k=l}^{u}y_i[k]\conj{y}_j[k+m]
\end{align*}
where $l=-\max\{0,m\}$ and $u=KL-1-\min\{0,m\}$ \cite{hayes1996statistical}. It is worthwhile to notice that $E(\hat{r}_{y_i,y_j}[m])=r_{y_i,y_j}[m]$, which means that $\hat{r}_{y_i,y_j}[m]$ is an unbiased estimator of $r_{y_i,y_j}[m]$. Similarly, we define
\begin{align*}
    \hat{\vec{r}}_{y_i,y_j} = \begin{bmatrix}
        \hat{r}_{y_i,y_j}[L] & \cdots & \hat{r}_{y_i,y_j}[-L]
    \end{bmatrix}.
\end{align*}
Then $E(\hat{\vec{r}}_{y_i,y_j})=\vec{r}_{y_i,y_j}$, which means that $\hat{\vec{r}}_{y_i,y_j}$ is an unbiased estimator of $\vec{r}_{y_i,y_j}$. Now consider \cref{eq:ry-R-rx}. Notice that
\begin{align*}
    E(\hat{\vec{r}}_{y_i,y_j}) = \vec{r}_{y_i,y_j} = \mat{R} \vec{r}_x.
\end{align*}
Thus $\hat{\vec{r}}_{y_i,y_j}$ can be used to estimate $\vec{r}_x$. Denote this estimation of $\vec{r}_x$ by $\hat{\vec{r}}_x$.

\subsection{Sparsity of the matrices}
\label{sub:reconstruction-sparsity}
- sparsity 

\subsection{Efficient generation of the matrices}
\label{sub:reconstruction-generation}
- generation of R


\end{document}