%!TEX program = xelatex

\documentclass[a4paper, openany, oneside]{memoir}
\usepackage[no-math]{fontspec}
\usepackage{pgfplots}
\pgfplotsset{compat=newest}
\usepackage{commath}
\usepackage{mathtools}
\usepackage{amssymb}
\usepackage{amsthm}
\usepackage{booktabs}
\usepackage{mathtools}
\usepackage{xcolor}
\usepackage[separate-uncertainty=true, per-mode=symbol]{siunitx}
\usepackage[noabbrev, capitalize]{cleveref}
\usepackage{listings}
\usepackage[american inductor, european resistor]{circuitikz}
\usepackage{amsmath}
\usepackage{amsfonts}
\usepackage{ifxetex}
\usepackage[dutch,english]{babel}
\usepackage[backend=bibtexu,texencoding=utf8,bibencoding=utf8,style=ieee,sortlocale=en_GB,language=auto]{biblatex}
\usepackage[strict,autostyle]{csquotes}
\usepackage{parskip}
\usepackage{import}
\usepackage{standalone}
\usepackage{hyperref}
%\usepackage[toc,title,titletoc]{appendix}

\ifxetex{} % Fonts laden in het geval dat je met Xetex compiled
    \usepackage{fontspec}
    \defaultfontfeatures{Ligatures=TeX} % To support LaTeX quoting style
    \setromanfont{Palatino Linotype} % Tover ergens in Font mapje in root.
    \setmonofont{Source Code Pro}
\else % Terug val in standaard pdflatex tool chain. Geen ondersteuning voor OTT fonts
    \usepackage[T1]{fontenc}
    \usepackage[utf8]{inputenc}
\fi
\newcommand{\references}[1]{\begin{flushright}{#1}\end{flushright}}
\renewcommand{\vec}[1]{\boldsymbol{\mathbf{#1}}}
\newcommand{\uvec}[1]{\boldsymbol{\hat{\vec{#1}}}}
\newcommand{\mat}[1]{\boldsymbol{\mathbf{#1}}}
\newcommand{\fasor}[1]{\boldsymbol{\tilde{\vec{#1}}}}
\newcommand{\cmplx}[0]{\mathrm{j}}
\renewcommand{\Re}[0]{\operatorname{Re}}
\newcommand{\Cov}{\operatorname{Cov}}
\newcommand{\Var}{\operatorname{Var}}
\newcommand{\proj}{\operatorname{proj}}
\newcommand{\Perp}{\operatorname{perp}}
\newcommand{\col}{\operatorname{col}}
\newcommand{\rect}{\operatorname{rect}}
\newcommand{\sinc}{\operatorname{sinc}}
\newcommand{\IT}{\operatorname{IT}}
\newcommand{\F}{\mathcal{F}}

\newtheorem{definition}{Definition}
\newtheorem{theorem}{Theorem}


\DeclareSIUnit{\voltampere}{VA} %apparent power
\DeclareSIUnit{\pii}{\ensuremath{\pi}}

\hypersetup{%setup hyperlinks
    colorlinks,
    citecolor=black,
    filecolor=black,
    linkcolor=black,
    urlcolor=black
}

% Example boxes
\usepackage{fancybox}
\usepackage{framed}
\usepackage{adjustbox}
\newenvironment{simpages}%
{\AtBeginEnvironment{itemize}{\parskip=0pt\parsep=0pt\partopsep=0pt}
\def\FrameCommand{\fboxsep=.5\FrameSep\shadowbox}\MakeFramed{\FrameRestore}}%
{\endMakeFramed}

% Impulse train
\DeclareFontFamily{U}{wncy}{}
\DeclareFontShape{U}{wncy}{m}{n}{<->wncyr10}{}
\DeclareSymbolFont{mcy}{U}{wncy}{m}{n}
\DeclareMathSymbol{\Sha}{\mathord}{mcy}{"58}
\addbibresource{../../../../includes/bibliography.bib}

\begin{document}

\section{Implementational details}
\label{sec:reconstruction-implementation}
This section will discuss some details of the algorithm which should be considered when implementing the algorithm.

\subsection{Restrictions on the sampling signals}
\label{sub:reconstruction-ci}
This subsection discusses sufficient restrictions on the sampling signal $c_i[n]$ which yield a correct estimation. Remember that the sampling signal determines the way a coset samples the input signal. The derivation of these restrictions can be found in \cref{sec:reconstruction-derivation}. The restrictions are as follows.

\begin{enumerate}
    \item Every $c_i[n]=0$ for $n < 0$ and $n > N-1$.
    \item All $c_i[n]=1$ for a single $n=n_i$ and otherwise zero.
    \item The circular sparse ruler problem must be satisfied.
\end{enumerate}

\begin{description}
    \item[Circular sparse ruler problem] The set consisting of $|n_i - n_j|$ and $N-|n_i-n_j|$ for all combinations of cosets $i$ and $j$ must make up $0,\ldots,N-1$.
\end{description}

The circular sparse ruler problem will be further discussed in \cref{cha:sampling_methods}.

\subsection{Estimation of the cross-correlations}
\label{sub:reconstruction-estimation}
In Step 4 of the algorithm we have to estimate $r_{y_i,y_j}[m]$. Suppose that $y_i[n]$ is known for $n = 0,\ldots,KL-1$. We assumed that $r_{y_i,y_j}[m]=0$ for $|m|>L$. It then remains to estimate $r_{y_i,y_j}[m]$ for $|m| \le L$. An estimator of $r_{y_i,y_j}[m]$ is given by
\begin{align} \label{eq:ryij-est}
    \hat{r}_{y_i,y_j}[m] = \frac{1}{KL-|m|}\sum_{k=l}^{u}y_i[k]\conj{y}_j[k+m]
\end{align}
where $l=-\max\{0,m\}$ and $u=KL-1-\min\{0,m\}$ \cite{hayes1996statistical}. Notice that $E(\hat{r}_{y_i,y_j}[m])=r_{y_i,y_j}[m]$, which means that $\hat{r}_{y_i,y_j}[m]$ is an unbiased estimator of $r_{y_i,y_j}[m]$.
% Similarly, we define
% \begin{align*}
%     \hat{\vec{r}}_{y_i,y_j} = \begin{bmatrix}
%         \hat{r}_{y_i,y_j}[L] & \cdots & \hat{r}_{y_i,y_j}[-L]
%     \end{bmatrix}^T.
% \end{align*}
% Then $E(\hat{\vec{r}}_{y_i,y_j})=\vec{r}_{y_i,y_j}$, which means that $\hat{\vec{r}}_{y_i,y_j}$ is an unbiased estimator of $\vec{r}_{y_i,y_j}$. Now consider \cref{eq:ry-R-rx}. Notice that
% \begin{align*}
%     E(\hat{\vec{r}}_{y_i,y_j}) = \vec{r}_{y_i,y_j} = \mat{R} \vec{r}_x.
% \end{align*}
% Thus $\hat{\vec{r}}_{y_i,y_j}$ can be used to estimate $\vec{r}_x$. Denote this estimation of $\vec{r}_x$ by $\hat{\vec{r}}_x$.

\subsection{Sparsity}
\label{sub:reconstruction-sparsity}
Operations on large matrices can be computationally expensive. If a matrix is sparse, then functions designed for sparse matrices can be used. These functions can provide significant speedups. In the discussion on the unicity of the estimation during the derivation of the algorithm, we argued that every row of $\mat{R}$ has exactly one nonzero element. Remember that $\mat{R}$ describes the relationship between the autocorrelation of the input signal and the observed quantities. Since $\mat{R}$ has $2LN+1$ columns, the fraction of nonzero elements of $\mat{R}$ is given by
\begin{align*}
    \rho_{\text{nz}}=\frac{1}{2LN+1}.
\end{align*}
Since $LN$ determines the resolution of the estimated power spectral density of the input signal, the product $LN$ is usually chosen large. This means that $\mat{R}$ is usually a sparse matrix. This property will be exploited \cref{sec:sparse-matrices}.

\subsection{Efficient generation of the matrices}
\label{sub:reconstruction-generation}
It turns out that $\mat{R}$ can be generated efficiently using commonly available functions. To generate $\mat{R}$, we make use of the functions $\operatorname{toeplitz}$ and $\operatorname{tril}$. Given a vector, $\operatorname{toeplitz}$ generates a Toeplitz matrix from this vector. Given a matrix, $\operatorname{tril}$ keeps the lower triangular matrix of this matrix and sets the other elements to zero. The functions $\operatorname{toeplitz}$ and $\operatorname{tril}$ are often found in software used for digital signal processing. The operation of these functions is illustrated in \cref{sec:reconstruction-generation-algorithm}. The algorithm to construct $\mat{R}$ is as follows.

\begin{enumerate}[labelindent=0pt,labelwidth=\widthof{\ref{last-item2}},label=Step \arabic*:,itemindent=1em,leftmargin=!]
    \item Let \begin{align*}
                \vec{c} = \begin{bmatrix} c_{i,j}[N-1] \; \cdots \; c_{i,j}[-N+1] \end{bmatrix}
\end{align*} where \begin{align*}
                c_{i,j}[m] = \sum_{k=-\infty}^{\infty}c_i[k] \conj{c}_j[k+m].
            \end{align*}
    \item Append $\vec{c}$ with $2(L-1)N+1$ zeros.
    \item Calculate $\operatorname{tril}[\operatorname{toeplitz}(\vec{c})]$.
    \item Omit the first $N-1$ rows and the last $N-1$ columns.
    \item Keep rows $1,N,\ldots,2LN+1$. This yields $\mat{R}_{i,j}$
    \item Construct \begin{align*}
        \mat{R} = \begin{bmatrix}
            \mat{R}_{1,1} \\ \vdots \\ \mat{R}_{M,M}
        \end{bmatrix}.
    \end{align*}
    \label{last-item2}
\end{enumerate}

An example of the algorithm which illustrates its correctness is given in \cref{sec:reconstruction-generation-algorithm}. This algorithm can take advantage of highly optimised implementations of $\operatorname{toeplitz}$ and $\operatorname{tril}$. Furthermore, Step 4 and 5 can be vectorised. Therefore, the algorithm allows for fast reconstruction of $\mat{R}$.
\end{document}