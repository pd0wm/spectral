%!TEX program = xelatex

\documentclass[a4paper, openany, oneside]{memoir}
\usepackage[no-math]{fontspec}
\usepackage{pgfplots}
\pgfplotsset{compat=newest}
\usepackage{commath}
\usepackage{mathtools}
\usepackage{amssymb}
\usepackage{amsthm}
\usepackage{booktabs}
\usepackage{mathtools}
\usepackage{xcolor}
\usepackage[separate-uncertainty=true, per-mode=symbol]{siunitx}
\usepackage[noabbrev, capitalize]{cleveref}
\usepackage{listings}
\usepackage[american inductor, european resistor]{circuitikz}
\usepackage{amsmath}
\usepackage{amsfonts}
\usepackage{ifxetex}
\usepackage[dutch,english]{babel}
\usepackage[backend=bibtexu,texencoding=utf8,bibencoding=utf8,style=ieee,sortlocale=en_GB,language=auto]{biblatex}
\usepackage[strict,autostyle]{csquotes}
\usepackage{parskip}
\usepackage{import}
\usepackage{standalone}
\usepackage{hyperref}
%\usepackage[toc,title,titletoc]{appendix}

\ifxetex{} % Fonts laden in het geval dat je met Xetex compiled
    \usepackage{fontspec}
    \defaultfontfeatures{Ligatures=TeX} % To support LaTeX quoting style
    \setromanfont{Palatino Linotype} % Tover ergens in Font mapje in root.
    \setmonofont{Source Code Pro}
\else % Terug val in standaard pdflatex tool chain. Geen ondersteuning voor OTT fonts
    \usepackage[T1]{fontenc}
    \usepackage[utf8]{inputenc}
\fi
\newcommand{\references}[1]{\begin{flushright}{#1}\end{flushright}}
\renewcommand{\vec}[1]{\boldsymbol{\mathbf{#1}}}
\newcommand{\uvec}[1]{\boldsymbol{\hat{\vec{#1}}}}
\newcommand{\mat}[1]{\boldsymbol{\mathbf{#1}}}
\newcommand{\fasor}[1]{\boldsymbol{\tilde{\vec{#1}}}}
\newcommand{\cmplx}[0]{\mathrm{j}}
\renewcommand{\Re}[0]{\operatorname{Re}}
\newcommand{\Cov}{\operatorname{Cov}}
\newcommand{\Var}{\operatorname{Var}}
\newcommand{\proj}{\operatorname{proj}}
\newcommand{\Perp}{\operatorname{perp}}
\newcommand{\col}{\operatorname{col}}
\newcommand{\rect}{\operatorname{rect}}
\newcommand{\sinc}{\operatorname{sinc}}
\newcommand{\IT}{\operatorname{IT}}
\newcommand{\F}{\mathcal{F}}

\newtheorem{definition}{Definition}
\newtheorem{theorem}{Theorem}


\DeclareSIUnit{\voltampere}{VA} %apparent power
\DeclareSIUnit{\pii}{\ensuremath{\pi}}

\hypersetup{%setup hyperlinks
    colorlinks,
    citecolor=black,
    filecolor=black,
    linkcolor=black,
    urlcolor=black
}

% Example boxes
\usepackage{fancybox}
\usepackage{framed}
\usepackage{adjustbox}
\newenvironment{simpages}%
{\AtBeginEnvironment{itemize}{\parskip=0pt\parsep=0pt\partopsep=0pt}
\def\FrameCommand{\fboxsep=.5\FrameSep\shadowbox}\MakeFramed{\FrameRestore}}%
{\endMakeFramed}

% Impulse train
\DeclareFontFamily{U}{wncy}{}
\DeclareFontShape{U}{wncy}{m}{n}{<->wncyr10}{}
\DeclareSymbolFont{mcy}{U}{wncy}{m}{n}
\DeclareMathSymbol{\Sha}{\mathord}{mcy}{"58}
\addbibresource{../../../includes/bibliography.bib}

\title{Introduction}

\author{W.P. Bruinsma \and R.P. Hes \and H.J.C. Kroep \and T.C. Leliveld \and W.M. Melching \and T.A. aan de Wiel}

\raggedbottom

\begin{document}

\section{Introduction}
This chapter will discuss the results which were presented in \cref{cha:system_evaluation_theory}. We will carefully reconsider the specifications stated in \cref{cha:overview}.

\section{Sampling}
Recall that the techniques used for sampling \emph{must} allow for
\begin{enumerate}
    \item correct operation of the detection;
    \item usage of a single sampling device and
    \item usage of multiple sampling devices such that the workload can be distributed across the devices.
\end{enumerate}
In addition, the sampling techniques \emph{should} allow for
\begin{enumerate}
    \item sampling as efficient as possible.
\end{enumerate}

We have refined three sampling techniques to fit our system. We saw that together, these techniques met their requirements. First, all techniques where refined to enable correct operation of the reconstructor. Second, coprime sampling and circular sparse sampling allow for the usage of one device, and collaborative sampling allows for the multiple devices. Finally, we have abstracted the correct operation of the reconstruction to the satisfaction of the circular sparse ruler problem. Since the solution of this problem influences the efficiency of the sampling technique, we can optimise to find solutions such that sampling techniques are as efficient as possible. This optimisation was discussed in \cref{ap:derivation_ILP}.

A comparison between the different sampling techniques has shown that each sampling technique has its own advantages. For example, circular sparse ruler sampling allows for the biggest compression, coprime sampling for the least amount of samplers and collaborative sampling for the usage of multiple devices. We can therefore conclude that the sampling technique must be chosen according to the situation, and there is no `one size which fits them all'. 

In addition to meeting the requirements, we did some research on the circular sparse ruler problem to further optimise finding solutions. This was discussed in \cref{ap:circ-ruler}. The circular sparse ruler problem allowed us to construct solutions for multi-coset sampling with comparably higher compression than the solutions proposed in \cite{ariananda2012compressive}.


We can also assess the similarities of the different sampling techniques. For example, circular sparse sampling can be considered a form of collaborative sampling with one device. This observation arises the question if is possible to generalise all sampling techniques to an universal technique. This universal technique could then be optimised. This is considered future work.  

% For collaborative sampling we looked at a method where the different devices are used to sense more bandwidth, but collaborative sampling can also be used for other purposes, like increasing accuracy, or sensing in a larger radius. We did not look at these applications and this should be considered future work.

\section{Reconstruction}
Recall that the reconstruction \textit{must} be able to
\begin{enumerate}
    \item operate in conjunction with the sampling techniques and
    \item operate in real-time.
\end{enumerate}
In addition, the reconstruction \textit{should} should
\begin{enumerate}
    \item as efficient as possible and
    \item as fast as possible.
\end{enumerate}

We have designed a reconstruction algorithm which is able to reconstruct the power spectral density of the sampled signal. We will reconsider the specifications one by one. First, the sampling techniques were carefully designed such that they adhere to this design. Therefore, the reconstruction algorithm is able to operate in conjunction with the sampling techniques. Second, the algorithm requires a specified amount of samples of the outputs of all cosets to operate. Since this amount can be specified, which is a trade-off between sampling time and accuracy, the reconstruction arguably operates in real-time. Finally, we have argued that the algorithm is computationally less intensive and less complex than other existing method based on optimisation or complex algorithms. We have optimised the operation of the reconstruction by exploiting sparsity and by providing a fast algorithm for generation of the matrices. By this means we tried to make the reconstruction as efficient and as fast as possible. However, this is subject to discussion.

In addition to meeting the requirements, we have done extensive research on reformulating the algorithms in terms of using a biased estimate for the cross-correlations of the output of the cosets. In comparison to the proposed unbiased estimator, a biased estimator can have less variance, which can result in less mean squared error of the estimation \cite{hayes1996statistical}. Although this analysis has been very thorough, its conclusion was not definite. We therefore conclude to not include the results in this thesis. We consider research on using a biased estimator to be future work.

\section{Detection}
Recall that the detection of signals \emph{must} be such that
\begin{enumerate}
    \item detection of signals consists of determination of the set of frequencies which are occupied by signals other than noise;
    \item the resolution of these frequencies can be specified and
    \item the correctness of operation can be specified.
\end{enumerate}
In addition, detection of signals \emph{should} be such that
\begin{enumerate}
    \item the resolution of detection can be as high as possible;
    \item its operation can be as correct as possible and
    \item its operation is as efficient as possible.
\end{enumerate}

For the detection on the output of the reconstruction we have looked at two blind detection methods: modified energy detection and CAV. Based on  a theoretical analysis we have decided to adopt the modified energy detector in our final system. This energy detector has been modified to work with our reformulated algorithm. This detector fulfills the must-haves: it is designed to detect per element of the reconstructed power spectral
density whether noise or another signal is present. This also implies that it fulfills the should-have of the maximum possible resolution.  From this set of comparisons, the set of frequencies occupied by some signal can be derived. Furthermore, it is possible to adjust the false-alarm probability of this detector and thus the correctness of operation can be specified. 

From the results as presented in \cref{sec:results_theory}, we conclude that the detector does not fulfill the should-have that its detection should be as correct possible. The results indicate that relative small noise uncertainty considerably degrades the  performance. Namely, the false alarm and the detection probability are affected by this problem. Furthermore, we have seen that this detector is not able to provide a reasonable\footnote{A detection probability of $0.9$ is given in the 802.22 EEE standard.} detection probability at low SNR's. Whether its operation is as efficient as possible is subject  to some discussion. From a computational point of view, the detector is very efficient due to the fact that computation of its test statistic is relatively simple. From an operational point of view, however, there is room for improvement and other detectors may provide better performance in the sense of detection probability and false alarm probability. This point of view has also motivated the analysis of the CAV detector in \cref{cha:detection}. Due to time constraints the current CAV implementation is limited in its detection abilities. That is, it cannot distinguish between frequency bands. Therefore CAV has not been included in our final system. Initial tests, however, indicate that the CAV detector outperforms the modified energy detector on the testing signal as described in \cref{ssec:test_signal}. These test motivate further research in the detection algorithm to be used in detection module. 
\end{document}
