\section{Algorithm}
In this section we will present a practical approach to the energy detector, if only an estimate of the autocorrelation of the received signal is present.

 \subsection{Power spectral density \& autocorrelation}
Consider a discrete time signal $x[n]$. We define its \emph{expected} power as $E\left[\left|x[n]\right|^2\right]$, which equals the autocorrelation function of $x$ at zero lag. 

By the Wiener-Khinchin Theorem we have that 

 \begin{align*}
S(j\omega) = \sum_{k=-\infty}^{\infty} r_{xx}[k] \exp \left( -j2\pi \omega k) \right)
 \end{align*}

where $r_{xx}[k]$ denotes the autocorrelation function of $x$ at lag $k$ and $S(\omega)$ denotes the spectral power density (PSD).
By definition of the PSD 

\begin{align*}
E\left[\left|x[n]\right|^2\right] = \int_{\omega = -\infty}^{\infty} S(j\omega)\text{d}\omega.
\end{align*}

\subsubsection{PSD of Bandlimited signals}
By our assumption that the noise samples are i.i.d. circular complex gaussian distributed
\begin{align*}
    r_{nn}[k] = \sigma^2_n \delta[k]
\end{align*}
and 
\begin{align*}
    S(j\omega) = \sigma^2_n.
\end{align*}

If just \emph{one} bandlimited signal $s$ is present in the received signal $x$, then under the assumption that it is independent from the noise, we can add the PSD of $s$ to that of the $n$ to obtain the PSD of $x$ (as there is no correlation between $x$ and $s$. If the lowest frequency present in $s$ is $\omega_1$ and the highest frequency $\omega_2$, then, given a bandpass filter with a frequency response 

\begin{align*}
H(j\omega) &= \begin{cases}1 &\text{$\omega_1 \leq |\omega| \leq \omega_2$} \\
0 & \text{elsewhere} \end{cases} 
\end{align*}
the PSD of $x$ is given by
\begin{align}
S_x(j\omega) &= H(j\omega) \cdot S(j\omega) + \sigma_n^2 \cdot \left(-H(j\omega+1) \right).
\end{align}

% Let us now consider the vector $\vec{x} \in \mathbb{C}^{2M+1}$ for which $(\vec{x})_{i+M+1} = x[i]$.
\subsection{The algorithm}

To detect the presence of a signal in a certain frequency band, first of all, we will use the modified energy detector test static $\Lambda^* = \frac{1}{N}\sum_{i=1}^N \left|(\vec{x})_i\right|^2$. Notice how $\Lambda^*$ represents an estimate of the expected power.  Given an estimate of the autocorrelation of the received signal, $r_{xx}$,  we propose the following algorithm

\begin{enumerate}   
    \item Estimate the PSD of $x$, $\hat{S}_x(j\Omega)$  by taking the absolute value of the DFT of $r_{xx}$, see Theorem \ref{th:abs_psd}.
    \item Estimate the noise variance $\sigma_n^2$ from the estimated PSD.
    \item Summate $S(j\Omega)$ in the desired frequency band.
    \item Add $\sigma_n^2 \cdot W$ to the result of the previous step, where $W$ denotes the bandwidth of
    $x$ minus that of the certain frequency band. 
    \item Compare the result of the previous step with the threshold $\frac{\eta}{N}$.
\end{enumerate}

\subsection{Estimating noise variance}
Under the assumption that there is at least one frequency band in the estimated PSD containing only noise, it is possible to obtain an estimate of the noise variance
by taking the average value of the PSD in that band. (Under the assumption that the band only contains noise, then the PSD should have a value of $\sigma_n^2$ in that particular band).


TODO: proof for both of these (second one is actually a lemma for the first)

\begin{blockTheorem} \label{th:abs_psd}
Let $\vec{x} \in \mathbb{C}^N$, then:
\begin{align*}
(\vec{x} \circ \vec{x})_N = \sum_{k=1}^{2N-1} \left| \sum_{n=1}^N (\vec{x} \circ \vec{x})_n \exp \left(\frac{-2\pi j (n-1)(k-1)}{2N-1}\right) \right | 
\end{align*}
 \end{blockTheorem}

\begin{blockTheorem} \label{th:abs_psd2}
Let $\vec{x} \in \mathbb{C}^N$, then:

\begin{align*}
\left| \sum_{n=1}^N (\vec{x} \circ \vec{x})_n \exp \left(\frac{-2\pi j (n-1)(k-1)}{2N-1}\right) \right | &= \sum_{n=1}^N (\vec{x} \circ \vec{x})_n \exp \left(\frac{2\pi j (N-n)(k-1)}{2N-1}\right) 
\end{align*}
 \end{blockTheorem}

