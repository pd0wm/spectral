
\documentclass[report, oneside, a4paper, openany]{memoir}
\usepackage[no-math]{fontspec}
\usepackage{pgfplots}
\pgfplotsset{compat=newest}
\usepackage{commath}
\usepackage{mathtools}
\usepackage{amssymb}
\usepackage{amsthm}
\usepackage{booktabs}
\usepackage{mathtools}
\usepackage{xcolor}
\usepackage[separate-uncertainty=true, per-mode=symbol]{siunitx}
\usepackage[noabbrev, capitalize]{cleveref}
\usepackage{listings}
\usepackage[american inductor, european resistor]{circuitikz}
\usepackage{amsmath}
\usepackage{amsfonts}
\usepackage{ifxetex}
\usepackage[dutch,english]{babel}
\usepackage[backend=bibtexu,texencoding=utf8,bibencoding=utf8,style=ieee,sortlocale=en_GB,language=auto]{biblatex}
\usepackage[strict,autostyle]{csquotes}
\usepackage{parskip}
\usepackage{import}
\usepackage{standalone}
\usepackage{hyperref}
%\usepackage[toc,title,titletoc]{appendix}

\ifxetex{} % Fonts laden in het geval dat je met Xetex compiled
    \usepackage{fontspec}
    \defaultfontfeatures{Ligatures=TeX} % To support LaTeX quoting style
    \setromanfont{Palatino Linotype} % Tover ergens in Font mapje in root.
    \setmonofont{Source Code Pro}
\else % Terug val in standaard pdflatex tool chain. Geen ondersteuning voor OTT fonts
    \usepackage[T1]{fontenc}
    \usepackage[utf8]{inputenc}
\fi
\newcommand{\references}[1]{\begin{flushright}{#1}\end{flushright}}
\renewcommand{\vec}[1]{\boldsymbol{\mathbf{#1}}}
\newcommand{\uvec}[1]{\boldsymbol{\hat{\vec{#1}}}}
\newcommand{\mat}[1]{\boldsymbol{\mathbf{#1}}}
\newcommand{\fasor}[1]{\boldsymbol{\tilde{\vec{#1}}}}
\newcommand{\cmplx}[0]{\mathrm{j}}
\renewcommand{\Re}[0]{\operatorname{Re}}
\newcommand{\Cov}{\operatorname{Cov}}
\newcommand{\Var}{\operatorname{Var}}
\newcommand{\proj}{\operatorname{proj}}
\newcommand{\Perp}{\operatorname{perp}}
\newcommand{\col}{\operatorname{col}}
\newcommand{\rect}{\operatorname{rect}}
\newcommand{\sinc}{\operatorname{sinc}}
\newcommand{\IT}{\operatorname{IT}}
\newcommand{\F}{\mathcal{F}}

\newtheorem{definition}{Definition}
\newtheorem{theorem}{Theorem}


\DeclareSIUnit{\voltampere}{VA} %apparent power
\DeclareSIUnit{\pii}{\ensuremath{\pi}}

\hypersetup{%setup hyperlinks
    colorlinks,
    citecolor=black,
    filecolor=black,
    linkcolor=black,
    urlcolor=black
}

% Example boxes
\usepackage{fancybox}
\usepackage{framed}
\usepackage{adjustbox}
\newenvironment{simpages}%
{\AtBeginEnvironment{itemize}{\parskip=0pt\parsep=0pt\partopsep=0pt}
\def\FrameCommand{\fboxsep=.5\FrameSep\shadowbox}\MakeFramed{\FrameRestore}}%
{\endMakeFramed}

% Impulse train
\DeclareFontFamily{U}{wncy}{}
\DeclareFontShape{U}{wncy}{m}{n}{<->wncyr10}{}
\DeclareSymbolFont{mcy}{U}{wncy}{m}{n}
\DeclareMathSymbol{\Sha}{\mathord}{mcy}{"58}

\title{Documentation}

\author{W.P. Bruinsma \and R.P. Hes \and H.J.C. Kroep \and T.C. Leliveld \and W.M. Melching \and T.A. aan de Wiel}

\begin{document}
\chapter{Reconstruction}
The reconstruction uses the output of the sampling block to estimate the power spectral density.

\section{Multicoset Sampling Reconstruction -Ariananda}
\subsection{concept}
The expected input of this reconstruction method is a multicoset sampled signal.
\\ \\
The expected input of this block is
$$
\sum_{l=0}^{M-1}z_l[kN]
$$
With $M$ the amount of sampling devices, N = sample frequency/nyquist rate and k the sampling index.\\
\\
The Cross Spectral Density (CSD) is given by
$$
P_{yi,yj}(\omega) = \sum_{k=-\infty}^\infty r_{yi,yj}[k]e^{-jk\omega}, \quad  \leq \omega < 2\pi
$$ 
where $r_{yi,yj}[k]$ is the cross-corelation function of $y_i[k]$ and $y_j[k]$.\\
\\
Now there are three different methods proposed to reconstruct the Power Spectral Density (PSD)

\subsubsection{Time-Domain Reconstruction Approach}
We desire to reconstruct $r_x[n]$, the auto correlation of $x[n]$. If we are able to reconstruct $r_x[n]$ we have enough information to determine the PSD.\\
The big idea of this method is to get $M^2$ equations to calculate the $N$ parts of the autocorrelation function. We want to pick the $M$ different measurement device outputs, combine these into $M^2$ different cross-correlation functions, and use these to reconstruct the auto-correlation with a sampling that is N lower than nyquist. \\
\\
For example: if  one would use 10 different measurement devices, one would be able to reconstruct a Power Spectral Density with a samplingrate that is 100 times lower than nyquist!
$$
r_{yi,yj}[k] = r_{zi,zj}[kN] = \sum_{m=-N+1}^{N-1}r_{ci,cj}[m]r_x[kN-m] = \sum_{l=0}^1\mathbf{r}^T_{ci,cj}[l]\mathbf{r}_x[k-l]
$$
This is basically a rewrite with new defined variables. Note that $l$ is a boolean here. $\mathbf{r}_{ci,cj}[0]$ is the part where $m\leq 0$ and $\mathbf{r}_{ci,cj}[1]$ is the part where $m > 0$ 
$$
\mathbf{r}_{ci,cj}[l] = [r_{ci,cj}[Nl], r_{ci,cj}[Nl-1], \dots, r_{ci,cj}[Nl-N+1]]^T
$$
$$
\mathbf{r}_x[k]= [r_x[kN],r_x[kN+1],\dots,r_x[(k+1)N-1]]^T
$$
Note that $\mathbf{r}_t[k]$ is the autocorrelationfunction we want to get, consisting of N parts. Recall that we need N parts because we want to sample at a rate N times slower than nyquist ratre.\\
\\
Now we make a vector $\mathbf{r}_y[k]$ by combining every possible combination of the $M$ auto_correlations into $M^2$ cross-correlations.
$$
\mathbf{r}_y[k]=\sum_{l=0}^1\mathbf{R}_c[l]\mathbf{r}_x[k-l]
$$
Again recall that $l$ is a boolean and that $l=0$ covers $m\leq 0$ and $l=1$ covers $m > 0$.




\subsection{implementation}

\subsection{verification}

\end{document}
