%!TEX program = xelatex

\documentclass{memoir}
\usepackage[no-math]{fontspec}
\usepackage{pgfplots}
\pgfplotsset{compat=newest}
\usepackage{commath}
\usepackage{mathtools}
\usepackage{amssymb}
\usepackage{amsthm}
\usepackage{booktabs}
\usepackage{mathtools}
\usepackage{xcolor}
\usepackage[separate-uncertainty=true, per-mode=symbol]{siunitx}
\usepackage[noabbrev, capitalize]{cleveref}
\usepackage{listings}
\usepackage[american inductor, european resistor]{circuitikz}
\usepackage{amsmath}
\usepackage{amsfonts}
\usepackage{ifxetex}
\usepackage[dutch,english]{babel}
\usepackage[backend=bibtexu,texencoding=utf8,bibencoding=utf8,style=ieee,sortlocale=en_GB,language=auto]{biblatex}
\usepackage[strict,autostyle]{csquotes}
\usepackage{parskip}
\usepackage{import}
\usepackage{standalone}
\usepackage{hyperref}
%\usepackage[toc,title,titletoc]{appendix}

\ifxetex{} % Fonts laden in het geval dat je met Xetex compiled
    \usepackage{fontspec}
    \defaultfontfeatures{Ligatures=TeX} % To support LaTeX quoting style
    \setromanfont{Palatino Linotype} % Tover ergens in Font mapje in root.
    \setmonofont{Source Code Pro}
\else % Terug val in standaard pdflatex tool chain. Geen ondersteuning voor OTT fonts
    \usepackage[T1]{fontenc}
    \usepackage[utf8]{inputenc}
\fi
\newcommand{\references}[1]{\begin{flushright}{#1}\end{flushright}}
\renewcommand{\vec}[1]{\boldsymbol{\mathbf{#1}}}
\newcommand{\uvec}[1]{\boldsymbol{\hat{\vec{#1}}}}
\newcommand{\mat}[1]{\boldsymbol{\mathbf{#1}}}
\newcommand{\fasor}[1]{\boldsymbol{\tilde{\vec{#1}}}}
\newcommand{\cmplx}[0]{\mathrm{j}}
\renewcommand{\Re}[0]{\operatorname{Re}}
\newcommand{\Cov}{\operatorname{Cov}}
\newcommand{\Var}{\operatorname{Var}}
\newcommand{\proj}{\operatorname{proj}}
\newcommand{\Perp}{\operatorname{perp}}
\newcommand{\col}{\operatorname{col}}
\newcommand{\rect}{\operatorname{rect}}
\newcommand{\sinc}{\operatorname{sinc}}
\newcommand{\IT}{\operatorname{IT}}
\newcommand{\F}{\mathcal{F}}

\newtheorem{definition}{Definition}
\newtheorem{theorem}{Theorem}


\DeclareSIUnit{\voltampere}{VA} %apparent power
\DeclareSIUnit{\pii}{\ensuremath{\pi}}

\hypersetup{%setup hyperlinks
    colorlinks,
    citecolor=black,
    filecolor=black,
    linkcolor=black,
    urlcolor=black
}

% Example boxes
\usepackage{fancybox}
\usepackage{framed}
\usepackage{adjustbox}
\newenvironment{simpages}%
{\AtBeginEnvironment{itemize}{\parskip=0pt\parsep=0pt\partopsep=0pt}
\def\FrameCommand{\fboxsep=.5\FrameSep\shadowbox}\MakeFramed{\FrameRestore}}%
{\endMakeFramed}

% Impulse train
\DeclareFontFamily{U}{wncy}{}
\DeclareFontShape{U}{wncy}{m}{n}{<->wncyr10}{}
\DeclareSymbolFont{mcy}{U}{wncy}{m}{n}
\DeclareMathSymbol{\Sha}{\mathord}{mcy}{"58}
\addbibresource{../includes/bibliography.bib}

\begin{document}

% Stoch processes:
% Correlatie
% PSD
% Wiener-Kichine
% Wide sense stationary
% Ergodisch

% Linear Algebra:
% Orthogonality
% Kwadratische vorm
% Hermitisch etc

% Signal processing:
% FT
% DTFT
% Sampling
% Aliasing
% DFT
% Windowing
% SnR
% Matched Filter
% Complex envelopes



\newcommand{\Cov}{\operatorname{Cov}}
\newcommand{\Var}{\operatorname{Var}}


\pagestyle{empty}
\begingroup
    \vspace*{0.12\textheight}
    {\hfill \HUGE Review of Basic Concepts}
    \vfill
    {\Large\noindent H.J.C Kroep \\ R.P. Hes \\ W.M. Melching \\ W.P. Bruinsma \\ T.A. aan de Wiel \\ T.C. Leliveld}
\endgroup

\chapter{Preliminaries}
\section{Convex functions}
\begin{definition}[Convex function]
    Let $f:S\to \mathbb{R}$ where $S$ is a convex subset of $\mathbb{R}$. Then $f$ is convex if for all $x_1,x_2 \in S$ and $\lambda \in [0,1]$
    \begin{align*}
        \lambda f(x_1) + (1- \lambda)f(x_2) \ge f( \lambda x_1 + (1- \lambda) x_2).
    \end{align*}
\end{definition}

\begin{theorem}
    Let $f:S \to \mathbb{R}$ be a convex function and let $\lambda_1,\ldots,\lambda_n \in \mathbb{R}$ be such that
    \begin{align*}
        \lambda_1 + \lambda_2 + \ldots + \lambda_n = 1.
    \end{align*}
    Then for all $x_1,\ldots,x_n \in S$
    \begin{align*}
        \lambda_1 f(x_1) + \lambda_2 f(x_2) + \ldots + \lambda_n f(x_n) \ge f(\lambda_1 x_1 + \lambda_2 x_2 + \ldots + \lambda_n x_n).
    \end{align*}
\end{theorem}
\begin{proof}
    We will proof by induction. Since $f$ is convex, the base case holds. Suppose that the statement is true for a certain $n$. Then, since $f$ is convex
    \begin{align*}
        &\lambda_1 f(x_1) + \lambda_2 f(x_2) + \ldots + \lambda_{n+1} f(x_{n+1}) \\
        &= (\lambda_1+\lambda_2)\left[\frac{\lambda_1}{\lambda_1 + \lambda_2} f(x_1) + \frac{\lambda_2}{\lambda_1+\lambda_2}f(x_2)\right] + \ldots + \lambda_{n+1} f(x_{n+1})  \\
        &\ge (\lambda_1 + \lambda_2)f\left( \frac{\lambda_1}{\lambda_1 + \lambda_2} x_1 + \frac{\lambda_2}{\lambda_1+\lambda_2}x_2\right) + \ldots + \lambda_{n+1} f(x_{n+1}) \\
        &\ge f\left[(\lambda_1 + \lambda_2) \frac{\lambda_1}{\lambda_1 + \lambda_2} x_1 + (\lambda_1 + \lambda_2)\frac{\lambda_2}{\lambda_1+\lambda_2}x_2\ldots + \lambda_{n+1}\right] \\
        &= f(\lambda_1 x_1 + \lambda_2 x_2 + \ldots + \lambda_{n+1} x_{n+1}).
    \end{align*}
\end{proof}

\chapter{Stochastic Processes}
\section{Basic Definitions}
\begin{definition}
    A \textbf{random variable} consists of an experiment with a probability measure $P[.]$ defined on a sample space $S$ and a function that assigns a real number to each outcome in the sample space of the experiment.
\end{definition}

\begin{definition}
    Subsets of the sample space are called \textbf{events}.
\end{definition}

\begin{definition}
    The \textbf{probability mass function} (PMF) of the discrete random variable $X$ is
    \begin{align*}
        P_X(x) = P[X=x].
    \end{align*}
\end{definition}

\begin{definition}
    The \textbf{cumulative distribution function} (CDF) of a random variable $X$ is
    \begin{align*}
        F_X(x) = P[X \le x].
    \end{align*}
\end{definition}
The CDF of a derived random variable $Y=g(X)$ is
\begin{align*}
    F_{Y}(x) = P[Y \le x] = P[g(X) \le x] = P[X \le g^{-1}(x)].
\end{align*}

\begin{definition}
    The \textbf{expected value} of a random variable $X$ is
    \begin{align*}
        E[X]=\mu_X = \sum_{x \in S_X} x P_X(x).
    \end{align*}
\end{definition}

\begin{definition}
    The \textbf{variance} of a random variable $X$ is
    \begin{align*}
        \Var[X] = E[(X-\mu_X)^2].
    \end{align*}
\end{definition}
The variance of a random variable $X$ simplifies to
\begin{align*}
    \Var[X] = E[X^2] - E^2[X].
\end{align*}

\section{Basic theorems}

\begin{theorem}[Jensen's Inequality]
    Let $f$ be a convex function and $X$ a random variable. Then
    \begin{align*}
        E[f(X)] \ge f(E[X]).
    \end{align*}
\end{theorem}
\begin{proof}
    Note that
    \begin{align*}
        \sum_{x \in S_X} P_X(x) = 1.
    \end{align*}
    Then, by the definition of the expected value and the convexity of $f$
    \begin{align*}
        E[f(X)] &= \sum_{x \in S_X} f(x) P_X(x) \\
        &\ge f\left[\sum_{x \in S_X} x P_X(x) \right] \\
        &=f(E[X]).
    \end{align*}
\end{proof}

\begin{theorem}[Cauchy-Schwarz Inequality]
    Let $X$ and $Y$ be random variables. Then
    \begin{align*}
        (E[XY])^2 \le E[X^2]E[Y^2].
    \end{align*}
\end{theorem}
\begin{proof}
    Let $a \in \mathbb{R}$, then
    \begin{align*}
        E[(a X + Y)^2] = E[X^2]a^2  + 2E[XY]a + E[Y^2].
    \end{align*}
    Consider this as a function of $a$. Note that  $E[(aX+Y)^2]\ge0$. So the obtained polynomial will have no zero-crossings. Therefore
    \begin{align*}
        (2 E[XY])^2-4 E[X^2] E[Y^2] \le 0.
    \end{align*}
\end{proof}

\section{Dependence}

\begin{definition}
    The \textbf{conditional probability} of an event $A$ given event $B$ is
    \begin{align*}
        P[A|B] = \frac{P[A \cap B]}{P[B]}.
    \end{align*}
\end{definition}

\begin{definition}
    Events $A$ and $B$ are \textbf{independent} if
    \begin{align*}
        P[A \cap B]=P[A] P[B].
    \end{align*}
\end{definition}

\begin{theorem}
    $A$ and $B$ are two independent events if and only if
    \begin{align*}
        P[A | B] = P[A].
    \end{align*}
\end{theorem}
\begin{proof}
    By the definitions
    \begin{align*}
        P[A | B] = \frac{P[A \cap B]}{P[B]} = \frac{P[A]P[B]}{P[B]} = P[A].
    \end{align*}
\end{proof}

\begin{definition}
    The \textbf{covariance} of two random variables $X$ and $Y$ is
    \begin{align*}
        \Cov[X,Y]=E[(X- \mu_X)(Y- \mu_Y)].
    \end{align*}
\end{definition}

\begin{definition}
    The \textbf{correlation} of two random variables $X$ and $Y$ is
    \begin{align*}
        r_{X,Y} = E[XY].
    \end{align*}
\end{definition}
The covariance of two random variables $X$ and $Y$ simplifies to
\begin{align*}
    \Cov[X,Y] = r_{X,Y} - \mu_X \mu_Y.
\end{align*}
The variance of the addition of two random variables $X$ and $Y$ simplifies to
\begin{align*}
    \Var[X+Y] = \Var[X] + \Var[Y] + 2 \Cov[X,Y].
\end{align*}
Consider two random variables $X$ and $Y$. If $\Cov[X,Y]=0$, then $X$ and $Y$ are \textbf{uncorrelated}. Correlation is a measure of dependence. If $\Cov[X,Y]>0$, then $X>\mu_X$ implies that we expect $Y>\mu_Y$.

If $r_{X,Y}=0$, then $X$ and $Y$ are \textbf{orthogonal}. In the case of zero-mean variables, orthogonal variables are uncorrelated.

\begin{definition}
    The \textbf{correlation coefficient} of two random variables $X$ and $Y$ is
    \begin{align*}
        \rho_{X,Y} = \frac{\Cov[X,Y]}{\sqrt{\Var[X]\Var[Y]}}.
    \end{align*}
\end{definition}

\begin{theorem}
    $|\rho_{X,Y}|\le1$ holds.
\end{theorem}
\begin{proof}
    We will proof that $\rho_{X,Y}^2 \le 1$. To this end, by the Cauchy-Schwarz Inquality
    \begin{align*}
        \Cov[X,Y]^2 &= E[(X- \mu_X)(Y- \mu_Y)]^2 \\
        &\le E[(X-\mu_X)^2] E[(Y- \mu_Y)^2] \\
        &= \Var[X] \Var[Y]
    \end{align*}
    Therefore
    \begin{align*}
        \rho_{X,Y}^2 = \frac{\Cov[X,Y]^2}{\Var[X]\Var[Y]} \le 1.
    \end{align*}
\end{proof}

\section{Normal Distribution}
\subsection{Gauss' Derivation of the Univariate Normal Distribution}
In deriving the normal distribution, Gauss assumed the following:
\begin{enumerate}
    \item Small error are more likely than large errors.
    \item The likelihood of errors of magnitudes $\varepsilon$ and $-\varepsilon$ are equal.
    \item When a quantity is measured several times, the most likely value is given by the average of the measurements.
\end{enumerate}
Let $p$ be the true value of the measured quantity. Let $n$ independent observations yield measurements $M_1,\ldots,M_n$ and let the probability density function of the error by given by $\phi(x)$. Then the assumptions translate to the following:
\begin{enumerate}
    \item $\phi(x)$ has a maximum at $x=0$.
    \item $\phi(x)=\phi(-x)$.
    \item The maximum likelihood estimate of $p$ is given by
    \begin{align*}
        \hat{p} = \frac{M_1 + \ldots + M_n}{n} = \mu.
    \end{align*}
\end{enumerate}
Based on our conclusions, the joint probability density function is given by
\begin{align*}
    f(x) = \phi(M_1 - x)\cdots\phi(M_n-x).
\end{align*}
Therefore the maximum likelihood estimate yields
\begin{align*}
    \left.\frac{\partial f}{\partial p}\right|_{p=\hat{p}}=
    -\left[\frac{\phi'(M_1-\hat{p})}{\phi(M_1-\hat{p})} + \ldots + \frac{\phi'(M_n-\hat{p})}{\phi(M_n-\hat{p})}\right]f(\hat{p})=0.
\end{align*}
Remember that the measurements are arbitrary. In further exploring the structure of $\phi(x)$, assume that
\begin{align*}
    M_1 = M, \qquad M_2=\ldots=M_n=M-nN
\end{align*}
for a real $N$. Then
\begin{align*}
    \hat{p}=\frac{nM - (n-1)nN}{n} =M-(n-1)N.
\end{align*}
Now our maximum likelihood estimate yields
\begin{align*}
    \frac{\phi'[(n-1)N]}{\phi[(n-1)N]} + (n-1)\frac{\phi'(-N)}{\phi(-N)} = \frac{\phi'[(n-1)N]}{\phi[(n-1)N]} - (n-1)\frac{\phi'(N)}{\phi(N)}=0.
\end{align*}
Using the fact that this holds for any $N$ and $\phi(x)$ obtains a maximum at $x=0$, it can be shown that
\begin{align*}
    \frac{\phi'(x)}{\phi(x)} = -\frac{x}{ \sigma^2}
\end{align*}
for any real $\sigma$. Finally, solving this differential equation and normalising the solution yields that
\begin{align*}
    \phi(x)=\frac{1}{\sigma \sqrt{2 \pi}}\exp\left[-\frac{x^2}{2 \sigma^2}\right].
\end{align*}
If $\phi(x)$ is the error probability density function of a random variable $X$, then we denote $X \sim N(\mu,\sigma^2)$.

\subsection{The Multivariate Normal Distribution}
Suppose that $\vec{Z}$ is a $n$ vector of independent random variables each $N(0,1)$ distributed. Then the joint probability density function is given by
\begin{align*}
    f_Z(\vec{Z}) = \prod_i \frac{1}{\sqrt{2 \pi}} \exp\left[-\frac{z_i^2}{2}\right] = \frac{1}{(2 \pi)^{n/2}} \exp\left[-\frac{1}{2}\vec{Z}^T \vec{Z}\right].
\end{align*}
Let $\mat{\Sigma}$ be a $n \times n$ symmetric positive-definite matrix. $\mat{\Sigma}$ is called the \textbf{covariance matrix}. Since $\mat{\Sigma}$ is symmetric positive-definite, we can write
\begin{align*}
    \mat{\Sigma} = \mat{Q}^T \mat{\Lambda} \mat{Q} = (\mat{Q}^T \mat{\Lambda}^{1/2} \mat{Q}) (\mat{Q}^T \mat{\Lambda}^{1/2} \mat{Q}) = \mat{\Sigma}^{1/2} \mat{\Sigma}^{1/2}.
\end{align*}
Note that $(\mat{\Sigma}^{1/2})^T = \mat{\Sigma}^{1/2}$
Let $\vec{X}$ be a $n$ vector such that
\begin{align*}
    \vec{X} = \mat{\Sigma}^{1/2} \vec{Z} + \vec{\mu}.
\end{align*}
Thus $\mat{\Sigma}$ defines a dependency between the variables of $\vec{X}$. We require $\mat{\Sigma}$ to be positive definite analogous to the requirement that $\sigma^2$ is positive. We require $\mat{\Sigma}$ to be symmetric since we assume the dependencies between the variables of $\vec{X}$ to be commutative. The inverse transformation is given by
\begin{align*}
    \vec{Z} = T(\vec{X}) = \mat{\Sigma}^{-1/2}(\vec{X}-\vec{\mu}).
\end{align*}
with Jacobian $J_T = \mat{\Sigma}^{-1/2}$. We can now use the change-of-variables theorem to write down the joint probability density function of $\vec{X}$
\begin{align*}
    f_X(\vec{X}) &= f_Z(T(\vec{X})) |J_T| \\
    &= \frac{1}{(2 \pi)^{n/2}} \exp\left\{-\frac{1}{2}[\mat{\Sigma}^{-1/2}(\vec{X}-\vec{\mu})]^T [\mat{\Sigma}^{-1/2}(\vec{X}-\vec{\mu})]\right\} |\mat{\Sigma}^{-1/2}| \\
    &= \frac{1}{(2 \pi)^{n/2} |\mat{\Sigma}|^{1/2}} \exp\left[-\frac{1}{2}(\vec{X}-\vec{\mu})^T \mat{\Sigma}^{-1} (\vec{X}-\vec{\mu})\right].
\end{align*}
This probability density function is known as the multivariate normal distribution.

\section{Stochastic Processes}

\begin{definition}
    A \textbf{stochastic process} $X(t)$ consists of an experiment with a probability measure $P[.]$ defined on a sample space $S$ and a function that assigns a time function $x(t,s)$ to each outcome $s$ in the sample space of the experiment.
\end{definition}

Suppose that we observe a random process $X(t)$ at a particular time $t_1$. Then each observation yields a $x(t_1,s)$. Therefore $x(t_1,s)$ is the sample value of a random variable. We denote this random variable by $X(t_1)$.

\begin{definition}
    A stochastic process $X(t)$ is \textbf{independently identically distributed} if
    \begin{enumerate}
        \item $X(t_k)$ for all $k$ are independent random variables,
        \item $X(t_k)$ for all $k$ have identical probability distribution functions.
    \end{enumerate}
\end{definition}    

\begin{definition}
    The \textbf{expected value} of a stochastic process $X(t)$ is the function
    \begin{align*}
        \mu_X(t) = E[X(t)].
    \end{align*}
\end{definition}

\begin{definition}
    The \textbf{autocovariance} function of a stochastic process $X(t)$ is
    \begin{align*}
        C_X(t,\tau)=\Cov[X(t),X(t+\tau)].
    \end{align*}
\end{definition}

\begin{definition}
    The \textbf{autocorrelation} function of a stochastic process $X(t)$ is
    \begin{align*}
        R_X(t,\tau)=E[X(t),X(t + \tau)].
    \end{align*}
\end{definition}
The autocovariance of a stochastic process simplifies to
\begin{align*}
    C_X(t,\tau) = R_X(t,\tau) - \mu_X(t) \mu_X(t + \tau).
\end{align*}

\begin{definition}
    A stochastic process $X(t)$ is \textbf{wide sense stationary} if for all $t$
    \begin{enumerate}
        \item $\mu_X(t) = \mu_X$,
        \item $R_X(t,\tau)  =R_X(\tau)$.
    \end{enumerate}
\end{definition}

\begin{theorem}
    For a wide sense stationary process $X(t)$, the following hold:
    \begin{enumerate}
        \item $R_X(0) \ge 0$,
        \item $R_X(\tau) = R_X(-\tau)$,
        \item $R_X(0) \ge |R_X(\tau)|$.
    \end{enumerate}
\end{theorem}
\begin{proof}
    The first property follows from the definition. To prove the second property, let us evaluate the autocorrelation function at $\tilde{t} = t + \tau$. Then
    \begin{align*}
        R_X(\tau) &= R_X(t,\tau)\\
        &= E[X(t)X(t+\tau)]\\
        &= E[X(\tilde{t} - \tau) X(\tilde{t})]\\
        &= E[X(\tilde{t}) X(\tilde{t} - \tau)]\\
        &= R_X(\tilde{t}, - \tau)\\
        &= R_X(-\tau).
    \end{align*}
    To prove the third property, we make use of the correlation coefficient. Note that
    \begin{align*}
        \Var[X(t)] = E[X^2(t)] -E^2[X(t)] = R_X(0) - \mu_X^2.
    \end{align*}
    So $\Var[X(t)]$ is constant for all $t$. Therefore
    \begin{align*}
        C_X(t,\tau) &= \rho_{X(t),X(t+\tau)} \sqrt{\Var[X(t)]\Var[X(t+\tau)]}\\
        &\le \sqrt{\Var[X(t)]\Var[X(t+\tau)]}\\
        &= C_X(0).
    \end{align*}
    Then
    \begin{align*}
        |R_X(\tau)| = [ C_X(t,\tau) + \mu_X^2]^2 \le [C_X(0) + \mu_X^2]^2 = R_X(0).
    \end{align*}
\end{proof}

\begin{definition}
    The \textbf{average power} of a wide sense stationary process $X(t)$ is $R_X(0)$.
\end{definition}

\begin{definition}
    A stochastic process $X(t)$ is \textbf{mean-ergodic} if the time average estimate
    \begin{align*}
        \hat{\mu}_X = \frac{1}{T} \int_0^T X(t) dt
    \end{align*}
    has zero mean square error in the limit.
\end{definition}

\begin{theorem}
    If a stochastic process $X(t)$ has a autocovariance function $C_X(\tau)$ which is absolutely integrable, then $X(t)$ is mean-ergodic.
\end{theorem}
\begin{proof}
    First we note that
    \begin{align*}
        E[\hat{\mu}_X] = \frac{1}{T} E\left[\int_0^T X(t) dt\right] = \frac{1}{T} \int_0^T E[X(t)] dt= \mu_X.
    \end{align*}
    So $\mu_X$ is unbiased. Next, since $C_X(\tau)$ is absolutely integrable, we calculate
    \begin{align*}
        \Var[\hat{\mu}_X] &= E\left[\left(\frac{1}{T}\int_0^T (X(t) - \mu_X)^2 dt\right)^2\right] \\
        &= \frac{1}{T^2} \int_0^T \int_0^T E[(X(t)-\mu_X)(X(t')-\mu_X)]dt' dt \\
        &= \frac{1}{T^2} \int_0^T \int_0^T C_X(t' - t)dt' dt \\
        &\le \frac{K}{T}.
    \end{align*}
    for some real $K$. Therefore $\Var[\hat{\mu}_X] \to 0$ if $T \to \infty$.
\end{proof}


\section{Power Spectral Density}
Consider a realisation $x(t)$ of a stochastic process $X(t)$. Its average power is calculated by
\begin{align*}
    P &= \lim_{T \to \infty} \frac{1}{T} \int_{-T/2}^{T/2} x(t) dt \\
    &= \lim_{T \to \infty} \frac{1}{T} \int_{-\infty}^{\infty} x_T(t) dt
\end{align*}
We now define $X_T(f) = \mathcal{F}[x_T(t)]$. Then by Parseval's theorem
\begin{align*}
    P &= \lim_{T \to \infty} \frac{1}{T} \int_{-\infty}^{\infty} |X_T(f)|^2 df \\
    &= \int_{-\infty}^{\infty} \lim_{T \to \infty} \frac{|X_T(f)|^2}{T} df.
\end{align*}
This motivates the following definition.
\begin{definition}
    The \textbf{power spectral density} (PSD) of a deterministic waveform $x(t)$ is 
    \begin{align*}
        \mathcal{P}(f) = \lim_{T \to \infty} \frac{|X_T(f)|^2}{T}.
    \end{align*}
\end{definition}

\begin{theorem}[Wiener-Khinchin]
    The power spectral density of a wide sense stationary stochastic process $X(t)$ is calculated by
    \begin{align*}
        \mathcal{P}(f) = \mathcal{F}[R_X(\tau)] = \int_{-\infty}^{\infty} R_X(\tau) \exp(-j2\pi f \tau) d \tau.
    \end{align*}
\end{theorem}

\chapter{Linear Algebra}
\section{Orthogonality}


\chapter{Signal Processing}



\end{document}