%!TEX program = xelatex

\documentclass{memoir}
\usepackage[no-math]{fontspec}
\usepackage{pgfplots}
\pgfplotsset{compat=newest}
\usepackage{commath}
\usepackage{mathtools}
\usepackage{amssymb}
\usepackage{amsthm}
\usepackage{booktabs}
\usepackage{mathtools}
\usepackage{xcolor}
\usepackage[separate-uncertainty=true, per-mode=symbol]{siunitx}
\usepackage[noabbrev, capitalize]{cleveref}
\usepackage{listings}
\usepackage[american inductor, european resistor]{circuitikz}
\usepackage{amsmath}
\usepackage{amsfonts}
\usepackage{ifxetex}
\usepackage[dutch,english]{babel}
\usepackage[backend=bibtexu,texencoding=utf8,bibencoding=utf8,style=ieee,sortlocale=en_GB,language=auto]{biblatex}
\usepackage[strict,autostyle]{csquotes}
\usepackage{parskip}
\usepackage{import}
\usepackage{standalone}
\usepackage{hyperref}
%\usepackage[toc,title,titletoc]{appendix}

\ifxetex{} % Fonts laden in het geval dat je met Xetex compiled
    \usepackage{fontspec}
    \defaultfontfeatures{Ligatures=TeX} % To support LaTeX quoting style
    \setromanfont{Palatino Linotype} % Tover ergens in Font mapje in root.
    \setmonofont{Source Code Pro}
\else % Terug val in standaard pdflatex tool chain. Geen ondersteuning voor OTT fonts
    \usepackage[T1]{fontenc}
    \usepackage[utf8]{inputenc}
\fi
\newcommand{\references}[1]{\begin{flushright}{#1}\end{flushright}}
\renewcommand{\vec}[1]{\boldsymbol{\mathbf{#1}}}
\newcommand{\uvec}[1]{\boldsymbol{\hat{\vec{#1}}}}
\newcommand{\mat}[1]{\boldsymbol{\mathbf{#1}}}
\newcommand{\fasor}[1]{\boldsymbol{\tilde{\vec{#1}}}}
\newcommand{\cmplx}[0]{\mathrm{j}}
\renewcommand{\Re}[0]{\operatorname{Re}}
\newcommand{\Cov}{\operatorname{Cov}}
\newcommand{\Var}{\operatorname{Var}}
\newcommand{\proj}{\operatorname{proj}}
\newcommand{\Perp}{\operatorname{perp}}
\newcommand{\col}{\operatorname{col}}
\newcommand{\rect}{\operatorname{rect}}
\newcommand{\sinc}{\operatorname{sinc}}
\newcommand{\IT}{\operatorname{IT}}
\newcommand{\F}{\mathcal{F}}

\newtheorem{definition}{Definition}
\newtheorem{theorem}{Theorem}


\DeclareSIUnit{\voltampere}{VA} %apparent power
\DeclareSIUnit{\pii}{\ensuremath{\pi}}

\hypersetup{%setup hyperlinks
    colorlinks,
    citecolor=black,
    filecolor=black,
    linkcolor=black,
    urlcolor=black
}

% Example boxes
\usepackage{fancybox}
\usepackage{framed}
\usepackage{adjustbox}
\newenvironment{simpages}%
{\AtBeginEnvironment{itemize}{\parskip=0pt\parsep=0pt\partopsep=0pt}
\def\FrameCommand{\fboxsep=.5\FrameSep\shadowbox}\MakeFramed{\FrameRestore}}%
{\endMakeFramed}

% Impulse train
\DeclareFontFamily{U}{wncy}{}
\DeclareFontShape{U}{wncy}{m}{n}{<->wncyr10}{}
\DeclareSymbolFont{mcy}{U}{wncy}{m}{n}
\DeclareMathSymbol{\Sha}{\mathord}{mcy}{"58}
\addbibresource{../includes/bibliography.bib}

\begin{document}

% Correlatie
% PSD
% Wiener-Kichine
% Wide sense stationary
% Sampling
% Aliasing
% DFT
% Compression Ratio
% SnR
% Matched Filter
% Complex envelopes
% Periodogram estimate
% Ergodisch
% Semidefinite programming
% Hermitisch etc
% Subspaces, affine spaces

\newcommand{\Cov}{\operatorname{Cov}}
\newcommand{\Var}{\operatorname{Var}}


\pagestyle{empty}
\begingroup
    \vspace*{0.12\textheight}
    {\hfill \HUGE Review of Basic Concepts}
    \vfill
    {\hfill \Large Wessel Bruinsma}
\endgroup

\chapter{Preliminaries}
\section{Convex functions}
\begin{definition}[Convex function]
    Let $f:S\to \mathbb{R}$ where $S$ is a convex subset of $\mathbb{R}$. Then $f$ is convex if for all $x_1,x_2 \in S$ and $\lambda \in [0,1]$
    \begin{align*}
        \lambda f(x_1) + (1- \lambda)f(x_2) \ge f( \lambda x_1 + (1- \lambda) x_2).
    \end{align*}
\end{definition}

\begin{theorem}
    Let $f:S \to \mathbb{R}$ be a convex function and let $\lambda_1,\ldots,\lambda_n \in \mathbb{R}$ be such that
    \begin{align*}
        \lambda_1 + \lambda_2 + \ldots + \lambda_n = 1.
    \end{align*}
    Then for all $x_1,\ldots,x_n \in S$
    \begin{align*}
        \lambda_1 f(x_1) + \lambda_2 f(x_2) + \ldots + \lambda_n f(x_n) \ge f(\lambda_1 x_1 + \lambda_2 x_2 + \ldots + \lambda_n x_n).
    \end{align*}
\end{theorem}
\begin{proof}
    We will proof by induction. Since $f$ is convex, the base case holds. Suppose that the statement is true for a certain $n$. Then, since $f$ is convex
    \begin{align*}
        &\lambda_1 f(x_1) + \lambda_2 f(x_2) + \ldots + \lambda_{n+1} f(x_{n+1}) \\
        &= (\lambda_1+\lambda_2)\left[\frac{\lambda_1}{\lambda_1 + \lambda_2} f(x_1) + \frac{\lambda_2}{\lambda_1+\lambda_2}f(x_2)\right] + \ldots + \lambda_{n+1} f(x_{n+1})  \\
        &\ge (\lambda_1 + \lambda_2)f\left( \frac{\lambda_1}{\lambda_1 + \lambda_2} x_1 + \frac{\lambda_2}{\lambda_1+\lambda_2}x_2\right) + \ldots + \lambda_{n+1} f(x_{n+1}) \\
        &\ge f\left[(\lambda_1 + \lambda_2) \frac{\lambda_1}{\lambda_1 + \lambda_2} x_1 + (\lambda_1 + \lambda_2)\frac{\lambda_2}{\lambda_1+\lambda_2}x_2\ldots + \lambda_{n+1}\right] \\
        &= f(\lambda_1 x_1 + \lambda_2 x_2 + \ldots + \lambda_{n+1} x_{n+1}).
    \end{align*}
\end{proof}

\chapter{Stochastic Processes}
\section{Basic Definitions}
\begin{definition}
    A \textbf{random variable} consists of an experiment with a probability measure $P[.]$ defined on a sample space $S$ and a function that assigns a real number to each outcome in the sample space of the experiment.
\end{definition}

\begin{definition}
    Subsets of the sample space are called \textbf{events}.
\end{definition}

\begin{definition}
    The \textbf{probability mass function} (PMF) of the discrete random variable $X$ is
    \begin{align*}
        P_X(x) = P[X=x].
    \end{align*}
\end{definition}

\begin{definition}
    The \textbf{cumulative distribution function} (CDF) of a random variable $X$ is
    \begin{align*}
        F_X(x) = P[X \le x].
    \end{align*}
\end{definition}
The CDF of a derived random variable $Y=g(X)$ is
\begin{align*}
    F_{Y}(x) = P[Y \le x] = P[g(X) \le x] = P[X \le g^{-1}(x)].
\end{align*}

\begin{definition}
    The \textbf{expected value} of a random variable $X$ is
    \begin{align*}
        E[X]=\mu_X = \sum_{x \in S_X} x P_X(x).
    \end{align*}
\end{definition}

\begin{definition}
    The \textbf{variance} of a random variable $X$ is
    \begin{align*}
        \Var[X] = E[(X-\mu_X)^2].
    \end{align*}
\end{definition}
The variance of a random variable $X$ simplifies to
\begin{align*}
    \Var[X] = E[X^2] - E^2[X].
\end{align*}

\section{Basic theorems}

\begin{theorem}[Jensen's Inequality]
    Let $f$ be a convex function and $X$ a random variable. Then
    \begin{align*}
        E[f(X)] \ge f(E[X]).
    \end{align*}
\end{theorem}
\begin{proof}
    Note that
    \begin{align*}
        \sum_{x \in S_X} P_X(x) = 1.
    \end{align*}
    Then, by the definition of the expected value and the convexity of $f$
    \begin{align*}
        E[f(X)] &= \sum_{x \in S_X} f(x) P_X(x) \\
        &\ge f\left[\sum_{x \in S_X} x P_X(x) \right] \\
        &=f(E[X]).
    \end{align*}
\end{proof}

\begin{theorem}[Cauchy-Schwarz Inequality]
    Let $X$ and $Y$ be random variables. Then
    \begin{align*}
        (E[XY])^2 \le E[X^2]E[Y^2].
    \end{align*}
\end{theorem}
\begin{proof}
    Let $a \in \mathbb{R}$, then
    \begin{align*}
        E[(a X + Y)^2] = E[X^2]a^2  + 2E[XY]a + E[Y^2].
    \end{align*}
    Consider this as a function of $a$. Note that  $E[(aX+Y)^2]\ge0$. So the obtained polynomial will have no zero-crossings. Therefore
    \begin{align*}
        (2 E[XY])^2-4 E[X^2] E[Y^2] \le 0.
    \end{align*}
\end{proof}

\section{Dependence}

\begin{definition}
    The \textbf{conditional probability} of an event $A$ given event $B$ is
    \begin{align*}
        P[A|B] = \frac{P[A \cap B]}{P[B]}.
    \end{align*}
\end{definition}

\begin{definition}
    Events $A$ and $B$ are \textbf{independent} if
    \begin{align*}
        P[A \cap B]=P[A] P[B].
    \end{align*}
\end{definition}

\begin{theorem}
    $A$ and $B$ are two independent events if and only if
    \begin{align*}
        P[A | B] = P[A].
    \end{align*}
\end{theorem}
\begin{proof}
    By the definitions
    \begin{align*}
        P[A | B] = \frac{P[A \cap B]}{P[B]} = \frac{P[A]P[B]}{P[B]} = P[A].
    \end{align*}
\end{proof}

\begin{definition}
    The \textbf{covariance} of two random variables $X$ and $Y$ is
    \begin{align*}
        \Cov[X,Y]=E[(X- \mu_X)(Y- \mu_Y)].
    \end{align*}
\end{definition}

\begin{definition}
    The \textbf{correlation} of two random variables $X$ and $Y$ is
    \begin{align*}
        r_{X,Y} = E[XY].
    \end{align*}
\end{definition}
The covariance of two random variables $X$ and $Y$ simplifies to
\begin{align*}
    \Cov[X,Y] = r_{X,Y} - \mu_X \mu_Y.
\end{align*}
The variance of the addition of two random variables $X$ and $Y$ simplifies to
\begin{align*}
    \Var[X+Y] = \Var[X] + \Var[Y] + 2 \Cov[X,Y].
\end{align*}
Consider two random variables $X$ and $Y$. If $\Cov[X,Y]=0$, then $X$ and $Y$ are \textbf{uncorrelated}. Correlation is a measure of dependence. If $\Cov[X,Y]>0$, then $X>\mu_X$ implies that we expect $Y>\mu_Y$.

If $r_{X,Y}=0$, then $X$ and $Y$ are \textbf{orthogonal}. In the case of zero-mean variables, orthogonal variables are uncorrelated.

\begin{definition}
    The \textbf{correlation coefficient} of two random variables $X$ and $Y$ is
    \begin{align*}
        \rho_{X,Y} = \frac{\Cov[X,Y]}{\sqrt{\Var[X]\Var[Y]}}.
    \end{align*}
\end{definition}

\begin{theorem}
    $|\rho_{X,Y}|\le1$ holds.
\end{theorem}
\begin{proof}
    We will proof that $\rho_{X,Y}^2 \le 1$. To this end, by the Cauchy-Schwarz Inquality
    \begin{align*}
        \Cov[X,Y]^2 &= E[(X- \mu_X)(Y- \mu_Y)]^2 \\
        &\le E[(X-\mu_X)^2] E[(Y- \mu_Y)^2] \\
        &= \Var[X] \Var[Y]
    \end{align*}
    Therefore
    \begin{align*}
        \rho_{X,Y}^2 = \frac{\Cov[X,Y]^2}{\Var[X]\Var[Y]} \le 1.
    \end{align*}
\end{proof}

\section{Stochastic Processes}

\begin{definition}
    A \textbf{stochastic process} $X(t)$ consists of an experiment with a probability measure $P[.]$ defined on a sample space $S$ and a function that assigns a time function $x(t,s)$ to each outcome $s$ in the sample space of the experiment.
\end{definition}

Suppose that we observe a random process $X(t)$ at a particular time $t_1$. Then each observation yields a $x(t_1,s)$. Therefore $x(t_1,s)$ is the sample value of a random variable. We denote this random variable by $X(t_1)$.

\begin{definition}
    A stochastic process $X(t)$ is \textbf{independently identically distributed} if
    \begin{enumerate}
        \item $X(t_k)$ for all $k$ are independent random variables,
        \item $X(t_k)$ for all $k$ have identical probability distribution functions.
    \end{enumerate}
\end{definition}

\section{Stationary Process}







\end{document}