\documentclass[oneside, a4paper, openany]{article}
\usepackage[no-math]{fontspec}
\usepackage{pgfplots}
\usepackage{float}
\pgfplotsset{compat=newest}
\usepackage{commath}
\usepackage{mathtools}
\usepackage{amssymb}
\usepackage{amsthm}
\usepackage{booktabs}
\usepackage{todonotes}
\usepackage{mathtools}
\usepackage{xcolor}
\usepackage[separate-uncertainty=true, per-mode=symbol]{siunitx}
\usepackage{listings}
\usepackage[american inductor, european resistor]{circuitikz}
\usepackage{amsmath}
\usepackage{amsfonts}
\usepackage{ifxetex}
\usepackage[dutch,english]{babel}
\usepackage[backend=bibtexu,texencoding=utf8,bibencoding=utf8,style=ieee,sortlocale=en_GB,language=auto]{biblatex}
\usepackage[strict,autostyle]{csquotes}
\usepackage{import}
\usepackage{standalone}
\usepackage{bookmark,hyperref}
\usepackage{xcolor,mdframed}
\usepackage{tikz}
\usepackage{framed}
\usepackage{float}
\usepackage{tabularx}
\usepackage{graphicx,adjustbox}
\usepackage{rotating}
\usepackage{pdfpages}
\usepackage{enumitem}
\usepackage{calc}
\usepackage{pgfplots}
\usepackage{filecontents}
\usepackage{caption}
\usepackage{subcaption}
\usepackage{lettrine}

\newcolumntype{Y}{>{\raggedright\arraybackslash}X} % Left-justified text in tabularx environment

\ifxetex{} % Fonts laden in het geval dat je met Xetex compiled
    \usepackage{fontspec}
    \defaultfontfeatures{Scale=MatchLowercase, Ligatures=TeX} % To support LaTeX quoting style
    %\setromanfont{Palatino Linotype} % Tover ergens in Font mapje in root.
    \setsansfont{Avenir Next LT Pro}
    \setromanfont{Adobe Caslon Pro} % Tover ergens in Font mapje in root.
    \setmonofont{Source Code Pro}
\else % Terug val in standaard pdflatex tool chain. Geen ondersteuning voor OTT fonts
    \usepackage[T1]{fontenc}
    \usepackage[utf8]{inputenc}
\fi
\usepackage[noabbrev, capitalize]{cleveref}
\usepackage{ifthen}
\usepackage{titlesec}
\usepackage{titlecaps}

\newcommand{\references}[1]{\begin{flushright}{#1}\end{flushright}}
\renewcommand{\vec}[1]{\boldsymbol{\mathbf{#1}}}
\newcommand{\uvec}[1]{\boldsymbol{\hat{\vec{#1}}}}
\newcommand{\mat}[1]{\boldsymbol{\mathbf{#1}}}
\newcommand{\fasor}[1]{\boldsymbol{\tilde{\vec{#1}}}}
\newcommand{\cmplx}[0]{\mathrm{j}}
\renewcommand{\Re}[0]{\operatorname{Re}}
\newcommand{\Cov}{\operatorname{Cov}}
\newcommand{\Var}{\operatorname{Var}}
\newcommand{\proj}{\operatorname{proj}}
\newcommand{\Perp}{\operatorname{perp}}
\newcommand{\col}{\operatorname{col}}
\newcommand{\rect}{\operatorname{rect}}
\newcommand{\sinc}{\operatorname{sinc}}
\newcommand{\lcm}{\operatorname{lcm}}
%\newcommand{\gcd}{\operatorname{gcd}}
\newcommand{\F}{\mathcal{F}}
\newcommand{\DTFT}{\mathcal{F}_*}
\newcommand{\conj}[1]{#1^*}
\renewcommand{\mod}{\operatorname{mod}}
\newcommand{\rot}{\operatorname{rot}}
\newcommand{\vecsc}[1]{\vec{\textsc{\textbf{#1}}}}
\renewcommand{\ss}[1]{_{#1}}

% Label without linebreak breaker
\newcommand{\lab}[1]{\label{#1}\nolinebreak}

\newtheorem{definition}{Definition}
\newtheorem{theorem}{Theorem}


\DeclareSIUnit{\voltampere}{VA} %apparent power
\DeclareSIUnit{\pii}{\ensuremath{\pi}}

\hypersetup{%setup hyperlinks
    colorlinks,
    citecolor=black,
    filecolor=black,
    linkcolor=black,
    urlcolor=black
}

% Example boxes
\usepackage{fancybox}
\usepackage{framed}
\usepackage{adjustbox}
\newenvironment{simpages}%
{\AtBeginEnvironment{itemize}{\parskip=0pt\parsep=0pt\partopsep=0pt}
\def\FrameCommand{\fboxsep=.5\FrameSep\shadowbox}\MakeFramed{\FrameRestore}}%
{\endMakeFramed}

% Impulse train
\DeclareFontFamily{U}{wncy}{}
\DeclareFontShape{U}{wncy}{m}{n}{<->wncyr10}{}
\DeclareSymbolFont{mcy}{U}{wncy}{m}{n}
\DeclareMathSymbol{\Sha}{\mathord}{mcy}{"58}

\setlength{\parindent}{0pt}
\nonzeroparskip

% Block environment configuration
\newcommand{\BlockLeftMargin}{20pt}
\newcommand{\BlockLeftMarginText}{25pt}
\newcommand{\BlockLeftMarginTextSpacing}{10pt}

% Own colours
\definecolor{gray75}{gray}{0.75}

% Block environment
\newenvironment{block}[3]{%
\makebox{\hspace{-\spinemargin}%
\begin{tikzpicture}[overlay]
    \draw [thick,color=gray75] (\BlockLeftMargin, 0) -- (\paperwidth - \spinemargin, 0);
    \node at (\BlockLeftMarginText, -0.9) [align=left, text width=\spinemargin - \BlockLeftMarginText - \BlockLeftMarginTextSpacing, anchor=west, text depth=1cm] {\textbf{\textsc{#1}}\newline\textit{#3}};
\end{tikzpicture}}%
\nopagebreak\\[0.25em]\ifthenelse{\equal{#2}{}}{}{(\textit{#2}.) }\nopagebreak\nolinebreak}
{\nopagebreak\\[-0.25em]%
\makebox{\hspace{-\spinemargin}%
\begin{tikzpicture}[overlay, remember picture]
    \draw [thick,color=gray75] (\spinemargin,0) -- (\paperwidth - \spinemargin,0);
\end{tikzpicture}} \vspace{0.5em}}

% Theorem
\newcounter{blockTheoremCounter}
\crefname{blockTheoremCounter}{Theorem}{Theorems}
\Crefname{blockTheoremCounter}{Theorem}{Theorems}

\newenvironment{blockTheorem}[1][]{%
\refstepcounter{blockTheoremCounter}%
\begin{block}{theorem \theblockTheoremCounter}{#1}{}}
{\end{block}}

% Definition
\newcounter{blockDefinitionCounter}
\crefname{blockDefinitionCounter}{Definition}{Definitions}
\Crefname{blockDefinitionCounter}{Definition}{Definitions}

\newenvironment{blockDefinition}[1][]{%
\refstepcounter{blockDefinitionCounter}%
\begin{block}{definition \theblockDefinitionCounter}{#1}{}}
{\end{block}}

% Proof
\newcounter{blockProofTheoremCounter}
\crefname{blockProofTheoremCounter}{Proof}{Proofs}
\Crefname{blockProofTheoremCounter}{Proof}{Proofs}

\newenvironment{blockProofTheorem}[1]{%
\refstepcounter{blockProofTheoremCounter}%
\begin{block}{proof of \\ theorem #1}{}{}}
{\qed\end{block}}

% Detail
\newcounter{blockDetailCounter}
\crefname{blockDetailCounter}{Detail}{Details}
\Crefname{blockDetailCounter}{Detail}{Details}

\newenvironment{blockDetail}[1][]{%
\refstepcounter{blockDetailCounter}%
\begin{block}{detail \theblockDetailCounter}{#1}{}}
{\end{block}}

% Redesign chapter headings
\newcommand{\chapternumber}{\thechapter}
\newcommand{\hsp}{\hspace{20pt}}
\titleformat{\chapter}[hang]{\Huge\bfseries}{\chapternumber\hsp\textcolor{gray75}{|}\hsp}{0pt}{\Huge\bfseries}

% Remove headers
% \addtopsmarks{headings}{}{
%   \createmark{chapter}{left}{nonumber}{}{}
% }
% \pagestyle{headings} % Activate changes

% Capitalise headers in a regular way
\renewcommand*{\memUChead}[1]{\titlecap{#1}}

% \hfill for math mode
\newcommand{\pushright}[1]{\intertext{\hfill$\displaystyle #1$}}
\newcommand{\pushline}{\hskip \textwidth minus \textwidth}
\newcommand{\matlab}{\textsc{Matlab}}

\definecolor{code-grey}{HTML}{DDDDDD}
\newcommand{\lib}[1]{\textsf{#1}}
\newcommand{\file}[1]{\textsf{#1}}
\newcommand{\func}[1]{\colorbox{code-grey}{\texttt{#1}}}
\newcommand{\class}[1]{\colorbox{code-grey}{\texttt{#1}}}

% Setup actiepunten
\newenvironment{important}[1][]{%
   \begin{mdframed}[%
      backgroundcolor={red!15}, hidealllines=true,
      skipabove=0.7\baselineskip, skipbelow=0.7\baselineskip,
      splitbottomskip=2pt, splittopskip=4pt, #1]%
   \makebox[0pt]{% ignore the withd of !
      \smash{% ignor the height of !
         \fontsize{32pt}{32pt}\selectfont% make the ! bigger
         \hspace*{-19pt}% move ! to the left
         \raisebox{-2pt}{% move ! up a little
            {\color{red!70!black}\sffamily\bfseries !}% type the bold red !
         }%
      }%
   }%
}{\end{mdframed}}
\newcommand{\excl}[1]{
\begin{important}
  \textbf{#1}
\end{important}
}

\makeatletter
\newcommand\footnoteref[1]{\protected@xdef\@thefnmark{\ref{#1}}\@footnotemark}
\makeatother

% Allow page breaks in display environments
%\allowdisplaybreaks
% S unit for use in Mega Samples per second
\DeclareSIUnit\sample{S}

\newcommand{\CC}{C\nolinebreak\hspace{-.05em}\raisebox{.3ex}{ \textbf{+}}\nolinebreak\hspace{-.10em}\raisebox{.3ex}{\textbf{+}}}
\def\CC{{C\nolinebreak[4]\hspace{-.05em}\raisebox{.3ex}{\textbf{++}}}}


\newcommand{\partauthor}[1]{\gdef\@partauthor{#1}}
\renewcommand{\printparttitle}[1]{
  \parttitlefont #1\\
  \vspace{1.5cm}
  \textnormal{\Large \@partauthor}
}
\title{Achtergrondinformatie \textit{Slimme Stoorzender}}
\author{Wessel Bruinsma \and Robin Hes \and Kees Kroep \and Dorus Leliveld \and Willem Melching \and Tom aan de Wiel}

\begin{document}
\maketitle

\clearpage

\tableofcontents

\clearpage

\section{Inleiding}
Wij zijn een groep studenten Electrical Engineering van de Technische Universiteit Delft. Op het moment houden wij ons bezig met ons Bachelor afstudeerproject, de laatste horde tot ons Bachelor-diploma. Onze groep bestaat uit zes studenten, van wie iedereen zijn Propedeuse cum laude behaald heeft en verscheidene het honoursprogramma hebben afgerond.

In dit document zullen we kort uiteenzetten...

\section{Bachelor afstudeerproject}
Het Bachelor afstudeerproject geeft ons de kans om de kennis die we tijdens onze studie hebben opgedaan toe te passen in het ontwerp van een product. We krijgen hierbij de mogelijkheid om onderzoek te doen naar state-of-the-art technologie. Dit maakt het mogelijk om bepaalde technologische grenzen te verleggen.


% Tijdspanne

\subsection{Begeleider}
% Geert Leus, awesome gast!

\section{Huidige problemen met stoorzenders}
\subsection{Eigen communicatie}
\subsection{Makkelijk te detecteren}
\subsection{Lastig groot deel spectrum te storen}
\subsection{Kost veel vermogen}


\section{Onze oplossing}
\subsection{Methode}
\subsubsection{Smart sensing}
Hier wat zeggen over compressive sensing
\subsubsection{Detectie van signalen}
\subsubsection{Transmissie}

\subsection{Waarom nu?}


\section{Ons product}
\subsection{Toepassingen}
\subsection{Vergelijkbare producten}

\section{Discussie}
% Hoe nuttig is ons product?
%     Komen de problemen die wij schetsen overeen met bestaande problemen?
%     Welke problemen ziet defensie graag opgelost?
%     In welke gebieden van defensie zou ons product interessant zijn?
% Wat zijn vergelijkbare producten?
%    Wordt hier binnen defensie al onderzoek naar gedaan? Wat gebeurt er precies?
%    Price range? Wat is nog interessant?
% Hoe ziet de bedrijfskolom er uit?
%     Wat is de markt binnen defensie
% Regelgeving?


\clearpage
\section{Oude meuk}
Als onderdeel van onze Bachelor Electrical Engineering worden we opgedeeld in groepen van zes studenten. Samen moeten we een klein bedrijf starten en werken aan een product met commerciele mogelijkheden. onze begeleider is prof. Geert Leus. Hij doet onderzoek in het gebied van Cognitive Radio.

\subsection{Cognitive Radio}
Het spectrum is dan wel opgekocht en verdeeld, het wordt wel schaars gebruikt. Grote delen van het spectrum worden niet vaak benut. Uit deze observatie komt het idee van Cognitive Radio voort. Het idee is om het spectrum te scannen en vast te leggen welke bandjes er in gebruik zijn en welke niet. Vervolgens kiest de cognitive radio een ongebruikte band uit en gaat daar op communiceren. Mocht de eigenaar van de betreffende band beginnen met zenden, dan zoekt de Cognitive Radio een nieuw leeg stuk spectrum uit en gaat daar communiceren.

\subsection{Cognitive Jammer}
Het concept van Cognitive Radio kan je ook omdraaien. Je kan juist gaan uitzenden en dus storen op de schaarse hoeveelheid bandjes die in gebruik is. Daarmee kan je dus met een minimale hoeveelheid energie de communicatie op een groot stuk spectrum plat leggen. Dit concept vormt de basis van ons product.

\subsection{Vergelijking Conventionele Jammer}
Conventionele jammers hebben een andere aanpak. Van te voren wordt een band bepaald waarop communicatie gestoord moet worden. Vervolgens wordt een grote hoeveelheid vermogen in die band gestopt zodat er geen communicatire meer mogelijk is. Ons product zou een aantal voordelen kunnen bieden ten opzichte van de conventionele jammer.

\subsection{Weinig vermogen}
Omdat de cognitive jammer slechts een minimaal deel van het spectrum hoeft te storen verbruikt het veel minder energie. Dit opent deuren voor meer en kleinere toepassingen. Een dronre met cognitive jamming ingebouwd zou hierdoor meer re\"eel worden.

\subsection{Moeilijk te detecteren}
Waar een conventionele jammer zeer makkelijk te detecteren is, omdat het vanuit \'e\'en punt een grote hoeveelheid vermogen uitzend, is een cognitive jammer veel moeilijker te detecteren. Omdat de cognitive jammer schaars jamt is het moeilijk te onderscheiden van normale communicatie.

\subsection{Eigen communicatie mogelijk}
Als er gegevens van eigen communicatie bekend zijn kunnen deze uitgefilterd worden als er gescant wordt naar vijandelijke communicatie. Hierdoor blijft eigen communicatie nog steeds mogelijk.

\subsection{State-of-the-Art Technologie}
Recent zijn er veel vorderingen gemaakt op het gebied van spectrum scannen. Hierdoor is het mogelijk om spectrum scanners substantieel goedkoper te kunnen produceren door toepassing van een aantal moderne algoritmes. Wij implementeren deze algoritmes in onze producten. Hierbij wordt gebruik gemaakt van het feit dat we niet de inhoud willen weten van de communicatie, maar slechts of er wordt gecommuniceerd of niet.

\end{document}
