%!TEX program = xelatex

\documentclass[a4paper, openany, oneside]{memoir}
\usepackage[no-math]{fontspec}
\usepackage{pgfplots}
\pgfplotsset{compat=newest}
\usepackage{commath}
\usepackage{mathtools}
\usepackage{amssymb}
\usepackage{amsthm}
\usepackage{booktabs}
\usepackage{mathtools}
\usepackage{xcolor}
\usepackage[separate-uncertainty=true, per-mode=symbol]{siunitx}
\usepackage[noabbrev, capitalize]{cleveref}
\usepackage{listings}
\usepackage[american inductor, european resistor]{circuitikz}
\usepackage{amsmath}
\usepackage{amsfonts}
\usepackage{ifxetex}
\usepackage[dutch,english]{babel}
\usepackage[backend=bibtexu,texencoding=utf8,bibencoding=utf8,style=ieee,sortlocale=en_GB,language=auto]{biblatex}
\usepackage[strict,autostyle]{csquotes}
\usepackage{parskip}
\usepackage{import}
\usepackage{standalone}
\usepackage{hyperref}
%\usepackage[toc,title,titletoc]{appendix}

\ifxetex{} % Fonts laden in het geval dat je met Xetex compiled
    \usepackage{fontspec}
    \defaultfontfeatures{Ligatures=TeX} % To support LaTeX quoting style
    \setromanfont{Palatino Linotype} % Tover ergens in Font mapje in root.
    \setmonofont{Source Code Pro}
\else % Terug val in standaard pdflatex tool chain. Geen ondersteuning voor OTT fonts
    \usepackage[T1]{fontenc}
    \usepackage[utf8]{inputenc}
\fi
\newcommand{\references}[1]{\begin{flushright}{#1}\end{flushright}}
\renewcommand{\vec}[1]{\boldsymbol{\mathbf{#1}}}
\newcommand{\uvec}[1]{\boldsymbol{\hat{\vec{#1}}}}
\newcommand{\mat}[1]{\boldsymbol{\mathbf{#1}}}
\newcommand{\fasor}[1]{\boldsymbol{\tilde{\vec{#1}}}}
\newcommand{\cmplx}[0]{\mathrm{j}}
\renewcommand{\Re}[0]{\operatorname{Re}}
\newcommand{\Cov}{\operatorname{Cov}}
\newcommand{\Var}{\operatorname{Var}}
\newcommand{\proj}{\operatorname{proj}}
\newcommand{\Perp}{\operatorname{perp}}
\newcommand{\col}{\operatorname{col}}
\newcommand{\rect}{\operatorname{rect}}
\newcommand{\sinc}{\operatorname{sinc}}
\newcommand{\IT}{\operatorname{IT}}
\newcommand{\F}{\mathcal{F}}

\newtheorem{definition}{Definition}
\newtheorem{theorem}{Theorem}


\DeclareSIUnit{\voltampere}{VA} %apparent power
\DeclareSIUnit{\pii}{\ensuremath{\pi}}

\hypersetup{%setup hyperlinks
    colorlinks,
    citecolor=black,
    filecolor=black,
    linkcolor=black,
    urlcolor=black
}

% Example boxes
\usepackage{fancybox}
\usepackage{framed}
\usepackage{adjustbox}
\newenvironment{simpages}%
{\AtBeginEnvironment{itemize}{\parskip=0pt\parsep=0pt\partopsep=0pt}
\def\FrameCommand{\fboxsep=.5\FrameSep\shadowbox}\MakeFramed{\FrameRestore}}%
{\endMakeFramed}

% Impulse train
\DeclareFontFamily{U}{wncy}{}
\DeclareFontShape{U}{wncy}{m}{n}{<->wncyr10}{}
\DeclareSymbolFont{mcy}{U}{wncy}{m}{n}
\DeclareMathSymbol{\Sha}{\mathord}{mcy}{"58}
\addbibresource{../includes/bibliography.bib}

%% Scroll door naar 'Vragen'
\usepackage{fullpage}
\usepackage{minibox}
\pagenumbering{gobble}
\newcounter{counter}
\newcommand{\docount}{\stepcounter{counter} \thecounter}
\renewcommand{\section}[1]{
    \begin{flushright}
        \fontsize{0.5cm}{1em}
        \textsc{#1}
    \end{flushright}
}
\newcolumntype{Y}{>{\raggedright\arraybackslash}X} % Left-justified text in tabularx environment

% Opinion environment
\newcommand{\opinionoptionwidth}{0.8cm}
\newenvironment{opinion}{
    \tabularx{\textwidth}{lYccccc}
    &&
        \makebox[\opinionoptionwidth]{\minibox[c]{zeer\\oneens}} &
        \makebox[\opinionoptionwidth]{oneens} &
        \makebox[\opinionoptionwidth]{neutraal} &
        \makebox[\opinionoptionwidth]{eens} &
        \makebox[\opinionoptionwidth]{\minibox[c]{zeer\\eens}} \\
}{
    \endtabularx
}
\newcommand{\opinionoptions}{& $\square$ & $\square$ & $\square$ & $\square$ & $\square$ \\}
\newcommand{\opinionquestion}[1]{
    \docount & #1 \opinionoptions
}

% Fill gap environment
\renewenvironment{fill}{
    \tabularx{\textwidth}{lY}
}{
    \endtabularx
}
\newcommand{\fillgap}{\underline{\hspace{3cm}}~}
\newcommand{\fillgaplarge}{\underline{\hspace{6cm}}~}
\newcommand{\fillgapline}{\underline{\hspace{\linewidth}} ~ }
\newcommand{\fillquestion}[1]{
    \docount & #1 \\
}

\begin{document}
% Title
\makebox[\textwidth]{
    \centering
    \fontsize{1cm}{1em}
    \textsc{Questionnaire}
} \\

Dit document bevat aan aantal vragen die u tijdens de bijeenkomst kunt beantwoorden. De informatie zal worden gebruikt om de interesse en staat van ons product te onderzoeken. Uw antwoorden zullen nooit persoonlijk aan u gerelateerd worden, met uitzondering van de voorwaarde hieronder. Het invullen van deze questionnaire is vrijwillig.

\begin{tabularx}{\textwidth}{lX}
    $\square$ & Mijn antwoorden mogen worden verwerkt tot informatie die Spectral tijdens hun eindpresentatie mag presenteren. \\
    $\square$ & Ik mag worden benaderd voor nadere toelichting op mijn antwoorden.
\end{tabularx}

%% Vragen

\section{algemeen}
\begin{opinion}
    \opinionquestion{Rechtstreekse aankoop van producten uit een startup is geen probleem.}
    \opinionquestion{Aankoop van producten is alleen mogelijk via een samenwerking met een gerenommeerde partij.}
    \opinionquestion{De kans op succes is groter door samen te werken met een gerenomeerde partij.}
    \opinionquestion{Ik zie mogelijkheden om een samenwerkingsverband met de TU Delft te beginnen.}
    \opinionquestion{De presentatie van Spectral heeft duidelijk gemaakt wat ik wilde weten.}
    \opinionquestion{Spectral heeft professioneel gehandeld.}
\end{opinion}
\begin{fill}
    \fillquestion{Ik zie andere mogelijkheden om de voorgestelde technologie toe te passen, namelijk \fillgapline \fillgapline}
\end{fill}
\section{slimme stoorzender}
\begin{opinion}
    \opinionquestion{De techniek achter de slimme stoorzender maakt kans binnen de markt van Defensie.}
    \opinionquestion{De slimme stoorzender maakt kans binnen de markt van Defensie.}
    \opinionquestion{De slimme stoorzender biedt mogelijkheden die er nog niet zijn.}
    \opinionquestion{Ik zie kansen om de slimme stoorzender toe te passen in het storen van de vijandige communicatie terwijl de eigen communicatie behouden blijft.}
    \opinionquestion{Ik zie kansen om de slimme stoorzender toe te passen in het onschadelijk maken van IED's.}
    \opinionquestion{Ik zie kansen om de slimme stoorzender toe te passen in combinatie met drones.}
\end{opinion}
\begin{fill}
    \fillquestion{De slimme stoorzender is interessant als deze \fillgap kost.}
    \fillquestion{De slimme stoorzender is vergelijkbaar met \fillgaplarge .}
    \fillquestion{Ik heb een opmerking over de slimme stoorzender, namelijk \fillgapline \fillgapline}
\end{fill}

\end{document}
