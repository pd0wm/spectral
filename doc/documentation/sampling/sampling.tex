
\documentclass[report, oneside, a4paper, openany]{memoir}
\usepackage[no-math]{fontspec}
\usepackage{pgfplots}
\pgfplotsset{compat=newest}
\usepackage{commath}
\usepackage{mathtools}
\usepackage{amssymb}
\usepackage{amsthm}
\usepackage{booktabs}
\usepackage{mathtools}
\usepackage{xcolor}
\usepackage[separate-uncertainty=true, per-mode=symbol]{siunitx}
\usepackage[noabbrev, capitalize]{cleveref}
\usepackage{listings}
\usepackage[american inductor, european resistor]{circuitikz}
\usepackage{amsmath}
\usepackage{amsfonts}
\usepackage{ifxetex}
\usepackage[dutch,english]{babel}
\usepackage[backend=bibtexu,texencoding=utf8,bibencoding=utf8,style=ieee,sortlocale=en_GB,language=auto]{biblatex}
\usepackage[strict,autostyle]{csquotes}
\usepackage{parskip}
\usepackage{import}
\usepackage{standalone}
\usepackage{hyperref}
%\usepackage[toc,title,titletoc]{appendix}

\ifxetex{} % Fonts laden in het geval dat je met Xetex compiled
    \usepackage{fontspec}
    \defaultfontfeatures{Ligatures=TeX} % To support LaTeX quoting style
    \setromanfont{Palatino Linotype} % Tover ergens in Font mapje in root.
    \setmonofont{Source Code Pro}
\else % Terug val in standaard pdflatex tool chain. Geen ondersteuning voor OTT fonts
    \usepackage[T1]{fontenc}
    \usepackage[utf8]{inputenc}
\fi
\newcommand{\references}[1]{\begin{flushright}{#1}\end{flushright}}
\renewcommand{\vec}[1]{\boldsymbol{\mathbf{#1}}}
\newcommand{\uvec}[1]{\boldsymbol{\hat{\vec{#1}}}}
\newcommand{\mat}[1]{\boldsymbol{\mathbf{#1}}}
\newcommand{\fasor}[1]{\boldsymbol{\tilde{\vec{#1}}}}
\newcommand{\cmplx}[0]{\mathrm{j}}
\renewcommand{\Re}[0]{\operatorname{Re}}
\newcommand{\Cov}{\operatorname{Cov}}
\newcommand{\Var}{\operatorname{Var}}
\newcommand{\proj}{\operatorname{proj}}
\newcommand{\Perp}{\operatorname{perp}}
\newcommand{\col}{\operatorname{col}}
\newcommand{\rect}{\operatorname{rect}}
\newcommand{\sinc}{\operatorname{sinc}}
\newcommand{\IT}{\operatorname{IT}}
\newcommand{\F}{\mathcal{F}}

\newtheorem{definition}{Definition}
\newtheorem{theorem}{Theorem}


\DeclareSIUnit{\voltampere}{VA} %apparent power
\DeclareSIUnit{\pii}{\ensuremath{\pi}}

\hypersetup{%setup hyperlinks
    colorlinks,
    citecolor=black,
    filecolor=black,
    linkcolor=black,
    urlcolor=black
}

% Example boxes
\usepackage{fancybox}
\usepackage{framed}
\usepackage{adjustbox}
\newenvironment{simpages}%
{\AtBeginEnvironment{itemize}{\parskip=0pt\parsep=0pt\partopsep=0pt}
\def\FrameCommand{\fboxsep=.5\FrameSep\shadowbox}\MakeFramed{\FrameRestore}}%
{\endMakeFramed}

% Impulse train
\DeclareFontFamily{U}{wncy}{}
\DeclareFontShape{U}{wncy}{m}{n}{<->wncyr10}{}
\DeclareSymbolFont{mcy}{U}{wncy}{m}{n}
\DeclareMathSymbol{\Sha}{\mathord}{mcy}{"58}

\title{Documentation}

\author{W.P. Bruinsma \and R.P. Hes \and H.J.C. Kroep \and T.C. Leliveld \and W.M. Melching \and T.A. aan de Wiel}

\begin{document}
\chapter{Sampling}
This block takes the output of the generation block and takes samples of it to be send to the reconstruction block. 
This section is the simulation of hardware with less samplerate than the input signal.


\section{Mimial Sparse Ruler Sampling}
\subsection{Concept}
The minimal sarse ruler sampling is design to use the sampling filters in such a way, that the reconstruction method performs optimally. This optimal solution is achieved using the minimal sparse ruler problem.
\subsection{Theory}
This Sampling method is a kind of multicoset sampling. This means that multiple (cheaper) devices are used to aquire the data. Every device measures the signal $x[n]$, samples it with a function $c_l[n]$, and then integrate and dumps the signal with over a period of $N$ samples. We define 
$$
c_l[n] = \delta[-n -n_i]
$$
$$
z_l[n] = c_l[n]\ast x[n] 
$$
Where
\begin{itemize}
\item $x[n]$ the input signal
\item $c_l[n]$ the filter of device $l$
\item $n_i$ the location of the dirac pulse
\item $z_l[n]$ the output of the sampling
\end{itemize}
Note that the dirac pulses make sure that
$$z_l[n]=0 \quad \quad \text{for }\;n\,\text{mod}\,N\neq n_i$$
When we Integrate and dump the signal we get output $y_l[k]$, which basically is the one sampled value in the period $kN\leq n<(k+1)N$.\\
\\
Now we want to choose the $n_i$ in such a way, that the cross-correlations of the different devices are lineary independent (see reconstruction to understand this). We also largest decimationfactor. we can get with $M$ devices.\\
\\
The cross-correlation between the sampling filters is 
$$
r_{ci,cj}[n] = \delta [n-n_i+n_j]
$$
We want to get enough different values for $n_i+n_j$ to solve the problem. This is a minimal sparse ruler problem. For a decimation of $N$ we want to get the least amount of $n_i \in \{0,1,\dots,N-1\}$, that are able to combine into $N$ different values of $n_i+n_j$. 

For example, for $N=6$ we get
\begin{equation*}
\begin{split}
n_i&\in\{0,1,4,6\}\\
1 &= 1-0\\
2 &= 6-4\\
3 &= 4-1\\
4 &= 4-0\\
5 &= 6-1\\
6 &= 6-0
\end{split}
\end{equation*}


There are algoritms to calculate this exactly, but if we limit the $N$ to a set of values, we can simply use a look-up table to improve performance dramatically.

When the set of $n_i$ is aqcuired we build the filter-matrix $c$ by picking an identity matrix $\mathbf{I}_N$ and choose columns of it with index $n_i$ to get matrix $c$ with size $M \times N$.

\subsection{Implementation}

\subsection{Verification}


\end{document}
