
\documentclass[report, oneside, a4paper, openany]{memoir}
\usepackage[no-math]{fontspec}
\usepackage{pgfplots}
\pgfplotsset{compat=newest}
\usepackage{commath}
\usepackage{mathtools}
\usepackage{amssymb}
\usepackage{amsthm}
\usepackage{booktabs}
\usepackage{mathtools}
\usepackage{xcolor}
\usepackage[separate-uncertainty=true, per-mode=symbol]{siunitx}
\usepackage[noabbrev, capitalize]{cleveref}
\usepackage{listings}
\usepackage[american inductor, european resistor]{circuitikz}
\usepackage{amsmath}
\usepackage{amsfonts}
\usepackage{ifxetex}
\usepackage[dutch,english]{babel}
\usepackage[backend=bibtexu,texencoding=utf8,bibencoding=utf8,style=ieee,sortlocale=en_GB,language=auto]{biblatex}
\usepackage[strict,autostyle]{csquotes}
\usepackage{parskip}
\usepackage{import}
\usepackage{standalone}
\usepackage{hyperref}
%\usepackage[toc,title,titletoc]{appendix}

\ifxetex{} % Fonts laden in het geval dat je met Xetex compiled
    \usepackage{fontspec}
    \defaultfontfeatures{Ligatures=TeX} % To support LaTeX quoting style
    \setromanfont{Palatino Linotype} % Tover ergens in Font mapje in root.
    \setmonofont{Source Code Pro}
\else % Terug val in standaard pdflatex tool chain. Geen ondersteuning voor OTT fonts
    \usepackage[T1]{fontenc}
    \usepackage[utf8]{inputenc}
\fi
\newcommand{\references}[1]{\begin{flushright}{#1}\end{flushright}}
\renewcommand{\vec}[1]{\boldsymbol{\mathbf{#1}}}
\newcommand{\uvec}[1]{\boldsymbol{\hat{\vec{#1}}}}
\newcommand{\mat}[1]{\boldsymbol{\mathbf{#1}}}
\newcommand{\fasor}[1]{\boldsymbol{\tilde{\vec{#1}}}}
\newcommand{\cmplx}[0]{\mathrm{j}}
\renewcommand{\Re}[0]{\operatorname{Re}}
\newcommand{\Cov}{\operatorname{Cov}}
\newcommand{\Var}{\operatorname{Var}}
\newcommand{\proj}{\operatorname{proj}}
\newcommand{\Perp}{\operatorname{perp}}
\newcommand{\col}{\operatorname{col}}
\newcommand{\rect}{\operatorname{rect}}
\newcommand{\sinc}{\operatorname{sinc}}
\newcommand{\IT}{\operatorname{IT}}
\newcommand{\F}{\mathcal{F}}

\newtheorem{definition}{Definition}
\newtheorem{theorem}{Theorem}


\DeclareSIUnit{\voltampere}{VA} %apparent power
\DeclareSIUnit{\pii}{\ensuremath{\pi}}

\hypersetup{%setup hyperlinks
    colorlinks,
    citecolor=black,
    filecolor=black,
    linkcolor=black,
    urlcolor=black
}

% Example boxes
\usepackage{fancybox}
\usepackage{framed}
\usepackage{adjustbox}
\newenvironment{simpages}%
{\AtBeginEnvironment{itemize}{\parskip=0pt\parsep=0pt\partopsep=0pt}
\def\FrameCommand{\fboxsep=.5\FrameSep\shadowbox}\MakeFramed{\FrameRestore}}%
{\endMakeFramed}

% Impulse train
\DeclareFontFamily{U}{wncy}{}
\DeclareFontShape{U}{wncy}{m}{n}{<->wncyr10}{}
\DeclareSymbolFont{mcy}{U}{wncy}{m}{n}
\DeclareMathSymbol{\Sha}{\mathord}{mcy}{"58}

\title{Documentation}

\author{W.P. Bruinsma \and R.P. Hes \and H.J.C. Kroep \and T.C. Leliveld \and W.M. Melching \and T.A. aan de Wiel}

\begin{document}
\chapter{Reconstruction}
The reconstruction uses the output of the sampling block to estimate the power spectral density.

\section{Multicoset Sampling Reconstruction -Ariananda}
\subsection{Concept}
The expected input of this reconstruction method is a multicoset sampled signal of the form 
$$z_l[kN]=y_l[k] \quad l \in \{0,1,\dots, M-1\}$$
Where
\begin{itemize}
\item $M$  = amount of sampling devices
\item $N = \frac{\text{Nyquist rate}}{\text{Sample Frequency}}$
\item $k$ = the sampling index
\item $z[k]$ = sampled signal if on Nyquist
\item $y[k]$ = sampled signal on actual sample rate
\end{itemize}

Now there are three different methods proposed to reconstruct the Power Spectral Density (PSD) of $x[n]$.

\subsubsection{Time-Domain Reconstruction Approach}
We desire to reconstruct $r_x[n]$, the auto correlation of $x[n]$, because this contains enough information to determine the PSD.
We want to pick the $M$ different device measurements, combine these into $M^2$ different cross-correlation functions and use these to reconstruct the auto-correlation with a sampling that is $N$ lower than nyquist. 
This is only possible when $M^2>N$.

For example: if  one would use 10 different measurement devices, one would theoretically be able to reconstruct a PSD with a sampling rate that is 100 times less than nyquist! This would however require a lot of processing power, which limits the application of this algoritm.
$$
r_{yi,yj}[k] = r_{zi,zj}[kN]
$$
here $r_{yi,yj}[k]$ is the cross-corelation function of $y_i[k]$ and $y_j[k]$.
$z[n]$ is the signal we would have if we sampled on nyquist frequency. Therefore $z[n]$ has N times more samples than $y[n]$.
$$
r_{yi,yj}[k] = r_{zi,zj}[kN] = \sum_{m=-N+1}^{N-1}r_{ci,cj}[m]r_x[kN-m] = \sum_{l=0}^1\mathbf{r}^T_{ci,cj}[l]\mathbf{r}_x[k-l]
$$
This is basically a rewrite with new defined variables. Note that $l$ is a boolean. $\mathbf{r}_{ci,cj}[0]$ is the part where $m\leq 0$ and $\mathbf{r}_{ci,cj}[1]$ is the part where $m > 0$ 
$$
\mathbf{r}_{ci,cj}[l] = [r_{ci,cj}[Nl], r_{ci,cj}[Nl-1], \dots, r_{ci,cj}[Nl-N+1]]^T
$$
$$
\mathbf{r}_x[k]= [r_x[kN],r_x[kN+1],\dots,r_x[(k+1)N-1]]^T
$$
Note that $\mathbf{r}_t[k]$ is the auto-correlation function we want to get, consisting of N parts. Recall that we need N parts because we want to sample at a rate N times less than nyquist rate.\\
\\
Now we make a vector $\mathbf{r}_y[k]$ by combining every possible combination of the $M$ auto-correlations into $M^2$ cross-correlations. $\mathbf{r}_y[k]$ is basically a vector containing all available $\mathbf{r}_{yi,yj}[k]$
$$
\mathbf{r}_y[k]=\sum_{l=0}^1\mathbf{R}_c[l]\mathbf{r}_x[k-l]
$$
Again recall that $l$ is a boolean and that $l=0$ covers $m\leq 0$ and $l=1$ covers $m > 0$. $\mathbf{R}_c$ is an $M^2 \times N \times 2$ matrix.
$$
\mathbf{R}_c[l] = [\dots,\mathbf{r}_{ci,cj}[l],\dots]^T \quad l \in \{0,1\}, i,j \in \{0,1, \dots, M-1\}
$$
Note that $\mathbf{r}_{ci,cj}[l]$ is a matrix of $N \times 2$! $\mathbf{R}_c[l]$ is therefore a matrix with elements that are time dependent.\\
\\
Next up we introduce a new variable $L$. This is the amount of samples of $y[k]$ that we use to determine the PSD. This value can be choosen freely, 
but has influence on the accuracy and processing of the algoritm. 
Note that the cross-correlation between two samples of length $L$ has a length of $2L+1$. Because of this, we only need to calculate $\mathbf{r}_x[k]$ for $L<k\leq L$. We can use this to pick $\mathbf{r}_y[k]$ and $\mathbf{r}_x[k]$, and put them into two big vectors.
$$
\mathbf{r}_y = [\mathbf{r}^T_y[0],\mathbf{r}^T_y[1],\dots,\mathbf{r}^T_y[L],\mathbf{r}^T_y[-L],\dots,\mathbf{r}^T_y[-1]]^T
$$
$$
\mathbf{r}_x = [\mathbf{r}^T_x[0],\mathbf{r}^T_x[1],\dots,\mathbf{r}^T_x[L],\mathbf{r}^T_x[-L],\dots,\mathbf{r}^T_x[-1]]^T
$$
where $\mathbf{r}_y$ has size $(2L+1)M^2\times 1$ and $\mathbf{r}_x$ has size $(2L+1)N \times 1$. \\

After this batch of new definitions and variables we finally arrive at the key point of the previous math:
$$
\mathbf{r}_y = \mathbf{R}_c\mathbf{r}_x
$$ 
Where
\begin{itemize}
\item $\mathbf{r}_y$ represents the auto-correlations of the different measurement devices
\item $\mathbf{R}_c$ represents the cross-correlations of the different measurement devices 
\item $\mathbf{r}_x$ represents the auto-correlation of the original signal
\end{itemize}
This equation is solvable for $\mathbf{r}_x$ if $M^2\geq N$. \\
Now we apply LS to solve the equation. We use
\begin{equation*}
\begin{split}
\mathbf{q}_x &= (\mathbf{F}_{2L+1}\otimes \mathbf{I}_N)\mathbf{r}_x \\
\mathbf{q}_y &= (\mathbf{F}_{2L+1}\otimes \mathbf{I}_M)\mathbf{r}_y \\
\mathbf{Q}_c[\omega] &= \sum_{k=0}^1\mathbf{R}_c[k]e^{-jk\omega}
\end{split}
\end{equation*}
Here $\mathbf{Q}_c$ is a diagonalisation of $\mathbf{R}_c$. We use this to rewrite our equation
$$
\mathbf{q}_y = \mathbf{Q}_c\mathbf{q}_x
$$
This is enough to calculate $\mathbf{r}_x$ with the initial input of the block. Finally we calculate the PSD with
$$
\mathbf{s}_x = \mathbf{F}_{(2L+1)N}\mathbf{r}_x
$$ 

\subsubsection{Alternative Time-Domain Approach}
For this alternate method we start with 
$$
\mathbf{y}[k] = \mathbf{Cx}[k]
$$
Where 
\begin{equation*}
\begin{alignedat}{2}
\mathbf{y}[k] &= [y_0[k], y_1[k],\dots,y_{M-1}[k]]^T \quad &\text{size}: M \times 1\\
\mathbf{x}[k] &= [x[kN],x[kN+1],\dots, x[kN+N-1]]^T \quad &\text{size}: N \times 1\\
\mathbf{C} &= [\mathbf{c}_0,\mathbf{c}_1,dots,\mathbf{c}_{M-1}] \quad &\text{size}: M \times N
\end{alignedat}
\end{equation*}
Next, we construct the autocorrelation matrix of $\mathbf{y}[k]$ and $\mathbf{x}[k]$
\begin{equation*}
\begin{alignedat}{2}
\mathbf{R}_y[0] &= E(\mathbf{y}[k]\mathbf{y}^H[k]) \quad &\text{size}: M \times M \\
\mathbf{R}_x^a &= E(\mathbf{x}[k]\mathbf{x}^H[k]) \quad &\text{size}: N \times N
\end{alignedat}
\end{equation*}
Their relation can be expressed as
$$
\mathbf{R}_y[0] = \mathbf{CR}_x^a\mathbf{C}^H
$$
An element of $\mathbf{R}_x^a$ is defined as 
$$[\mathbf{R}_x^a]_{ij} = r_x[i-j]$$ 
Because $x[n]$ is Wide Sense Stationary, $\mathbf{R_x^a}$ is a Toeplitz matrix. 
Recall a Toeplitz matrix is of the form
$$
\begin{bmatrix}
d &c & b & a & \dots \\
e & d & c & b & ~ \\
f & e & d & c & ~ \\
g & f & e & d & ~ \\
\vdots & ~ & ~ & ~& \ddots
\end{bmatrix}
$$
\todo[inline, author = Kees]{waarom is dit zo logisch? :S}

$\mathbf{y}[n]$ is not Wide Sense Stationary, however and therefore $\mathbf{R}_y[0]$ is not necessarily a Toeplitz Matrix.
We can exploit the nature of a Toeplitz matrix by only estimating one column of $\mathbf{R}_x^a$. \\

The operation $vec(.)$ stacks a matrix into one big vector. For example:
$$
vec(\begin{bmatrix}
a & b \\
c & d
\end{bmatrix})
= 
\begin{bmatrix}
a \\ b \\ c \\ d
\end{bmatrix}
$$ 
The operation $\otimes$ is the Kronecker product operation. For example:
$$
\begin{bmatrix}
a \\ b
\end{bmatrix} \otimes
\begin{bmatrix}
c \\ d
\end{bmatrix} = 
\begin{bmatrix}
a\cdot c \\ a\cdot d \\b\cdot c \\ b\cdot d
\end{bmatrix}
$$ 
We now define
$$
\mathbf{r}_y[0] = vec(\mathbf{R}_y[0])
$$

\todo[inline, author=Kees]{Hier gebleven}
The paper states that this method is inferior to the main method in some ways without clearly indicating significant advantages.

\subsubsection{Frequency-Domain Reconstruction Approach}
This method tries to reconstruct the spectrum instead of the PSD.

\subsection{Implementation}

\subsection{Verification}

\end{document}
